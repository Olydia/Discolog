\documentclass{llncs}

\usepackage{subcaption}
\usepackage{subfig} 
\usepackage{usual}
\usepackage{graphicx}
\pagestyle{plain}

%
\begin{document}
\title{Agent utterances in the dialogue}
\maketitle 

The dialogue model constructed so far allows the user to choose any utterance of the five modeled utterances at each speaking slot. For this reason, The agent has to be able de to respond consistently regardless to the utterance chosen by the user. In this report, I present my model of utterances responses that the agent can choose for each type of received utterance. This model take in account the current mental state when the user utterance was received and its perception of the relationship. We focus on the relation of dominance that takes three values: \{Dominant, submissive, peer\}.

\par In the following, I present the agent possible responses for each utterance. 
\section{State Preference (C, less, more)}

	\par A user can state a preference on values of a certain criterion \emph{C}. As explained, the semantic of the utterance allows the speaker to express three different cases:
	\begin{itemize}
		\item StatePreference(C, less, more) : I like "less" less than "more"
		\item StatePreference(C, *, more): I like the most More
		\item StatePreference(C, less, *): I like the least less. 
	\end{itemize}
	
	\par The agent will analyses the value of the received utterance and depending on the perception of the relationship, the agent will choose  an adequate answer. The possibles responses are depicted in Figure 1, and explained in the following: 
	\begin{enumerate}
		\item \textbf{StatePreference (C, less, more)}:The agent can react to the the stated values by observing its preferences on the stated criteria. This utterance is selected if the agent's previous utterance doesn't concern a statement about (less, more). This condition avoid the agent to fall in infinite loop.  Depending on the values of the input preference, the agent will respond with one the the following values. 
			\subitem input: \textit{(*, more)} $\rightarrow$ 
				output: \textit{(more, mostPreferred)} if \textit{more $\not = $ mostPreffered,} else \textit{(*, more)}.
			\subitem input: \textit{(less, *)} $\rightarrow$ 
				output: \textit{(leastPreferred,less)} if \textit{less $\not = $ leastPreffered,} else \textit{(less, *)}.
			\subitem input: \textit{(less, more)} $\rightarrow$ Output:  \textit{(less, more)} if(score(less) $<$ score(more)), \textit{else (more,less)}.
			
		\item \textbf{StatePreference (C, less1, more1)}: The agent can state a preference about other values than expressed in the user utterance provided that the agent is not submissive. In the case of all the preferences related to the criterion \emph{C} are already expressed (all the preferences are in OAS). The agent expresses a preference about another criterion \emph{C1} such that $\emph{C} \not = \emph{C1}$.
		
		\item \textbf{AskPreference(C, less1, more1)}: THis response is chosen in the case where the agent is submissive so he doesn't want to propose new values. Therefore, he uses the utterance ask to invite the user to choose values. In the same way as the previous responses, if the agent is aware of all the user preferences about the criterion \emph{C}, he will ask the user about his preferences about another criterion \emph{C1}.
		
		
		\item  \textbf{Propose(C, more)}: the choice of the propose utterance depends on the perception of the relationship of dominance : 
			\subitem if the agent is not submissive; then  the value "more" of the expressed preference has to match the most preferred value of the agent preferences on the criterion \emph{C} \textit{(i.e more = mostPreferred)}. Thus, the agent constructs a proposal that takes as value its mostPreferred value of  \emph{C}.
			\subitem If the agent is submissive, he prioritizes the user preferences. Thus, if the statement is of the type (StatePreference (*, more)). The agent proposes to choose the value "more" for \emph{C}.
	\end{enumerate}

\section{Ask Preference (C, less, more)}
	\par In the case of user asks the agent about a certain preference. The only possible response is to express its opinion about the asked preference.  The content of the response is generated using a method that calculates the values from the input of the ask utterance. We distinguish the following cases:
		\begin{enumerate}
			\item If \textit{(C, less, more)=(*,*)} thus \textit{reactToAsk(C,less,more) = mostPreferred(C)}. The agent express its most preferred value of the criterion \emph{C}.
				\subitem \textbf{For example :} 
				\subitem U: what kind of cuisine do you prefer ?
				\subitem A: I like Japanese cuisine. (which corresponds to the agent mostPreferred value for the criterion \textit{Cuisine}).
 
			\item  If \textit{(less=*)} and \textit{(more $\not =$ *)}, thus \textit{reactToAsk(*,more)=(*,more) } if \textit{(more=mostPreferred)}, else \textit{reactToAsk(*,more)=(*,mostPreferred)}.
			\subitem For example : 
				\subitem U: Do you like Chinese cuisine ?
				\subitem A: I like Chinese less than Japanese cuisine. (which corresponds to the agent mostPreferred value for the criterion \textit{Cuisine}).
			\item In the same idea, if \textit{(more=*)} and \textit{(less $\not =$ *)}, thus \textit{reactToAsk(less,*)=(less,*)} if \textit{(less=leastPreferred)}, else \textit{reactToAsk(less,*)=(leastPreferred,*)}.
			
		\end{enumerate}

\section{Propose (value)}
	\par When the agent receive a proposal, he can either accept it or reject this proposal. However, we model different ways to express the rejection of a proposal that is influenced by the perception of the relationship. The value of proposal can be either a criterion or an option.
		\begin{enumerate}
			
			\item \textbf{Accept(value)}: A proposal is accepted by the agent only if its score of preference is acceptable. To calculate the acceptance of a proposal, we modeled a simple method that checks if  the score of the value of proposal is acceptable. In the case of a criterion proposal  preference if the score of a criterion is positive then it is acceptable. While options are sorted by their utility rate, and an option is acceptable if its rank in the list is bellow the middle.  Note that, when an option is accepted, the negotiation is closed. 
			
			\item \textbf{Reject(value)}: We suppose that a submissive person is not comfortable with expressing a clear reject. Therefore, we decide that only a non submissive agent can express a reject if the proposed values are not acceptable. 
		
			\item \textbf{StatePreference(C,value,*)}: a submissive agent can express an implicit reject using a statePreference. In the case of a criterion proposal, the proposed value is assigned to the less argument of the preference and the more is calculated from the preference base of the agent, such that $more \notin Rejected$. In the case of option, we extract the criterion that gets the worst score and we generate a preference in the same way than the criterion proposal.
			
			\item \textbf{Propose(value 2)} : This utterance is defined to allow the agent to counter propose another value if the value proposed by the user is not acceptable. This utterance is allowed only for a non submissive agent such that $value2 \not = value$ and $value2 \notin Rejected$. This value can either be a criterion or an option.

\section{Reject (value)}
\par When the agent receive a proposal, he can either accept it or reject this proposal. However, we model different ways to express the rejection of a proposal that is influenced by the perception of the relationship. The value of proposal can be either a criterion or an option.
\begin{enumerate}
	
	\item \textbf{Accept(value)}: A proposal is accepted by the agent only if its score of preference is acceptable. To calculate the acceptance of a proposal, we modeled a simple method that checks if  the score of the value of proposal is acceptable. In the case of a criterion proposal  preference if the score of a criterion is positive then it is acceptable. While options are sorted by their utility rate, and an option is acceptable if its rank in the list is bellow the middle.  Note that, when an option is accepted, the negotiation is closed. 
	
	\item \textbf{Reject(value)}: We suppose that a submissive person is not comfortable with expressing a clear reject. Therefore, we decide that only a non submissive agent can express a reject if the proposed values are not acceptable. 
	
	\item \textbf{StatePreference(C,value,*)}: a submissive agent can express an implicit reject using a statePreference. In the case of a criterion proposal, the proposed value is assigned to the less argument of the preference and the more is calculated from the preference base of the agent, such that $more \notin Rejected$. In the case of option, we extract the criterion that gets the worst score and we generate a preference in the same way than the criterion proposal.
	
	\item \textbf{Propose(value 2)} : This utterance is defined to allow the agent to counter propose another value if the value proposed by the user is not acceptable. This utterance is allowed only for a non submissive agent such that $value2 \not = value$ and $value2 \notin Rejected$. This value can either be a criterion or an option.
				
			
			 
		\end{enumerate}		

\end{document}
\documentclass{llncs}
\usepackage[noend]{algpseudocode}
\usepackage{subcaption}
\usepackage{subfig} 
\usepackage{usual}
\usepackage{graphicx}
\usepackage[rflt]{floatflt}
%\pagestyle{plain}
%
\begin{document}
\title{  \vskip -10pt}

\author{Lydia Ould Ouali\inst{1}, Charles Rich\inst{2} \and
Nicolas Sabouret\inst{1} }

\institute{LIMSI-CNRS, UPR 3251, Orsay, France \\
Universit\'e Paris-Sud, Orsay, France \\
\email{\{ouldouali, nicolas.sabouret\}@limsi.fr}
\and
Worcester Polytechnic Institute\\ Worcester, Massachusetts, USA\\
\email{rich@wpi.edu}
}
\maketitle 
\begin{abstract}\vskip -20pt
  
\end{abstract}

\section{Introduction}
Increasingly new technologies are used to assist humans in their daily life. For example, a robot companion for isolated elder. In order to facilitate the help, theses technologies need to interact with the user. Therefore, they were allowed with conversational skills [always on always]. 
\par In the beginning, conversations with user were task centered (gives directives, ask for information to achieve the task) and the social aspect of dialogue was completely neglected for a long time. However, several researchers proved that social aspect cannot be ignored during the dialogue since the dialogue is social by definition. Moreover, it has be found that users prefer to interact with  conversational agents with social skills \cite{moon}, in addition this skill allows the agent to build a long term relationship with the user\cite{bickmore}. 
Social conversational agents have in addition to their usual task goals, social goals to achieve. Indeed, Bickmore demonstrated  that social goals participate in the satisfaction of task goals. For example, to achieve the task make the user take his medicine, the agent has to satisfy the social goal "put the user in good mood".
different social aspect were studied (trust bickmore, emotions ..)
different research show interest in constructing social agents that take in account the social aspect of dialogue.
explain that the way we interact with a person that we meet newly is different of the way we talk with a close friend or a hierarchical supervisor ...
Thus, the dialogue strategy is directly influenced by the relationship that we construct with the other.  
In this paper we are interested in studying the impact of 
interest: how the social relation ship of the agent with user will influence its dialogue strategies to satisfy its goals
In this article we address a new aspect of user modeling – assessing the psychosocial
relationship between the person and the computer. And we introduce new methods
for adapting the computer’s behavior to the user model, as well as for explicitly and
dynamically changing this relationship through the use of social talk. 

\section{Related works}
\subsection{Interpersonal relationship}
brief presentation on what is interpersonal relationship in dialogue and how it could evolve during dialogue and affect the strategies
\subsection{Dimensions of interpersonal relationship}
present the different dimensions briefly and focus more on the dimension of dominance because it is the one studied in this article
\subsection{Social conversational agents}
 Existing works on social conversational agents
\noindent 
\vskip 4pt
\bibliographystyle{plain}
\bibliography{abbrevs,Library}
\end{document}

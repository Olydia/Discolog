\documentclass{llncs}
\usepackage[noend]{algpseudocode}
\usepackage{subcaption}
\usepackage{subfig} 
\usepackage{usual}
\usepackage{graphicx}
\usepackage[rflt]{floatflt}
%\pagestyle{plain}
%
\begin{document}
\title{  \vskip -10pt}

\author{Lydia Ould Ouali\inst{1}, Charles Rich\inst{2} \and
Nicolas Sabouret\inst{1} }

\institute{LIMSI-CNRS, UPR 3251, Orsay, France \\
Universit\'e Paris-Sud, Orsay, France \\
\email{\{ouldouali, nicolas.sabouret\}@limsi.fr}
\and
Worcester Polytechnic Institute\\ Worcester, Massachusetts, USA\\
\email{rich@wpi.edu}
}
\maketitle 

\section{Interpersonal relationship}
Social relationship and its effects on behavior lies at the heart of social science. It was proved that understanding interpersonal relationship is crucial for social cognition \cite{reis2000relationship}. Most of the literature that get interested in the conceptual analysis of interpersonal relationship have agreed that the essence of relationship appears in the nature of interaction that occurs between relationship partners. Moreover, social relationship is a dynamic system that may develop and change continuously over interactions \cite{reis2000relationship,svennevig2000getting}.
Communication between relationship partner will grow in stages from the initial interaction where partners share superficial information to a more deeper relationship where partners can share more personal information. Therefore, the social relationship of partners affects their behavior and their strategy of dialogue.

\section{Representation of interpersonal relationship}

The aim of this section is to relate the work of N.HASLAM who get interested on the mental representation of social relationship. In summary, there are three different representation in the literature. 
\par The first is the dimensional representation. It is the most common representation that consists on represent relationships in a dimensional circle (c.f wiggins model). Therefore, any relationship can be situated and valuated  in this  \textit{continuous} dimensional space. 

The second representation is the lawful representation. Laws are defined in the same circle's dimension of affiliation and control. The main difference with the dimensional representation is that laws try to make discrete prediction about the other behavior. For each behavior, complementarity and symmetry make discontinuous prediction about the the other interact behavior. 

Finally, categorical representation  make a discrete prediction on which kind of social relationship are well performed. In addition the categorical representation focus only on local prediction ( prediction in a small region within a dimensional scheme).

\begin{tabular}{|c|c|c|}
  \hline
  Dimensions & Laws & Categories \\
  \hline
  	Continuous &   discontinuous   &   discontinuous  \\
 	Local & Global & Local\\
  \hline
\end{tabular}

\subsection{Dimensions of interpersonal relationship}
The definition of dimensions was widely studied under different labels. However, we distinguish four dimensions that are always used for the representation of interpersonal relationship. 
\subsubsection{Dominance and power}
Scholars from different fields converge to define power as the ability to influence the other behavior \cite{svennevig2000getting}. Power may be latent (Komter, 1989), which is in contrast with the definition of dominance which is inevitably manifest (Dunber, 2004). It is an asymmetric variable in which one interactant's assertion of control is met by acquiescence from another (Rogers-Millar \& Millar, 1979). 
\subsubsection{Familiarity}
In Svennevig’s relational model \cite{svennevig2000getting}, the definition of familiarity is based on social penetration theory (Berscheid and Reis, 1998) which describe the grades of relationship evolution through mutual exchange of information both in depth (superficial information to personal and intimate information) and breadth(from narrow to a broad range of personal topic).   
\subsubsection{Affect}
This dimension represent the degree of liking that have one interact for the other. This dimension allows interactants to create personal attachment and improve the relationship of interactants \cite{nicholson2001role}
\subsubsection{Solidarity}
The solidarity dimension is in the opposite of power dimension. It is a symmetrical dimension where two individuals share equal obligations and rights \cite{svennevig2000getting}. Is is identified as ‘like-mindedness’ \cite{bickmore2005establishing} where interactants have the same behaviors and share for example the same preferences.
\subsection{Dialogue utterances}
\par The first utterance is related to the expression of preferences. One of the behavior that shows up during the last experiment is that depending on the role of individual in relationship, he will introduce a discussion on preference in a particular way. For example, a powerful person will explicitly 
talk about its preference. In the contrast a submissive person or individual that shares affect will ask the other if he shares the same preferences. 
To define the utterances semantic, we will use the language definition defined in \cite{sidner}. \textbf{BEL} is short for belief, \textbf{INT} for intention and \textbf{MB} for mutual belief.
 \begin{itemize}
 \item State.Preference(\textit{Pref}) : I like \textit{Pref} 
 \par Agent express \textit{Pref} to user. After communicating the message, the mental state of the agent will be updated as the following: 
  \newline (Bel Agent \textit{Pref})
  \newline (Bel Agent (communicated Agent \textit{Pref} User)).
  
 \item Ask.Preference(\textit{Pref}) : Do you like \textit{Pref} ?
 \newline  Here the intention of the agent is to identify if the user believes pref. 
 \newline (INT Agent (Identify user \textit{Pref}))
 \item Propose.Preference(\textit{Pref}): I think that \textit{Pref} would be great.
  \newline (Bel Agent \textit{Pref})
  \newline(INT Agent (Achieve Agent(Bel User \textit{Pref})))
  \newline (Bel Agent (communicated Agent \textit{Pref} User)).
 \item Accept.Preference(\textit{Pref}): Okay, let's choose \textit{Pref}. After receiving a propose utterance from the user, the agent might accept the proposal. The mental state of the agent is: 
 \newline (MB Agent User \textit{Pref})) 

 \item reject.Preference(\textit{Pref}): Sorry, I would choice something else.
  \newline (Not (Bel Agent \textit{Pref}))
  \newline (Bel Agent (communicated Agent (Not (Bel \textit{Pref})) User)).
  
\par 
  	In the experiment and in the modeling of the first dialogue, the dominant was able to insist and propose several times the same proposition. I'm not sure if a special utterance will be needed or we can simply use the utterance \textit{ProposeStrongly}. 
Dans la définition dans ces utterances. est que l'agent exprime dans ce cas sa dominance. et si c'est le cas, l'expression de sa dominance est implicite. 
 \item StateStrongly.Preference(\textit{Pref}): I really like \textit{Pref}.
 \item ProposeStrongly.Preference(\textit{Pref}): \textit{Pref} is the best choice for you.
 \item AcceptStrongly.Preference(\textit{Pref}): Yes of course. \textit{Pref} is a great choice.
  \item rejectStrongly.Preference(\textit{Pref}): I really don't like \textit{Pref}. I'd rather prefer something else.
 \end{itemize} 
 \subsection{Find utterances in dialogues}	
 \subsection{Synthetic dialogue with utterances}
 Synthetic dialogue to illustrate the language definition. The Agent and the user shared the goal to find a restaurant. The Agent has a predefined list of preferences of type of food (AgentPreferences = \{ +Japanese, +Italian, -spicy food\}) and the agent has no information on user preferences (UserPreferences = \{\}). In this example the Agent is peer with the user. 
 
 
 dialogue without utterances and dialogue only with utterances


\noindent 
\vskip 4pt
\bibliographystyle{plain}
\bibliography{abbrevs,Library}
\end{document}

\documentclass{llncs}
\usepackage[noend]{algpseudocode}
\usepackage{subcaption}
\usepackage{subfig} 
\usepackage{usual}
\usepackage{amsmath}
\usepackage{graphicx}
\usepackage{eulervm}
\usepackage{fontenc}

\usepackage[rflt]{floatflt}
%\pagestyle{plain}
%
\begin{document}
\title{  \vskip -10pt}

\author{Lydia Ould Ouali\inst{1}, Charles Rich\inst{2} \and
Nicolas Sabouret\inst{1} }

\institute{LIMSI-CNRS, UPR 3251, Orsay, France \\
Universit\'e Paris-Sud, Orsay, France \\
\email{\{ouldouali, nicolas.sabouret\}@limsi.fr}
\and
Worcester Polytechnic Institute\\ Worcester, Massachusetts, USA\\
\email{rich@wpi.edu}
}
\maketitle 

\section{Interpersonal relationship}
Social relationship and its effects on behavior lies at the heart of social science. It was proved that understanding interpersonal relationship is crucial for social cognition \cite{reis2000relationship}. Most of the literature that get interested in the conceptual analysis of interpersonal relationship have agreed that the essence of relationship appears in the nature of interaction that occurs between relationship partners. Moreover, social relationship is a dynamic system that may develop and change continuously over interactions \cite{reis2000relationship,svennevig2000getting}.
Communication between relationship partner will grow in stages from the initial interaction where partners share superficial information to a more deeper relationship where partners can share more personal information. Therefore, the social relationship of partners affects their behavior and their strategy of dialogue.

\section{Representation of interpersonal relationship}

The aim of this section is to relate the work of N.HASLAM who get interested on the mental representation of social relationship. In summary, there are three different representation in the literature. 
\par The first is the dimensional representation. It is the most common representation that consists on represent relationships in a dimensional circle (c.f wiggins model). Therefore, any relationship can be situated and valuated  in this  \textit{continuous} dimensional space. 

The second representation is the lawful representation. Laws are defined in the same circle's dimension of affiliation and control. The main difference with the dimensional representation is that laws try to make discrete prediction about the other behavior. For each behavior, complementarity and symmetry make discontinuous prediction about the the other interact behavior. 

Finally, categorical representation  make a discrete prediction on which kind of social relationship are well performed. In addition the categorical representation focus only on local prediction ( prediction in a small region within a dimensional scheme).

\begin{tabular}{|c|c|c|}
  \hline
  Dimensions & Laws & Categories \\
  \hline
  	Continuous &   discontinuous   &   discontinuous  \\
 	Local & Global & Local\\
  \hline
\end{tabular}

\subsection{Dimensions of interpersonal relationship}
The definition of dimensions was widely studied under different labels. However, we distinguish four dimensions that are always used for the representation of interpersonal relationship. 
\subsubsection{Dominance and power}
Scholars from different fields converge to define power as the ability to influence the other behavior \cite{svennevig2000getting}. Power may be latent (Komter, 1989), which is in contrast with the definition of dominance which is inevitably manifest (Dunber, 2004). It is an asymmetric variable in which one interactant's assertion of control is met by acquiescence from another (Rogers-Millar \& Millar, 1979). 
\subsubsection{Familiarity}
In Svennevig’s relational model \cite{svennevig2000getting}, the definition of familiarity is based on social penetration theory (Berscheid and Reis, 1998) which describe the grades of relationship evolution through mutual exchange of information both in depth (superficial information to personal and intimate information) and breadth(from narrow to a broad range of personal topic).   
\subsubsection{Affect}
This dimension represent the degree of liking that have one interact for the other. This dimension allows interactants to create personal attachment and improve the relationship of interactants \cite{nicholson2001role}
\subsubsection{Solidarity}
The solidarity dimension is in the opposite of power dimension. It is a symmetrical dimension where two individuals share equal obligations and rights \cite{svennevig2000getting}. Is is identified as ‘like-mindedness’ \cite{bickmore2005establishing} where interactants have the same behaviors and share for example the same preferences.
\subsection{Dialogue utterances}
In this paper we are interested in  modeling a collaborative negotiation of preferences in the context of social dialogue. The negotiation takes its values during the dialogue when messages are exchanged between interlocutors. A message \emph{M = $\textless i \rightarrow j, s(cont), F  \textgreater$} is defined as triple where \emph{i, j}$\in$ \{agent, user\} are the agents participating in the dialogue,\emph{s $\in \wp$}  is the utterance used to express a message and $\wp = \{ Ask, Propose, Reject, Accept, State\}$  represents the set of utterances that agents can express to exchange messages. 
 $F \in \Im $  where $\Im = \{ Strongly, weakly, yelling, with hesititation ...\}$ is a set of multimodal features that are applied to the utterance to express a personal linguistic style or social move.
 
 \par The social relationship evolves during the interaction and influence this later. We focus on the relation of dominance in this paper. $Dom = \{+, -, =\}$. For example, when $Dom = +$ represents the fact that the agent is dominant and the user is submissive. 
 We assume a social relationship: $2^\wp$ $x$  $context \rightarrow Dom $ that tells which feature says in term of social move. The context represent all the previous knowledge of the speaker. 
 
\subsubsection{Preferences}
\par Now, that  the model of communication is defined, we introduce the notion of negotiation on preferences. First, lets define the domain of preferences. We assume that the agent expresses its preference on a defined object based on one or multiple criteria.   
 \begin{itemize}
 \item Objects $O$ : Set of all possible objects of negotiation. For example, negotiate to find at which restaurant have dinner.
 \item Criteria $C$: represents the criteria or features of preferences on a defined object. For example, we assume that we can choose a restaurant based on one or several of the following criteria = \{cuisine, ambiance, quality of food, price, location\}. Each criterion has its domain of values that we note: $\forall c \in C$, $D_{c}$ is its domain. For example $cuisine = \{chineese, italian\}$.

 \item $\forall o \in O, \forall c \in C$, we define $v(c,o) \in D_{c}$ as the objective value of preference attributed for the object $o$ in the context of the criteria $c$. For example, Ginza is an expensive Japanese restaurant. Thus $v(prix, Ginza) = exprensive$ and $(cuisine, Ginza) = japanese$. 
 
 \item Lets now define interlocutor's preferences. $\forall agent_{i}$ that has to define its preference for an object, for example a restaurant. First, the agent has to know his preferences on quality criteria of this object represented as follows: 
 $Pref_{i}^C $ is a total ordered set of criteria.   $ cuisine \textgreater _{c} price$ means that the criteria of cuisine is more important for the $agent_{i}$  than the price. Second, when the criteria is defined, the agent\emph{i} defines his preferences on the domain of this criteria. Thus, $\forall v_{1} , v_{2} \in D_{C}$, the agent has a  partially ordered preference on these values noted $Pref_{i}^C (v_{1}, v_{2})$  can be represented as $v_{1}>_{C} v_{2}$ . For example, the agent prefers the cuisine criteria, and for the cuisine the $agent_{i}:  Pref_{i}^{cuisine} (Japanese , Chinese)$ , means that  $agent_{i}$ prefers the Japanese cuisine than the Chinese.  Second, once the criteria were defined, $agent_{i}$ can define its preferences on the object "restaurant". Based on what we defined above, we conclude that defining a preference on an object is a multi criteria decision. Assume a function of decision  $Dec$ such that $\forall o_{1}, o_{2} \in O x O, Dec$ define an order of preference on these objects where  $\{o_{1}\prec o_{2}, o_{1} \succ o_{2}, o_{1} \approx o_{2}\}$.   
 \end{itemize}
\subsubsection{Belief base and intentions}
In this section we present the agent belief on his preferences. We situate our definition in the context of social conversation between two people. Suppose $i,j \in \{Agents\}$
\par We note $pref_{i} (c,v)$ to define the degree of agent preference for $v$ on the criteria $c$.

 We define $ B_{i} \varphi$ the agent $i$  belief that an entity $\varphi$ is true. For example, $Agent_{i}$ like Chinese cuisine is represented $ B_{j} Pref_{i}(cuisine, Chinese)$.
 
  $I_{i} \varphi$ means that the agent $i$ has the intention to do or communicate $\varphi$. We define: $U_{i} \varphi = \neg B_{i} \varphi \land \neg B_{i} \neg \varphi$, means that the agent $i$ has no belief on $\varphi$. For example What type of cuisine do you like? is represented $U_{j} Pref_{i}(cuisine,*)$. We define axioms of cooperation that allows an agent to respond to a communication. Thus, $ B_{i} \varphi \land  B_{i} U_{j} \varphi \land  B_{i} I_{j}  B_{j} \varphi \implies I_{i}  B_{j} \varphi  $
 
\subsubsection{Utterances semantic}

 \begin{itemize}
 \item State.Preference(\textit{$P_{i}, P_{j}$}) : I prefer $P_{i}$ to $P_{j}$. We define two variant valuations on stating preferences as follows: 
 \subitem State.Preference(\textit{$P_{i}, *$}): I prefer the most $P_{i}$.
 \subitem State.Preference(\textit{$*, P_{i}$}): I don't like /hate $P_{i}$.

 \item Ask.Preference(\textit{$P_{i}, P_{j}$}) : Do you prefer $P_{i}$ to $P_{j}$ ?. We define two variant valuations as follows: 
 \subitem Ask.Preference(\textit{$P_{i}, *$}): Do you like $P_{i}$?
 \subitem Ask.Preference(\textit{$*$}): What do you like ?. This case appear when the speaker has no belief on the hear preferences. 
 
 \item Propose.Preference(\textit{Pref}): I think that \textit{Pref} would be great.
  
 \item Accept.Preference(\textit{Pref}): Okay, let's choose \textit{Pref}. After receiving a propose utterance from the user, the agent might accept the proposal. The 
 \item Reject.Preference(\textit{Pref}): Sorry, I would choice something else.
 \end{itemize} 
 \subsection{Find utterances in dialogues}	
 
 In the following I represented the utterances in the hand made dialogues. When analyzing the recorded dialogues, the utterances appears in a more implicit manner. For example, when Lauriane says: " Sinon j'aime bien japonais". here its a State.preference(Lauriane, leonor,japonais).Loenor perceives it as a propose and make a reject by saying : "Je n'aime pas du tout le japonais."
 Dialogue	Utterance

 \subsection{Synthetic dialogue with utterances}
 In this section, I present a synthetic dialogue to illustrate the language definition.Yhe goal of the agent is to invite the user to a restaurant. The Agent has a predefined list of preferences of types of food (AgentPreferences = \{ +Indian, +Italian, -Japanese\}) and the agent has no information on user preferences (UserPreferences = \{\}). In this example the Agent is peer with the user. 


\begin{minipage}{0.45\textwidth}
 \begin{enumerate}
	\item A: Would like to have dinner with me ?
	\item U: Yes, that would be great.
	\item A: What kind of food do you prefer ?
	\item U: I like Japanese food
	\item A: Oh, I really don't like japanese food.
	\item U: Ok. What do you prefer ?
	\item A: I like italian food
	\item U: Yeah, I like italian food too.
	\item A: So let's have dinner at an italian restaurant. 
	\item U: perfect for me!
\end{enumerate}
\end{minipage}%
\hfill
\begin{minipage}{0.45\textwidth}
 \begin{enumerate}
	\item Propose.Preference(A,U,dinner).
	\\
	\item Accept.Preference(U,A,dinner).

	\item Ask.Preferences(A,U,UserPreferences)
	\\
	\item State.Preference(U,Japanese)
	\item RejectStrongly.Preference(A,U,Japaneese)
			\\
	\item Ask.Preferences(U,A,AgentPreferences)
	\item State.Preference(A,italian).
	\item State.Preference(U,italian).
	\item Propose.Preference(A,U,Italian).
		\\
	\item Accept.Preference(U,A,Italian).
\end{enumerate}
\end{minipage}%


\noindent 
\vskip 4pt
\bibliographystyle{plain}
\bibliography{abbrevs,Library}
\end{document}

\chapter{Experiments, test and results} % Main chapter title

\label{Chapter 6} % For referencing the chapter elsewhere, use \ref{Chapter1} 

\lhead{Chapter 6. \emph{Experiments, test and results}}
The experimental evaluation was devised in two mains parts, the first consisted on constructing the HTN model to evaluate the Discolog system. The second point was to variate the level of knowledge to test the robustness of  Discolog system against different types of breakdown. 
%\section{experimental environment}

\section{experimental model creation}
In order to test efficiently the Discolog system, we must test the Discolog system on different type of HTN. in the absence of accurate models. we had to create our own evaluation data. Therefore, an algorithm was constructed to generate different types of HTN models. 

The evaluation HTN was constructed using synthetic data as demonstrated in the algorithm ...
\begin{itemize}
	\item Each compound task in the HTN has a set of [$r_1$, ...,$r_\text{recipes}$].
	\item Each recipe is constituted by [$r_1$, ...,$r_\text{length}$] children to decompose the parent task.
	\item the preconditions of the first child are the same as its parent, and the postconditions of the last child are the same as its parent. 
	\item Conditions defined in the primitive tasks are chained in each recipe. Example: the task a is decomposed to \{$a_1$, $a_2$, $a_3$\}using the recipe R1. Thus, the preconditions of $a_1$ are the same as its parent a and the postconditions of $a_3$ are the same as its parent a. We define the postcondition of $a_1$ as "P1" then the preconditions of the next task $a_2$ are "P1", we also define the postconditions of $a_2$ as "P2" then the preconditions of $a_3$ task are "P2".
	\item each recipe has its applicability condition, therefore, we generated primitives tasks whose postconditions turns to true the applicability condition of each recipe, in order to be able to recover from a breakdown caused by an applicability condition failure. 
\end{itemize}
\begin{algorithm}
\caption{DiscoLog algorithm }\label{tree}
\begin{algorithmic}[]
	\Procedure{CreateHTN}{depth,length,recipes,top}
	\State $\textit{ConstructHTNTree(depth,length,recipes,top)} $

	\State $\textit{DefineLevelOfKnwoledge(top, level)} $
	\EndProcedure \textbf{EndProcedure}
	\State
	\State
	
	\Procedure{ConstructHTNTree}{depth,length,recipes,top}
	\If {$\textit{depth} > 1 $}
\For{$ \textbf{each r} \in \text{ recipes} $}
\State $\text{addRecipe(r, top)}$
\For{$ \textbf{each l} \in \text{ length} $}
\State $\text{addchild(top,childl})$
\State $ConstructHTNTree(depth,length,recipes,childl)$
\EndFor
\EndFor
\EndIf
	\EndProcedure \textbf{EndProcedure}
\end{algorithmic}
\end{algorithm}
\subsection{Test algorithm}
The Discolog system was tested on each generated model several times, and for each model, we variate the type of breakdown to recover from and the level of knowledge used in the model. We calculate the percentage of recover relative to the level of knowledge. The generated algorithm is described bellow.
\begin{algorithm}
	\caption{Test algorithm }\label{test}
	\begin{algorithmic}[]
		\Loop \text{ (10) }
\Loop (\text{Random breakdowns } $\gets$ 100)
\State(Discolog(HTN,Goal))

\EndLoop\textbf{EndLoop} \\
\EndLoop\textbf{EndLoop} \\
	\end{algorithmic}
\end{algorithm}
 
\section{ results of the Experiments}
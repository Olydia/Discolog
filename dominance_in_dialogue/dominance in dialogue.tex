\documentclass{llncs}
\usepackage[noend]{algpseudocode}
\usepackage{subcaption}
\usepackage{subfig} 
\usepackage{usual}
\usepackage{graphicx}
\usepackage[rflt]{floatflt}
%\pagestyle{plain}
%
\begin{document}
\title{  \vskip -10pt}

\author{Lydia Ould Ouali\inst{1}, Charles Rich\inst{2} \and
Nicolas Sabouret\inst{1} }

\institute{LIMSI-CNRS, UPR 3251, Orsay, France \\
Universit\'e Paris-Sud, Orsay, France \\
\email{\{ouldouali, nicolas.sabouret\}@limsi.fr}
\and
Worcester Polytechnic Institute\\ Worcester, Massachusetts, USA\\
\email{rich@wpi.edu}
}
\maketitle 
\begin{abstract}\vskip -20pt
  
\end{abstract}

\section{Introduction}
Increasingly new technologies are used to assist humans in their daily life. For example, a robot companion for isolated elder. One condition for a successful application of these technologies is the ability to  interact easily with users. Therefore, these technologies were allowed with conversational skills \cite{sidner2013always}. 
\par In the beginning, conversations with users were task centered (gives directives, ask for information to achieve tasks, etc ...) and the social aspect of dialogue was completely neglected for a long time. However, several researchers proved that social aspect cannot be ignored during the dialogue since the dialogue is social by definition \cite{markopoulos2005case}. Moreover, it has be found that users prefer to interact with  conversational agents which have  social skills \cite{moon1998intimate} and these skills allow the agent to build a long term relationship with the user\cite{bickmore2005establishing}. 
\par Social conversational agents have in addition to their usual task goals, social goals to achieve. Indeed, achieving social goals may contribute to achieve task goals \cite{bickmore2000weather}. For example, to achieve the task \textit{"remind the user take his medicine"}, the agent would first achieve the social goal to \textit{"put the user in good mood"}.

In this paper, we will discuss the impact of interpersonal relationship in the strategy of the agent during the dialogue. The mental representation of social relationship is crucial for social intelligence \cite{haslam1994mental}. Therefore, depending on the interpersonal relationship that the agent will build with the user during the dialogue, it will affect its strategy to achieve its social (and task) goals. The way that we discuss with a person that we newly meet is different from a dialogue with a close friend or a supervisor. 
\par For this paper we will focus on one single dimension of social relationship which is \textit{dominance}. Our interest is to study how the relation of dominance can affect agent decisions during dialogue and the evolution of its strategy and behavior specially when it comes to talk about preferences and negotiate about them with the user. Moreover, as relationship is dynamic and can progress during dialogue, the agent has to adapt its strategy to the observable changes. Let's take an example of an elder  who has dietary restrictions and wants to order diner. The agent knows that the elder hates this dietary restrictions. Thus, the agent defines as goal to make the user respect this food preference. Depending on its relationship with the user, the agent will adopt a specific strategy (food suggestions, linguistic style, ask for user preferences etc..) to find a type of food that the user appreciates and respects the user's diet.



\section{Related works}
\subsection{Interpersonal relationship}
Social relationship and its effects on behavior lies at the heart of social science. It was proved that understanding interpersonal relationship is crucial for social cognition \cite{reis2000relationship}. Most of the literature that get interested in the conceptual analysis of interpersonal relationship have agreed that the essence of relationship appears in the nature of interaction that occurs between relationship partners. Moreover, social relationship is a dynamic system that may develop and change continuously over interactions \cite{reis2000relationship,svennevig2000getting}.
Communication between relationship partner will grow in stages from the initial interaction where partners share superficial information to a more deeper relationship where partners can share more personal information. Therefore, the social relationship of partners affects their behavior and their strategy of dialogue.

\subsection{Representation of interpersonal relationship}

The aim of this section is to relate the work of N.HASLAM who get interested on the mental representation of social relationship. In summary, there are three different representation in the literature. 
\par The first is the dimensional representation. It is the most common representation that consists on represent relationships in a dimensional circle (c.f wiggins model). Therefore, any relationship can be situated and valuated  in this  \textit{continuous} dimensional space. 

The second representation is the lawful representation. Laws are defined in the same circle's dimension of affiliation and control. The main difference with the dimensional representation is that laws try to make discrete prediction about the other behavior. For each behavior, complementarity and symmetry make discontinuous prediction about the the other interact behavior. 

Finally, categorical representation  make a discrete prediction on which kind of social relationship are well performed. In addition the categorical representation focus only on local prediction ( prediction in a small region within a dimensional scheme).

\begin{tabular}{|c|c|c|}
  \hline
  Dimensions & Laws & Categories \\
  \hline
  	Continuous &   discontinuous   &   discontinuous  \\
 	Local & Global & Local\\
  \hline
\end{tabular}

\subsection{Dimensions of interpersonal relationship}
The definition of dimensions was widely studied under different labels. We distinguish four dimensions that are always used for the representation of interpersonal relationship. 
\subsubsection{Dominance and power}
Scholars from different fields converge to define power as the ability to influence the other behavior \cite{svennevig2000getting}. Power may be latent (Komter, 1989), which is in contrast with the definition of dominance which is inevitably manifest (Dunber, 2004). It is an asymmetric variable in which one interactant's assertion of control is met by acquiescence from another (Rogers-Millar \& Millar, 1979). 
\subsubsection{Familiarity}
In Svennevig’s relational model \cite{svennevig2000getting}, the definition of familiarity is based on social penetration theory (Berscheid and Reis, 1998) which describe the grades of relationship evolution through mutual exchange of information both in depth (superficial information to personal and intimate information) and breadth(from narrow to a broad range of personal topic).   
\subsubsection{Affect}
This dimension represent the degree of liking that have one interact for the other. This dimension allows interactants to create personal attachment and improve the relationship of interactants \cite{nicholson2001role}
\subsubsection{Solidarity}
The solidarity dimension is in the opposite of power dimension. It is a symmetrical dimension where two individuals share equal obligations and rights \cite{svennevig2000getting}. Is is identified as ‘like-mindedness’ \cite{bickmore2005establishing} where interactants have the same behaviors and share for example the same preferences.
\subsection{Dialogue acts and dialogue strategies}
 \subsubsection{Propose/ Ask:} One of the behavior that come up during the experiment is that depending on the role of individual in relationship, he will introduce a discussion on preference in a particular way. For example, a powerful person will explicitly 
talk about its preference. In the contrast a submissive person or individual that shares affect will ask the other if he shares this preference. 
\subsubsection{Accept/ Reject:} For these acts, It is still unclear how to model them. Depending on the relationship the Accept/Reject formulation will change. 
The interest of these acts is the way that can reflect the strategy depending on the relationship. Therefore, I think that this act should be adapted to the relationship. For example the propose act is more a dominant "behavior" and ask is a submissive" behavior. 

\noindent 
\vskip 4pt
\bibliographystyle{plain}
\bibliography{abbrevs,Library}
\end{document}

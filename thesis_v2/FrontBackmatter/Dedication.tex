%*******************************************************
% Dedication
%*******************************************************
\chapter{Remerciements}
\thispagestyle{empty}
\phantomsection
\pdfbookmark[1]{Remerciements}{Remerciements}

\vspace*{3cm}

\epigraph{L’évolution n’est pas une simple éclosion sans peine et sans lutte, comme celle de la vie organique, mais le travail dur et forcé sur soi même.}{\textit{    --- Friedrich Hegel	}}

\medskip

Nombreux les doctorants qui m'ont prévenus que la thèse ne serait pas un simple projet sur lequel on pouvait avancer facilement. 4 ans après je les rejoins, la thèse est un projet tumultueux d'où on en sort grandit professionnellement et humainement.

Pour cela je tiens à remercier les personnes qui m'ont aidé tout au long de ce parcours.

Tout d'abord je voudrais remercier Catherine Pelachaud et Alexande Pauchert d’avoir accepté de relire cette thèse et d’en être rapporteurs. Leurs lectures attentives et de leurs remarques précieuses ont beaucoup participé à l'amélioration du manuscrit. Je tiens à remercier Sophie Rosset d’avoir accepté d’être présidente du jury. Je remercie également tous les membres du jury, Sylvie Pesty, d’avoir accepté d’assister à la présentation de ce travail, particulièrement Candace Sidner qui s’est déplacée depuis Boston.

Je tiens tout particulièrement à Remercier Nicolas Sabouret qui a dirigé mes travaux tout au long de ces quatre années de thèse. 
Il a toujours été disponible, malgré un planning très chargé, il a su être à l’écoute de mes nombreuses questions, et s’est toujours intéressé à l’avancée de mes travaux. 
Merci de m'avoir toujours encouragé durant les moments de doutes, d'avoir trouvé le temps de travailler avec moi quand j'en avais besoin. 

J'ai beaucoup appris en travaillant avec toi, tant sur la rigueur du travail et nous savons que je partais de loin, que sur la réflexion que demande la recherche. 
Cette thèse n'aurait pas abouti sans ton encadrement. 

Je remercie aussi mon co-encadrant Charles Rich \emph{Chuck}. C'est avec émotion que je le remercie aujourd'hui après qu'il nous ait quitté l'an dernier. 
Bien que tu m'encadrais à distance, tu as consacré beaucoup de temps à suivre l’évolution de ma thèse, nos réunions hebdomadaires du mercredi en témoignent.  Nos discussions m’ont tellement aidé à avancer sur ma thèse, à me reconcentrer sur le cœur de la thèse quand souvent je me perdais, et en prime, à perfectionner mon anglais. Tes nombreuses relectures et corrections de mon anglais m’ont beaucoup aidé et ont participé à la nomination d’un papier au prix de « Best paper ».
J’ai étais chanceuse d’avoir des encadrants qui m’ont toujours poussé à aller de l’avant, qui m’ont tant appris durant mon parcours et a qui je dois l’aboutissement de cette thèse.

Après avoir passée cinq ans au Limsi, j’ai eu le temps de connaitre des collègues que je voudrais remercier. Je tiens à remercier les anciens doctorants, Asma, Léonor et Caroline qui m’ont conseillé durant le début de thèse et m’ont rassuré.
Je tiens aussi à remercier Guillaume, l’expert stat qui m’a tout appris sur le fantastique monde des stats. Alya et Delphine pour toutes nos discussions et nos pauses Gouter / shopping. Gibet, Adrien et Léo pour toutes les pauses café.
Merci à tous les membres de l’équipe CPU qui ont rendu la venue jusqu’à Orsay supportable. 
Merci à Laurence, Carole, Sophie, Sophie Havard, Bénédicte, Anne, Stéphanie pour leur disponibilité et leur aide sur l’administratif.

Sur le plan personnel, je tiens d’abord à remercier ma famille pour son soutien inconditionnel durant tout mon parcours universitaire. Je vous doit absolument tout. 

J’aimerais aussi remercier mes amis de Paris (Macylia, Tahar, Khalil et Sonia) et d’Alger (Lounna, Manel et Wissam) qui ont toujours été à mon écoute et qui m’ont fait sortir la tête de la thèse afin d’oublier le travail le temps d’une soirée. 

Enfin, je tiens à remercier mon meilleur ami et mon mari Amine. Sans ton soutien et tes encouragements je ne sais pas si j’aurais supporté d’aller au bout de cette thèse. Merci pour ton soutien moral, ta présence et aussi ton aide d’informaticien, statisticien relecteur, cuisinier … et j’en passe.  Cette thèse et moi te devons beaucoup. Merci.
\chapter[Relation interpersonnelle de dominance]{Modèle de la relation interpersonnelle de dominance}
\label{chap:Tom}
\begingroup
\parindent=0em
\etocsettocstyle{\rule{\linewidth}{\tocrulewidth}\vskip0.5\baselineskip}{\rule{\linewidth}{\tocrulewidth}}
\localtableofcontents 
\clearpage
\endgroup

Comme nous l'avion mentionné dans l'introduction, la principale contribution de cette thèse est d'étudier l'impact d'une relation interpersonnelle de dominance sur les stratégies de négociations entre un agent conversationnel et un humain. 
Pour ce faire, l'agent doit être en mesure de simuler une relation de dominance entre lui et l'utilisateur humain. Nous proposons dans ce chapitre un algorithme pour simuler cette relation de dominance. 

A partir de notre modèle de négociation collaborative présenté dans la chapitre \ref{chap:dec}, nous proposons une extension qui permet à l'agent de raisonner sur les comportements de dominance de son interlocuteur, et d'automatiquement adapter ses comportements à ceux perçus chez son interlocuteur dans le but de créer une relation complémentaire de dominance.

Dans la section 1, nous présentons brièvement le modèle théorique de la théorie de l'esprit sur lequel se base notre approche pour raisonner sur les comportements de l'interlocuteur. Dans la suite, nous présentons une première solution naïve et nous discuterons ses limites. Ensuite, dans la section 3, nous détaillerons notre solution et nous présenterons dans la section 5 son évaluation.  




\section{Croyances sur l'autre: Théorie de l'esprit}
La théorie de l'esprit (ToM) est un processus cognitif permettant d'attribuer des états mentaux à autrui. En effet, un individu doté de ToM est capable d'attribuer
des états mentaux tels que des croyances, intentions et désirs à autrui afin de mieux comprendre, expliquer, prédire ou manipuler leurs comportements  \cite{harbers2012modeling}. Il a été démontré que la ToM est un concept crucial dans la compréhensions des interactions humaines. Pour cette raison, une large communauté étudie les mécanismes de la ToM. Cependant, un débat subsiste sur la nature des mécanismes de théorie de l’esprit partagé entre deux approches: une approche \textit{théorie-théorie}  \emph{(TT)} et une approche \textit{simulation théorie} \emph{(ST)}.
Les défenseurs de la théorie-théorie(TT) postulent que la ToM s’appuie sur une représentation implicite du raisonnement de l'autre construite à partir d'un ensemble de concepts tels que les désirs, les croyances ou les plans de l'autre \cite{harbers2012modeling}. En effet, cette théorie se base sur une représentation basique de l'environnement \emph{"folk theory"}. A partir de comportements observés chez l'autre, l'individu fait des inférences théoriques sur ses états mentaux \cite{shanton2010simulation}.

En revanche, les partisans de simulation théorie(ST)  suggèrent que  que les humains ont la capacité de se projeter dans la perspective d'une autre personne \cite{shanton2010simulation}.
Par conséquent, ils peuvent simuler l'activité mentale d'autrui avec leurs propres capacités de raisonnement pratiques. Cela leurs permet d'imiter l'état mental de leurs partenaire interactionnel \cite{harbers2009modeling}.
Par ailleurs, cette approche stipule qu'il n'est pas pas nécessaire d'être capable d'introspection complète de l'autre pour simuler ses processus mentaux. En d'autres mots, il n'est pas nécessaire de catégoriser toutes les croyances et les désirs attribués à cette personne pour pouvoir simuler son état mental \cite{harbers2012modeling}.

Enfin, les partisans de la \emph{ST} ajoutent que simulation est plus efficace que l'acquisition d'une théorie complète. Pour ces raisons, certains partisans de la théorie-théorie admettent qu'au moins une certaine forme de simulation doit avoir lieu lorsque les gens raisonnent au sujet d'autrui, et incorporent des aspects de simulation dans une approche théorie-théorie \cite{harbers2012modeling}.

Dans le contexte de cette thèse, nous utiliserons l'approche de \emph{simulation théorie (ST)} qui permet d'utiliser le modèle de décision de l'agent pour raisonner sur les comportements de dominance de l'interlocuteur. 

Dans la suite de ce chapitre, nous présentons notre algorithme pour simuler le comportement de l'interlocuteur.

%A cause de ces divergences de représentation, plusieurs propositions ont été faite pour représenter la théorie de l'esprit comme un mix entre la théorie-théorie et simulation-théorie. 

\section{Approche naïve}
	Notre but est de proposer un modèle de négociation collaborative capable de simuler une relation interpersonnelle de dominance. La relation de dominance étant complémentaire, l'agent doit être capable de prédire les comportements de dominance de son interlocuteur afin d'adopter un comportement complémentaire comme présenté dans la figure. Nous proposons une solution basé sur la \emph{ST}, pour laquelle nous adaptons le modèle décisionnel de l'agent pour raisonner sur les comportements de son interlocuteur. 
	
	Le modèle décisionnel proposé est régi par l'état mental: les préférences du négociateur ainsi que sa position dans le spectre de dominance. Une proposition naïve consisterait à formuler des hypothèses sur l'état mental de l'interlocuteur, pour chaque hypothèse, appeler le modèle décisionnel de l'agent pour simuler les réponses possibles dans le contexte courent de la négociation. Ensuite, nous comparons les réponses produites par la simulation avec l'acte de dialogue de l'utilisateur $utterance_{other}$ produite dans l'étape 3 (voir la figure). 
	La dernière étape consiste à mettre à jours les hypothèses jusqu'à converger à une valeur de dominance précise. 
	
	Cette approche repose toutefois sur quelques hypothèses fortes. Tout d'abord, nous supposons que le modèle de décision est une représentation précise du processus décisionnel de l'utilisateur. Il n'y a aucun moyen de garantir cette hypothèse. Cependant, dans le chapitre \ref{chap:dec}, nous avons montré que les comportements de dominance exprimés par les agents sont correctement perçus par les utilisateurs humains. La seconde hypothèse repose sur la capacité de notre système à générer toutes les hypothèses sur l'état mental de l'interlocuteur pour n'importe quel sujet de négociation. C'est à dire, pour chaque hypothèse sur une valeur de dominance, générer l'ensemble de modèles de préférences $\prec_i$ pour chaque critère.
	
	Sur la base de ces hypothèses, nous présentons l'algorithme général du modèle d'état mental de l'utilisateur comme suit:

		\begin{enumerate}
			\item Construire l'ensemble $H_{dom}$ des hypothèse sur la valeur de dominance: $h\in H_{dom}$ représente l'hypothèse $dom=h$. Nous considérons 9 valeurs de dominance: $H_{dom}=\{0.1, 0.2, \ldots, 0.9\}$.
			\item Pour chaque hypothèse $h$, construire l'ensemble des préférences possibles $Prec_h$: les éléments $p\in Prec_h$ sont des préférences d'ordre partiels définis sur les critères du sujet de négociation.
			
			\item Après chaque acte de dialogue reçu $u$, supprimer tout les élements de $Prec_h$ qui ne sont pas compatibles avec $u$. Concrètement, si la condition d'applicabilité d' $u$ n'est pas satisfaite dans $p \ dans Prec_h$, alors $p$ doit être retiré des états mentaux candidats.
			\item Pour chaque $h$, générer l'acte de dialogue correspondant en utilisant $h$ comme état mental pour le processus décisionnel.
			\item Calculer le $score(h)$ comme le nombre d'hypothèses restantes $|Prec_h|$ générant le même acte de dialogue que celle produite par l'interlocuteur $Utterance_{other}$. 
			\item 	L'hypothèse avec le score le plus élevé est est considéré comme la plus proche de l'état mental de l'interlocuteur.
			$$dom_{other} = \operatorname*{arg\,max}_{h} (score(h))$$
		\end{enumerate}
		
		\subsection{Limites de l'approche naïve}
			L'approche naïve repose sur deux hypothèses fortes, la première sur la validité du processus décisionnel. Cependant, comme le préconise la \emph{Simulation théorie}, se projeter à la place d'autrui est une solution plausible pour raisonner sur l'autre.
			
			La seconde hypothèse repose sur la capacité de notre système a générer toutes les hypothèses et à les réviser à chaque tour de parole en temps réel. Cependant, la génération de toutes les relations de préférences est coûteuse. En effet, afin de générer l'ensemble des préférences pour chaque hypothèse, nous devons considérer tout les ordres partiels $\prec_i$ pour chaque critère $C_i$.
			
			Nous pouvons calculer la taille de l'ensemble des relations de préférences binaires en fonction du nombre de valeurs, fixé à $(|C_i|+1)! $ ordre partiel possibles pour chaque critère. Par conséquent, pour un sujet de négociation avec $n$ critères , il y a $\prod_{i=1}^n (|C_i|+1)!$ ensembles de préférences possibles.
			
			

\section{Approche adaptative: Simulation de l'autre}
	
\section{Évaluation}

\section{Exemple}


Approche computationnelle

Exemples

Etude

conculsion
\begin{appendix}

\chapter{Dialogues générés pour l'étude Agent/Agent}
\label{annexe:crowd}
	Nous avons proposé quatre dialogues avec des initialisations différentes concernant les valeurs de dominance et les préférence respectives des agents A et B
	
	\subsection{Dialogue 1}
		\begin{minipage}{\textwidth}
			{\ttfamily
				\begin{addmargin}[1em]{2em}% 1em left, 2em right
					A: "Let's go to a Chinese restaurant."
					
					\hspace*{3mm}B: "I don't like Chinese restaurants, let's choose something else."
					
					A: "Let's go to the Shanghai. It's a quiet, cheap Chinese restaurant on the south side."
					
					\hspace*{3mm}B: "Do you like Italian restaurants?"
					
					A: "I don't like Italian restaurants."
					
					\hspace*{3mm}B: "Do you like French restaurants?"
					
					A: "Let's go to the Shanghai. It's a quiet, cheap Chinese restaurant on the south side."
					
					\hspace*{3mm}B: "Do you like French restaurants?"
					
					A: "I don't like French restaurants."
					
					\hspace*{3mm}B: "Do you like Korean restaurants?"
					
					A: "Let's go to a cheap restaurant."
					
					\hspace*{3mm}B: "Okay, let's go to a cheap restaurant."
					
					A: "Let's go to a restaurant on the south side."
					
					\hspace*{3mm}B: "Okay, let's go to a restaurant on the south side."
					
					A: "Let's go to the Shanghai. It's a quiet, cheap Chinese restaurant on the south side."
					
					\hspace*{3mm}B: "Okay, let's go to the Shanghai restaurant."
					
				\end{addmargin}
			}

		\end{minipage}
	
	\subsection{Dialogue 2}
	
	\begin{minipage}{\textwidth}
		{\ttfamily
			\begin{addmargin}[1em]{2em}% 1em left, 2em right
				A: "Let's go to a Chinese restaurant."
				
				\hspace*{3mm}B: "I don't like Chinese restaurants, let's choose something else."
				
				A: "Let's go to a cheap restaurant."
				
				\hspace*{3mm}B: "Do you like expensive restaurants?"
				
				A: "I don't like expensive restaurants."
				
				\hspace*{3mm}B: "Do you like affordable restaurants?"
				
				A: "Let's go to a cheap restaurant."
				
				\hspace*{3mm}B: "Do you like affordable restaurants?"
				
				A: "I don't like affordable restaurants."
				
				\hspace*{3mm}B: "What kind of atmosphere do you like?"
				
				A: "Let's go to a cheap restaurant."
				
				\hspace*{3mm}B: "Okay, let's go to a cheap restaurant."
				
				A: "Let's go to the Sap. It's a quiet, cheap Japanese restaurant on the south side."
				
				\hspace*{3mm}B: "Okay, let's go Sap restaurant".
				
			\end{addmargin}
		}
	\end{minipage}
	
		
	\subsection{Dialogue 3}
	
		\begin{minipage}{\textwidth}
			{\ttfamily
				\begin{addmargin}[1em]{2em}% 1em left, 2em right
					
					A: "Let's go to a Chinese restaurant."
					
					\hspace*{3mm}B: "I don't like Chinese restaurants, let's choose something else."
					
					A: "Let's go to a cheap restaurant."
					
					\hspace*{3mm}B: "Do you like expensive restaurants?"
					
					A: "I don't like expensive restaurants."
					
					\hspace*{3mm}B: "Do you like affordable restaurants?"
					
					A: "Let's go to a cheap restaurant."
					
					\hspace*{3mm}B: "Okay, let's go to a cheap restaurant."
					
					A: "Let's go to the Sap. It's a quiet, cheap Japanese restaurant on the south side."
					
					\hspace*{3mm}B: "Okay, let's go to the Sap restaurant."
					
					
				\end{addmargin}
			}
		\end{minipage}
		
	
	\subsection{Dialogue 4}
		
		\begin{minipage}{\textwidth}
			{\ttfamily
				\begin{addmargin}[1em]{2em}% 1em left, 2em right
					A: "Let's go to a restaurant on the north side."
					
					\hspace*{3mm}B: "Okay, let's go to a restaurant on the north side."
					
					A: "Let's go to the Paris bistro . It's a romantic, cheap French restaurant on the north side."
					
					\hspace*{3mm}B: "Okay, let's go to the Paris bistro restaurant."
					
					
				\end{addmargin}
			}
		\end{minipage}
		
		
		
\chapter{Complémentarité Vs Similarité dans la relation de dominance}	
\label{chap:Annexe}
\section{Gain commun atteint dans la négociation}

	\subsection{Perception du gain commun}

\begin{table} [h]
	\centering
	\begin{tabular}{ |p{2cm}| p{2cm} |p{2cm}| p{2cm}| p{2cm}|}
		\cline{3-5}
		\multicolumn{2}{c|}{ } & \textbf{Comp. > Similaire} & \textbf{Comp. > Neutre} & \textbf{Neutre > Similaire} \\ 
		\hline
		\multirow{3}{*} {Équité}  &  Z-Wilcoxon &  - 0.23&- 0.34 & - 0.06 \\ 	
		& p-value & 0.47 & 0.4 & 0.43\\ 
		\hline
		
		\multirow{3}{*} {Satisfaction}  &   Z-Wilcoxon  &  - 0.67& - 0.22& - 0.69 \\ 	
		& p-value & 0.24& 0.41& 0.29 \\ 
		\hline

	\end{tabular}
	\caption{Analyse du gain commun atteint par tous les agents}
\end{table}

\subsection{Satisfaction du choix final}

	\begin{table}[h]
		
		\begin{tabular}{ c c c c }
			\hline\hline
			 & \textbf{Comp. >Similaire} & \textbf{Comp. >Neutre} & \textbf{Neutre >Similaire} \\ 
			\hline \hline
			
				T-test  & 8,9 & 6,4 &  2,3 \\ 	
				p-value & 1,74E-12 &  2,884E-08 & 0,025  \\ 
			\hline
			\hline
		\end{tabular}
		\caption{Analyse du gain commun atteint par tous les agents}
	\end{table}
\vspace{-1 em}

\section{Tours de paroles par négociation}
\begin{table}[h]
	
	\begin{tabular}{ c c c c }
		\hline\hline
		& \textbf{Comp. < Similaire} & \textbf{Comp. < Neutre} & \textbf{Neutre < Similaire} \\ 
		\hline\hline
		
		T-test  & 3.33 & 1.49 &  5.56 \\ 	
		p-value & 0.0007 &  0,06 & 2.595e-07 \\ 
		\hline
		\hline	
	\end{tabular}
	\caption{Analyse de la différence entre le nombre de tours de paroles}
\end{table}


\section{Appréciation et confort durant la négociation}

\begin{table*}[h]
	\begin{adjustbox}{angle=90}
	\begin{tabular}{ l c p{3 cm} p{3 cm} p{3 cm} }

		\hline\hline
		\textbf{ }& & \textbf{Comp.>Sim.} & \textbf{Comp.>Neutre} & \textbf{Neutre>Sim.} \\ 
		\hline
		
		\multirow{3}{*} {Appréciation}  & Z-Wilcoxon& -3,17 & -0,32 & -3,29	\\ 	
										& p-value &	0,0005 & 0,6298 & 0,0003  \\ 
										& Effect size & -0,20 & -0,02 & -0,21 \\ 
		\hline
		
		\multirow{3}{*} {Confort}  &Z-Wilcoxon& -2,74 & -0,81 & -2,62 \\
									& p-value & 0,0026 & 0,2 & 1 \\ 
									& Effect size & -0,18 & -0,05 & -0,17 \\ 
		
		
		\hline\multirow{3}{*} {Collaboration}  &  Z-Wilcoxon  & -3,86 & -0,91 & -3,61\\ 	
		& p-value & 4,28E-05 & 0,1718 & 0,0001\\ 
		& Effect size & -0,35 & -0,08 & -0,33 \\ 
		\hline
		\hline
		
	\end{tabular}
	\caption{Les scores d'appréciation pour tous les agents}
	\end{adjustbox}
\end{table*}

\end{appendix}



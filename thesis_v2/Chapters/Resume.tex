%*******************************************************
% Abstract
%*******************************************************
%\renewcommand{\abstractname}{Abstract}
\pdfbookmark[1]{Abstract}{Abstract}
% \addcontentsline{toc}{chapter}{\tocEntry{Abstract}}
\begingroup
\let\cleardoublepage\relax
\let\cleardoublepage\relax


\chapter*{Résumé}
\begin{otherlanguage}{french}
L'essor des travaux en informatique affective voit la naissance de diverses questions de recherches pour étudier les interactions agents /humains. Parmi elles, se pose la question de l'impact des relations interpersonnelles sur les stratégies de communications. 
Les interactions prennent généralement place dans des environnements collaboratifs dans lesquels l'agent et l'utilisateur humain partagent des buts communs. La relation interpersonnelle que les individus créent durant leurs interactions affecte leurs stratégies de communications. Par ailleurs, des individus qui collaborent pour atteindre un but commun sont généralement amené à négocier. Ce type de négociation permet aux négociateurs d'échanger des informations et leurs expertises afin de mieux collaborer.

L'objectif cette thèse est d'étudier l'impact de la relation interpersonnelle de dominance sur les stratégies de négociation collaborative entre un agent et un humain. 
Ce travail se base sur les travaux en psychologie sociale pour définir la manifestation de la dominance dans une négociation.
Nous proposons un modèle de négociation collaborative dont le modèle décisionnel est régi par la relation de dominance.
En effet, en fonction de sa position dans le spectre de dominance, l'agent est capable d'exprimer une stratégie de négociation spécifique.
En parallèle, l'agent simule une relation interpersonnelle de dominance avec son interlocuteur. Pour ce faire, nous avons doté l'agent d'un modèle de théorie de l'esprit qui permet à l'agent de raisonner sur le comportements de son interlocuteur afin de prédire sa position dans le spectre de dominance. Ensuite, il adapte sa stratégie de négociation vers une stratégie complémentaire à celle détectée chez son interlocuteur. 

Nos résultats ont montré que les comportements de dominance exprimés par notre agent sont correctement perçus. De plus, la simulation de la relation interpersonnelle de dominance a un impact positif sur la négociation: les négociateurs atteignent de bon taux de gains communs. De plus, la relation de dominance augment le sentiment d'appréciation entre les négociateurs et la négociation est perçue comme confortable.

\end{otherlanguage}


\vfill
\clearpage
\chapter*{Abstract}

The rise of work in affective computing sees the emergence of various research questions to study agent / human interactions. Among them,raises the question of the impact of interpersonal relations on the strategies of communication.
Human/agent interactions usually take place in collaborative environments in which the agent and the user share common goals.

The interpersonal relationship which individuals create during their interactions affects their communications strategies. Moreover, individuals who collaborate to achieve a common goal are usually brought to negotiate. This type of negotiation allows the negotiators to  exchange information and their expertise in order to better collaborate.

The objective of this thesis is to study the impact of the interpersonal relationship of dominance on collaborative negotiation strategies between an agent and a human.
This work is based on work in social psychology to define the manifestation of dominance in a negotiation.
We propose a collaborative negotiation model whose decision model is governed by the dominance relation.
Indeed, depending on its position in the dominance spectrum, the agent is able to express a specific negotiation strategy.
In parallel, the agent simulates an interpersonal relationship of dominance with his interlocutor. To this aim, we provided the agent with a model of theory of mind that allows him to reason about the behaviour of his interlocutor in order to predict his position in the dominance spectrum. Afterwards, the agent adapts his negotiation strategy to complement the negotiation strategy detected in the interlocutor.

Our results showed that the dominance behaviours expressed by our agent are correctly perceived by human participants. In addition, the simulation of the interpersonal relationship of dominance has a positive impact on the negotiation: the negotiators reach a good rate of common gains and the negotiation is perceived comfortable and appreciable.


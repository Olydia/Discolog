\chapter{Conclusion}

Néanmoins, il est nécessaire de garder à l’esprit que l’expérimen-
tation informatique n’est pas une "validation" en soi d’un modèle
psychologique. Si l’implémentation d’un processus n’amène pas aux
comportements prédits par la théorie, cela permet de pointer d’éven-
tuels problèmes au sein de ladite théorie (ou dans l’interprétation qui
en a été faite par le concepteur du programme informatique). Mais
en cas de succès, il faut être conscient qu’il peut exister des modèles
concurrents en psychologie qui pourraient également amener aux
mêmes comportements observables. De même, le succès d’une étude
perceptive (e. g. un agent virtuel est perçu comme possédant la carac-
téristique implémentée) n’est pas une "validation" en soi d’un choix
d’implémentation. En effet, une implémentation différente (potentiel-
lement même dénuée de base théorique) pourrait peut-être entraîner
la même perception chez l’utilisateur. L’intérêt de l’expérimentation
ici ne réside donc pas dans la "validation" en tant que telle, mais dans
la réflexion bidirectionnelle qu’elle nourrit (aussi bien par rapport au
concept implémenté que par rapport aux choix computationnels qui
ont été faits).
Notre deuxième objectif est donc de proposer une méthodologie
permettant d’articuler modèle psychologique et modèle informatique
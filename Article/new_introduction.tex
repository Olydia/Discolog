%% bare_conf.tex
%% V1.3
%% 2007/01/11
%% by Michael Shell
%% See:
%% http://www.michaelshell.org/
%% for current contact information.
%%
%% This is a skeleton file demonstrating the use of IEEEtran.cls
%% (requires IEEEtran.cls version 1.7 or later) with an IEEE conference paper.
%%
%% Support sites:
%% http://www.michaelshell.org/tex/ieeetran/
%% http://www.ctan.org/tex-archive/macros/latex/contrib/IEEEtran/
%% and
%% http://www.ieee.org/

%%*************************************************************************
%% Legal Notice:
%% This code is offered as-is without any warranty either expressed or
%% implied; without even the implied warranty of MERCHANTABILITY or
%% FITNESS FOR A PARTICULAR PURPOSE! 
%% User assumes all risk.
%% In no event shall IEEE or any contributor to this code be liable for
%% any damages or losses, including, but not limited to, incidental,
%% consequential, or any other damages, resulting from the use or misuse
%% of any information contained here.
%%
%% All comments are the opinions of their respective authors and are not
%% necessarily endorsed by the IEEE.
%%
%% This work is distributed under the LaTeX Project Public License (LPPL)
%% ( http://www.latex-project.org/ ) version 1.3, and may be freely used,
%% distributed and modified. A copy of the LPPL, version 1.3, is included
%% in the base LaTeX documentation of all distributions of LaTeX released
%% 2003/12/01 or later.
%% Retain all contribution notices and credits.
%% ** Modified files should be clearly indicated as such, including  **
%% ** renaming them and changing author support contact information. **
%%
%% File list of work: IEEEtran.cls, IEEEtran_HOWTO.pdf, bare_adv.tex,
%%                    bare_conf.tex, bare_jrnl.tex, bare_jrnl_compsoc.tex
%%*************************************************************************

% *** Authors should verify (and, if needed, correct) their LaTeX system  ***
% *** with the testflow diagnostic prior to trusting their LaTeX platform ***
% *** with production work. IEEE's font choices can trigger bugs that do  ***
% *** not appear when using other class files.                            ***
% The testflow support page is at:
% http://www.michaelshell.org/tex/testflow/



% Note that the a4paper option is mainly intended so that authors in
% countries using A4 can easily print to A4 and see how their papers will
% look in print - the typesetting of the document will not typically be
% affected with changes in paper size (but the bottom and side margins will).
% Use the testflow package mentioned above to verify correct handling of
% both paper sizes by the user's LaTeX system.
%
% Also note that the "draftcls" or "draftclsnofoot", not "draft", option
% should be used if it is desired that the figures are to be displayed in
% draft mode.
%
\documentclass[conference]{IEEEtran}
\usepackage{blindtext, graphicx}
\usepackage{algorithm,algpseudocode}
%\usepackage{color} 
%\usepackage{amsmath}
%\usepackage{algorithm}
%\usepackage{amsmath}
%\usepackage{subfigure}
%%\usepackage[notref,notcite]{showkeys}  % use this to temporarily show labels
%\usepackage{overcite}
%\usepackage{footnpag}
% Add the compsoc option for Computer Society conferences.
%
% If IEEEtran.cls has not been installed into the LaTeX system files,
% manually specify the path to it like:
% \documentclass[conference]{../sty/IEEEtran}





% Some very useful LaTeX packages include:
% (uncomment the ones you want to load)


% *** MISC UTILITY PACKAGES ***
%
%\usepackage{ifpdf}
% Heiko Oberdiek's ifpdf.sty is very useful if you need conditional
% compilation based on whether the output is pdf or dvi.
% usage:
% \ifpdf
%   % pdf code
% \else
%   % dvi code
% \fi
% The latest version of ifpdf.sty can be obtained from:
% http://www.ctan.org/tex-archive/macros/latex/contrib/oberdiek/
% Also, note that IEEEtran.cls V1.7 and later provides a builtin
% \ifCLASSINFOpdf conditional that works the same way.
% When switching from latex to pdflatex and vice-versa, the compiler may
% have to be run twice to clear warning/error messages.






% *** CITATION PACKAGES ***
%
%\usepackage{cite}
% cite.sty was written by Donald Arseneau
% V1.6 and later of IEEEtran pre-defines the format of the cite.sty package
% \cite{} output to follow that of IEEE. Loading the cite package will
% result in citation numbers being automatically sorted and properly
% "compressed/ranged". e.g., [1], [9], [2], [7], [5], [6] without using
% cite.sty will become [1], [2], [5]--[7], [9] using cite.sty. cite.sty's
% \cite will automatically add leading space, if needed. Use cite.sty's
% noadjust option (cite.sty V3.8 and later) if you want to turn this off.
% cite.sty is already installed on most LaTeX systems. Be sure and use
% version 4.0 (2003-05-27) and later if using hyperref.sty. cite.sty does
% not currently provide for hyperlinked citations.
% The latest version can be obtained at:
% http://www.ctan.org/tex-archive/macros/latex/contrib/cite/
% The documentation is contained in the cite.sty file itself.






% *** GRAPHICS RELATED PACKAGES ***
%
\ifCLASSINFOpdf
% \usepackage[pdftex]{graphicx}
% declare the path(s) where your graphic files are
% \graphicspath{{../pdf/}{../jpeg/}}
% and their extensions so you won't have to specify these with
% every instance of \includegraphics
% \DeclareGraphicsExtensions{.pdf,.jpeg,.png}
\else
% or other class option (dvipsone, dvipdf, if not using dvips). graphicx
% will default to the driver specified in the system graphics.cfg if no
% driver is specified.
% \usepackage[dvips]{graphicx}
% declare the path(s) where your graphic files are
% \graphicspath{{../eps/}}
% and their extensions so you won't have to specify these with
% every instance of \includegraphics
% \DeclareGraphicsExtensions{.eps}
\fi
% graphicx was written by David Carlisle and Sebastian Rahtz. It is
% required if you want graphics, photos, etc. graphicx.sty is already
% installed on most LaTeX systems. The latest version and documentation can
% be obtained at: 
% http://www.ctan.org/tex-archive/macros/latex/required/graphics/
% Another good source of documentation is "Using Imported Graphics in
% LaTeX2e" by Keith Reckdahl which can be found as epslatex.ps or
% epslatex.pdf at: http://www.ctan.org/tex-archive/info/
%
% latex, and pdflatex in dvi mode, support graphics in encapsulated
% postscript (.eps) format. pdflatex in pdf mode supports graphics
% in .pdf, .jpeg, .png and .mps (metapost) formats. Users should ensure
% that all non-photo figures use a vector format (.eps, .pdf, .mps) and
% not a bitmapped formats (.jpeg, .png). IEEE frowns on bitmapped formats
% which can result in "jaggedy"/blurry rendering of lines and letters as
% well as large increases in file sizes.
%
% You can find documentation about the pdfTeX application at:
% http://www.tug.org/applications/pdftex





% *** MATH PACKAGES ***
%
%\usepackage[cmex10]{amsmath}
% A popular package from the American Mathematical Society that provides
% many useful and powerful commands for dealing with mathematics. If using
% it, be sure to load this package with the cmex10 option to ensure that
% only type 1 fonts will utilized at all point sizes. Without this option,
% it is possible that some math symbols, particularly those within
% footnotes, will be rendered in bitmap form which will result in a
% document that can not be IEEE Xplore compliant!
%
% Also, note that the amsmath package sets \interdisplaylinepenalty to 10000
% thus preventing page breaks from occurring within multiline equations. Use:
%\interdisplaylinepenalty=2500
% after loading amsmath to restore such page breaks as IEEEtran.cls normally
% does. amsmath.sty is already installed on most LaTeX systems. The latest
% version and documentation can be obtained at:
% http://www.ctan.org/tex-archive/macros/latex/required/amslatex/math/





% *** SPECIALIZED LIST PACKAGES ***
%
%\usepackage{algorithmic}
% algorithmic.sty was written by Peter Williams and Rogerio Brito.
% This package provides an algorithmic environment fo describing algorithms.
% You can use the algorithmic environment in-text or within a figure
% environment to provide for a floating algorithm. Do NOT use the algorithm
% floating environment provided by algorithm.sty (by the same authors) or
% algorithm2e.sty (by Christophe Fiorio) as IEEE does not use dedicated
% algorithm float types and packages that provide these will not provide
% correct IEEE style captions. The latest version and documentation of
% algorithmic.sty can be obtained at:
% http://www.ctan.org/tex-archive/macros/latex/contrib/algorithms/
% There is also a support site at:
% http://algorithms.berlios.de/index.html
% Also of interest may be the (relatively newer and more customizable)
% algorithmicx.sty package by Szasz Janos:
% http://www.ctan.org/tex-archive/macros/latex/contrib/algorithmicx/




% *** ALIGNMENT PACKAGES ***
%
%\usepackage{array}
% Frank Mittelbach's and David Carlisle's array.sty patches and improves
% the standard LaTeX2e array and tabular environments to provide better
% appearance and additional user controls. As the default LaTeX2e table
% generation code is lacking to the point of almost being broken with
% respect to the quality of the end results, all users are strongly
% advised to use an enhanced (at the very least that provided by array.sty)
% set of table tools. array.sty is already installed on most systems. The
% latest version and documentation can be obtained at:
% http://www.ctan.org/tex-archive/macros/latex/required/tools/


% Also highly recommended is Mark Wooding's extremely powerful MDW tools,
% especially mdwmath.sty and mdwtab.sty which are used to format equations
% and tables, respectively. The MDWtools set is already installed on most
% LaTeX systems. The lastest version and documentation is available at:
% http://www.ctan.org/tex-archive/macros/latex/contrib/mdwtools/


% IEEEtran contains the IEEEeqnarray family of commands that can be used to
% generate multiline equations as well as matrices, tables, etc., of high
% quality.


%\usepackage{eqparbox}
% Also of notable interest is Scott Pakin's eqparbox package for creating
% (automatically sized) equal width boxes - aka "natural width parboxes".
% Available at:
% http://www.ctan.org/tex-archive/macros/latex/contrib/eqparbox/




% *** SUBFIGURE PACKAGES ***
%\usepackage[tight,footnotesize]{subfigure}
% subfigure.sty was written by Steven Douglas Cochran. This package makes it
% easy to put subfigures in your figures. e.g., "Figure 1a and 1b". For IEEE
% work, it is a good idea to load it with the tight package option to reduce
% the amount of white space around the subfigures. subfigure.sty is already
% installed on most LaTeX systems. The latest version and documentation can
% be obtained at:
% http://www.ctan.org/tex-archive/obsolete/macros/latex/contrib/subfigure/
% subfigure.sty has been superceeded by subfig.sty.



%\usepackage[caption=false]{caption}
%\usepackage[font=footnotesize]{subfig}
% subfig.sty, also written by Steven Douglas Cochran, is the modern
% replacement for subfigure.sty. However, subfig.sty requires and
% automatically loads Axel Sommerfeldt's caption.sty which will override
% IEEEtran.cls handling of captions and this will result in nonIEEE style
% figure/table captions. To prevent this problem, be sure and preload
% caption.sty with its "caption=false" package option. This is will preserve
% IEEEtran.cls handing of captions. Version 1.3 (2005/06/28) and later 
% (recommended due to many improvements over 1.2) of subfig.sty supports
% the caption=false option directly:
%\usepackage[caption=false,font=footnotesize]{subfig}
%
% The latest version and documentation can be obtained at:
% http://www.ctan.org/tex-archive/macros/latex/contrib/subfig/
% The latest version and documentation of caption.sty can be obtained at:
% http://www.ctan.org/tex-archive/macros/latex/contrib/caption/




% *** FLOAT PACKAGES ***
%
%\usepackage{fixltx2e}
% fixltx2e, the successor to the earlier fix2col.sty, was written by
% Frank Mittelbach and David Carlisle. This package corrects a few problems
% in the LaTeX2e kernel, the most notable of which is that in current
% LaTeX2e releases, the ordering of single and double column floats is not
% guaranteed to be preserved. Thus, an unpatched LaTeX2e can allow a
% single column figure to be placed prior to an earlier double column
% figure. The latest version and documentation can be found at:
% http://www.ctan.org/tex-archive/macros/latex/base/



%\usepackage{stfloats}
% stfloats.sty was written by Sigitas Tolusis. This package gives LaTeX2e
% the ability to do double column floats at the bottom of the page as well
% as the top. (e.g., "\begin{figure*}[!b]" is not normally possible in
% LaTeX2e). It also provides a command:
%\fnbelowfloat
% to enable the placement of footnotes below bottom floats (the standard
% LaTeX2e kernel puts them above bottom floats). This is an invasive package
% which rewrites many portions of the LaTeX2e float routines. It may not work
% with other packages that modify the LaTeX2e float routines. The latest
% version and documentation can be obtained at:
% http://www.ctan.org/tex-archive/macros/latex/contrib/sttools/
% Documentation is contained in the stfloats.sty comments as well as in the
% presfull.pdf file. Do not use the stfloats baselinefloat ability as IEEE
% does not allow \baselineskip to stretch. Authors submitting work to the
% IEEE should note that IEEE rarely uses double column equations and
% that authors should try to avoid such use. Do not be tempted to use the
% cuted.sty or midfloat.sty packages (also by Sigitas Tolusis) as IEEE does
% not format its papers in such ways.





% *** PDF, URL AND HYPERLINK PACKAGES ***
%
%\usepackage{url}
% url.sty was written by Donald Arseneau. It provides better support for
% handling and breaking URLs. url.sty is already installed on most LaTeX
% systems. The latest version can be obtained at:
% http://www.ctan.org/tex-archive/macros/latex/contrib/misc/
% Read the url.sty source comments for usage information. Basically,
% \url{my_url_here}.





% *** Do not adjust lengths that control margins, column widths, etc. ***
% *** Do not use packages that alter fonts (such as pslatex).         ***
% There should be no need to do such things with IEEEtran.cls V1.6 and later.
% (Unless specifically asked to do so by the journal or conference you plan
% to submit to, of course. )


% correct bad hyphenation here
\hyphenation{op-tical net-works semi-conduc-tor}


\begin{document}
	%
	% paper title
	% can use linebreaks \\ within to get better formatting as desired
	\title{Plan recovery in reactive HTNs using symbolic planning}
	
	
	% author names and affiliations
	% use a multiple column layout for up to three different
	% affiliations
	\author{\IEEEauthorblockN{Lydia OULD OUALI}
		\IEEEauthorblockA{ %School of Electrical and\\Computer Engineering\\
			LIMSI-CNRS, UPR 3251, Orsay, France \\
			Univ. Paris-Sud, Orsay, France \\
			Email: ouldouali@limsi.fr
		}
		\and
		\IEEEauthorblockN{Charles RICH}
		\IEEEauthorblockA{
			Worcester Polytechnic Institute\\ Worcester, MA, USA\\
			Email: rich@wpi.edu/
		}
		\and
		\IEEEauthorblockN{Nicolas SABOURET}
		\IEEEauthorblockA{ LIMSI-CNRS, UPR 3251, Orsay, France \\
			Univ. Paris-Sud, Orsay, France \\
			Email: Nicolas.Sabouret@limsi.fr}
	}
	
	% conference papers do not typically use \thanks and this command
	% is locked out in conference mode. If really needed, such as for
	% the acknowledgment of grants, issue a \IEEEoverridecommandlockouts
	% after \documentclass
	
	% for over three affiliations, or if they all won't fit within the width
	% of the page, use this alternative format:
	% 
	%\author{\IEEEauthorblockN{Michael Shell\IEEEauthorrefmark{1},
	%Homer Simpson\IEEEauthorrefmark{2},
	%James Kirk\IEEEauthorrefmark{3}, 
	%Montgomery Scott\IEEEauthorrefmark{3} and
	%Eldon Tyrell\IEEEauthorrefmark{4}}
	%\IEEEauthorblockA{\IEEEauthorrefmark{1}School of Electrical and Computer Engineering\\
	%Georgia Institute of Technology,
	%Atlanta, Georgia 30332--0250\\ Email: see http://www.michaelshell.org/contact.html}
	%\IEEEauthorblockA{\IEEEauthorrefmark{2}Twentieth Century Fox, Springfield, USA\\
	%Email: homer@thesimpsons.com}
	%\IEEEauthorblockA{\IEEEauthorrefmark{3}Starfleet Academy, San Francisco, California 96678-2391\\
	%Telephone: (800) 555--1212, Fax: (888) 555--1212}
	%\IEEEauthorblockA{\IEEEauthorrefmark{4}Tyrell Inc., 123 Replicant Street, Los Angeles, California 90210--4321}}
	
	
	
	
	% use for special paper notices
	%\IEEEspecialpapernotice{(Invited Paper)}
	
	
	
	
	% make the title area
	\maketitle
	
	%	
	\begin{abstract}
		%\boldmath
		
	\end{abstract}
	%	% IEEEtran.cls defaults to using nonbold math in the Abstract.
	%	% This preserves the distinction between vectors and scalars. However,
	%	% if the journal you are submitting to favors bold math in the abstract,
	%	% then you can use LaTeX's standard command \boldmath at the very start
	%	% of the abstract to achieve this. Many IEEE journals frown on math
	%	% in the abstract anyway.
	%	
	%	% Note that keywords are not normally used for peerreview papers.
	%	\begin{IEEEkeywords}
	%		
	%	\end{IEEEkeywords}
	
	
	
	
	
	
	% For peer review papers, you can put extra information on the cover
	% page as needed:
	% \ifCLASSOPTIONpeerreview
	% \begin{center} \bfseries EDICS Category: 3-BBND \end{center}
	% \fi
	%
	% For peerreview papers, this IEEEtran command inserts a page break and
	% creates the second title. It will be ignored for other modes.
	\IEEEpeerreviewmaketitle
	
	
	
	\section{Introduction}
	
	\par Automatic planning is an important field of controlling artificial agents in complex and dynamic environments where research built two different approaches. The first one is symbolic planning: this approach consists in constructing a complete symbolic and logical model of the environment that allows the agent to reason about this model and define a complete plan to carry out its goals.  
	The most popular architecture used to describe the environment is the hierarchical architecture HTN (Hierarchical Taks Network) \cite{erol1996hierarchical}, which allows a recursive decomposition of complex goals into sub-goals or primitive actions. The HTN architecture eases the design of the environment and gives more expressiveness. Symbolic planning assumes that the environment is fully defined. By consequence,  the agent is able to predict all the possible situations to plan in advance. Nevertheless, it becomes clear that authoring a complete representation of a dynamic and complex environment such as simulation of human behavior \cite{conte1998mas} or the definition of dialog systems \cite{allen2002human} requires significant  knowledge-engineering effort \cite{zhuo2009learning}, and even reveals to be impossible \cite{maes1990designing}. However, with incomplete knowledge the agent cannot anticipate the future and the generated plan might be not executed as expected. Therefore, if at any point of the execution the plan breaksdown (i.e action execution fails), the planner has to stop the execution and build another plan that achieves the agent's goals. Such operation might be costly in terms of time and resources. 
	
	\par Because of these limitations, another planning approach called reactive planning was proposed \cite{firby1987investigation}. Reactive planning avoids long-term prediction in order to make the execution faster. For this reason, it leaves all the planning during the execution phase: the agent plans only for the next step to be executed from the current defined state of the environment. Thus, it can adapt the next step according to the observed changes. The main advantage of reactive planning systems is they don't need a complete definition of the environment. Instead, they aim to define the policy of the agent in its environment by running through a pres-authored HTN structure with  procedural  knowledge. Procedural knowledge defines conditions in the HTN domain knowledge as black-box procedures (for exmaple : JavaScript code) that contains no logical information (i.e no symbolic knowledge). This type of reactive HTN  eases the design, reduce the complexity of planning and still can cope with complex dynamic environments \cite {brom2005hierarchical}. They are used in numerous application domains, such as dialog systems \cite{bohus2003ravenclaw} and simulating human behavior \cite{brom2005hierarchical}.
	\par Nevertheless, breakdowns can still appear in reactive planning. An action execution can fail and leads the HTH to a state where no action can be applicable to achieve the goal. 
	In such situation, the agent has to stop and think about a new solution to reach its goal. However, without symbolic knowledge, the agent has nothing to reason about. The execution thus stops and the agent cannot recover from its breakdown.
	
	\par  In order to deal with this limitation, we propose in this paper to extend reactive HTNs with a linear symbolic planner. For this reason, we propose to  the HTN author to extend the procedurale konwledge of the HTN with some symbolic knowledge that allows the symbolic planner to  compute local recovery plans. We study the capacity of such  model to recover from breakdowns in reactive planning. 
	
	\par In section 2, we briefly present existing works in this domain. In section 3 we formalize the proposed solution \textit{Discolog} and describe its implementation. Section 4, presents the expriments and discusses the obtained solutions. At the end, we discuss the futures works to validate and extends our solution to differents domains and uses.
	
	\section{Backgorund and related works}
	The proposed work aims to present a contribution to repair breakdowns in reactive HTNs. Thus, we propose a hybrid planning system that combines a reactive HTN with a basic linear symbolic planning system. In this section, we present the definition of the two systems involved  and the existing approaches related to ouw work. 
	\subsection{Hierachical and linear symbolic planning}
	linear planning system attempts to generate a plan to reach a goal state i.e. a sequence of actions such that starting from an initial state, the plan leads to the goal state.  STRIPS planner \cite{fikes1972strips} is the first contribution based on the linear planning methods. STRIPS is based on  mean-end planner relied on a simple backward chaining search in the state space.
	\par Another contribution in planning is the Hierarchical planning approach called HTN (Hierarchical Task Network)  \cite{erol1996hierarchical}. HTN was created as extension of the linear planner STRIPS  allowing the planner to add more information and expressibility to the domain knowledge. 
	HTNs can be represented as AND/OR tree where AND nodes represents the tasks and OR nodes defines the recipes of tasks. 
	\begin{itemize}
		\item[-] Primitives tasks represented as leaf nodes in the tree, are similar to linear planning actions  and can be directly executed in the world. In addition, HTN tasks have postconditions that specify when a task execution successes or fails i.e(a task execution is considered as complete when its post-conditions are satisfied).
		
		\item[-] Compound tasks involve several tasks and can be performed by decomposing them into a sequence of subtasks using a specific recipe. 
		
		\item[-] Recipe represents  a method to achieve or decompose a compound task. Each recipe is defined with an applicability condition that helps the planner choosing the appropriate decomposition for a task if there is more than  one.
		
	\end{itemize}
	\par The HTN planning systems entents to plan for one or more goal task. Planning proceeds using task decomposition that starts from the initial goal task , decomposes it using a corresponding recipe, and breaksdown the goal into sequence of simpler subtasks. This process is applied recursively until until a conflict-free plan can be found. the plan consists on sequence of fully ordered primitive tasks that can make the goal task successful. 
	\par The HTN planners becomes popular this last decade and in different domains where several sys-
	tems were developed these recent years such as SHOP \cite{nau1999shop}, SIPE \cite{wilkins1988practical} or NOAH  \cite{sacerdoti1975structure}.
	
	
	
	\subsection{Related works and previous approachs}
	
	\subsubsection{Reactive planning}
	Reactive planning becomes very popular in AI such in controlling mobile agents \cite{beaudry2005reactive}, simulating human behaviour \cite{bryson2001modularity} or Gaming with the name f behaviour trees. Behaviours trees have the same hiearchical structure of HTNs and do Real time decision which can be seen as reactive planning.
	Nevertheless, As reactive planning is used for highly dynamic environements it presents certain limits as discussed below: 
	
	C. Brom \cite{brom2005hierarchical} proposes in his work an educational toolkit for prototyping human-like behaviour. the proposed reative architecture was based on the work proposed in \cite{bryson2001modularity}. Nevertheless, this reactive planning system faces some limits such as : 
	impossibility to add new goals during the execution or inhebit an undesirable subtask.unatural switching between behaviour. 
	
	R. James Firby  \cite{firby1987investigation} 
	
	
	\subsubsection{Plan repair}
	plan repair by extending the generated plan with graph containing the causal links between the HTN's tasks
	\cite{ayan2007hotride}  \cite{warfield2007adaptation} \cite{van2005plan} \cite{boella2002replanning}
	plan repair using heurestic \cite{hayashi2006dynagent}
	
	
	%A great deal of research has been done in the field of planning. Most of theses works were interested in constructing symbolic planning systems especially symbolic HTNs that include plan recovery from breakdwons and replanning. A common approach \cite{ayan2007hotride} \cite{van2005plan} \cite{warfield2007adaptation}. (revoir la phrase) is in addition to the plan, the HTN procduces a graph that captures the causal links between the HTN's tasks. Therefore, if during  plan execution, a breakdown occurs, the HTN uses the structure to detect all the tasks compromised by the failure and therefore needs to be repaired. The HTN is called to propose a local plan recovory to repair the failed taks instead of replanning from the scartch. Nevertheless, theses systems are always dependent to the HTN and inhirit its limitations. For example, if the HTN cannot propose a plan recovery for the failed tasks then it has to replan from the scratch or blocked.  
	%\par \cite{hayashi2006dynagent} proposes to construct a planning system applicated to musium tour guide. this planning system uses A* heurestic to define the most fiable plan. If this later still breakdwons it uses the heurstic to call other plans to recover from breakdwons.  
	%
	% In this last decate, there has been an interest on defineing reactive planning that can deal with high dynamic environments. Present articles in different field of IA and their solutions to handle limits of reactive planning. 
	\section{motivation example }
	\par In this section, we describe a typical reactive HTN execution. For comprehensive convenience, we define a toy example of a robot that moves objects to (load/unload) them from trucks.
	
	In the following example, we assume that the robot goal is to load an object which is initially on the floor to a truck. Figure \ref{Example} depicts the HTN decomposition to achieve this goal using the domain knowledge defined in figure. The robot takes as goal to load an object which weight 12 Kg. Thus, the robot starts by executing the first task (i.e Move task). It first evaluates the first recipe ( MoveWithOneArm), and its applicability condition is not hold because the object weight more than 5 kg. The robot will then try the second recipe, which is also inapplicable because the object weight more than 10 kg. By consequence, the robot has no task to proceed with, and a breakdown case is detected. The execution stops and the goal is not achieved. In such situation the agent must be able to think in order to construct a strategy for recovering from this breakdown. For that matter, we propose to extend the reactive HTN of the robot with a simple linear planner that can construct a plan as strategy to locally recover from this breakdown. 
	
	\begin{figure*}[h]
		\centering
		\includegraphics[width=\textwidth]{robotExample.png}
		\caption{\label{Example}Example of reactive planning execution for a robot}
	\end{figure*}
	\fbox{\parbox{\columnwidth}{\centering \textbf{ HTN domain knowledge:} \\
			
			
			\par \textbf{Tasks} \\
			\raggedright Load (Object X, Place Y, Place Z) :  Precond: isOn(X,Y) . \\
			Postcond : isOn(X,Z).\\
			
			Unload (Object X, Place Y, Place Z) :  Precond: isOn(X,Y) .\\ 
			Postcond : isOn(X,Z).\\
			
			Move (Object X, Place Y) :  Precond: isOn(X,Y) . \\
			Postcond :  ! isOn(X,Y).\\
			
			PutDown(Object X,Place Y) :  Precond:  ! isOn(X,Y) .\\ 
			Postcond :   isOn(X,Y).\\
			
			\centering \textbf{ Primitive tasks\\}
			\raggedright Hold(Object X, RobotArm Arm) :  Precond:  ! isOn(X,Arm) . \\
			Postcond :   isOn(X,Arm).\\
			
			RaiseUp(Object X, Floor) : Precond:   isOn(X,Floor) . \\
			Postcond :   !isOn(X,Floor).\\
			
			Release(Object X, RobotArm arm) : Precond:   isOn(X,Arm) . \\
			Postcond :   ! isOn(X,Arm).\\
			
			put(Object X, Place Y): Precond:   !isOn(X,Y) . \\
			Postcond :   isOn(X,Y).\\
			
			\centering \textbf{Recipes} \\
			\raggedright holdwithOneArm(X) : Applicability cond:  X.weight $<$ 5 Kg \\
			
			ReleasewithOneArm(X) : Applicability cond: X.weight $>$ 5 Kg and weight $<$ 10 Kg \\
		}}
		\section{Solution }
		
		In this section we present our proposition based on extending the reactive HTN with symbolic planner to recover from breakdown.
		\par Reactive HTNs do tasks decomposition and plan only for the next primitive task to execute from the current state, using  for that a procedural domain knowledge. This later represents the policy of the agent in its environment as a set of procedures which can only be executed. Nevertheless, as the domain knowledge is never complete breakdowns might happens.Thus, the HTN can no longer proceed with the execution. In order to recover from such breakdown, the agent has to think of a strategy, but without symbolic knowledge and planner the agent has no way to think. 
		\par To overcome this problem, we propose to extend the reactive HTN with a symbolic linear planner that uses  symbolic knowledge extracted from the HTN. 
		
		\par Some procedures in the procedural knowledge have the form of boolean procedures. this later can thought as predicate. For example, the  
		\emph{isOn(X,Y)} procedure return either true or false depending on the object X  is on the place Y. This procedure can be represented as a predicate that takes two variables. Therefor, We propose to the HTN designer to covert theses procedures  to a symbolic predicates and the primitives actions containing theses predicates. 
		
		\par Thus, if a breakdown occurs (The agent has nothing to execute and the goal is not reached yet). The hybrid system calls the recover procedure describe in the algorithm \ref{pseudoPSO}. It first traverse the HTN to detects all the tasks in the HTN affected by the breakdown \emph{(FindCandidate procedure)}. The affected tasks by the breakdown have one of their pre/post conditions which are no longer supported in the current state.
		\begin{algorithm}
			\caption{ Reactive planning and plan recovery algorithm}
			\label{pseudoPSO}
			\begin{algorithmic}[1]
				\Procedure{Hybrid system}{DomainKnowlege,Goal}
				\State $\pi \gets Reactive_HTN \textit{(DomainKnowlege,Goal)}$
				\If {$ \pi \gets\textit{Success} $} 
				\State \Return $\textit{Success} $
				\Else 
				\State$ plan \gets Recover(Goal)$
				\If {(plan = \textit{null})}
				\State \Return Failure
				\Else 
				\For{$ \textbf{each}$  action \textit{$a_i$} $\in$ plan }
				\State  $\textit{Discolog} (HTN,a_i) $
				\EndFor
				\EndIf
				\EndIf
				
				\EndProcedure 
				
				
				
				\Procedure{Recover}{Goal}
				\State $\textit{Candidates}\gets\textit{findCandidate}{(G)} $
				\If {$ \textit{Candidates = }\emptyset $} 
				\State \Return $\textit{null} $
				\Else 
				\State $\Pi \gets \emptyset$
				
				\For{$ \textbf{each} \textit{ candidate} \in \textit{Candidates}$}
				\State $\Pi$ += LinearPlanner(candidate, CurrentState)
				\State  $Cost \gets \{ cost(\pi) | \pi \in  \Pi \} $
				\EndFor
				\EndIf
				\State \Return $\pi \in \Pi$  with minimum cost$(\pi)$
				
				\EndProcedure
				
				
				\Procedure{FindCandidates}{Goal}
				\For{$ \textbf{each} \textit{ child} \in \textit{Goal}$}
				
				\If {$ \textbf{ status}$ (child) = failed}
				\State   add condition(child) to Candidates
				\EndIf
				
				\State $\textit{findCandidates} (children(child))$
				\EndFor
				\State \Return Candidates
				
				\EndProcedure 
			\end{algorithmic}
		\end{algorithm}
		Once all the task candidates are detected, the linear planner is called for each candidate. The linear planner uses the symbolic knowledge to  construct a plan that takes as goal repairing the failed condition in the task candidate. Finally, the hybrid system calculates the best and most promising  plan to be passed to the reactive HTN in order to be executed. 
		
		
		\subsection{Implementation of the solution}
		
		\par In this section we present the implementation of our solution called Discolog. Discolog is based on reactive HTN called Disco \cite{rich2009building} and a simple STRIPS planner implemented in Prolog. We choose the Prolog implementation because of its inference engine that allow the planner reasoning on the possible links between the predicates.
		\par  Disco uses the ANSI/CEA-2018 standard for the procedural definition of its domain knowledge. Thus, Tasks are modeled using XML
		format. Primitive tasks contains grounding script parameter defined	as JavaScript program which represent the effect of the primitive task execution
		in the environment.
		\par  Like reactive HTN, disco has no logical knowledge and plans only for the next step. Each task has a status which represents its state in the HTN:
		\begin{itemize}
			\item  if the preconditions of the task are hold in the current state, the status of this task becomes \textit{live} else the status of the task is \textit{notDone} .
			\item  Once the execution of a task is terminated, Disco evaluates its postconditions, if their are valid then the task execution is successful and its status becomes \textit{Done}. Nevertheless, if task's postconditions  are not valid, then the status of the task becomes \textit{Failed}.
		\end{itemize}
		The status of the tasks are used to calculate the tasks candidates for the recovery procedure as described in the algorithm \ref{findCandidates} 		
		
		\par For example, taking back the example of the robot, Both recipes to decompose the load task are not applicable which make the execution breaks down. Discolog calculates all the tasks and recipes affected by this breakdown by checking. In this case, all the remaining tasks to decompose the load task are taken as candidates. the planner is then called for each one. Using the symbolic knowledge the planner proposes as plan to decompose the object into two objects and load them separately as represented in the figure (attacher une figure de la reparation de plan). 
		\label{findCandidates}
		\begin{algorithm}
			
			\begin{algorithmic}[]
				\Procedure{FindCandidates}{Goal}
				\For{$ \textbf{each} \textit{ child} \in \textit{Goal}$}
				\If {$(precondition(child)!= \emptyset$  and  \textbf{ status} (child)$\notin$\{Done, Live, Blocked\})}
				\State add precondition(child) to Candidates
				
				\ElsIf{(postcondition(child)!=$\emptyset $ and  \textbf{ status} (child)$ \in$ \{Failed\}}
				\State add postcondition(child) to Candidates
				\EndIf
				\If {(\textbf{status} $\in$ \{Live\} and nonPrimitive(child) and applicability(child)!=  $\emptyset)$}
				\State add Applicability-condition(child) to Candidates
				\EndIf
				\State $\textit{findCandidates} (children(child))$
				\EndFor
				\Return Candidates
				
				\EndProcedure 
			\end{algorithmic}
		\end{algorithm}
		
		\section{Experiments and results}
		
		
		1. Approach of the expriments: Test the capability of Discolog to recover from a breakdown given a certain amount of symbolic knowlege. 
		
		
		2. Benshmark creation:  
		
		Random HTNs with synthetic data. Breakdown caused in each primiive task. The purpose is to study the abbility of Discolog to find a plan recovery for all possible breakdowns in the HTN. 
		Symbolic data generation : the variation of the level of symbolic knowledge to insert in the linear planner domain knowledge
		
		
		3. Present the obtained results and discuss them. 
		
		results obtaind of tree (5,4,1) (2,3,2) (3,3,3) 
		
		
		Discuss the fact that the more symbolic knowledge we have the more recovery we get. 
		Expose the fact that we can not have a 100 of symbolic knowelge and its is limited to the representation of the HTN hauthor which is also incomplete.
		
		\section{Conclusion}
		Remind the context of our work. the proposition and its adventages. the future work : 
		
		1. present system support  for authoring reactive HTNs. 
		
		
		2. dialog system using Discolog
		
		
		% if have a single appendix:
		%\appendix[Proof of the Zonklar Equations]
		% or
		%\appendix  % for no appendix heading
		% do not use \section anymore after \appendix, only \section*
		% is possibly needed
		
		% use appendices with more than one appendix
		% then use \section to start each appendix
		% you must declare a \section before using any
		% \subsection or using \label (\appendices by itself
		% starts a section numbered zero.)
		%
		
		
		%	\appendices
		%	\section{Proof of the First Zonklar Equation}
		%	\
		%	
		%	% use section* for acknowledgement
		%	\section*{Acknowledgment}
		%	
		
		
		
		
		% Can use something like this to put references on a page
		% by themselves when using endfloat and the captionsoff option.
		\ifCLASSOPTIONcaptionsoff
		\newpage
		\fi
		
		
		
		% trigger a \newpage just before the given reference
		% number - used to balance the columns on the last page
		% adjust value as needed - may need to be readjusted if
		% the document is modified later
		%\IEEEtriggeratref{8}
		% The "triggered" command can be changed if desired:
		%\IEEEtriggercmd{\enlargethispage{-5in}}
		
		% references section
		
		% can use a bibliography generated by BibTeX as a .bbl file
		% BibTeX documentation can be easily obtained at:
		% http://www.ctan.org/tex-archive/biblio/bibtex/contrib/doc/
		% The IEEEtran BibTeX style support page is at:
		% http://www.michaelshell.org/tex/ieeetran/bibtex/
		%\bibliographystyle{IEEEtran}
		% argument is your BibTeX string definitions and bibliography database(s)
		%\bibliography{IEEEabrv,../bib/paper}
		%
		% <OR> manually copy in the resultant .bbl file
		% set second argument of \begin to the number of references
		% (used to reserve space for the reference number labels box)
		\label{Bibliography}
		\bibliography{Bibliography} % The references (bibliography) information are stored in the file named "Bibliography.bib"
		\bibliographystyle{alpha}
		
		% biography section
		% 
		% If you have an EPS/PDF photo (graphicx package needed) extra braces are
		% needed around the contents of the optional argument to biography to prevent
		% the LaTeX parser from getting confused when it sees the complicated
		% \includegraphics command within an optional argument. (You could create
		% your own custom macro containing the \includegraphics command to make things
		% simpler here.)
		%\begin{biography}[{\includegraphics[width=1in,height=1.25in,clip,keepaspectratio]{mshell}}]{Michael Shell}
		% or if you just want to reserve a space for a photo:
		
		\begin{IEEEbiography}[{\includegraphics[width=1in,height=1.25in,clip,keepaspectratio]{picture}}]{John Doe}
			\
		\end{IEEEbiography}
		
		% You can push biographies down or up by placing
		% a \vfill before or after them. The appropriate
		% use of \vfill depends on what kind of text is
		% on the last page and whether or not the columns
		% are being equalized.
		
		%\vfill
		
		% Can be used to pull up biographies so that the bottom of the last one
		% is flush with the other column.
		%\enlargethispage{-5in}
		
		
		
		
		% that's all folks
	\end{document}
	
	
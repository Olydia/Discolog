%What is dominance in social psychology
%Nonverbal Behavior and the Vertical Dimension of Social Relations: A Meta-Analysis
%
%\textbf{several representation of dominance :} 
%For example, dominance can be defined as a personality trait involving the motive to control others, the self-
%perception of oneself as controlling others, and/or as a behavioral
%outcome (success in controlling others or their resources). Status,
%involving an ascribed or achieved quality implying respect and
%privilege, does not necessarily include the ability to control others
%or their resources.
%
% Similarly, power defined as the capacity or
%structurally sanctioned right to control others or their resources
%does not necessarily imply prestige or respect. Other distinctions
%have also been drawn, including different functional bases of
%power, such as reward power, expert power, referent power, or
%coercive power (French and Raven, 1959), and outcome dependency
%(Stevens  Fiske, 2000). Some writers conceptualize dominance
%in terms of social skill (e.g., Burgoon  Dunbar, 2000; Byrne,
%2001). Some writers define dominance as the enactment of certain
%nonverbal or verbal behaviors (Rosa  Mazur, 1979). Authors do
%not use the various verticality terms such as power, dominance,
%and status in consistent ways, and often the terms are used without
%a clear definition.
%
%\textbf{Burgoon and dunber 2010}
%
%le pouvoir est généralement défini comme la capacité de produire les effets voulus et, en particulier, la capacité d'influencer le comportement d'une autre personne même face à la résistance  %(Bachrach & Lawler, 1981;
%%Berger, 1994; Burgoon & Dunbar, 2006; Foa & Foa, 1974; Dunbar, 2004; Emerson,
%%1962; French & Raven, 1959; Gray-Little & Burks, 1983; Henley, 1995; Huston, 1983;
%%Neff & Harter, 2002; Olson & Cromwell, 1975).
%
%
%La dominance, d'un autre côté, se réfère à des comportements interactionnels dépendant du contexte et de la relation, dans lesquels le pouvoir est rendu saillant et l'influence est atteinte %(Burgoon et al., 1998; Mack, 1974; Rogers-Millar & Millar, 1979)
%dans le contexte de l'échange interpersonnel, la dominance est basée sur une combinaison de tempérament individuel et de caractéristiques situationnelles qui encouragent un comportement dominant. 
%Le dominance interpersonnelle est un comportement de communication basé sur la relation qui dépend du contexte et des motivations des individus impliqués
%
%Le pouvoir fournit le contexte pour les comportements de dominance. Bien que la domination puisse parfois être utilisée pour acquérir du pouvoir, elle ne doit pas être considérée comme synonyme de pouvoir %Keltner, Gruenfeld, & Anderson, 2003).
%
%Dans un contexte social, la dominance et le pouvoir sont relatifs au partenaire social et ne sont pas absolus (Dunbar, 2004). 
%Comme l'ont noté Emerson (1962) et d'autres théoriciens, le pouvoir équivaut à dépendre des autres et suppose donc une relation sociale.
%
%\textbf{Dyadic power theory:}
%
%
%Elle se base sur les échanges sociaux de pouvoir sur 4 éléments
%
%\begin{enumerate}
%	\item cette définition s'étend pour prendre en compte les échanges sociaux qui inclut les stratégies de communications qui deviennent manifestes durant l'interaction
%	\item Contrôle relationnel : elle reconnaît que l'autorité d'utiliser ou d'échanger des ressources dans les interactions est souvent accordée aux individus par les normes sociétales ainsi que par l'histoire relationnelle propre aux individus impliqués dans l'interaction.
%	
%	\item Cette représentation insiste sur l'idée que le pouvoir doit être vu relativement, ou par rapport aux autres. Ainsi, il reconnaît que le pouvoir est une construction dynamique et multidimensionnelle qui incorpore les perspectives des deux individus dans l'interaction.
%	\item En cohérence avec une perspective de communication, il place l'interaction elle-même au centre des préoccupations. Les tentatives de prendre le contrôle de toute interaction donnée par la dominance, bien que déterminées par le pouvoir, sont au centre de la théorie car elles déterminent le résultat du processus  (la décision finale qui a des ramifications pour l'avenir de la relation).
%%\end{enumerate}
%
%
%
%
%\textbf{Interpersonal dominance paper}
%En préface, notre intention n'est pas de représenter les trois comme des perspectives exhaustives ou mutuellement exclusives sur la construction de la domination-soumission, mais plutôt de mettre en lumière les nombreuses facettes de cette dimension fondamentale des relations humaines.


\subsection{Trait de personnalité}
%Les chercheurs qui se sont intéressés à l'étude de la personnalité tendent à voir la personnalité comme stable \cite{burgoon2006nonverbal} et par conséquent, cherchent a définir des indicateurs de dominance stable permettant de de former  des profils comportementaux stables. 

La dominance comme trait de personnalité a connu une évolution où la littérature en psychologie a élargit le concept de la dominance au-delà des comportements d'agressivité.
En effet, les travaux en Éthologie et psychologie évolutive ont d'abord étudié la dominance comme un trait de personnalité innée qui confère à la personne la capacité d'exercer son pouvoir et sa domination, susciter la déférence et l'acquiescement, ou apaiser et se soumettre à un comportement conspécifique plus fort \cite{keltner1995signs,burgoon2006nonverbal}. Cependant, les travaux en Éthologie contemporaine s'éloignent de cette définition en faveur d'études pour la compréhension de l'interaction entre les organismes et leur environnement social \cite{burgoon2006nonverbal}.

Pour les psychologues de la personnalité, la dominance est considérée comme un trait individuel durable qui désigne le tempérament de la personne et les prédispositions comportementales de chacun \cite{cattell1970handbook,ridgeway1987nonverbal} tels que l'agression, l'ambition, l'argumentation,
assertivité, vantardise, confiance et détermination.
Les adjectifs communs utilisés pour décrire la personne dominante sont affirmés, agressifs, compétitifs, exigeant, égoïste et têtu. Plus précisément, la dominance en tant que variable de personnalité «montre un ajustement réaliste constant au succès et à l'échec de l'individu, à la santé ou à la maladie, aux capacités ou handicaps et aux forces extérieures relatives» \cite{cattell1970handbook,burgoon1998nature}.
En effet, selon  Emmons et McAdams \cite{emmons1991personal}, les individus avec un motif de haut pouvoir sont décrits comme voulant contrôler et influencer les autres, désirant la célébrité ou l'attention du public, et ayant la capacité de susciter l'émotion chez les autres. Dans la même veine, \cite{jackson1974personality} décrivent une personne dominante comme une personne qui cherche et maintient un rôle de leader dans un groupe. Ces personnes prennent en charge et guident les membres du groupe vers la réalisation d'objectifs louables à travers des tactiques d'influence, des manœuvres de contrôle de l'environnement et l'expression énergique de l'opinion \cite{burgoon1998nature}. 
A l'opposé, les individus non dominants ou soumis sont présentés comme coopératifs, modestes et obéissants, mais aussi obséquieux, doux, faibles, peu sûrs et évitant les situations qui nécessitent une affirmation de soi \cite{burgoon1998nature}. 

%Les habiletés sociales font partie de cette équation, car la capacité d'être énergique, de prendre des initiatives et d'être expressif mais détendu et équilibré \cite{burgoon2006nonverbal}. 
Par ailleurs, la dominance et soumission sont généralement révélés à travers le style de communication \cite{burgoon1998nature}. Une personne dominante va avoir plus d'assurance dans sa manière de communiquer, à être plus confiante, enthousiaste, énergique, active, compétitive, sûre d'elle, vaniteuse et direct. Ceci transparaît dans les comportements verbaux et non verbaux utilisés. Une personne dominante va  utiliser plus d'espace, avec un temps de conversation plus long et être capable d'avoir plus d'interruption réussies \cite{burgoon1998nature}. 

En revanche, une personne soumise va utiliser un mouvement contraint, une utilisation limitée de l'espace, de grandes quantités de regard tout en écoutant  \cite{burgoon1998nature}. Ces comportements sont présentés plus en détails dans la section ... 


\subsection{Statut social}
Néanmoins, une hypothèse unificatrice dans la littérature éthologique est que la dominance représente une caractéristique universelle de l'organisation sociale reflétée dans le rang ou la position dans une hiérarchie sociale (Sebeok, 1972) et l'accès préférentiel aux ressources (Liska, 1988).

Lorsque la concurrence pour l'accès prioritaire à la ressource s'installe entre des individus ou des groupes qui ne se connaissent pas, le classement de dominance des individus doit être établi et signalé (Gauthreaux, 1981). L'universalité de ces signaux est implicite dans le fait que les hiérarchies de dominance sociale si communes dans les groupes de primates (Shively, 1985) sont également communes dans les groupes humains.
Les membres des groupes primates et humains savent tous deux où ils se situent et qui a le statut le plus élevé ou le plus de dominance (Hogan, 1979, cité dans Smither, 1993).
Deux formes alternatives de dominance ont été identifiées dans la littérature. On associe la domination à des facteurs tels que l'héritabilité, l'âge et l'ordre de naissance qui confèrent un plus grand contrôle ou un meilleur accès à des ressources privilégiées (Cattell, Eber et Tatsuoka, 1970). Liska (1988) appelle cela une domination acquise.

Chez les humains, la domination acquise est souvent assimilée à des rôles sociaux. Il peut se manifester par des caractéristiques immuables (signaux statiques) ou des indicateurs qui changent lentement (signaux lents) de facteurs tels que la position dans une hiérarchie de statut, l'âge, la maturité, la parenté, l'ordre de naissance, et rôles institutionnalisés. La deuxième forme est la domination sociale {Liska., 1988), qui est acquise à travers des capacités, des stratégies ou des possibilités d'affiliation démontrées. Contrairement au caractère immuable des manifestations de domination acquise, la dominance sociale peut se manifester par des indicateurs dynamiques tels que la proximité, la posture, le regard, l'expression faciale, la vocalisation, la durée de la conversation ou l'usage de la langue (Keating, 1985; 1988). Particulièrement pertinents pour ceux qui s'intéressent à la communication interpersonnelle, de tels affichages peuvent être liés par les relations, les contextes et le temps.

Pour les biologistes et sociobiologistes, la dominance désigne la position d'un organisme dans une hiérarchie sociale (Sebeok, 1972) qui lui accorde un accès préférentiel aux ressources (Omark, 1980).

Pour les sociologues, le pouvoir et la domination sont étroitement liés au statut, qui désigne sa position dans une hiérarchie socialement acceptée, ce qui est répandu dans tous les types de sociétés (Lips, 1991). La position sociale d'une personne est souvent basée sur la possession de biens évalués par la société (par exemple, l'argent, la profession, la beauté) ou la position dans une hiérarchie de prestige des relations dans une unité sociale.

Un statut élevé donne souvent l'apparence d'un pouvoir et peut faciliter la domination parce qu'on est doté d'une autorité légitime, et l'autorité légitime confère à l'individu le potentiel d'une plus grande influence (Rollins and Bahr, 1976). Mais un statut élevé ne garantit pas l'exercice du pouvoir ou l'affichage d'un comportement dominant, et les manifestations de dominance en l'absence de pouvoir légitime peuvent ne pas réussir à exercer une influence (Ridgeway, Diekema, and Johnson, 1995).

Bien que le statut soit parfois confondu avec le pouvoir et la domination, il désigne généralement sa position dans une hiérarchie sociale (Ellyson et Dovidio, 1985). Le statut peut accorder un certain degré de pouvoir et le «droit» d'exercer une domination, mais les individus de haut rang ne sont pas forcément puissants ou dominants, et les démonstrations de dominance ne placent inévitablement pas haut dans une hiérarchie de statut\cite{burgoon1998nature}.

Status (or rank) is most frequently defined as priority of access to resources in competitive situations.



%What is dominance in social psychology
%Nonverbal Behavior and the Vertical Dimension of Social Relations: A Meta-Analysis
%
%\textbf{several representation of dominance :} 
%For example, dominance can be defined as a personality trait involving the motive to control others, the self-
%perception of oneself as controlling others, and/or as a behavioral
%outcome (success in controlling others or their resources). Status,
%involving an ascribed or achieved quality implying respect and
%privilege, does not necessarily include the ability to control others
%or their resources.
%
% Similarly, power defined as the capacity or
%structurally sanctioned right to control others or their resources
%does not necessarily imply prestige or respect. Other distinctions
%have also been drawn, including different functional bases of
%power, such as reward power, expert power, referent power, or
%coercive power (French and Raven, 1959), and outcome dependency
%(Stevens  Fiske, 2000). Some writers conceptualize dominance
%in terms of social skill (e.g., Burgoon  Dunbar, 2000; Byrne,
%2001). Some writers define dominance as the enactment of certain
%nonverbal or verbal behaviors (Rosa  Mazur, 1979). Authors do
%not use the various verticality terms such as power, dominance,
%and status in consistent ways, and often the terms are used without
%a clear definition.
%
%\textbf{Burgoon and dunber 2010}
%

%
%
%La dominance, d'un autre côté, se réfère à des comportements interactionnels dépendant du contexte et de la relation, dans lesquels le pouvoir est rendu saillant et l'influence est atteinte %(Burgoon et al., 1998; Mack, 1974; Rogers-Millar & Millar, 1979)
%dans le contexte de l'échange interpersonnel, la dominance est basée sur une combinaison de tempérament individuel et de caractéristiques situationnelles qui encouragent un comportement dominant. 
%Le dominance interpersonnelle est un comportement de communication basé sur la relation qui dépend du contexte et des motivations des individus impliqués
%
 
%Comme l'ont noté Emerson (1962) et d'autres théoriciens, le pouvoir équivaut à dépendre des autres et suppose donc une relation sociale.
%
%\textbf{Dyadic power theory:}
%
%
%Elle se base sur les échanges sociaux de pouvoir sur 4 éléments
%
%\begin{enumerate}
%	\item cette définition s'étend pour prendre en compte les échanges sociaux qui inclut les stratégies de communications qui deviennent manifestes durant l'interaction
%	\item Contrôle relationnel : elle reconnaît que l'autorité d'utiliser ou d'échanger des ressources dans les interactions est souvent accordée aux individus par les normes sociétales ainsi que par l'histoire relationnelle propre aux individus impliqués dans l'interaction.
%	
%	\item Cette représentation insiste sur l'idée que le pouvoir doit être vu relativement, ou par rapport aux autres. Ainsi, il reconnaît que le pouvoir est une construction dynamique et multidimensionnelle qui incorpore les perspectives des deux individus dans l'interaction.
%	\item En cohérence avec une perspective de communication, il place l'interaction elle-même au centre des préoccupations. Les tentatives de prendre le contrôle de toute interaction donnée par la dominance, bien que déterminées par le pouvoir, sont au centre de la théorie car elles déterminent le résultat du processus  (la décision finale qui a des ramifications pour l'avenir de la relation).
%%\end{enumerate}
%
%
%
%
%\textbf{Interpersonal dominance paper}
%En préface, notre intention n'est pas de représenter les trois comme des perspectives exhaustives ou mutuellement exclusives sur la construction de la domination-soumission, mais plutôt de mettre en lumière les nombreuses facettes de cette dimension fondamentale des relations humaines.


\subsection{Trait de personnalité}
%Les chercheurs qui se sont intéressés à l'étude de la personnalité tendent à voir la personnalité comme stable \cite{burgoon2006nonverbal} et par conséquent, cherchent a définir des indicateurs de dominance stable permettant de de former  des profils comportementaux stables. 

La dominance comme trait de personnalité a connu une évolution où la littérature en psychologie a élargit le concept de la dominance au-delà des comportements d'agressivité.
En effet, les travaux en Éthologie et psychologie évolutive ont d'abord étudié la dominance comme un trait de personnalité innée qui confère à la personne la capacité d'exercer son pouvoir et sa domination, susciter la déférence et l'acquiescement, ou apaiser et se soumettre à un comportement conspécifique plus fort \cite{keltner1995signs,burgoon2006nonverbal}. Cependant, les travaux en Éthologie contemporaine s'éloignent de cette définition en faveur d'études pour la compréhension de l'interaction entre les organismes et leur environnement social \cite{burgoon2006nonverbal}.

Pour les psychologues de la personnalité, la dominance est considérée comme un trait individuel durable qui désigne le tempérament de la personne et les prédispositions comportementales de chacun \cite{cattell1970handbook,ridgeway1987nonverbal} tels que l'agression, l'ambition, l'argumentation,
assertivité, vantardise, confiance et détermination.
Les adjectifs communs utilisés pour décrire la personne dominante sont affirmés, agressifs, compétitifs, exigeant, égoïste et têtu. Plus précisément, la dominance en tant que variable de personnalité «montre un ajustement réaliste constant au succès et à l'échec de l'individu, à la santé ou à la maladie, aux capacités ou handicaps et aux forces extérieures relatives» \cite{cattell1970handbook,burgoon1998nature}.
En effet, selon  Emmons et McAdams \cite{emmons1991personal}, les individus avec un motif de haut pouvoir sont décrits comme voulant contrôler et influencer les autres, désirant la célébrité ou l'attention du public, et ayant la capacité de susciter l'émotion chez les autres. Dans la même veine, \cite{jackson1974personality} décrivent une personne dominante comme une personne qui cherche et maintient un rôle de leader dans un groupe. Ces personnes prennent en charge et guident les membres du groupe vers la réalisation d'objectifs louables à travers des tactiques d'influence, des manœuvres de contrôle de l'environnement et l'expression énergique de l'opinion \cite{burgoon1998nature}. 
A l'opposé, les individus non dominants ou soumis sont présentés comme coopératifs, modestes et obéissants, mais aussi obséquieux, doux, faibles, peu sûrs et évitant les situations qui nécessitent une affirmation de soi \cite{burgoon1998nature}. 

%Les habiletés sociales font partie de cette équation, car la capacité d'être énergique, de prendre des initiatives et d'être expressif mais détendu et équilibré \cite{burgoon2006nonverbal}. 
Par ailleurs, la dominance et soumission sont généralement révélés à travers le style de communication \cite{burgoon1998nature}. Une personne dominante va avoir plus d'assurance dans sa manière de communiquer, à être plus confiante, enthousiaste, énergique, active, compétitive, sûre d'elle, vaniteuse et direct. Ceci transparaît dans les comportements verbaux et non verbaux utilisés. Une personne dominante va  utiliser plus d'espace, avec un temps de conversation plus long et être capable d'avoir plus d'interruption réussies \cite{burgoon1998nature}. 

En revanche, une personne soumise va utiliser un mouvement contraint, une utilisation limitée de l'espace, de grandes quantités de regard tout en écoutant  \cite{burgoon1998nature}. Ces comportements sont présentés plus en détails dans la section ... 


\subsection{Statut social}

	Le statut et le pouvoir sont étroitement liés et parfois confondus \cite{burgoon1998nature}. Pour les sociologues, le statut désigne la position d'une personne dans une hiérarchie sociale \cite{ellyson1985power}, ce qui est répandu dans tous les types de sociétés \cite{lips1991women}.
	Ceci est confirmé  dans la littérature éthologique qui fait l'hypothèse unificatrice que la dominance représente une caractéristique universelle de l'organisation sociale reflétée dans le rang ou la position dans une hiérarchie sociale \cite{burgoon1998nature} et l'accès préférentiel aux ressources \cite{liska1990dominance}.
	
	Ces comportements ont d'abord été observé chez les primates. En effet, lorsque la concurrence pour l'accès prioritaire à la ressource s'installe entre des individus ou des groupes qui ne se connaissent pas, le classement de dominance des individus doit être établi et signalé \cite{burgoon1998nature}. Les membres des groupes primates savent tous où ils se situent et qui a le statut le plus élevé ou le plus de dominance \cite{smither1993authoritarianism}.
	L'universalité de ces signaux  dans les hiérarchies de dominance sociale communes dans les groupes de primates sont également communes dans les groupes humains \cite{burgoon1998nature}.
	
	Deux formes alternatives de dominance ont été identifiées dans la littérature.
	La première forme est la dominance acquise \cite{liska1990dominance}. Elle est associée à des facteurs tels que l'héritabilité, l'âge et l'ordre de naissance qui confèrent un plus grand contrôle ou un meilleur accès à des ressources privilégiées \cite{cattell1970handbook}.
	Par ailleurs, chez les humains, la dominance acquise est souvent assimilée à des rôles sociaux qui peut se manifester par des caractéristiques immuables \textit{(signaux statiques)} ou des indicateurs qui changent lentement \textit{(signaux lents)} de facteurs tels que la position dans une hiérarchie de statut, l'âge, la maturité, la parenté, l'ordre de naissance, et rôles institutionnalisés \cite{burgoonnonverbal}. 
	
	La seconde forme est la dominance sociale \cite{liska1990dominance}, qui est acquise à travers des capacités, des stratégies ou des possibilités d'affiliation démontrées \cite{burgoon1998nature}. Contrairement au caractère immuable des manifestations de domination acquise, la dominance sociale peut se manifester par des indicateurs dynamiques tels que la proximité, la posture, le regard, l'expression faciale, la vocalisation, la durée de la conversation ou l'usage de la langue \cite{keating1985human}. Elle se révèle être particulièrement pertinente pour ceux qui s'intéressent à la communication interpersonnelle, de tels affichages peuvent être liés par les relations, les contextes et le temps\cite{burgoon1998nature}.
		
	Un statut élevé accorde un certain degrés de pouvoir et peut faciliter la dominance parce qu'on est doté d'une autorité légitime, et l'autorité légitime confère à l'individu le potentiel d'une plus grande influence \cite{burgoon2006nonverbal}. 
	
	Cependant, un statut élevé ne garantit pas l'exercice du pouvoir ou l'affichage d'un comportement dominant, et les manifestations de dominance en l'absence de pouvoir légitime peuvent ne pas réussir à exercer une influence \cite{ridgeway1995legitimacy}.
	Par ailleurs, les individus de haut rang ne sont pas forcément puissants ou dominants et les démonstrations de dominance ne placent inévitablement pas haut dans une hiérarchie de statut\cite{burgoonnonverbal}.
	
	\subsection{Relation interpersonnelle}
		Burgoon \emph{et al} \cite{burgoon1998nature} défendent le concept de la dominance et la soumission comme propriétés d'une relation interpersonnelle plutôt que individuelle, et mettent l'accent sur les habiletés sociales et les pratiques de communication qui contribuent à la dominance plutôt que sur les comportements hérités, biologiquement déterminés, contrôlés par la personnalité.
		
		Cependant, la relation interpersonnelle de dominance est souvent confondue avec le pouvoir. Étant fortement corrélés \cite{dunbar2005perceptions}, ils sont souvent regroupés sous une même rubrique \cite{ellyson1985power} et souvent utilisés comme synonymes.
		
		Plusieurs recherches en psychologie sociale \cite{burgoon1998nature,dunbar2005perceptions} se sont intéressés à définir clairement la relation interpersonnelle de dominance, le pouvoir et la relation entre ces deux concepts. 
		
		
		La définition du pouvoir est  restée longtemps conflictuelles dans les différents domaines d'applications. Cependant, la littérature en psychologie sociale et en communication interpersonnelle, converge à définir le pouvoir comme la capacité à produire les effets voulus et, en particulier, la capacité d'influencer le comportement d'une autre personne même face à la résistance  \cite{burgoon2000interactionist,burgoon2006nonverbal,huston1983power}
		\emph{Burgoon et Dunbar} \cite{burgoon1998nature,dunbar2005perceptions} ajoutent que comme le pouvoir est une capacité, il est latent. Par conséquent, il n'est pas toujours exercé et une personne puissante peut ne pas avoir conscience de son pouvoir. De plus, quand il est exercé, il n'est pas toujours couronné de succès et même réussi, il peut avoir une magnitude limitée en fonction de l'environnement \cite{huston1983power}.
		
		French et Raven (1959) ont distingué cinq types de pouvoir: \textit{le pouvoir coercitif, récompense, légitime, expert et référent}. 
		\textit{Le pouvoir coercitif} est basé sur la capacité à utiliser des menaces pour forcer une personne à agir contre son gré et d'administrer une punition pour un comportement indésirable.
		
		Inversement, \textit{le pouvoir de récompense} découle de la capacité à offrir es incitations positives pour récompenser les gens pour un comportement désiré. Par exemple, une augmentation de salaire ou une promotion.
		\textit{Le pouvoir légitime} représente la position d'autorité qui est basé sur les croyances des subalternes selon lesquelles un supérieur a le droit de prescrire et de contrôler leur comportements. Par exemple en fonction sa position dans l'organisation. 
		\textit{Le pouvoir d'expert} peut être dérivé de l'expérience, des connaissances ou de l'expertise dans un domaine donné \cite{van2006power}. 
		Enfin \textit{le pouvoir référent} découle du sentiment d'affiliation au groupe, il est basé sur l'attraction interpersonnelle des subordonnés, leur admiration et leur identification avec un supérieur \cite{van2006power}.
	
		Les spécialistes de la communication et de la psychologie sociale considèrent en grande partie la dominance comme une variable sociale plutôt qu'organismique, mais qui est définie au niveau interpersonnel. Par ailleurs, la dominance et la soumission sont placés dans le contexte des partis multiples, c'est à dire que la dominance est une variable dyadique plutôt que monadique et est définit en fonction des résultats auxquels elle se rapporte  \cite{burgoon1998nature,burgoon2006nonverbal}.
		Comme spécifié par \emph{Mitchell et Maple} \cite{smither1993authoritarianism}, la dominance est le résultat d'une interaction d'événements et dépend donc des individus impliqués dans l'interaction et non de l'individu seul.
		
	 	Contrairement au pouvoir qui peut être latent, la dominance est forcément manifeste et comportementale, elle consiste en des stratégies expressives, basées sur la relation. Elle s'exprimer par un ensemble de transactions complémentaires à travers des actes communicatifs par lesquels l'influence est exercée et accueillie avec acquiescement d'un autre \cite{burgoon2000interactionist,millar1987relational}. 
		
		De plus, même si la soumission-dominance est conceptualisée comme une dimension universelle sur laquelle toutes les relations sociales peuvent être réparties, l'expression de dominance peut être provoquée par une combinaison de tempéraments individuels et de caractéristiques situationnelles	qui encouragent un comportement dominant \cite{burgoon2000interactionist}. Ces derniers, sont sensibles à l'évolution des objectifs, des interlocuteurs, des situations et du temps, entre autres facteurs.
		Ainsi, au cours d'un même épisode, les individus peuvent ajuster leurs expressions de soumission de dominance à des circonstances changeantes.
		
		Comme nous l'avons expliqué plus haut, la dominance et le pouvoir sont deux concepts distincts. Cependant, cette présentation démontre aussi une forte corrélation entre eux. 
		Par exemple, \emph{Burgoon et Dillman} \cite{burgoon1995interpersonal} soutiennent que la dominance n'est qu'un moyen pour exprimer le pouvoir et d'atteindre l'influence voulue.
		Ils ajoutent que la dominance n'est donc pas la voie exclusive du pouvoir, mais plutôt l'un des moyens alternatifs par lesquels le pouvoir est opéré. D'un autre coté, le pouvoir fournit le contexte pour les comportements de dominance. %Par conséquent, elle ne doit pas être considérée comme synonyme de pouvoir %Keltner, Gruenfeld,  Anderson, 2003).
		
		Une autre distinction est que le pouvoir se réfère souvent à de simples potentialités d'influence (telles que reflétées dans des concepts comme les bases de pouvoir, les motivations de pouvoir et le locus de contrôle),
		alors que la dominance est plus souvent liée au comportements réels \cite{dunbar2005perceptions,burgoon1998nature}. 
		Ainsi, alors que le pouvoir permet l'affichage de la dominance, et que le comportement dominant peut solidifier le pouvoir, la dominance et le pouvoir, bien que corrélés, ne sont pas des concepts interchangeables \cite{burgoon1995interpersonal}.
		
		Enfin, la dominance et le pouvoir, appliqués à un contexte social, sont relatifs au partenaire social et ne sont pas absolus \cite{dunbar2005perceptions}.
		
	\subsection{Conclusion}
		Converger vers le fait que nous nous intéressons à la relation interpersonnelle dans le cadre d'une négociation collaborative avec un agent conversationnel.
	\section{Manifestations de la dominance dans les interactions}
		La relation interpersonnelle de dominance se manifeste à travers des actes communicatifs durant l'interactions. Ces actes sont produits soit à travers des comportements verbaux ou non verbaux.
		Nous présentons dans cette section, les comportements communicatifs liés à la relation de dominance. 
		\subsection{Comportements verbaux}
		
		\subsection{Comportements non-verbaux}
			Les indicateurs non verbaux de dominances sont riches et ont été longtemps étudié \cite{burgoon1995interpersonal,burgoon1998nature}. Une manière de diviser les indices non verbaux est par le code général utilisé pour les classer (tels que kinésique, vocalique, proxémique, haptique, etc.) \cite{burgoon2006nonverbal}. En effet, Dunbar et Burgoon \cite{dunbar2005perceptions} résume qu'un communicateur non verbal dominant prototypique serait plus dynamique sur le plan kinésique et vocal (en utilisant plus de gestes, plus de regard, plus d'animation vocale et plus de bavardages) tout en donnant l'impression de relaxation et de confiance .
			
			 Les indices kinésiques étant les plus riches (expression faciale, direction du regard, posture, mouvements corporels et gestes \emph{etc.}).
			Par exemple, Dunbar et Burgoon \cite{dunbar2005perceptions} ont constaté que plus une femme avait un contrôle corporel, plus elle était perçue comme dominante par les observateurs. De plus, les personnes en position de pouvoir  étaient les plus expressifs et  contrôlaient de leur corps. Dans la même veine,  Carney\emph{et al}\cite{hall2005nonverbal} ont constaté que les individus de grande puissance étaient perçus comme se penchant en avant, avaient des positions ouvertes, s'orientaient vers l'autre et avaient une posture plus érigée que ceux de moindre puissance. Par ailleurs, les gestes se rapportent aussi à la perception de dominance.  Carney\emph{et al}  \cite{hall2005nonverbal} ont constaté que les personnes de grande puissance étaient plus susceptibles d'utiliser des gestes, de déclencher plus de tremblements de la main et d'avoir une plus grande fréquence de contact invasif. Dunbar et Burgoon \cite{dunbar2005perceptions} ont constaté que les observateur considéraient les hommes avec une utilisation accrue des gestes d'illustrations comme étant plus dominants.
			
			Le regard apporte aussi un impact sur la perception de la dominance. Dunbar et Burgoon \cite{dunbar2005perceptions} ont constaté que des rapports de dominance visuelle plus élevés étaient corrélés à une dominance plus élevée perçue. En outre, Carney\emph{et al} \cite{hall2005nonverbal} ont trouvé que le regard plus mutuel, plus long, et regarder son interlocuteur en parlant seraient plus appropriés venant d'un individu avec plus de dominance.
			
			Les indicateurs de dominance vocaux incluent la fréquence et la longueur de temps de parole \cite{mast2002dominance}, le volume de la parole, le tempo de la parole, la hauteur, le contrôle vocal, et les interruptions \cite{dunbar2005perceptions}.
		
		\subsection{Conclusion}

\section{Négociation et dominance}
	La négociation a été définie comme un processus décisionnel interpersonnel par lequel deux personnes ou plus s'entendent sur la façon d'allouer des ressources rares \cite{thompson2000mind}. 
	
	La dominance joue un rôle intégral dans la négociation, pour cette raison, une large littérature s'est concentrée sur son impact dans divers domaines et disciplines, notamment la communication, l'économie, la psychologie sociale et la sociologie. Il a été largement prouvé que la dominance, ou dans le cadre de la littérature en négociation le pouvoir, avait une grande influence sur le comportement et les stratégies des négociateurs, et par conséquent sur leurs performances et le résultat de la négociation.
	Premièrement, des recherches ont montré que les négociateurs dominants (avec un pouvoir élevé) avaient des aspirations plus élevées, exigeait plus, concédait moins et à utiliser plus souvent les menaces et les bluffs (Lawler, 1992) en comparaison avec des négociateurs soumis (avec un pouvoir faible) \cite{de1995impact}.
	
	
	 Le pouvoir augmente également l'orientation vers l'action et le comportement axé sur les objectifs \cite{van2006power}. En effet, Galinsky \cite{galinsky2003power} affirme que les négociateurs dominants sont plus susceptibles d'initier la négociation, de faire la première offre et de contrôler le flux de la négociation.  De plus, il a montré que les négociateurs dominants manifestaient une plus grande propension à agir comparés au négociateurs soumis. De plus, ils agissaient d'une manière cohérente avec le but final.
	 
	 
	 En outre, le pouvoir affecte la manière dont les négociateurs collectent des informations sur leurs partenaires \cite{de2004influence}. Les négociateurs de faible puissance ont un fort désir de développer une compréhension précise de leur partenaire de négociation, ce qui les amène à poser plus de questions  de diagnostiques \emph {(diagnostic questions)}.
	 En revanche, les négociateurs dominants sont susceptibles de poser plus de questions tendancieuses \emph {(leading questions)} que des questions de diagnostique. Ce type de question suggère une réponse  qui semble cohérente avec une croyance ou une hypothèse, que cette croyance ou hypothèse soit correcte ou non \cite{galinsky2003power}.
	
	En conséquences de ces comportements relatifs à la dominance, la littérature suggère que les négociateurs dominants sont égocentriques et ont tendance à ne pas prêter attention aux préférences des négociateurs les moins dominants   \cite{fiske1993controlling, de1995impact}. L'idée est que les individus dominants disposent de nombreuses ressources et peuvent souvent agir à leur guise sans conséquences graves \cite{van2006power}. Inversement, les négociateurs moins dominants s'intéressent à leurs partenaires, exhibent un plus grand niveau d'équité et de considération \cite{de1995impact}. Cette stratégie consiste à acquérir ou à reprendre le contrôle de leurs résultats en accordant une attention particulière aux personnes dont elles dépendent \cite{fiske1993controlling}
	
	Giebels \cite{giebels2000interdependence} montre que l'ensemble de ces comportements conduit les négociateurs dominants à se retrouver avec la part la plus grande du gâteau.
	
	Les travaux présentés ci dessus s'intéresse à l'impact du pouvoir individuel de chaque négociateur sur le résultat de la négociation. Cependant, pour comprendre la dynamique de la négociation, il faut s'intéresser au pouvoir à un niveau dyadique ou interpersonnel. 
	Nous présenterons dans la section suivante, la littérature en psychologie qui s'est intéressé à l'influence de la dominance interpersonnelle sur la négociation. 

\subsection{Complémentarité de dominance dans la négociation}

	qu'est la complémentarité
	dyad power
	qu'est qu'il amene
	avec quoi il a été comparé
Intro de négociation et pouvoir %https://msbfile03.usc.edu/digitalmeasures/kimpeter/intellcont/Power%20dynamics%20in%20negotiation%20(AMR,%202005)-1.pdf



What is dominance in social psychology
Nonverbal Behavior and the Vertical Dimension of Social Relations: A Meta-Analysis

\textbf{several representation of dominance :} 
For example, dominance can be defined as a personality trait involving the motive to control others, the self-
perception of oneself as controlling others, and/or as a behavioral
outcome (success in controlling others or their resources). Status,
involving an ascribed or achieved quality implying respect and
privilege, does not necessarily include the ability to control others
or their resources. Similarly, power defined as the capacity or
structurally sanctioned right to control others or their resources
does not necessarily imply prestige or respect. Other distinctions
have also been drawn, including different functional bases of
power, such as reward power, expert power, referent power, or
coercive power (French and Raven, 1959), and outcome dependency
(Stevens  Fiske, 2000). Some writers conceptualize dominance
in terms of social skill (e.g., Burgoon  Dunbar, 2000; Byrne,
2001). Some writers define dominance as the enactment of certain
nonverbal or verbal behaviors (Rosa  Mazur, 1979). Authors do
not use the various verticality terms such as power, dominance,
and status in consistent ways, and often the terms are used without
a clear definition.

\textbf{Burgoon and dunber 2010}

le pouvoir est généralement défini comme la capacité de produire les effets voulus et, en particulier, la capacité d'influencer le comportement d'une autre personne même face à la résistance  %(Bachrach & Lawler, 1981;
%Berger, 1994; Burgoon & Dunbar, 2006; Foa & Foa, 1974; Dunbar, 2004; Emerson,
%1962; French & Raven, 1959; Gray-Little & Burks, 1983; Henley, 1995; Huston, 1983;
%Neff & Harter, 2002; Olson & Cromwell, 1975).


La dominance, d'un autre côté, se réfère à des comportements interactionnels dépendant du contexte et de la relation, dans lesquels le pouvoir est rendu saillant et l'influence est atteinte %(Burgoon et al., 1998; Mack, 1974; Rogers-Millar & Millar, 1979)
dans le contexte de l'échange interpersonnel, la dominance est basée sur une combinaison de tempérament individuel et de caractéristiques situationnelles qui encouragent un comportement dominant. 
Le dominance interpersonnelle est un comportement de communication basé sur la relation qui dépend du contexte et des motivations des individus impliqués

Le pouvoir fournit le contexte pour les comportements de dominance. Bien que la domination puisse parfois être utilisée pour acquérir du pouvoir, elle ne doit pas être considérée comme synonyme de pouvoir %Keltner, Gruenfeld, & Anderson, 2003).

Dans un contexte social, la dominance et le pouvoir sont relatifs au partenaire social et ne sont pas absolus (Dunbar, 2004). 
Comme l'ont noté Emerson (1962) et d'autres théoriciens, le pouvoir équivaut à dépendre des autres et suppose donc une relation sociale.

\textbf{Dyadic power theory:}


Elle se base sur les échanges sociaux de pouvoir sur 4 éléments

\begin{enumerate}
	\item cette définition s'étend pour prendre en compte les échanges sociaux qui inclut les stratégies de communications qui deviennent manifestes durant l'interaction
	\item Contrôle relationnel : elle reconnaît que l'autorité d'utiliser ou d'échanger des ressources dans les interactions est souvent accordée aux individus par les normes sociétales ainsi que par l'histoire relationnelle propre aux individus impliqués dans l'interaction.
	
	\item Cette représentation insiste sur l'idée que le pouvoir doit être vu relativement, ou par rapport aux autres. Ainsi, il reconnaît que le pouvoir est une construction dynamique et multidimensionnelle qui incorpore les perspectives des deux individus dans l'interaction.
	\item En cohérence avec une perspective de communication, il place l'interaction elle-même au centre des préoccupations. Les tentatives de prendre le contrôle de toute interaction donnée par la dominance, bien que déterminées par le pouvoir, sont au centre de la théorie car elles déterminent le résultat du processus  (la décision finale qui a des ramifications pour l'avenir de la relation).
\end{enumerate}




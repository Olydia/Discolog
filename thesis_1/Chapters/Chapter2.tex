%What is dominance in social psychology
%Nonverbal Behavior and the Vertical Dimension of Social Relations: A Meta-Analysis
%
%\textbf{several representation of dominance :} 
%For example, dominance can be defined as a personality trait involving the motive to control others, the self-
%perception of oneself as controlling others, and/or as a behavioral
%outcome (success in controlling others or their resources). Status,
%involving an ascribed or achieved quality implying respect and
%privilege, does not necessarily include the ability to control others
%or their resources.
%
% Similarly, power defined as the capacity or
%structurally sanctioned right to control others or their resources
%does not necessarily imply prestige or respect. Other distinctions
%have also been drawn, including different functional bases of
%power, such as reward power, expert power, referent power, or
%coercive power (French and Raven, 1959), and outcome dependency
%(Stevens  Fiske, 2000). Some writers conceptualize dominance
%in terms of social skill (e.g., Burgoon  Dunbar, 2000; Byrne,
%2001). Some writers define dominance as the enactment of certain
%nonverbal or verbal behaviors (Rosa  Mazur, 1979). Authors do
%not use the various verticality terms such as power, dominance,
%and status in consistent ways, and often the terms are used without
%a clear definition.
%
%\textbf{Burgoon and dunber 2010}
%

%
%
%La dominance, d'un autre côté, se réfère à des comportements interactionnels dépendant du contexte et de la relation, dans lesquels le pouvoir est rendu saillant et l'influence est atteinte %(Burgoon et al., 1998; Mack, 1974; Rogers-Millar & Millar, 1979)
%dans le contexte de l'échange interpersonnel, la dominance est basée sur une combinaison de tempérament individuel et de caractéristiques situationnelles qui encouragent un comportement dominant. 
%Le dominance interpersonnelle est un comportement de communication basé sur la relation qui dépend du contexte et des motivations des individus impliqués
%
 
%Comme l'ont noté Emerson (1962) et d'autres théoriciens, le pouvoir équivaut à dépendre des autres et suppose donc une relation sociale.
%
%\textbf{Dyadic power theory:}
%
%
%Elle se base sur les échanges sociaux de pouvoir sur 4 éléments
%
%\begin{enumerate}
%	\item cette définition s'étend pour prendre en compte les échanges sociaux qui inclut les stratégies de communications qui deviennent manifestes durant l'interaction
%	\item Contrôle relationnel : elle reconnaît que l'autorité d'utiliser ou d'échanger des ressources dans les interactions est souvent accordée aux individus par les normes sociétales ainsi que par l'histoire relationnelle propre aux individus impliqués dans l'interaction.
%	
%	\item Cette représentation insiste sur l'idée que le pouvoir doit être vu relativement, ou par rapport aux autres. Ainsi, il reconnaît que le pouvoir est une construction dynamique et multidimensionnelle qui incorpore les perspectives des deux individus dans l'interaction.
%	\item En cohérence avec une perspective de communication, il place l'interaction elle-même au centre des préoccupations. Les tentatives de prendre le contrôle de toute interaction donnée par la dominance, bien que déterminées par le pouvoir, sont au centre de la théorie car elles déterminent le résultat du processus  (la décision finale qui a des ramifications pour l'avenir de la relation).
%%\end{enumerate}
%
%
%
%
%\textbf{Interpersonal dominance paper}
%En préface, notre intention n'est pas de représenter les trois comme des perspectives exhaustives ou mutuellement exclusives sur la construction de la domination-soumission, mais plutôt de mettre en lumière les nombreuses facettes de cette dimension fondamentale des relations humaines.

Dans ce chapitre, nous présenterons un état de l'art sur la dominance tant en psychologie sociale que les travaux en informatique affective qui s'intéressent à la modélisation de la dominance dans des dispositifs artificiels. 

Dans la première section, nous nous intéressons la dominance comme sujet de recherche en psychologie sociale. En effet, les études sur les comportements de dominance ont fait l'objet d'une pléthores de travaux dans différentes disciplines. Elle a été étudié comme trait individuel représentant la personnalité d'un individu ou son statut social, ou bien sur un aspect dyadique comme relation interpersonnelle. Ces différents travaux ont donné naissances à plusieurs définitions de la dominance ainsi que sa manifestation dans l'interaction.

La seconde section porte sur les travaux en informatique affective concernant l'ajout de relations sociales dans les agents conversationnel. Nous présenterons l'évolution des travaux vers l'ajout de comportements sociaux. Finalement, nous préciserons le positionnement de nos travaux au sein de notre champ de recherche.

\section{Dominance en psychologie sociale}
	La dominance a longtemps été au centre des travaux en psychologie sociale. En effet \emph{Burgoon} soutient que la construction de la dominance tient incontestablement une place primordiale dans la compréhension des actions des humains\cite{burgoon1995interpersonal}. Elle permet de comprendre les comportements des individus dans les interactions et définit la nature des relations interpersonnelle. 
	
	Néanmoins, il n'existe pas une définition claire du concept de la dominance. Souvent confondu avec le pouvoir \cite{burgoon2000interactionist,dunbar2005perceptions}, elle est définie comme trait de personnalité ou comme un statut, impliquant une qualité attribuée ou obtenue qui entraîne le respect et le privilège dans une hiérarchie \cite{hall2005nonverbal}. En parallèle, elle définit aussi une relation interpersonnelle. Nous présentons dans cette section les différentes représentation de la dominance dans la littérature en psychologie sociale et les comportements qui y sont associés.  
	
\subsection{Trait de personnalité}
%Les chercheurs qui se sont intéressés à l'étude de la personnalité tendent à voir la personnalité comme stable \cite{burgoon2006nonverbal} et par conséquent, cherchent a définir des indicateurs de dominance stable permettant de de former  des profils comportementaux stables. 

La dominance comme trait de personnalité a connu une évolution où la littérature en psychologie a élargit le concept de la dominance au-delà des comportements d'agressivité.
En effet, les travaux en Éthologie et psychologie évolutive ont d'abord étudié la dominance comme un trait de personnalité innée qui confère à la personne la capacité d'exercer son pouvoir et sa domination, susciter la déférence et l'acquiescement, ou apaiser et se soumettre à un comportement conspécifique plus fort \cite{keltner1995signs,burgoon2006nonverbal}. Cependant, les travaux en Éthologie contemporaine s'éloignent de cette définition en faveur d'études afin de mieux comprendre l'interaction entre les organismes et leur environnement social \cite{burgoon2006nonverbal}.

Pour les psychologues de la personnalité, la dominance est considérée comme un trait individuel durable qui désigne le tempérament de la personne et les prédispositions comportementales de chacun \cite{cattell1970handbook,ridgeway1987nonverbal} tels que l'agression, l'ambition, l'argumentation,
assertivité, vantardise, confiance et détermination.
Les adjectifs communs utilisés pour décrire la personne dominante sont affirmés, agressifs, compétitifs, exigeant, égoïste et têtu. Plus précisément, la dominance en tant que variable de personnalité «montre un ajustement réaliste constant au succès et à l'échec de l'individu, à la santé ou à la maladie, aux capacités ou handicaps et aux forces extérieures relatives» \cite{cattell1970handbook,burgoon1998nature}.
En effet, selon  Emmons et McAdams \cite{emmons1991personal}, les individus avec un motif de haut pouvoir sont décrits comme voulant contrôler et influencer les autres, désirant la célébrité ou l'attention du public, et ayant la capacité de susciter l'émotion chez les autres. Dans la même veine, \cite{jackson1974personality} décrivent une personne dominante comme une personne qui cherche et maintient un rôle de leader dans un groupe. Ces personnes prennent en charge et guident les membres du groupe vers la réalisation d'objectifs louables à travers des tactiques d'influence, des manœuvres de contrôle de l'environnement et l'expression énergique de l'opinion \cite{burgoon1998nature}. 
A l'opposé, les individus non dominants ou soumis sont présentés comme coopératifs, modestes et obéissants, mais aussi obséquieux, doux, faibles, peu sûrs et évitant les situations qui nécessitent une affirmation de soi \cite{burgoon1998nature}. 

%Les habiletés sociales font partie de cette équation, car la capacité d'être énergique, de prendre des initiatives et d'être expressif mais détendu et équilibré \cite{burgoon2006nonverbal}. 
Par ailleurs, la dominance et soumission sont généralement révélés à travers le style de communication \cite{burgoon1998nature}. Une personne dominante va avoir plus d'assurance dans sa manière de communiquer, à être plus confiante, enthousiaste, énergique, active, compétitive, sûre d'elle, vaniteuse et direct. Ceci transparaît dans les comportements verbaux et non verbaux utilisés. Une personne dominante va  utiliser plus d'espace, avec un temps de conversation plus long et être capable d'avoir plus d'interruption réussies \cite{burgoon1998nature}. 

En revanche, une personne soumise va utiliser un mouvement contraint, une utilisation limitée de l'espace, de grandes quantités de regard tout en écoutant  \cite{burgoon1998nature}. Ces comportements sont présentés plus en détails dans la section  \ref{sec:manifesationDom}


\subsection{Statut social}

	Le statut et le pouvoir sont étroitement liés et parfois confondus \cite{burgoon1998nature}. Pour les sociologues, le statut désigne la position d'une personne dans une hiérarchie sociale \cite{ellyson1985power}, ce qui est répandu dans tous les types de sociétés \cite{lips1991women}.
	Ceci est confirmé  dans la littérature éthologique qui fait l'hypothèse unificatrice que la dominance représente une caractéristique universelle de l'organisation sociale reflétée dans le rang ou la position dans une hiérarchie sociale \cite{burgoon1998nature} et l'accès préférentiel aux ressources \cite{liska1990dominance}.
	
	Ces comportements ont d'abord été observé chez les primates. En effet, lorsque la concurrence pour l'accès prioritaire à la ressource s'installe entre des individus ou des groupes qui ne se connaissent pas, le classement de dominance des individus doit être établi et signalé \cite{burgoon1998nature}. Les membres des groupes primates savent tous où ils se situent et qui a le statut le plus élevé ou le plus de dominance \cite{smither1993authoritarianism}.
	L'universalité de ces signaux  dans les hiérarchies de dominance sociale communes dans les groupes de primates sont également communes dans les groupes humains \cite{burgoon1998nature}.
	
	Deux formes alternatives de dominance ont été identifiées dans la littérature.
	La première forme est la dominance acquise \cite{liska1990dominance}. Elle est associée à des facteurs tels que l'héritabilité, l'âge et l'ordre de naissance qui confèrent un plus grand contrôle ou un meilleur accès à des ressources privilégiées \cite{cattell1970handbook}.
	Par ailleurs, chez les humains, la dominance acquise est souvent assimilée à des rôles sociaux qui peut se manifester par des caractéristiques immuables \textit{(signaux statiques)} ou des indicateurs qui changent lentement \textit{(signaux lents)} de facteurs tels que la position dans une hiérarchie de statut, l'âge, la maturité, la parenté, l'ordre de naissance, et rôles institutionnalisés \cite{burgoonnonverbal}. 
	
	La seconde forme est la dominance sociale \cite{liska1990dominance}, qui est acquise à travers des capacités, des stratégies ou des possibilités d'affiliation démontrées \cite{burgoon1998nature}. Contrairement au caractère immuable des manifestations de domination acquise, la dominance sociale peut se manifester par des indicateurs dynamiques tels que la proximité, la posture, le regard, l'expression faciale, la vocalisation, la durée de la conversation ou l'usage de la langue \cite{keating1985human}. Elle se révèle être particulièrement pertinente pour ceux qui s'intéressent à la communication interpersonnelle, de tels expressions peuvent être liées par les relations, les contextes et le temps\cite{burgoon1998nature}.
		
	Un statut élevé accorde un certain degrés de pouvoir et peut faciliter la dominance parce qu'on est doté d'une autorité légitime, et l'autorité légitime confère à l'individu le potentiel d'une plus grande influence \cite{burgoon2006nonverbal}. 
	
	Cependant, un statut élevé ne garantit pas l'exercice du pouvoir ou l'affichage d'un comportement dominant, et les manifestations de dominance en l'absence de pouvoir légitime peuvent ne pas réussir à exercer une influence \cite{ridgeway1995legitimacy}.
	Par ailleurs, les individus de haut rang ne sont pas forcément puissants ou dominants et les démonstrations de dominance ne placent inévitablement pas haut dans une hiérarchie de statut\cite{burgoonnonverbal}.
	
	\subsection{Relation interpersonnelle}
		Burgoon \emph{et al} \cite{burgoon1998nature} défendent le concept de la dominance et la soumission comme propriétés d'une relation interpersonnelle plutôt que individuelle, et mettent l'accent sur les habiletés sociales et les pratiques de communication qui contribuent à la dominance plutôt que sur les comportements hérités, biologiquement déterminés, contrôlés par la personnalité.
		
		Cependant, la relation interpersonnelle de dominance est souvent confondue avec le pouvoir. Étant fortement corrélés \cite{dunbar2005perceptions}, ils sont souvent regroupés sous une même rubrique souvent utilisés comme synonymes  \cite{ellyson1985power,burgoon1998nature}.
		
		Plusieurs recherches en psychologie sociale \cite{burgoon1998nature,dunbar2005perceptions,burgoon2006nonverbal} se sont intéressés à définir clairement la relation interpersonnelle de dominance, le pouvoir et la relation entre ces deux concepts. 
		
		
		La définition du pouvoir est  restée longtemps conflictuelles dans les différents domaines d'applications. Cependant, la littérature en psychologie sociale et en communication interpersonnelle, converge à définir le pouvoir comme la capacité à produire les effets voulus et, en particulier, la capacité d'influencer le comportement d'une autre personne même face à la résistance  \cite{burgoon2000interactionist,burgoon2006nonverbal,huston1983power}.
		\emph{Burgoon et Dunbar} \cite{burgoon1998nature,dunbar2005perceptions} ajoutent que comme le pouvoir est une capacité, il est latent. Par conséquent, il n'est pas toujours exercé et une personne avec un pouvoir élevé peut ne pas avoir conscience de son pouvoir. De plus, quand il est exercé, il n'est pas toujours couronné de succès et même réussi, il peut avoir une magnitude limitée en fonction de l'environnement \cite{huston1983power}.
		
		\emph{French et Raven}\cite{french1959bases} ont distingué cinq types de pouvoir: \textit{le pouvoir coercitif, récompense, légitime, expert et référent}. 
		\textit{Le pouvoir coercitif} est basé sur la capacité à utiliser des menaces pour forcer une personne à agir contre son gré et d'administrer une punition pour un comportement indésirable.
		
		Inversement, \textit{le pouvoir de récompense} découle de la capacité à offrir es incitations positives pour récompenser les gens pour un comportement désiré. Par exemple, une augmentation de salaire ou une promotion.
		\textit{Le pouvoir légitime} représente la position d'autorité qui est basé sur les croyances des subalternes selon lesquelles un supérieur a le droit de prescrire et de contrôler leur comportements (\emph{e.x.} en fonction sa position dans l'organisation). 
		\textit{Le pouvoir d'expert} peut être dérivé de l'expérience, des connaissances ou de l'expertise dans un domaine donné \cite{van2006power}. 
		Enfin \textit{le pouvoir référent} découle du sentiment d'affiliation au groupe, il est basé sur l'attraction interpersonnelle des subordonnés, leur admiration et leur identification au un supérieur \cite{van2006power}.
	
		Les spécialistes de la communication et de la psychologie sociale considèrent en grande partie la dominance comme une variable sociale plutôt qu'organismique, mais qui est définie au niveau interpersonnel. En outre, la dominance est présentée comme une variable dyadique plutôt que monadique et est définit en fonction des résultats auxquels elle se rapporte  \cite{burgoon1998nature,burgoon2006nonverbal}.
		Comme spécifié par \emph{Mitchell et Maple} \cite{smither1993authoritarianism}, la dominance est le résultat d'une interaction d'événements et dépend donc des individus impliqués dans l'interaction et non de l'individu seul.
		
	 	Contrairement au pouvoir qui peut être latent, la dominance est forcément manifeste et comportementale, elle consiste en des stratégies expressives, basées sur la relation. Elle s'exprime par un ensemble de transactions complémentaires à travers des actes communicatifs par lesquels l'influence est exercée et accueillie avec acquiescement d'un autre \cite{burgoon2000interactionist,millar1987relational}. 
		
		De plus, même si la soumission-dominance est conceptualisée comme une dimension universelle sur laquelle toutes les relations sociales peuvent être réparties, l'expression de dominance peut être provoquée par une combinaison de tempéraments individuels et de caractéristiques situationnelles	qui encouragent un comportement dominant \cite{burgoon2000interactionist}. Ces derniers, sont sensibles à l'évolution des objectifs, des interlocuteurs, des situations et du temps, entre autres facteurs.
		Ainsi, au cours d'un même épisode, les individus peuvent ajuster leurs expressions de soumission de dominance à des circonstances changeantes.
		
		Comme nous l'avons expliqué plus haut, la dominance et le pouvoir sont deux concepts distincts. Cependant, cette présentation démontre aussi une forte corrélation entre eux. 
		Par exemple, \emph{Burgoon et Dillman} \cite{burgoon1995interpersonal} soutiennent que la dominance n'est qu'un moyen pour exprimer le pouvoir et d'atteindre l'influence voulue.
		Ils ajoutent que la dominance n'est donc pas la voie exclusive du pouvoir, mais plutôt l'un des moyens alternatifs par lesquels le pouvoir est opéré. D'un autre coté, le pouvoir fournit le contexte pour les comportements de dominance. %Par conséquent, elle ne doit pas être considérée comme synonyme de pouvoir %Keltner, Gruenfeld,  Anderson, 2003).
		
		Une autre distinction est que le pouvoir se réfère souvent à de simples potentialités d'influence (telles que reflétées dans des concepts comme les bases de pouvoir, les motivations de pouvoir et le locus de contrôle),
		alors que la dominance est plus souvent liée au comportements réels \cite{dunbar2005perceptions,burgoon1998nature}. 
		Ainsi, alors que le pouvoir permet l'affichage de la dominance, et que le comportement dominant peut solidifier le pouvoir, la dominance et le pouvoir, bien que corrélés, ne sont pas des concepts interchangeables \cite{burgoon1995interpersonal}.
		
		Enfin, la dominance et le pouvoir, appliqués à un contexte social, sont relatifs au partenaire social et ne sont pas absolus \cite{dunbar2005perceptions}.
		

		
	\subsection{Manifestations de la dominance dans les interactions}
	\label{sec:manifesationDom}
		La dominance est une construction multi-facettes qui peut être démontrée interactivement de plusieurs façons et dont la signification dépend à la fois du contexte et de la perception \cite{dunbar2005perceptions}. Sa manifestation dans l'interaction se fait à travers des actes de communications qui s'expriment tant par des comportements verbaux que non verbaux.
		Nous présentons dans cette section, les comportements communicatifs liés à la relation de dominance. 
		
		\subsection{Comportements verbaux}
			La perception de dominance est liée à des comportement vocaux tel que le taux de paroles en terme de nombre de mots utilisés lors d'une conversation ainsi que la longueur du message à chaque tour de parole \cite{dunbar2005perceptions}. Le volume sonore mesuré en amplitude, la fréquence et la verbosité sont aussi des indicateurs vocaux de dominance \cite{dunbar2005perceptions}. 
			Un autre facteur important qui reflète la dominance est la fréquence d'interruptions fait par l'individu \cite{dunbar2005perceptions,hall2005nonverbal}. En effet, les individus qui réussissent à faire des interruptions sont perçus comme dominants. A l'opposé, ceux qui marquent des disfluences montrent un signe de manque de contrôle et de dominance. 

			\emph{Burgoon et Dunbar} \cite{dunbar2005perceptions} ont fait un état de l'art sur les comportements verbaux de dominance en les associant des stratégies d'interaction. En effet, ils ont identifié dans la littérature différentes stratégies assimilées à des comportements de dominance. Ils citent d'abord \emph{Canary et Spitzberg} \cite{canary1987} qui présentent trois stratégies: 
			\begin{itemize}
				\item Les stratégies intégratives qui représentent des stratégies coopératives.
				\item Les tactiques distributives qui sont concurrentielles et antagonistes.
				\item Les stratégies d'évitement dont le but est de dissiper les discussions conflictuelles. 
			\end{itemize}
			
			Parallèlement, ils présentent les travaux de \emph{Frieze et McHugh} \cite{frieze1992power} qui incluent six types de stratégies verbales directes et indirects. 
			Les stratégies directes incluent les stratégies	directes positives en abordant ouvertement le problème ou la question discutée. Les stratégies autres-directes par lesquelles l'interlocuteur se réfère à des expériences passées similaires ou se projette sur ce que feraient les autres dans la même situation. Enfin la stratégie directe coercitive en utilisant une coercition verbale et physique. 
			
			Concernant les stratégies indirectes, les auteurs citent trois stratégies : les stratégies indirectes positives en adoptant un comportement conciliant et gentil, les stratégies ignorer-indirecte en évitant les situations de conflits comme ignorer le problème ou nier l'existence d'un conflit. Enfin, les stratégies de retrait indirect, par exemple dans le contexte d'une relation de couple, exprimer un retrait émotionnel ou le désire de quitter la relation. 
		
			En règle générale, les stratégies directes sont considérées comme des stratégies plus dominantes que les stratégies indirectes \cite{dunbar2005perceptions}.
			
		\subsection{Comportements non-verbaux}
			Les indicateurs non verbaux de dominances sont riches et ont été longtemps étudié \cite{burgoon1995interpersonal,burgoon1998nature}. Une manière de diviser les indices non verbaux est par le code général utilisé pour les classer (tels que kinésique, vocalique, proxémique, haptique, etc.) \cite{burgoon2006nonverbal}. En effet, Dunbar et Burgoon \cite{dunbar2005perceptions} résume qu'un communicateur non verbal dominant prototypique serait plus dynamique sur le plan kinésique et vocal (en utilisant plus de gestes, plus de regard, plus d'animation vocale et plus de bavardages) tout en donnant l'impression de relaxation et de confiance .
			
			 Les indices kinésiques étant les plus riches (expression faciale, direction du regard, posture, mouvements corporels et gestes \emph{etc.}).
			Par exemple, Dunbar et Burgoon \cite{dunbar2005perceptions} ont constaté que plus une femme avait un contrôle corporel, plus elle était perçue comme dominante par les observateurs. De plus, les personnes en position de pouvoir  étaient les plus expressifs et  contrôlaient de leur corps.
			A l'opposé,  Carney\emph{et al}\cite{hall2005nonverbal} ont constaté que les individus dominants étaient perçus comme se penchant en avant, avaient des positions ouvertes, s'orientaient vers l'autre et avaient une posture plus érigée que ceux de moindre dominance, qui eux, avaient tendance à se resserrer et à occuper peu de place. Par ailleurs, les gestes se rapportent aussi à la perception de dominance.  
			Dans la même veine, les chercheurs ont constaté que les personnes perçues comme dominantes étaient plus susceptibles d'utiliser des gestes, de déclencher plus de tremblements de la main et d'avoir une plus grande fréquence de contact invasif \cite{hall2005nonverbal}.
			 Dunbar et Burgoon \cite{dunbar2005perceptions} ont constaté que les observateur considéraient les hommes avec une utilisation accrue des gestes d'illustrations comme étant plus dominants.
			
			Le regard apporte aussi un impact sur la perception de la dominance. Dunbar et Burgoon \cite{dunbar2005perceptions} ont constaté que des rapports de dominance visuelle plus élevés étaient corrélés à une dominance plus élevée perçue. En outre, Carney\emph{et al} \cite{hall2005nonverbal} ont trouvé que le regard plus mutuel, plus long, et regarder son interlocuteur en parlant seraient plus appropriés venant d'un individu avec plus de dominance.
			
			Les indicateurs de dominance vocaux incluent la fréquence et la longueur de temps de parole \cite{mast2002dominance}, le volume de la parole, le tempo de la parole, la hauteur, le contrôle vocal, et les interruptions \cite{dunbar2005perceptions}.
		
		\subsection{Conclusion}
			
			Ces travaux mettent en lumière l'importance de la relation de dominance dans les interaction. Plusieurs travaux ont avancé l'hypothèse que la relation de dominance était la première dimension qui s'installe lors d'une interaction. 
			
			Cet impact a poussé la communauté qui étudie la négociation à sy intéresser et à étudier comment les comportements de dominances impactent les stratégies de négociation et leur pouvoir dans la négociation. 
			Nous présenterons la section suivante un état de l'art des études de la communauté en psychologie sociale.
			
			
	\section{Négociation et dominance}
	La négociation a été définie comme un processus décisionnel interpersonnel par lequel deux personnes ou plus s'entendent sur la façon d'allouer des ressources rares \cite{thompson2000mind}. 
	
	La dominance joue un rôle intégral dans la négociation, pour cette raison, une large littérature s'est concentrée sur son impact dans divers domaines et disciplines, notamment la communication, l'économie, la psychologie sociale et la sociologie.
	 Il a été largement prouvé que la dominance, ou dans le cadre de la littérature en négociation le pouvoir, avait une grande influence sur le comportement et les stratégies des négociateurs, et par conséquent sur leurs performances et le résultat de la négociation.
	
	Premièrement, des recherches ont montré que les négociateurs dominants (avec un pouvoir élevé) avaient des aspirations plus élevées, exigeait plus, concédait moins et utilisaient plus souvent les menaces et les bluffs en comparaison avec des négociateurs soumis (avec un pouvoir faible) \cite{de1995impact}.
	
	
	 La dominance augmente également l'orientation vers l'action et encourage un comportement axé sur les objectifs \cite{van2006power}. En effet, Galinsky \cite{galinsky2003power} affirme que les négociateurs dominants sont plus susceptibles d'initier la négociation, de faire la première offre et de contrôler le flux de la négociation.  De plus, il a montré que les négociateurs dominants manifestaient une plus grande propension à agir comparés au négociateurs soumis et leurs agissements sont cohérents avec le but final.
	 
	 
	 En outre, la dominance affecte la manière dont les négociateurs collectent des informations sur leurs partenaires \cite{de2004influence}. Les négociateurs soumis ont un fort désir de développer une compréhension précise de leur partenaire de négociation, ce qui les amène à poser plus de questions  de diagnostiques \emph {(diagnostic questions)}.
	 En revanche, les négociateurs dominants sont susceptibles de poser plus de questions tendancieuses \emph {(leading questions)}. Ce type de question suggère une réponse  qui semble cohérente avec une croyance ou une hypothèse, que cette croyance ou hypothèse soit correcte ou non \cite{galinsky2003power}.
	
	En conséquences de ces comportements relatifs à la dominance, la littérature suggère que les négociateurs dominants sont égocentriques et ont tendance à ne pas prêter attention aux préférences des négociateurs moins dominants   \cite{fiske1993controlling, de1995impact}. L'idée est que les individus dominants disposent de nombreuses ressources et peuvent souvent agir à leur guise sans conséquences graves \cite{van2006power}. Inversement, les négociateurs moins dominants s'intéressent à leurs partenaires, exhibent un plus grand niveau d'équité et de considération \cite{de1995impact}. Cette stratégie consiste à acquérir ou à reprendre le contrôle de leurs résultats en accordant une attention particulière aux personnes dont elles dépendent \cite{fiske1993controlling}.
	
	Giebels \cite{giebels2000interdependence} montre que l'ensemble de ces comportements conduit les négociateurs dominants à se retrouver avec la part la plus grande du gâteau.
	
	Les travaux présentés ci-dessus s'intéressent à l'impact des comportements de dominance à une échelle individuelle. Cependant, pour comprendre la dynamique de la négociation, il faut s'intéresser à la dominance à un niveau dyadique ou interpersonnel. 
	Nous présenterons dans la section suivante, la littérature en psychologie qui s'est intéressée à l'influence de la dominance interpersonnelle sur la négociation. 

	\subsection{Complémentarité de dominance dans la négociation}
	Afin de comprendre pleinement la relation entre les comportements de dominance, les stratégies et les résultats de la négociation, les chercheurs doivent considérer simultanément la dominance absolue des deux négociateurs et leur dominance par rapport à leurs adversaires c'est à dire la dominance interpersonnelle. 
	
	Pour cette raison, les chercheurs en psychologie sociale se sont intéressés à la complémentarité des comportements de dominance exprimés par les négociateurs. 
	Les théoriciens de Circumplex affirment que les comportements des individus seront similaires aux partenaires interactionnels dans la dimension d'affiliation et inverse  de la dimension de contrôle qui est la dimension de la dominance \cite{tiedens2003power}.
	Par conséquent, quand cette complémentarité apparaît, elle permet d'obtenir une meilleure coordination entre les individus \cite{wiltermuth2015benefits}. \emph{Estroff et Nowicki} \cite{estroff1992interpersonal} ont constaté que des paires complémentaires de participants obtenaient ensemble de meilleurs résultats dans une tâche de puzzle que les couples non-complémentaires. Ces résultats suggèrent qu'une complémentarité entre deux partenaires renforce leur attractivité mutuelle.
	A l'opposé, cette dynamique de coordination n'est pas retrouvé dans des dyades où les deux partenaires sont dominants ou les deux sont soumis. 
	Dans le premier cas, les partenaires d'interaction luttent pour le contrôle, ce qui rend difficile la collaboration. Bien que le trait de contrôle de la tâche peut faciliter l'évolution du processus en cas d'accord, il peut facilement rendre difficile la coordination en cas de conflit causé par des affrontements mutuels \cite{wiltermuth2015benefits}. Dans le cas où les de deux partenaires d'interaction sont soumis, peu de choses peuvent être accomplies car aucune direction n'est définie et les partenaires ne prennent pas d'initiatives \cite{wiltermuth2015benefits}.
	 
	
	En parallèle, plusieurs études ont analysé l'apport de ces comportements à la création de valeurs dans la négociation.
	
	La création de valeurs permet de trouver des solutions qui répondent aux besoins et aux intérêts les plus importants des deux parties. Elle survient lorsqu'il existe des différences dans les préférences des négociateurs, y compris dans l'utilité attribuée  aux éléments en cours de négociation \cite{lax1986managerial}. 
	Il est important de mieux comprendre ce processus de création de valeur, car la capacité d'identifier et de mettre en œuvre des résultats mutuellement bénéfiques est la clé pour des accords durables \cite{wiltermuth2015benefits}.
	De plus, \emph{Olekalns et al} affirme que le succès ou l'échec de toute stratégie dans le processus de création de valeur est partiellement déterminée par le contexte social notamment la relation de dominance \cite{olekalns2013dyadic}.
	
	\emph{Wiltermuth et al} soutient que la création de valeur accrue se produit lorsque les comportements des négociateurs créent une dynamique de complémentarité de la dominance, caractérisée par le fait qu'une personne dans une interaction dyadique se comporte de manière dominante et que son homologue se comporte de façon soumise \cite{wiltermuth2015benefits}. De plus, les négociateurs exprimant de la dominance facilitent le processus de découverte de solutions mutuellement bénéfiques lorsque leur domination suscite la soumission de leurs homologues. De ce fait, les dyades complémentaires arrivent à atteindre un gain commun plus important comparés aux autres dyades.
	
	Cette amélioration du gain est aussi due à une meilleure distributions des rôles dans les dyades complémentaires qui leur permet d'augmenter et améliorer l'échange d'informations. 
	
	L'ensemble de ces résultats suggèrent que la complémentarité améliore le sentiment de confort et d'affection entre les négociateurs. Ceci est confirmé par les travaux de \emph{Tiedens et al} \cite{tiedens2003power} qui ont montré à travers des études que lorsqu'une complémentarité de dominance s'installe entre deux individus, ces derniers se sentent plus à l'aise dans l'interaction. 
	
	
	
%Intro de négociation et pouvoir %https://msbfile03.usc.edu/digitalmeasures/kimpeter/intellcont/Power%20dynamics%20in%20negotiation%20(AMR,%202005)-1.pdf


	\section{Modèles de négociations automatiques}
	
	Au cours des dernières années, de nombreux nouveaux modèles d'agents négociateurs ont été développés. Deux champs de recherche liés à ce domaine peuvent être distingués: les systèmes de soutien à la négociation \emph{ negotiation support system (NSS)} et les agents de négociation automatisés \emph{automated negotiation agents (ANA)}.
	
%	\textit{Jonker et al} \cite{jonker2012negotiating} présentent dans un travail un état de l'art sur les systèmes de support à la négociation.

	 les NNS sont utilisés pour aider les novices en négociation à améliorer leurs compétences. Les données collectées permettent aux chercheurs de mieux comprendre les difficultés que rencontrent les négociateurs au cours d'une négociation \cite{jonker2012negotiating}. Néanmoins, ces systèmes sont encore peu utilisés, nous citerons les systèmes tel que \emph{Inspire} [ref] un système complet avec une interface de communication, il permet de spécifier les préférences et de gérer les offres au cours de la négociation. De plus, il permet aux utilisateurs de faire une analyse post-accord. 
	 Il a été utilisé comme  un outils de simulation afin de mieux préparer les négociateurs au cas réels ce qui le rend un bon outils d'apprentissage de négociation. Un autre outils d'éducation est le système \emph{Athena}. Enfin, nous citons le système \emph{SmartSettle} qui permet d'assister les utilisateurs lors de négociations complexes. Il est fournis avec un support de propositions qui permet de faire des suggestions aux négociateurs. 
	  
	Concernant les recherches pour développer des ANAs, les travaux se sont principalement concentrés sur trois aspects à savoir la conception de stratégies de négociation, des protocoles, et la modélisation du domaine de négociation \cite{jonker2012negotiating}. De plus, depuis l'émergence de la compétition internationale \emph{International Automated Negotiating Agents Competition (ANAC)} \cite{reftoANAc}, une compétition qui regroupe les meilleurs agents négociateurs, de nouveaux modèles de négociations sont proposés chaque année. 
	
	Ces nouveaux modèles innovent sur l'aspect rationnel de l'agent, et proposent de nouvelles stratégies qui permettent à l'agent d'atteindre de meilleurs gains. 
	En effet, les stratégies de négociation prennent en compte trois modules : un module pour décider de l'acceptabilité des offres reçues, un module pour définir la stratégie d'enchères et enfin un modèle de l'adversaire afin calculer ses préférences et décider lesquels prendre en compte dans la prise de décision \cite{baarslag2014decoupling}.
	
	Par exemple, \textit{l'agent K} gagnant d'\emph{ANAC 10} dispose d'un mécanisme sophistiqué pour accepter une offre. En effet, il utilise la moyenne et la variance de l'utilité de toutes les offres reçues, puis détermine la probabilité de recevoir une meilleure offre dans le futur et définit sa cible de proposition. En conséquence, il accepte ou rejette l'offre. Sa stratégie d'enchère consiste à d'abord définir une valeur d'utilité cible pour les propositions et va aléatoirement sélectionner des offres dont l'utilité est dans l'intervalle de valeurs cibles. Si aucune proposition n'est trouvée, l'agent décroît valeur d'utilité cible.	Cependant, ce système ne prend pas en compte les préférences de son adversaire. 
	
	A l'opposé, plusieurs travaux ont mis en avant l'importance de définir un modèle de l'adversaire \cite{}. Par exemple, l'agent \textit{IAMhaggler2011} utilise une technique de régression gaussienne afin de prédire la stratégie de son adversaire. Un élément clé d'une négociation réussie est d'optimiser sa stratégie de concessions. A cet effet, l'agent prend en comptes la stratégie de concessions de son adversaire calculé à partir de la gaussienne et les contraintes temporelles liées à la négociation pour définir sa stratégie de concessions. 
	
	
	Les agents proposés ci-dessus sont conçus pour améliorer leurs gains personnels dans le contextes de négociations compétitives. Un autre champs de recherche vise à proposer des modèles de négociations pour optimiser les gains des deux négociateurs, ce type de négociation est nommé négociation coopérative ou collaborative. 
	
	Dans un contexte où des agents partagent des ressources importantes et limitées, ils peuvent faire face à des situations de conflit pour l'allocation de ces ressources. Dans ce cadre précis, une négociation coopérative peut être décrite comme un processus de prise de décision pour résoudre un conflit impliquant deux ou plusieurs parties sur plusieurs objectifs interdépendants, mais non mutuellement exclusifs \cite{lewicki2011essentials}. 
	
	Par exemple, \cite{wollkind2004automated} propose un système multi-agents pour gérer l'espace aérien national. Le modèle proposé utilise des stratégies de négociation collaborative pour la résolution des conflits de trafic aérien. Les agents utilisent un protocole de concessions monotoniques \cite{Zlotkin and Rosenschein} qui consiste à faire des propositions et contre-propositions en diminuant l'utilité de ces dernières jusqu'à atteindre une offre qui peut être acceptée par les deux agents. 
	
	Par ailleurs, \emph{Gutman et Maes} [13] discutent de la différence entre les modèles de négociation compétitive et coopérative. Ils montrent que les négociations compétitives sur les marchés de détail sont inutilement hostiles aux clients et n'offrent aucun avantage à long terme aux commerçants.
	Essentiellement, dans des négociations concurrentielles, le commerçant est opposé au client dans un tiraillement de guerre. Ils concluent que les commerçants se soucient souvent moins du profit sur une transaction donnée et se soucient davantage de la rentabilité à long terme, ce qui implique la satisfaction du client et des relations clients à long terme. 
	Dans cette continuité, \emph{Jonker et al} \cite{jonker2007agent} proposent un modèle de négociation multi-critères coopérative pour le e-commerce pour lequel ils utilisent une approche heuristique pour calculer l'utilité des critères et modélisent des agents capables d'explorer conjointement l'espace des résultats possibles avec des informations limitées. 
	
	Revenir sur 
	
%	Mixed stratégie
%	Cet agent joue une stratégie tit-for-tat par rapport à sa propre utilité. L'agent coopérera d'abord, puis répondra en nature à l'action précédente de l'adversaire, tout en visant le point Nash dans le scénario. Si l'offre de l'opposant améliore son utilité, l'agent concède en conséquence. L'agent est gentil dans le sens où il ne riposte pas. Par conséquent, lorsque l'adversaire fait une offre qui réduit l'utilité de l'agent, l'Agent Tit-for-Tat de Nice suppose que l'adversaire a fait une erreur et ne fait rien, attendant une meilleure offre. Cette approche est basée sur [14]. Nice Tit-for-Tat Agent maintient un modèle bayésien [15] de son adversaire, mis à jour après chaque coup par l'adversaire. Ce modèle est utilisé pour essayer d'identifier les enchères optimales de Pareto afin de pouvoir répondre à une concession de l'adversaire avec un bon mouvement. L'agent essayera de refléter la concession de l'adversaire en fonction de sa propre fonction d'utilité. L'agent détecte des scénarios très coopératifs pour viser un peu plus que l'utilitaire Nash. En outre, si le domaine est important, si le facteur d'actualisation est élevé ou si le temps est écoulé, l'agent fera des concessions plus importantes en ce qui concerne sa cible d'enchères. L'agent tente d'optimiser l'utilité de l'adversaire en effectuant un certain nombre d'enchères différentes avec approximativement cet utilitaire de cible d'enchères.
	
	%http://www.ifaamas.org/Proceedings/aamas08/proceedings/pdf/paper/AAMAS08_0385.pdf
	
	 %https://ac.els-cdn.com/S0004370212001105/1-s2.0-S0004370212001105-main.pdf?_tid=3fb41905-2d3a-4662-aa2a-5a608b9d8587&acdnat=1528135687_1d2be1d9ea1f434c06689901851602de

	
	Un aspect important dans la modélisation d'un agent négociateur est le module de communication qu'il utilise pour interagir avec son adversaire 
	

	
	
	
	  
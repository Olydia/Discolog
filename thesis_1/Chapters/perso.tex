\subsection{Trait de personnalité}
Le dominance a été longtemps étudié comme un trait individuel 

1.impliquant le motif de contrôler les autres, de percevoir soi comme ayant le contrôle les autres, et / ou en tant que résultat comportemental (succès dans le contrôle des autres ou de leurs ressources \cite{hall2005nonverbal}

2. Pour les psychologues de la personnalité, la dominance est considérée comme un trait individuel durable qui désigne le tempérament caractéristique et les prédispositions comportementales de chacun (par exemple, Cattell, Eber et Tatsuoka, 1970; Ridgeway, 1987).
Les habiletés sociales font partie de cette équation, car la capacité d'être énergique, de prendre des initiatives et d'être expressif mais détendu et équilibré sont autant de facettes de l'expression de dominance qui correspondent aux caractérisations d'un style de communication habile (Burgoon et Dunbar, 2000).


Caractéristiques en fonction du domaine d'application
Comportements de personne dominante versus comportement de personne soumise à travers leur style de communication

Après avoir défini la relation interpersonnelle de dominance dans le chapitre \ref{chap:2}, ainsi que sa manifestation dans l'interaction tant sur l'aspect verbal que non verbal, nous avons ensuite détaillé son impact sur les stratégies de négociations. 

Ce chapitre introduit le modèle de décision d'un agent négociateur qui lui permet d'adapter sa stratégie de négociation à la relation de dominance qu'il vise à instaurer avec son interlocuteur. Dans la section 1, nous définissons les principes de décisions basés sur les comportements de pouvoir inspirés des travaux en psychologie sociale. Dans la section 2, nous présentons un premier modèle décisionnel utilisant des règles de décisions.  Pour ce modèle, nous nous sommes basés sur la structure d'arbres défini dans \emph{DISCO} \cite{ri} et nous discuterons ses limites. Ensuite dans la section 3, nous présenterons notre modèle décisionnel final qui prends en compte les comportements de pouvoir de l'agent associés à ses préférences pour construire sa stratégie de négociation. 

\section{Comportements de pouvoir et stratégies de négociation}

 Dans le cadre de cette thèse, nous nous sommes essentiellement basés sur les travaux de Bugroon \emph{et al} pour la définition de la dominance. 
 La dominance comme relation interpersonnelle est présentée comme la capacité à exprimer des comportements de pouvoir où l'influence est atteinte. Prenant cette définition comme point de départ, nous nous sommes ensuite intéressé à
 la manifestation des comportements de pouvoir durant le processus de négociation et comment ces comportements influencé les stratégies de négociations dans le contexte d'interaction humain/humain. 
 
 Dans ce qui suit, nous présentons trois principes de comportements extrait des travaux en psychologie sociale qui ont étudiaient l'impact du pouvoir sur les négociateurs
 
 \begin{enumerate}
 	\item 
 \end{enumerate}
 
 
 
 

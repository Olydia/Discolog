Après avoir défini la relation interpersonnelle de dominance dans le chapitre \ref{chap:2}, ainsi que sa manifestation dans l'interaction tant sur l'aspect verbal que non verbal, nous avons ensuite détaillé son impact sur les stratégies de négociations. 

Ce chapitre introduit le modèle de décision d'un agent négociateur qui lui permet d'adapter sa stratégie de négociation à la relation de dominance qu'il vise à instaurer avec son interlocuteur. Dans la section 1, nous définissons les principes de décisions basés sur les comportements de pouvoir inspirés des travaux en psychologie sociale. Dans la section 2, nous présentons un premier modèle décisionnel utilisant des règles de décisions.  Pour ce modèle, nous nous sommes basés sur la structure d'arbres défini dans \emph{DISCO} \cite{ri} et nous discuterons ses limites. Ensuite dans la section 3, nous présenterons notre modèle décisionnel final qui prends en compte les comportements de pouvoir de l'agent associés à ses préférences pour construire sa stratégie de négociation. Ensuite, nous présenterons deux études visant à valider le modèle décisionnel dans les deux cas d'interaction agent/agent et agent/humain.

\section{Comportements de pouvoir et stratégies de négociation}

 Comme nous l'avons présenté dans le chapitre \ref{chap:2}, nous nous sommes essentiellement basés sur les travaux en psychologie sociale pour la définition de la dominance. 
 La dominance comme relation interpersonnelle est présentée comme la capacité à exprimer des comportements de pouvoir où l'influence est atteinte. Prenant cette définition comme point de départ, nous nous sommes ensuite intéressé à la manifestation des comportements de pouvoir durant le processus de négociation et comment ces comportements influencé les stratégies de négociations dans le contexte d'interaction humain/humain. 
 
 Dans ce qui suit, nous présentons \emph{trois principes} de comportements extraits des travaux en psychologie sociale qui ont étudiaient l'impact du pouvoir sur les négociateurs et leur stratégies.
 
	\begin{enumerate}
	\item \textbf{Niveau d'exigence et de concessions:} Les négociateurs avec un pouvoir élevé affichent un niveau d'exigence plus important comparés aux négociateurs avec un pouvoir plus faible. Par ailleurs, les exigences des négociateurs de faible pouvoir diminuent avec le temps. Ceci se traduit par des concessions plus importantes comparés aux négociateurs avec un pouvoir plus important. \cite{de1995impact}
	
	\item \textbf{Soi \emph{vs} autrui:} Les négociateurs de faible pouvoir prennent en compte les préférences de leur interlocuteur dans la négociation, tandis que les négociateurs avec un pouvoir plus grand sont  centrés sur eux-mêmes et s'intéressent uniquement à la satisfaction leurs propres préférences. \cite{fiske1993controlling,de1995impact}
	
	\item \textbf{Contrôle du flux de la négociation:}
	Les négociateurs avec un pouvoir élevé ont tendance à faire le premier pas et à prendre les devants dans la négociation \cite {magee2007power}. Ils sont centrés sur l'avancement du processus de prise de décision, en prenant des décisions rapides \cite{zablotskaya2012relating}.
	  A l'opposé, les négociateurs de faible pouvoir visent à construire un modèle précis des préférences du partenaire de négociation. 
	  Par conséquent,  ils posent plus de questions afin de collecter les informations nécessaires qui leurs permet de prendre la décision la plus équitable(\emph{e.g}  faire des propositions)~\cite{de2004influence}. 
	
\end{enumerate}


Le but est de construire un modèle de décision capable d'illustrer ces comportements de pouvoir et par conséquent, adapter la stratégie de négociation en fonction du pouvoir que l'agent veut exprimer.

Dans ce qui suit nous présenterons, le modèle décision de l'agent qui prend en compte la relation de pouvoir.

 
\section{Règles de décision}
	Nous avons construit des règles de décision modélisé sous forme d'arbres de dialogues. L'implémentation de notre système de dialogue étant géré par le logiciel \emph{DISCO}. Disco est une implémentation d'un ``collaborative discourse manager'' inspiré d'une théorie de dialogue collaboratif comme Collagen \cite{rich1997collagen}. Disco est un système qui permet la génération de dialogues orienté tâches pour lequel il utilise le formalisme des HTNs (Hierarchical Task Networks) \cite{erol1994htn} pour la gestion des tâches. Il est implémenté avec le standard ANSI/CEA-2018 : chaque tâche est définit avec des préconditions, des effets et des postconditions. Les tâches sont regroupées par \emph{recettes} munies de conditions d'applicabilité.
	
	De plus, Disco a été étendu avec un module génération d'arbres de dialogues afin de communiquer et collaborer avec l'utilisateur pour la réalisation des tâches. Ce module est nommé Disco for Games (D4g) et permet de définir des sémantiques d'actes de dialogue. D4g est déjà fourni avec un ensemble d'actes de dialogue.
	
	Nous avons complété ce système avec les actes de dialogues présenté dans la section précédente afin qu'il puisse supporter la négociation sur les préférences.
	
	Pour chaque acte de dialogue que l'agent reçois, nous modélisons l'ensemble des réponses que l'agent peut sélectionner. Par exemple, suite à un acte Propose énoncé par l'utilisateur, l'agent peut répondre par un Accept, un Reject ou un autre Propose ("User: Allons au Chinois. Agent: Et si nous allions plutôt au Japonais?"). 
	Chaque branche est définie avec des conditions d'applicabilités pour décider quelle réponse est adoptée. 
	Ces conditions prennent en compte le pouvoir de l'agent en plus du contexte courent de la négociation. Dans l'exemple précédent, l'agent doit être dominant pour répondre à un Propose par un autre Propose. 
	
	

	
	\subsection{Sélection de l'acte de dialogue}*
		Division du spectre de pouvoir sur deux. Agent dominant et agent soumis.
		pour chaque type d'acte de dialogue que l'agent reçoit, nous générons un ensemble de réponses possible. Chaque réponse dépend du pouvoir de l'agent.
		
		\subsubsection{State Preference}
			Quand l'utilisateur énonce un statePreference, l'agent a le choix entre 4 réponses:
			
		\subsection{Propose}
			A la réception d'un propose comme présenté dans la figure ..., l'agent choisis sa réponse en fonction de l'\emph{acceptabilité} de la proposition.
			Nous avons écrit l'algorithme d'acceptabilité afin qu'il s'adapte à la valeur de dominance de l'agent. Ce choix permet de refléter les comportements du principe 2 \emph{niveau d'exigences et concessions}.
			
			En effet, au fur et a mesure que la négociation évolue, l'agent décide de faire des concessions sur certains critères. Par exemple, il peut considérer que le critère de \emph{location} n'est plus important pour le choix d'un restaurant. Par conséquent, il considérera que toute valeur de location est désormais \emph{acceptable}.
			
			Par ailleurs, la notion d'exigence apparaît dans l'algorithme ci-dessous. Si l'agent est dominant, l'ensemble de valeurs acceptables est plus restreint qu'un agent soumis.
			
					\begin{figure}[]
				\begin{algorithmic}[1]\small
					\Function{isAcceptable}{$value$, $relation$}
					\If{(type de $value$ n'est pas un critère important)} 
					\State return $true$
					\EndIf
					
					\State List = trier les valeurs par ordre décroissant
					\If{($relation = dominant$)} 
					\State $return$ index($value$)< $size(List)/2$
					\EndIf
					\If{($relation = sub$)} 
					\State return $return$ index($value$)< $size(List)/4$
					\EndIf
					\EndFunction
				\end{algorithmic}
				\vskip 8pt
				\caption{pseudo}{Calcul d'acceptabilité d'une proposition $value$}
			\end{figure} 
			
			Par conséquent, l'arbre de dialogue est le suivant:
		\subsection{Accept }
				
		%----------------------------------------------------------------
			Le système de dialogue offre à l'utilisateur la liberté de choisir n'importe quel acte de dialogue pour son tour de parole. Disco déroule alors l'arbre de dialogue correspondant de gauche à droite (en commençant par la branche la plus à gauche). La première branche applicable rencontrée est directement exécutée sans vérifier les branches restantes.
		Par exemple, les réponses que l'agent peut générer quand il reçoit un \emph{StatePreference} de la part de l'utilisateur sont présentées dans la .
	
	
	
	
Après avoir défini la relation interpersonnelle de dominance dans le chapitre \ref{chap:2}, ainsi que sa manifestation dans l'interaction tant sur l'aspect verbal que non verbal, nous avons ensuite détaillé son impact sur les stratégies de négociations. 

Ce chapitre introduit le modèle de décision d'un agent négociateur qui lui permet d'adapter sa stratégie de négociation à la relation de dominance qu'il vise à instaurer avec son interlocuteur. Dans la section 1, nous définissons les principes de décisions basés sur les comportements de pouvoir inspirés des travaux en psychologie sociale. Dans la section 2, nous présentons un premier modèle décisionnel utilisant des règles de décisions.  Pour ce modèle, nous nous sommes basés sur la structure d'arbres défini dans \emph{DISCO} \cite{ri} et nous discuterons ses limites. Ensuite dans la section 3, nous présenterons notre modèle décisionnel final qui prends en compte les comportements de pouvoir de l'agent associés à ses préférences pour construire sa stratégie de négociation. Ensuite, nous présenterons deux études visant à valider le modèle décisionnel dans les deux cas d'interaction agent/agent et agent/humain.

\section{Comportements de pouvoir et stratégies de négociation}

 Comme nous l'avons présenté dans le chapitre \ref{chap:2}, nous nous sommes essentiellement basés sur les travaux en psychologie sociale pour la définition de la dominance. 
 La dominance comme relation interpersonnelle est présentée comme la capacité à exprimer des comportements de pouvoir où l'influence est atteinte. Prenant cette définition comme point de départ, nous nous sommes ensuite intéressé à la manifestation des comportements de pouvoir durant le processus de négociation et comment ces comportements influencé les stratégies de négociations dans le contexte d'interaction humain/humain. 
 
 Dans ce qui suit, nous présentons \emph{trois principes} de comportements extraits des travaux en psychologie sociale qui ont étudiaient l'impact du pouvoir sur les négociateurs et leur stratégies.
 
	\begin{enumerate}
	\item \textbf{Niveau d'exigence et de concessions:} Les négociateurs avec un pouvoir élevé affichent un niveau d'exigence plus important comparés aux négociateurs avec un pouvoir plus faible. Par ailleurs, les exigences des négociateurs de faible pouvoir diminuent avec le temps. Ceci se traduit par des concessions plus importantes comparés aux négociateurs avec un pouvoir plus important. \cite{de1995impact}
	
	\item \textbf{Soi \emph{vs} autrui:} Les négociateurs de faible pouvoir prennent en compte les préférences de leur interlocuteur dans la négociation, tandis que les négociateurs avec un pouvoir plus grand sont  centrés sur eux-mêmes et s'intéressent uniquement à la satisfaction leurs propres préférences. \cite{fiske1993controlling,de1995impact}
	
	\item \textbf{Contrôle du flux de la négociation:}
	Les négociateurs avec un pouvoir élevé ont tendance à faire le premier pas et à prendre les devants dans la négociation \cite {magee2007power}. Ils sont centrés sur l'avancement du processus de prise de décision, en prenant des décisions rapides \cite{zablotskaya2012relating}.
	  A l'opposé, les négociateurs de faible pouvoir visent à construire un modèle précis des préférences du partenaire de négociation. 
	  Par conséquent,  ils posent plus de questions afin de collecter les informations nécessaires qui leurs permet de prendre la décision la plus équitable(\emph{e.g}  faire des propositions)~\cite{de2004influence}. 
	
\end{enumerate}


Le but est de construire un modèle de décision capable d'illustrer ces comportements de pouvoir et par conséquent, adapter la stratégie de négociation en fonction du pouvoir que l'agent veut exprimer.

Dans ce qui suit nous présenterons, le modèle décision de l'agent qui prend en compte la relation de pouvoir.

 
\section{Règles de décision}
	Nous avons construit des règles de décision modélisées sous forme d'arbres de dialogues. L'implémentation de notre système de dialogue étant géré par le logiciel \emph{DISCO}. Disco est une implémentation d'un ``collaborative discourse manager'' inspiré d'une théorie de dialogue collaboratif comme Collagen \cite{rich1997collagen}. Disco est un système qui permet la génération de dialogues orienté tâches pour lequel il utilise le formalisme des HTNs (Hierarchical Task Networks) \cite{erol1994htn} pour la gestion des tâches. Il est implémenté avec le standard ANSI/CEA-2018 : chaque tâche est définit avec des préconditions, des effets et des postconditions. Les tâches sont regroupées par \emph{recettes} munies de conditions d'applicabilité.
	
	De plus, Disco a été étendu avec un module génération d'arbres de dialogues afin de communiquer et collaborer avec l'utilisateur pour la réalisation des tâches. Ce module est nommé Disco for Games (D4g) et permet de définir des sémantiques d'actes de dialogue. D4g est déjà fourni avec un ensemble d'actes de dialogue.
	
	Nous avons complété ce système avec les actes de dialogues présenté dans la section \label{sec:communication} afin qu'il puisse supporter la négociation sur les préférences.
	
	Pour chaque acte de dialogue que l'agent reçois, nous modélisons l'ensemble des réponses que l'agent peut sélectionner. Par exemple, suite à un acte \emph{Propose} énoncé par l'utilisateur, l'agent peut répondre par un \emph{Accept}, un \emph{Reject} ou un autre \emph{Propose} ("User: Allons au Chinois. Agent: Et si nous allions plutôt au Japonais?"). 
	Chaque branche est définie avec des conditions d'applicabilités pour décider quelle réponse est adoptée. 
	Ces conditions prennent en compte le pouvoir de l'agent en plus du contexte courent de la négociation. Dans l'exemple précédent, l'agent doit être dominant pour répondre à un Propose par un autre Propose. 
	
	

	
	\subsection{Sélection de l'acte de dialogue}
		Nous avons initialisé l'agent avec un comportement de pouvoir parmi trois types de comportements de pouvoir inspirés de la littérature en psychologies social.  L'agent peut suivre un comportement \emph{dominant, soumis} ou \emph{neutre}. 
		
		En fonction de l'acte de dialogue que l'agent reçoit, nous générons un ensemble de réponses possibles. Chaque réponse dépend du pouvoir de l'agent. Le système de dialogue offre à l'utilisateur la liberté de choisir n'importe quel acte de dialogue pour son tour de parole. Disco déroule alors l'arbre de dialogue correspondant de gauche à droite (en commençant par la branche la plus à gauche). La première branche applicable rencontrée est directement exécutée sans vérifier les branches restantes.
		
		Notons que dans la suite, chaque arbre de dialogue est défini avec une condition de sortie qui clos la négociation avec un échec. Cette dernière est activée seulement par agent \emph{dominant} dans la situation où toutes les valeurs restantes ne sont pas acceptables. 
		
		\subsubsection{AskPreference}
			A la réception d'un \emph{AskPreference}, l'agent répond en exprimant ses préférences sur la question demandée comme présenté dans la figure .. .
	
		\subsubsection{State Preference}
			Le comportement standard qu'un agent adopte à la réception d'un \emph{StatePreference(v)} est de donner son opinion sur la valeur exprimée (c-à-d que l'agent calcule la valeur de satisfiabilité de \textit{v}). Cependant, si l'agent a déjà exprimé ses préférences sur ces valeurs, il va vouloir choisir un autre acte de dialogue en fonction de la relation sociale:
			\begin{itemize}
				\item \emph{Propose(x)}: Si la valeur exprimée par l'utilisateur est acceptable (voir figure \ref{pseudo}) pour l'agent, il va proposer de choisir cette valeur. Ce comportement relate le principe 1.
				\item \emph{Propose}: Si le nombre de \emph{StatePreferences} autorisé est atteint (voir figure \ref{alg:maxtours}), l'agent doit respecter le principe 3 et faire évoluer la négociation. La valeur proposée doit respecter les préférences de l'agent. 
				\item \emph{StatePreference}:l'agent est \emph{neutre}, il va vouloir exprimer ses préférences afin de construire une meilleure connaissance.
				\item \emph{AskPreference}:l'agent est \emph{soumis}. Dans le cas où toutes les valeurs du critères courent ont été discutés, l'agent va respecter le principe 3 et faire évoluer la négociation en ouvrant la discussion sur un autre critère.
			\end{itemize}
			
			\begin{figure}[]
				\begin{algorithmic}[1]\small
					\Function{MaxStatements}{}
					\State $nbTours$ = Nombre de \emph{StatePreferences} exprimés successivement.
					\State $maxTours$ 
					\If{($dominant$)} 
					\State $maxTours = 1$
					\EndIf
					\If{($peer$)} \State $maxTours = 2$
					\EndIf
					\If{($soumis$)} 
					\State $maxTours = 4$
					\EndIf
					\State $retrun$ $nbTours\geq maxTours$
					\EndFunction
				\end{algorithmic}
				\vskip 8pt
				\label{alg:maxtours}
				\caption{Maximum de tours de \emph{StatePreference} autorisé en fonction du pouvoir de l'agent}
			\end{figure} 
		\subsubsection{Propose}
			A la réception d'un \emph{Propose} comme présenté dans la figure ..., l'agent choisit sa réponse en fonction de l'\emph{acceptabilité} de la proposition.
			Nous avons écrit l'algorithme d'acceptabilité afin qu'il s'adapte à la valeur de pouvoir de l'agent. Ce choix permet de refléter les comportements du principe 2 \emph{niveau d'exigences et concessions}.
			
			En effet, au fur et à mesure que la négociation évolue, l'agent va faire des concessions sur certains critères. Par exemple, il peut considérer que le critère de \emph{localisation} n'est plus important pour le choix d'un restaurant. Par conséquent, il considérera que toute valeur de \emph{localisation} est désormais \emph{acceptable}.
			
			Par ailleurs, la notion d'exigence apparaît dans l'algorithme ci-dessous. Si l'agent est \emph{dominant}, l'ensemble de valeurs acceptables est plus restreint qu'un agent \emph{soumis}.
			
			\begin{figure}[]
				\begin{algorithmic}[1]\small
					\Function{isAcceptable}{$proposal$}
					\If{(type de $proposal$ n'est pas un critère important)} 
					\State return $true$
					\EndIf
					
					\State List = trier les valeurs par ordre décroissant de préférences
					\If{($dominant$)} 
					\State $return$ index($proposal$)< $size(List)/2$
					\EndIf
					\If{($soumis$)} 
					\State return $return$ index($proposal$)< $size(List)/4$
					\EndIf
					\EndFunction
				\end{algorithmic}
				\vskip 8pt
				\label{pseudo}
				\caption{Calcul d'acceptabilité d'une proposition $value$}
			\end{figure} 
			
			
			L'arbre de décision produit pour répondre à un \emph{Propose(p)} est présenté dans la figure .... 
			\begin{itemize}
			
				\item  \emph{Accept(p)}: Si la proposition $p$ est acceptable, l'agent doit exprimer un \emph{Accept(p)}. Sinon, l'agent doit faire évoluer la négociation pour trouver un meilleurs compromis. En fonction de la relation de pouvoir l'agent choisit un acte de dialogue spécifique.
				\item \emph{Ask :} l'agent \emph{soumis} va suivre les comportements décrit dans les principes 1 et 3. En effet, il va essayer de collecter plus de connaissances sur les préférences de l'autre et ainsi prendre en compte ses préférences. De plus, en vue du principe 2 qui affirme qu'un agent soumis doit faire des concessions, l'agent soumis n'est pas autorisé à exprimer plus d'un nombre de rejets successifs.  
				\item \emph{Reject :} l'agent qui n'est pas dominant (\emph{i.e. soumis ou neutre}) rejete une proposition qui n'est pas acceptable.
				\item \emph{Propose :} suivant le principe 3, l'agent \emph{dominant} fait évoluer la négociation en proposant une autre valeur qui respecte mieux ses préférences. 
			\end{itemize}
	
			 
	
		\subsubsection{Accept }
			A la réception	d'un \emph{Accept}, nous séparons deux cas de réponses en fonction du type de la valeur acceptée $v$.
			Premièrement, $ v \in \mathcal{O}$ une \textit{option}, l'agent clos la négociation par un \emph{succès}.
			Sinon, $v \in C_i$ est une valeur de critère, l'agent entame la négociation sur un autre critère et sa réponse sera choisie en fonction de la relation de pouvoir à exprimer.
			
			\begin{itemize}
				\item \emph{Propose}: La condition d'applicabilité pour cet acte de dialogue dépend du pouvoir de l'agent. Un agent \emph{dominant} entamera la négociation sur le nouveau critère en proposant une valeur qui respecte ses préférences. Cependant, un agent \emph{soumis} n'est autorisé à faire une proposition que s'il a des connaissances sur les préférences de l'autres,  communiqués par des \emph{StatePreferences}, qui lui permettront de prendre une décision équitable. cet algorithme traduit les comportements présentés dans les principes 1 et 3. 
				
				\item \emph{AskPreference}: Comme présenté dans le principe 3, l'agent est \emph{soumis} collecte le plus d'informations possibles sur les préférences de son interlocuteur. 
				\item \emph{StatePreference}: L'agent est \emph{neutre} ouvre la négociation en communiquant ses préférences sur le nouveau critère à discuter. 
				
			\end{itemize}
			
%				\begin{figure}[]
%					\begin{algorithmic}[1]\small
%						\Function{CanPropose}{}
%						\If{($dominant$)} 
%						\State return $true$
%						\EndIf
%						
%						\State 
%						\If{($soumis$)} 
%						\State l'agent a des connaissances suffisantes sur les préférences de l'interlocuteur
%						\State return $true$
%						\EndIf
%						\EndFunction
%					\end{algorithmic}
%					\vskip 8pt
%					\label{alg:canPropose}
%					\caption{Calcul d'acceptabilité d'une proposition $value$}
%				\end{figure}
%				
				
		\subsubsection{Reject}
			Suite à un \emph{Reject}, l'agent choisis une réponse en suivant les trois principes comme présenté ci-dessous:
			\begin{itemize}
				\item \emph{AskPreference}: Si la proposition d'un agent \emph{soumis} est rejetée, il va considérer qu'il n'avait pas assez de connaissances pour prendre une bonne décision. Pour compenser, il va demander à l'utilisateur ses préférences sur des valeurs qu'il ne connaît pas déjà. 
				
				\item \emph{Propose}: Suivant les principes 1 et 3, l'agent \emph{dominant} fait avancer la négociation en proposant de nouvelles valeurs. Cependant, si la valeur rejetée se trouve être la valeur qu'il préfère le plus, il va refuser de concéder et donc proposer la valeur encore une fois. Ce comportement est fidèle au principe 2. 
				
				\item \emph{StatePreference}: l'agent \emph{peer} continue la négociation en exprimant ses préférences sur d'autres valeurs. 
			\end{itemize}	
		%----------------------------------------------------------------
			
	\subsubsection{Exemple}
		Nous présentons un dialogue généré en utilisant notre modèle de dialogue. Deux agents\textit{A} et \textit{B}
		A: "Let's go to a Chinese restaurant."
		
		B: "I like Italian more than Chinese restaurants."
		
		A: "Let's go to a quiet restaurant."
		
		B: "I don't like quiet restaurants."
		
		A: "Let's go to a cheap restaurant."
		
		B: "Do you like quiet or lively restaurants? "
		
		A: "I like quiet restaurants."
		
		B: "I like lively restaurants."
		
		A: "Let's go to the Samura restaurant. It's a lively, cheap Japanese restaurant."
		
		B: "What kind of cuisine do you like?"
		
		A: "I like Chinese restaurants."
		
		B: "I like Italian restaurants."
		
		A: "Let's go to the Dragon restaurant. It's a lively, cheap Chinese restaurant."
		
		B: "Okay. Let's go to the Dragon restaurant."
		
		A: "Okay. I'll call to book a table."
	
	\subsection{Limites des arbres de dialogue}
		Parler de l'étude qui a montré une ambiguité dans la perception des comportement et la capacité à isoler l'impact de chaque principe sur la prise de décision
		
		Comportement incongrue ou non attendue.
		
		Situation de \textit{breakdown}, du a une modèlisation manuelle, nous avions rencontrer des situations où aucune condition d'applicabilité n'étaient vrai. 
		
		Pour toutes ces raisons, nous avons repenser notre solution pour qu'elle soit plus fidèle aux principe de négociation et qu'elle puisse refléter les comportements de pouvoir. 
	
	\section{Modèle de décision basé sur les comportements de pouvoirs}
	
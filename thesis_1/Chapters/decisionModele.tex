Après avoir défini la relation interpersonnelle de dominance dans le chapitre \ref{chap:2}, ainsi que sa manifestation dans l'interaction tant sur l'aspect verbal que non verbal, nous avons ensuite détaillé son impact sur les stratégies de négociations. 

Ce chapitre introduit le modèle de décision d'un agent négociateur qui lui permet d'adapter sa stratégie de négociation à la relation de dominance qu'il vise à instaurer avec son interlocuteur. Dans la section 1, nous définissons les principes de décisions basés sur les comportements de pouvoir inspirés des travaux en psychologie sociale. Dans la section 2, nous présentons un premier modèle décisionnel utilisant des règles de décisions.  Pour ce modèle, nous nous sommes basés sur la structure d'arbres défini dans \emph{DISCO} \cite{ri} et nous discuterons ses limites. Ensuite dans la section 3, nous présenterons notre modèle décisionnel final qui prends en compte les comportements de pouvoir de l'agent associés à ses préférences pour construire sa stratégie de négociation. Ensuite, nous présenterons deux études visant à valider le modèle décisionnel dans les deux cas d'interaction agent/agent et agent/humain.

\section{Comportements de pouvoir et stratégies de négociation}

 Comme nous l'avons présenté dans le chapitre \ref{chap:2}, nous nous sommes essentiellement basés sur les travaux en psychologie sociale pour la définition de la dominance. 
 La dominance comme relation interpersonnelle est présentée comme la capacité à exprimer des comportements de pouvoir où l'influence est atteinte. Prenant cette définition comme point de départ, nous nous sommes ensuite intéressé à la manifestation des comportements de pouvoir durant le processus de négociation et comment ces comportements influencé les stratégies de négociations dans le contexte d'interaction humain/humain. 
 
 Dans ce qui suit, nous présentons \emph{trois principes} de comportements extraits des travaux en psychologie sociale qui ont étudiaient l'impact du pouvoir sur les négociateurs et leur stratégies.
 
	\begin{enumerate}
	\item \textbf{Niveau d'exigence et de concessions:} Les négociateurs avec un pouvoir élevé affichent un niveau d'exigence plus important comparés aux négociateurs avec un pouvoir plus faible. Par ailleurs, les exigences des négociateurs de faible pouvoir diminuent avec le temps. Ceci se traduit par des concessions plus importantes comparés aux négociateurs avec un pouvoir plus important. \cite{de1995impact}
	
	\item \textbf{Soi \emph{vs} autrui:} Les négociateurs de faible pouvoir prennent en compte les préférences de leur interlocuteur dans la négociation, tandis que les négociateurs avec un pouvoir plus grand sont  centrés sur eux-mêmes et s'intéressent uniquement à la satisfaction leurs propres préférences. \cite{fiske1993controlling,de1995impact}
	
	\item \textbf{Contrôle du flux de la négociation:}
	Les négociateurs avec un pouvoir élevé ont tendance à faire le premier pas et à prendre les devants dans la négociation \cite {magee2007power}. Ils sont centrés sur l'avancement du processus de prise de décision, en prenant des décisions rapides \cite{zablotskaya2012relating}.
	  A l'opposé, les négociateurs de faible pouvoir visent à construire un modèle précis des préférences du partenaire de négociation. 
	  Par conséquent,  ils posent plus de questions afin de collecter les informations nécessaires qui leurs permet de prendre la décision la plus équitable(\emph{e.g}  faire des propositions)~\cite{de2004influence}. 
	
\end{enumerate}


Le but est de construire un modèle de décision capable d'illustrer ces comportements de pouvoir et par conséquent, adapter la stratégie de négociation en fonction du pouvoir que l'agent veut exprimer.

Dans ce qui suit nous présenterons, le modèle décision de l'agent qui prend en compte la relation de pouvoir.

 

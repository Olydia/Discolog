
\documentclass [french]{article}
\usepackage[francais]{babel}
\usepackage[utf8]{inputenc}
\usepackage{color}




\begin{document}
	
	\section{Introduction}
		\begin{enumerate}
			\item Intérêt des agents relationnels 
			\subitem Prise en compte des relations sociales
			\item Présentation de la problématique : Étudier l'impact d'une relation interpersonnelle sur les stratégies de négociation.
				\subitem 
			\item Organisation du document
		\end{enumerate}
		
		
	\section{État de l'art}
	Dans cette partie, nous dressons un état de l'art des travaux qui s'intéressent à la dominance dans les interactions tant qu'en psychologie sociale qu'en informatique affective 
		\subsection{Dominance en psychologie sociale}
			La dominance a été largement étudié dans le domaine de la psychologie sociale. Nous retrouvons des travaux dans différentes veines du domaines qui étudient l'influence de la dominance sur les relations humaines, tant comme trait de personnalité, rôle social ou encore relation interpersonnelle que construisent les individus. Ces comportements se manifestent dans l'interaction avec des comportements verbaux, non verbaux ou encore sur un niveau stratégique.		
			Nous présentons dans cette section différentes études sur la dominance.
			
			\subsubsection{Relation interpersonnelle}
			\label{ri}
			\subsubsection{Trait de personnalité}
			\label{perso}
			\subsubsection{Statue social}
			\subsubsection{Attitude sociale} Travaux de Florian Pecune?
			\subsubsection{Conclusion} 
				Conclure sur les différents aspects de la dominance dans les interactions sociales.
				Définir le contexte qui nous intéresse à savoir la relation interpersonnelle de dominance dans le contexte de négociation collaborative présentée dans la prochaine section. 
		\subsection{Relation interpersonnelle de dominance et négociation}
			Il a été largement prouvé que la relation de dominance est la 1ère relation à s'établir dans le contexte de négociation. De  plus, elle  affecte directement les stratégies de négociation exprimés par les négociateurs. Nous présentons les comportements dans la négociation influencés par la dominance.
			
			\subsubsection{Comportements de pouvoir}
			
			\subsubsection{Complémentarité de dominance} 
				
			\subsubsection{Conclusion} Positionner mes travaux:
			
			 1) Étudier la dominance dans l'interaction comme relation interpersonnelle. 
			
			2) Nous nous intéressons plus précisément à l'impact de la dominance dans les dialogues la négociation collaborative.
				 
				 \textbf{A quel moment définir la négociation collaborative... ?} 
				
				
		\subsection{Agents conversationnels sociaux}
			Des travaux en informatique affective se sont intéressées aux différents comportements de dominance dans la modélisation d'agent conversationnels.
			(Présenter les différents travaux en IA qui ont modélisé des comportements liés à la dominance).
			
			\subsubsection{Modélisation de l'autre: Théorie de l'esprit} Modéliser l'état mental de l'autre afin de comprendre son raisonnement. Cela permet d'une part d'adapter le comportement de l'agent afin de simuler une relation interpersonnelle. D'une autre part, cela permet d'anticiper les futurs comportements de l'autre. 
		
		\subsubsection{Conclusion}
		
			Notre but est de modéliser les comportements de pouvoir qui traduisent une tendance de dominance. De plus, nous souhaitons simuler une relation interpersonnelle de dominance avec l'utilisateur. A cette fin, nous proposons un modèle de la théorie de l'esprit basé sur la simulation  pour deux raisons. En premier lieu, comprendre les comportements de pouvoir exprimés par l'utilisateur. Deuxièmement, adapter le comportement de l'agent aux comportements de pouvoir exprimés par l'utilisateur afin de créer une relation de dominance dite complémentaire. 	
			
	
	\section{Contributions}
	 % !TeX spellcheck = fr_FR

Comme mentionné dans l'introduction, la principale contribution de cette thèse est d'étudier l'impact de la relation de dominance sur les stratégies de négociation dans le cadre de négociation collaborative entre un agent conversationnel et un utilisateur humain. 
Pour ce faire, nous devons construire un modèle de négociation qui permettent aux négociateurs de présenter leurs stratégies de négociation. 

Dans ce chapitre, nous présentons notre modèle de négociation collaborative sur lequel sera construit notre modèle de décision basé sur la dominance. 

Afin de définir un système de dialogue dans lequel la relation de dominance régit le choix du prochain énoncé, nous avons d'abord enregistré des dialogues de négociation entre deux personnes afin d'observer leurs comportements dans un cadre de dialogue social de type "négociation collaborative". Nous avons annotés et analysé les dialogues. Cette étude nous a livré un ensemble de comportements  que nous présenterons dans la première section de ce chapitre. 

Les informations collectées grâce l'observation des comportements humains nous a guidés dans la conception de notre modèle de négociation.



Dans la seconde section, nous présenterons le domaine de négociation utilisé. Dans le cadre de cette thèse, nous nous basons sur les modèles de négociations multi-critères largement utilisé dans la mise en œuvre de systèmes de négociations automatiques. 

La troisième section présentera notre modèle de communication basé sur des actes de dialogues. En effet, nous nous sommes réapproprié les actes de dialogue de \cite{grosz1986attention} pour la négociation collaborative.


 \section{Collecte de données}
	 Afin de définir un système de dialogue social dans lequel la relation de dominance régit le choix du prochain énoncé, nous avons d'abord effectuer une étude dans laquelle nous avons analyser les comportements de deux interlocuteurs dans un cadre de dialogue social de type "négociation collaborative". 
	Le but du dialogue est de trouver un restaurant où dîner avec son interlocuteur. De plus, les interlocuteurs n'avaient pas de connaissances prérequis sur les préférences de leur interlocuteur.
	
	Une fois le dialogue enregistré, nous les avons annoté et analyser la structure du dialogue en suivant la théorie de \emph{Grosz et Sidner} \cite{sidner} qui stipule que la structure du dialogue est composée de trois éléments à savoir la structure linguistique, la structure intentionnelle et enfin l'état attentionnel.	
	Nous présenterons dans ce qui suit la procédure de l'analyse ainsi que les résultats obtenus. 
	
	\subsection{Analyse de la structure de dialogue}  
		La théorie présenté par \emph{Grosz et Sidner} propose que la structure d'un dialogue orienté tâche est constituée par trois éléments chacun agissant sur un aspect du dialogue. 
		
		D'abord, \emph{la structure linguistique} a pour but décomposé le dialogue en une séquence de segments de dialogue appelé (\textit{DS: discourse segment }) de tel sorte que chaque segment est composé d'un ensemble de sous-segments et d'une séquence d'énoncés appartenant uniquement au segment(n'appartient pas a un sous segment). De plus, les énoncés au sein d'un même segment contribuent à un même but. Cette décomposition proposée est non strict du fait qu'il est difficile de trouver des indices de segmentation. Des exemples d'indices proposés sont l'intention communicative commune à chaque DS, des expressions  linguistiques comme l'utilisation de termes "d'abord, finalement, $\ldots$". Des indices plus subtiles tels que le changement d'intonation, le temps de pause peuvent aussi être utilisés.
		
	\subsection{Structure intentionnelle}
		Un interlocuteur s'engage dans un dialogue poussé par une ou plusieurs intentions internes qu'on nomme \emph{DP: discourse purpose}. De plus, pour chaque DS, nous pouvons isolé un but spécifique noté \emph{DSP: Discourse segment purpose}. Le but est d'analyser comment les DSPs participent à la satisfaction du DS initial. En outre, cette structure comprend l'analyse des relations entre les différents DSPs. Deux relations ont été identifiées, \emph{dominance} et \emph{satisfaction-precedence}. Si la satisfaction de l'intention d'un DSP1 participe a la satisfaction de celle d'un DSP2, alors le DSP1 \textbf{contribue} au DSP2. Par opposition le DSP2 \textbf{domine} le DSP 1.


	\subsubsection{Structure attentionnelle}
	
		Cette structure représente “l'abstraction de l'attention des participants au fur et à mesure que leur dialogue avance”.
		% Dynamic stack that records salient objects, properties and relations Focusing – process of manipulating focus spaces on attentional (focus) stack
	 
	 \subsubsection{Exemple de l'analyse}
		 Nous présentons dans cette section un exemple de l'analyse dite en DSP que nous avons effectué sur les dialogues que nous avons enregistré.
		%	ajouter un exemple avec chaque dsp: l'intention trouver et l'indentation.
	
	
	
	\subsubsection{Résultats de l'analyse}
		L'analyse en DSPs nous a révélé un nombre de comportements intéressants tant sur l'aspect structurelle de la négociation que sur les stratégies de négociations déployés par les interlocuteurs. 	
		
		En effet, la décomposition du dialogue en \emph{DS} nous a confirmé que les négociateurs s'intéressaient a différents critères pour le choix d'une option (restaurant dans notre exemple). Ces critères sont négocié simultanément durant la négociation jusqu'a trouver un compromis qui satisfasse les négociateurs sur les critères qu'il jugent importants. 
		Par exemple, dans le premier dialogue, les interlocuteurs se sont plus intéressés a l'ambiance du restaurant et son emplacement pour le choix finale. Au contraire dans le second dialogue, les interlocuteurs se sont principalement intéressés aux type de la cuisine et la qualité.
		    
		 De plus, Les critères les plus importants sont les premiers à être abordés, et en cas de conflit d'autre critères sont abordés. 
		 Ceci est confirmé par des travaux en négociations automatiques qui mettent en avant l'intérêt de la modalisation multicritères dans les systèmes de négociation. Ce point sera abordé plus en détails en section ... 
		 
		 Sur l'aspect communicatif, notre modèle se basant sur des actes de dialogue, nous nous sommes intéressés au style d'informations échangés lors de la négociation. Nous avons récoltés des informations sur le style linguistique à affecter a nos actes de dialogues qui seront présentés en détails dans la section ... 
		 
		 Finalement, le dernier point et le plus important à être discuté est les stratégies que les négociateurs utilisent. En effet, l'analyse en DSP nous a permis d'analyser la corrélation entre différents comportements reliés à la dimension de la dominance. En effet, nous avons pu observé la relation entre la dominance et la fréquence de prise de parole. Le style linguistique est aussi influencé, l'argumentation, l'expression des préférences etc. 
		 (ajouter exemple)
		 
		  
	%	Les informations collectées grâce l'observation des comportements humains nous a guidés dans la conception de notre modèle dialogique qui comprend un modèle mental de l'environnement de l'agent et un modèle lui permettant de mener une négociation coopérative. 
		
	

\section{Domaine de négociation}


%	L'interet d'une négociation multi-critères dans la modélisation d'un sujet social
% voir intro :https://www.ri.cmu.edu/pub_files/pub4/lai_guoming_2008_1/lai_guoming_2008_1.pdf
%https://link.springer.com/content/pdf/10.1007/s10458-006-9009-y.pdf

	La recherche en négociation automatique peut être divisée en deux catégories en ce qui concerne la représentation du domaine: négociation sur un critère et la négociation multi-critères. Cependant, La littérature existante se concentre plus sur la négociation uni critère \cite{lai2008decentralized,lai2004literature}. 
	
	Dans le cadre d'une interaction avec un négociateur humain, la négociation multi-critère est cruciale. En effet, dans un environnement humain, les négociateurs peuvent discuter de plusieurs critères simultanément, ce fait est confirmé par l'analyse que nous avions effectué dans la section précédente.  Nous avons observé que les négociateurs s'intéressaient à plusieurs critères pour le choix d'un restaurants. Par exemple le type de cuisine, la location ou encore l'ambiance de ce dernier. Ces critères ont soit été abordé simultanément dans la négociation, ou bien un par un. C'est à dire que les négociateurs s'accordaient sur un premier critère avant d'aborder un autre, ou bien discuter des différents critères jusqu'a aboutir à un compromis.
	
	De plus, plusieurs travaux en négociation automatique ont mis en exergue que la négociation multi-critères augmente la coordination et collaboration durant le processus de négociation afin de rechercher un résultat qui apporte des gains communs pour les deux parties \cite{jonker2007agent,lai2008decentralized,lai2004literature} . 

	% 	Pour toutes ces raisons, notre choix s'est porter sur la négociation multi-critère

	Les résultats des précédents travaux nous ont motivé à utiliser une représentation multi-critère pour modéliser notre domaine de négociation collaborative. 
	
	\subsection{Représentation formelle des éléments de la négociation }	
	Le but de la négociation est de choisir une \textit{option} $O$ dans l'ensemble des options $\mathcal{O}$ comprenant toutes les options alternatives envisagé pour un sujet de négociation donnée. 
	
	L'évaluation de chaque option repose sur un ensemble de critères $\mathcal{C}$ reflétant les caractéristiques de l'option. Nous définissons l'ensemble $\mathcal{C}$ de $n$ critères, et $C_1,\ldots,C_n$ le domaine de valeurs de chaque critère de l'ensemble. 
	Par conséquent, $\mathcal{O}$ peut être défini comme le produit vectoriel de  $C_1\times\ldots\times C_n$ et chaque option $O \in \mathcal{O}$ est un tuple $(v_1,\ldots,v_n)$. 
	
		Par exemple, une négociation collaborative qui porte sur le choix d'un restaurant où dîner à pour but de choisir un restaurant parmi
		
		Continuer l'exemple complet du restaurant. 
	
	

les préférences sont censées régir le processus
décisionnel
, Satisfiabilité
Contexte de négociation: préférences de l'autre, historique de la négociation (proposition, préférences exprimés)
\textcolor{blue}{explication des préférences}
%https://pdfs.semanticscholar.org/f4ea/f27ea6f3f94b8e8d4d8d103eb7a1ebaf2324.pdf
%Strict preference is asymmetric: There is no pair of x and x’ in Ω such that x≺i
%x’ and
%x’≺i
%x;
%• Transitivity: For all x, x’ and x’’ in Ω , if xÉi
%x’ and x’Éi
%x’’, then xÉi
%x’’;
%• Completeness: For all x and x’ in Ω , either xÉi
%x’ or x’Éi
%x;
%• Strict convexity: For any solution x, the set of solutions that an agent prefers to x is
%strictly convex;
%where xÉi
%x’ (or x≺i
%x’) indicates that the offer x’ is at least as good as (or better than) x
%for agent i.
%The first two conditions ensure that the agents’ preferences are consistent in the
%negotiation domain; the third condition ensures that any pair of points in the negotiation 
%13
%domain can be compared; the last condition ensures that agents’ preferences on each
%issue are monotone if the values of the other issues are fixed, i.e., if the value of an issue
%increases, when the values of the other issues are fixed, the utility of an agent is
%monotonically increasing or decreasing. This last condition implies that each Pareto
%optimal solution of a multi-attribute negotiation is on a joint tangent hyperplane of a pair
%of indifference curves (or surfaces) 2
%of the two agents and the Pareto frontier is a
%continuous curve
\subsubsection{Communication}
Présentation des actes de dialogues avec leurs catégories et conditions d'applicabilité. 
\textcolor{red}{Expliquer que note choix d'utterances se basent sur les travaux de Candece Sidner. De plus, l'analyse en DSP nous a révélé que les participants utiliser des doubles utterances dans leur négociation, et ceci de manière réccurente. Ceci traduisait de plus leur stratégies de négociation influncé par des comportements de pouvoir}


		
	
	
	\section{Conclusion}
		\subsection{Résumé des contributions}
		
		\subsection{Limites observés ? (pas sure)}
			Faire une autocritique de notre solution parce que bien-sure elle n'est pas parfaite ni complète et qu'on en a conscience. 
		
		\subsection{Respectives et futurs travaux}
			(Je reprends tes idées ... )
			\subsubsection{Perspectives à court termes}
				Intégrer le modèle de dialogue dans un ACA pour prendre en compte les comportements non verbaux. Il a été montré que la dominance est véhiculé a travers des comportements non verbaux (comme présenté dans la section \ref{ri}).
			
			\subsubsection{Perspectives à moyen termes}
			\begin{itemize}
				\item	Ajouter la notion d'argumentation dans le processus de négociation
				\item	Étudier la dominance comme trait de personnalité (voir section \ref{perso})et son impact sur les stratégies de négociations et d'argumentation.

			\end{itemize}

			\subsubsection{Perspectives à long termes}
				Étudier les autres dimensions de relations [Svennivig] et leur impact sur le processus de négociation
				
				Projet ANR (Nicolas Sabouret et Magalie Ochs).
\end{document}
What is dominance in social psychology
Nonverbal Behavior and the Vertical Dimension of Social Relations: A Meta-Analysis

\textbf{several representation of dominance :} 
For example, dominance can be defined as a personality trait involving the motive to control others, the self-
perception of oneself as controlling others, and/or as a behavioral
outcome (success in controlling others or their resources). Status,
involving an ascribed or achieved quality implying respect and
privilege, does not necessarily include the ability to control others
or their resources. Similarly, power defined as the capacity or
structurally sanctioned right to control others or their resources
does not necessarily imply prestige or respect. Other distinctions
have also been drawn, including different functional bases of
power, such as reward power, expert power, referent power, or
coercive power (French and Raven, 1959), and outcome dependency
(Stevens  Fiske, 2000). Some writers conceptualize dominance
in terms of social skill (e.g., Burgoon  Dunbar, 2000; Byrne,
2001). Some writers define dominance as the enactment of certain
nonverbal or verbal behaviors (Rosa  Mazur, 1979). Authors do
not use the various verticality terms such as power, dominance,
and status in consistent ways, and often the terms are used without
a clear definition.
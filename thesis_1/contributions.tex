% !TeX spellcheck = fr_FR

Comme mentionné dans l'introduction, la principale contribution de cette thèse est d'étudier l'impact de la relation de dominance sur les stratégies de négociation dans le cadre de négociation collaborative entre un agent conversationnel et un utilisateur humain. 
Pour ce faire, nous devons construire un modèle de négociation qui permettent aux négociateurs de présenter leurs stratégies de négociation. 

Dans ce chapitre, nous présentons notre modèle de négociation collaborative sur lequel sera construit notre modèle de décision basé sur la dominance. 

Afin de définir un système de dialogue dans lequel la relation de dominance régit le choix du prochain énoncé, nous avons d'abord enregistré des dialogues de négociation entre deux personnes afin d'observer leurs comportements dans un cadre de dialogue social de type "négociation collaborative". Nous avons annotés et analysé les dialogues. Cette étude nous a livré un ensemble de comportements  que nous présenterons dans la première section de ce chapitre. 

Les informations collectées grâce l'observation des comportements humains nous a guidés dans la conception de notre modèle de négociation.



Dans la seconde section, nous présenterons le domaine de négociation utilisé. Dans le cadre de cette thèse, nous nous basons sur les modèles de négociations multi-critères largement utilisé dans la mise en œuvre de systèmes de négociations automatiques. 

La troisième section présentera notre modèle de communication basé sur des actes de dialogues. En effet, nous nous sommes réapproprié les actes de dialogue de \cite{grosz1986attention} pour la négociation collaborative.


 \section{Collecte de données}
	 Afin de définir un système de dialogue social dans lequel la relation de dominance régit le choix du prochain énoncé, nous avons d'abord effectuer une étude dans laquelle nous avons analyser les comportements de deux interlocuteurs dans un cadre de dialogue social de type "négociation collaborative". 
	Le but du dialogue est de trouver un restaurant où dîner avec son interlocuteur. De plus, les interlocuteurs n'avaient pas de connaissances prérequis sur les préférences de leur interlocuteur.
	
	Une fois le dialogue enregistré, nous les avons annoté et analyser la structure du dialogue en suivant la théorie de \emph{Grosz et Sidner} \cite{sidner} qui stipule que la structure du dialogue est composée de trois éléments à savoir la structure linguistique, la structure intentionnelle et enfin l'état attentionnel.	
	Nous présenterons dans ce qui suit la procédure de l'analyse ainsi que les résultats obtenus. 
	
	\subsection{Analyse de la structure de dialogue}  
		La théorie présenté par \emph{Grosz et Sidner} propose que la structure d'un dialogue orienté tâche est constituée par trois éléments chacun agissant sur un aspect du dialogue. 
		
		D'abord, \emph{la structure linguistique} a pour but décomposé le dialogue en une séquence de segments de dialogue appelé (\textit{DS: discourse segment }) de tel sorte que chaque segment est composé d'un ensemble de sous-segments et d'une séquence d'énoncés appartenant uniquement au segment(n'appartient pas a un sous segment). De plus, les énoncés au sein d'un même segment contribuent à un même but. Cette décomposition proposée est non strict du fait qu'il est difficile de trouver des indices de segmentation. Des exemples d'indices proposés sont l'intention communicative commune à chaque DS, des expressions  linguistiques comme l'utilisation de termes "d'abord, finalement, $\ldots$". Des indices plus subtiles tels que le changement d'intonation, le temps de pause peuvent aussi être utilisés.
		
	\subsection{Structure intentionnelle}
		Un interlocuteur s'engage dans un dialogue poussé par une ou plusieurs intentions internes qu'on nomme \emph{DP: discourse purpose}. De plus, pour chaque DS, nous pouvons isolé un but spécifique noté \emph{DSP: Discourse segment purpose}. Le but est d'analyser comment les DSPs participent à la satisfaction du DS initial. En outre, cette structure comprend l'analyse des relations entre les différents DSPs. Deux relations ont été identifiées, \emph{dominance} et \emph{satisfaction-precedence}. Si la satisfaction de l'intention d'un DSP1 participe a la satisfaction de celle d'un DSP2, alors le DSP1 \textbf{contribue} au DSP2. Par opposition le DSP2 \textbf{domine} le DSP 1.


	\subsubsection{Structure attentionnelle}
	
		Cette structure représente “l'abstraction de l'attention des participants au fur et à mesure que leur dialogue avance”. (a détailler)
		% Dynamic stack that records salient objects, properties and relations Focusing – process of manipulating focus spaces on attentional (focus) stack
	 
	 \subsubsection{Exemple de l'analyse}
		 Nous présentons dans cette section un exemple de l'analyse dite en DSP que nous avons effectué sur les dialogues que nous avons enregistré.
		 D'abord, nous nous sommes intéressés à la structure linguistique. En effet, nous avons extrait les actes de dialogues de chaque interlocuteur, que nous avons ensuite regroupé en \emph{DS}. 
		 
		 La seconde étape consistait à analyser le but ou l'intention caché derrière chaque \emph{DS}. Par conséquent, nous avons formulé les \emph{DSPs} comme présenté dans l'exemple ci-dessous.		%	ajouter un exemple avec chaque dsp: l'intention trouver et l'indentation.
		
%		\begin{verbatim}
%			[DSP0: Où aller manger ce soir] 
%
%B: ou est que tu veux aller manger ce soir ?
%
%A: ..., bah moi je vais souvent manger au même restaurant à paris, au fait.  
%
%
%[DSP1:  Proposition d'un restaurant breton] 
%
%A: Alors j'avoue j'aime bien les restaus breton.  
%
%A: J'aime bien manger des crêpes.
%
%B: j'ai passé 3 ans à Rennes...
%
%A: Souvent les bonnes crêperies sont à Montparnasse. 
%
%A:je vais souvent à une crêperie qui s'appelle le Josselin et le petit Josselin. y'en a deux au fait.  
%
%A:Les crêpes sont très bonnes, oué. Elles sont costaux mais elles sont très bonnes.
%
%
%
%[DSP 2 : Rejet de la proposition pour cause de comparaison avec ce que B connait]
%
%B: Ayant passé 3 ans à Rennes. Les crêperies sur paris sont un peu moyennes. 
%
%A: tu trouveras de meilleures crêperies à Rennes. 
%
%
%[DSP3: Proposer un restaurant japonais]
%
%B: Sinon j'aime bien japonais 
%
%A: Je n'aime pas du tout le japonais. 
%
%B: tu n'aimes pas tous ce qui est poisson cru ... 
%
%A: je n'aime pas trop la cuisine asiatique encore moins japonais. 
%
%A: Je n'aime pas trop les sushi déjà.
%A: Non je ne suis pas trop cuisine asiatique.
%
%
%[DSP 4: Proposer un restaurant ]
%
%B: du coup pas crêperie, pas japonais. 
%
%B: à Orsay; y'a un restau brasserie qui s'appelle le gramophone à côté de la gare. 
%
%
%[DSP5: Proposition: restaurant africain + italien]
%
%A: j'aime bien les restaus italiens et africains aussi. c'est très bon aussi.  
%
%B: les restaus africains. je ne suis pas fan.
%
%A: et les restaurants italiens ? 
%
%B: oué ... 
%
%
%
%...
%		\end{verbatim}
%\begin{figure}
%	\fbox{\begin{minipage}{.95\textwidth}
%			{\scriptsize\ttfamily
%				\begin{addmargin}[1em]{2em}
%
%		B: Où est que tu veux aller manger ce soir ?
%		A: ..., bah moi je vais souvent manger au même restaurant à paris, au fait.
%		
%		\hspace*{3mm}\textbf{[DSP1:  Proposition d'un restaurant breton] }
%		
%		\hspace*{3mm} A: Alors j'avoue j'aime bien les restaurants breton.
%		  
%		\hspace*{3mm} A: J'aime bien manger des crêpes.
%		
%		\hspace*{3mm} B: j'ai passé 3 ans à Rennes...
%		
%		\hspace*{3mm} A: Souvent les bonnes crêperies sont à Montparnasse. 
%		
%		\hspace*{3mm} A:je vais souvent à une crêperie qui s'appelle le Josselin et le petit Josselin. y'en a deux au fait. 
%		 
%		\hspace*{3mm} A:Les crêpes sont très bonnes, oué. Elles sont costaux mais elles sont très bonnes.					
%
%		\hspace*{3mm}\textbf{[DSP 2 : Rejet de la proposition pour cause de comparaison avec ce que B connaît]}
%		
%		\hspace*{3mm} B: Ayant passé 3 ans à Rennes. Les crêperies sur paris sont un peu moyennes. 
%		
%		\hspace*{3mm} A: tu trouveras de meilleures crêperies à Rennes. 					
%		
%		
%		\hspace*{3mm} \textbf{[DSP3: Proposer un restaurant japonais]}
%		\hspace*{3mm} B: Sinon j'aime bien japonais 
%		\hspace*{3mm} A: Je n'aime pas du tout le japonais. 
%		\hspace*{3mm}B: tu n'aimes pas tous ce qui est poisson cru ... 
%		\hspace*{3mm}A: je n'aime pas trop la cuisine asiatique encore moins japonais. 
%		\hspace*{3mm} A: Je n'aime pas trop les sushi déjà.
%		\hspace*{3mm} A: Non je ne suis pas trop cuisine asiatique.
%
%
%		\end{addmargin}
%	}
%\end{minipage}}
%
%\caption{\label{fig:ex-dialogue}Excerpt of Dialogue 2.}
%\end{figure}
	\subsubsection{Résultats de l'analyse}
		L'analyse en DSPs nous a révélé un nombre de comportements intéressants tant sur l'aspect structurelle de la négociation que sur les stratégies de négociations déployés par les interlocuteurs. 	
		
		Sur l'aspect structurelle, la décomposition du dialogue en \emph{DS} nous a confirmé que les négociateurs s'intéressaient à différents critères pour le choix d'une option (restaurant dans notre exemple). Ces critères sont négociés simultanément durant la négociation jusqu'a ce que les interlocuteurs trouvent un compromis qui les satisfasse sur les critères jugés importants. 
		Par exemple, dans le premier dialogue, les interlocuteurs se sont plus intéressés à l'ambiance du restaurant et son emplacement pour le choix final. En outre dans le second dialogue, les interlocuteurs se sont principalement intéressés aux type et la qualité de la cuisine.
		    
		De plus, Les critères les plus importants sont les premiers à être abordés, et en cas de conflit d'autre critères sont abordés. 
		Ceci est confirmé par des travaux en négociations automatiques qui mettent en avant l'intérêt de la modalisation multicritères dans les systèmes de négociation. Ce point sera abordé plus en détails en section suivante. 
		 
		 Nous nous sommes aussi intéressé à l'aspect dialogique de la négociation. En effet, notre modèle se basant sur des actes de dialogue, nous avons analysé  les différentes informations échangées lors de la négociation. 
		 Nous avons récoltés des informations sur le style linguistique à affecter à nos actes de dialogues qui seront présentés en détails dans la section ... 
		 
		 Finalement, nous avons utilisé la structure attentionnelle et intentionnelle afin d'étudier les stratégies de négociation adoptées par les négociateurs. nous avions analyser la corrélation entre différents comportements durant la négociation influencés par la dimension de la dominance.
		 
		 Le résultats obtenus montrent qu'une relation complémentaire de dominance s'installe entre les négociateurs. C'est à dire que dans la situation où un négociateur prend le pouvoir, l'autre parti accepte cette prise de pouvoir et adapte son comportement.
	
		 La prise de pouvoir se manifeste par prise de parole. Le négociateur avec un haut niveau de dominance avait tendance à prendre la parole plus fréquemment, et plus longtemps. Par exemple, en analysant le \emph{DS1} et \emph{DS3}, nous observons que l'interlocuteur \textit{B} prend plus de tour de parole et pour chaque tour, plusieurs actes dialogique sont énoncés. Par conséquent, en moyenne, la
		 
		 
		 De plus, le style linguistique traduit aussi un comportement de dominance, nous avons observé que la personne dominante avait tendance à facilement exprimer ses préférences (\emph{e.g.} voir \emph{DS3}), argumenter ses choix et décisions dans le but de convaincre l'autre. 
		 
		Ces résultats obtenus ont soutenu les comportements de dominance relayé dans les travaux en psychologie sociale et nous ont aidé a orienter la conception de notre modèle de dialogue
		
	

\section{Domaine de négociation}
\label{domaine}

%	L'interet d'une négociation multi-critères dans la modélisation d'un sujet social
% voir intro :https://www.ri.cmu.edu/pub_files/pub4/lai_guoming_2008_1/lai_guoming_2008_1.pdf
%https://link.springer.com/content/pdf/10.1007/s10458-006-9009-y.pdf

	La recherche en négociation automatique peut être divisée en deux catégories en ce qui concerne la représentation du domaine: négociation sur un critère et la négociation multi-critères. Cependant, La littérature existante se concentre plus sur la négociation uni critère \cite{lai2008decentralized,lai2004literature}. 
	
	Dans le cadre d'une interaction avec un négociateur humain, la négociation multi-critère est cruciale. En effet, dans un environnement humain, les négociateurs peuvent discuter de plusieurs critères simultanément, ce fait est aussi observé dans l'étude que nous avions effectué dans la section précédente.  Nous avons observé que les négociateurs s'intéressaient à plusieurs critères pour le choix d'un restaurant. Par exemple le type de cuisine, la location ou encore l'ambiance de ce dernier. Ces critères ont soit été abordé simultanément dans la négociation, ou bien un par un. C'est à dire que les négociateurs s'accordaient sur un premier critère avant d'aborder un autre, ou bien discuter des différents critères jusqu'a aboutir à un compromis.
	
	De plus, plusieurs travaux en négociation automatique ont mis en exergue l'apport de la négociation multi-critères. Elle permet d'augmenter la coordination et collaboration durant le processus de négociation afin de rechercher un résultat qui apporte des gains communs pour les deux parties \cite{jonker2007agent,lai2008decentralized,lai2004literature} . 

	% 	Pour toutes ces raisons, notre choix s'est porter sur la négociation multi-critère

	Les résultats des précédents travaux nous ont motivé à utiliser une représentation multi-critère pour modéliser notre domaine de négociation collaborative. 
	
	\subsection{Représentation formelle des éléments de la négociation }	
	Le but de la négociation est de choisir une \textit{option} $O$ dans l'ensemble des options $\mathcal{O}$ comprenant toutes les options alternatives envisagé pour un sujet de négociation donnée. 
	
	L'évaluation de chaque option repose sur un ensemble de critères $\mathcal{C}$ reflétant les caractéristiques de l'option. Nous définissons l'ensemble $\mathcal{C}$ de $n$ critères, et $C_1,\ldots,C_n$, comme le domaine de valeurs de chaque critère de l'ensemble. 
	Par conséquent, $\mathcal{O}$ peut être défini comme le produit vectoriel de  $C_1\times\ldots\times C_n$ et chaque option $O \in \mathcal{O}$ est un tuple $(v_1,\ldots,v_n)$. 
	
	Par exemple, une négociation collaborative qui porte sur le choix d'un restaurant où dîner peut être modélisé en prenant en compte quatre critères à savoir $\mathcal{C} = \{Cuisine, Prix, Emplacement, Athmosphère, \}$. La table \ref{tab:domain} résume un exemple de domaine possible pour chaque critère. Nous faisons l'hypothèse que l'agent connaisse toutes les options pour un domaine donné. Un exemple d'option est $ Anterprima(Italien, coûteux , animé, west side)$. Au total, $638$ options peuvent être généré à partir de ce domaine. 
	\begin{table}[h]
		\centering
		\begin{tabular}{|c|c|}
			\hline
			Critère $i $ & Domaine de valeur $C_i$ \\
			\hline
			Cuisine & \{Italien, Français, Japonais, Chinois, Mexicain, Turque, Coréen\} \\
			\hline
			Atmosphère & \{Animé, Calme, Romantique, Familial, Cosy, Moderne\} \\
			\hline
			Prix & \{Coûteux, abordable, a prix bas\} \\
			\hline
			Emplacement & \{West side, East side, Downtown, North side, South side\} \\
			\hline
			
		\end{tabular}
		\caption{Domaine de valeurs pour les critères de choix d'un restaurant} 
		\label{tab:domain}
	\end{table}
	
	\subsection{Préférences}
		L'agent conversationnel est défini avec un ensemble de préférences formalisé  par un ordre partiel $\prec_i$ défini sur chaque domaine de critères $C_i$. 
		Nous définissons la relation de préférence comme une relation binaire. Par exemple, $japonais \prec_{cuisine} italien$ signifie que l'agent préfère la cuisine italienne à la cuisine japonaise. Elle est aussi transitive, par exemple, l'agent dispose d'une autre préférence $italien \prec_{cuisine} français$, par conséquent, nous pouvons déduire que l'agent $japonais \prec_{cuisine} français$. Ces conditions garantissent que les préférences de l'agent sont cohérentes dans le domaine de la négociation; et la condition de transitivité assure que toutes les valeurs sont comparables.
		
		Les préférences étant un aspect essentiel dans la prise de décision durant la négociation, nous avons modélisé une fonction qui représente la valeur d'utilité ou satisfaction pour chaque valeur. Donc, pour un critère $i\in \mathcal{C}$, pour une valeur $v\in C_i$, l'agent calcule sa satisfaction \emph{satisfaction} $sat_{self}(v \prec_i)$ pour cette valeur comme le nombre de valeurs qu'il préfère moins dans l'ordre partiel des préférences $\prec_i$. La valeur est ensuite normalisé dans l'intervalle [0,1]:
		
		\begin{equation}
		sat_{self}(v, \prec_i) =	1 - \left( \frac{|\{v' : v' \neq v \  \wedge \ (v \prec_i v')\}| }{( |C_i| - 1 )}\right)
		\end{equation}
		
		La notion de satisfaction est généralisé pour chaque option $o= (v_1, \ldots, v_n)\in \mathcal{O}$ comme une moyenne des valeurs de satisfactions des différentes valeurs de critères: 
		\footnote{Il existe une grande quantité de travaux  dans le domaine de la prise de décision  qui traitent sur la combinaison de plusieurs critères pour le calcul d'utilité en utilisant par exemple des moyennes pondérées ou des intégrales de Choquet. Nous nous ne intéressons pas dans nos travaux à l'optimisation de la fonction de calcul, pour cette raison nous optons pour une fonction simple d'agrégation de préférences.}

		\begin{equation}
		sat_{self}(o, \prec) = \frac{\sum_{i=1}^{n} sat_{self}(v_i, \prec_i) }{n}
		\vspace{-1.5em} 
		\end{equation}
	


%https://pdfs.semanticscholar.org/f4ea/f27ea6f3f94b8e8d4d8d103eb7a1ebaf2324.pdf
%Strict preference is asymmetric: There is no pair of x and x’ in Ω such that x≺i
%x’ and
%x’≺i
%x;
%• Transitivity: For all x, x’ and x’’ in Ω , if xÉi
%x’ and x’Éi
%x’’, then xÉi
%x’’;
%• Completeness: For all x and x’ in Ω , either xÉi
%x’ or x’Éi
%x;
%• Strict convexity: For any solution x, the set of solutions that an agent prefers to x is
%strictly convex;
%where xÉi
%x’ (or x≺i
%x’) indicates that the offer x’ is at least as good as (or better than) x
%for agent i.
%The first two conditions ensure that the agents’ preferences are consistent in the
%negotiation domain; the third condition ensures that any pair of points in the negotiation 
%13
%domain can be compared; the last condition ensures that agents’ preferences on each
%issue are monotone if the values of the other issues are fixed, i.e., if the value of an issue
%increases, when the values of the other issues are fixed, the utility of an agent is
%monotonically increasing or decreasing. This last condition implies that each Pareto
%optimal solution of a multi-attribute negotiation is on a joint tangent hyperplane of a pair
%of indifference curves (or surfaces) 2
%of the two agents and the Pareto frontier is a
%continuous curve
\subsubsection{Communication}

Présentation des actes de dialogues avec leurs catégories et conditions d'applicabilité. 
\textcolor{red}{Expliquer que note choix d'utterances se basent sur les travaux de Candece Sidner. De plus, l'analyse en DSP nous a révélé que les participants utiliser des doubles utterances dans leur négociation, et ceci de manière réccurente. Ceci traduisait de plus leur stratégies de négociation influncé par des comportements de pouvoir}


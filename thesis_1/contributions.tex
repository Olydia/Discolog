% !TeX spellcheck = <none>

Comme mentionné dans l'introduction, la principale contribution de cette thèse est d'étudier l'impact de la relation de dominance sur les stratégies de négociation dans le cadre de négociation collaborative entre un agent conversationnel et un utilisateur humain. 
Pour ce faire, nous devons construire un modèle de négociation qui permettent aux négociateurs de présenter leurs stratégies de négociation. 

Dans ce chapitre, nous présentons notre modèle de négociation collaborative sur lequel sera construit notre modèle de décision basé sur la dominance. 
Dans la section 
 dans le chapitre 2
	La représentation du domaine de négociation en  informatique, présenter les différents papiers
	interet du win win
	Présenter notre expé
	
	
	La recherche en négociation automatique peut être divisée en deux catégories en ce qui concerne la representation du domaine: negociation sur un critère et la négociation multi-critères. La recherche multi-critères reste un champs moins étudié que la négociation uni critère. Cependant, dans le cadre d'une interaction avec un négociateur humain, la négociation multi-critère est crucial. En effet, dans un environnement humain, les négociateurs peuvent discuter de plusieurs critères simultannément. Un exemple présenté dans [] où des négociateurs
	

	L'interet d'une négociation multi-critères dans la modélisation d'un sujet social
	% voir intro :https://www.ri.cmu.edu/pub_files/pub4/lai_guoming_2008_1/lai_guoming_2008_1.pdf
	%https://link.springer.com/content/pdf/10.1007/s10458-006-9009-y.pdf
	
	De plus, nous avons mené une première expérience où nous avons enregistré deux dialogues de négociations entre deux humains qui avaient pour but de trouver un restaurant où diner.
	
	Nous avons analyser le dialogue en utilisant la décomposition en DSP (Candece). Un des premier résultat observé que les négociateurs s'intéréssaient à plusieurs critères pour le choix d'un restaurants, par exemple le type de cuisine, la location ou encore l'ambiance de ce dernier. Ces critères été soit abordé simultanément dans la négociation, ou bien un par un. C'est a dire que les négociateurs s'accordaient sur un premier critère avant d'aborder un autre, ou bien discuter des différents critères jusqu'a aboutir à un compromis.
		inspiration des travaux en négociation qui se basent sur la représentation multi-critères ( ou multi-attribute) du domaine.
		
\subsubsection{Domaine de négociation }
Critère , Option, Préférences, Satisfiabilité
Contexte de négociation: préférences de l'autre, historique de la négociation (proposition, préférences exprimés)
\textcolor{blue}{explication des préférences}
%https://pdfs.semanticscholar.org/f4ea/f27ea6f3f94b8e8d4d8d103eb7a1ebaf2324.pdf
%Strict preference is asymmetric: There is no pair of x and x’ in Ω such that x≺i
%x’ and
%x’≺i
%x;
%• Transitivity: For all x, x’ and x’’ in Ω , if xÉi
%x’ and x’Éi
%x’’, then xÉi
%x’’;
%• Completeness: For all x and x’ in Ω , either xÉi
%x’ or x’Éi
%x;
%• Strict convexity: For any solution x, the set of solutions that an agent prefers to x is
%strictly convex;
%where xÉi
%x’ (or x≺i
%x’) indicates that the offer x’ is at least as good as (or better than) x
%for agent i.
%The first two conditions ensure that the agents’ preferences are consistent in the
%negotiation domain; the third condition ensures that any pair of points in the negotiation 
%13
%domain can be compared; the last condition ensures that agents’ preferences on each
%issue are monotone if the values of the other issues are fixed, i.e., if the value of an issue
%increases, when the values of the other issues are fixed, the utility of an agent is
%monotonically increasing or decreasing. This last condition implies that each Pareto
%optimal solution of a multi-attribute negotiation is on a joint tangent hyperplane of a pair
%of indifference curves (or surfaces) 2
%of the two agents and the Pareto frontier is a
%continuous curve
\subsubsection{Communication}
Présentation des actes de dialogues avec leurs catégories et conditions d'applicabilité. 
\textcolor{red}{Expliquer que note choix d'utterances se basent sur les travaux de Candece Sidner. De plus, l'analyse en DSP nous a révélé que les participants utiliser des doubles utterances dans leur négociation, et ceci de manière réccurente. Ceci traduisait de plus leur stratégies de négociation influncé par des comportements de pouvoir}


\documentclass{llncs}
\usepackage{subcaption}
\usepackage{subfig} 
\usepackage{usual}
\usepackage[utf8]{inputenc}

\usepackage{graphicx}
\usepackage[rflt]{floatflt}
\usepackage[noend]{algpseudocode}
\usepackage{subfig} 
\usepackage{frame, caption}
\usepackage{amsmath}
\usepackage{eulervm}
\usepackage{fontenc}
\usepackage{mathrsfs}
\usepackage{multirow, enumitem, longtable, rotating,lipsum, scrextend}
\usepackage{array}
\usepackage{makecell}
\usepackage{xcolor, soul}
\sethlcolor{yellow}	
\usepackage{floatrow}
%\pagestyle{plain}
%
\begin{document}
	plan of the paper
	1. We defined a model of dialogue make a schema that explains 
	The model produces an utterance based on the agent mental state
	the agent mental state includes : its preferences, its representation of power noted\emph{pow}, and the context of the dialogue(history of the dialogue)
	
	Previous studies validated our hypotheses that the choice of the utterance as built reflect the right behaviors of power.
	Our goal is to produce plausible social behaviors when building a relation of dominance with a human interlocutor. 
	(Reprendre la definition de dom comme ecrite) 
	
	We make the assumption that our model faithfully reproduce behaviors of power when selecting utterances to enunciate. On that assumption, we aim to build a model of ToM based on simulation, able to compute the current value of power \emph{pow} expressed by the interlocutor based on the utterance expressed. 
	
	\section{Model of dialogue}
	
		We defined a model of cooperative negotiation enabling a conversational agent to adapt its negotiation strategy to the power it intends to express.  Therefore, a conversational agent is initiated with a value of power $pow \in [0,1]$. In addition, the decisional process to produce an utterance is influenced by four main factors:
	\begin{enumerate}
%		\item \emph{pow}: the agent has a personal representation of the power that it wants to express. The value of power influence all the decisional process and the negotiation strategy from the choice of the utterance to the values associated to the utterances.
		
		\item \emph{Preferences:} For each topic of negotiation, decision is based on a set of criteria $\{C_1, ..., C_n\}$. Thus, the agent is defined with a set of preferences  called $\prec_i$ on each criterion $C_i$. The preference relation is binary and transitive. it's either partial or total ordered. 
		
		% for a fixed value of $pow$,and a topic of negotiation.  a mental state is also defined with a set of agent's preferences. 
		
		Preferences are essential in the decisional process because it allows the agent to compute the values which it likes, that we note \textit{satisfiable}.
		
		The satisfiability of a value $v \in C_i$ defined in the domain of the criterion is computed as follow: 
		
			\begin{equation}
			sat(v, \prec_i) =	1 - \left( \frac{|\{v' : v' \neq v \  \wedge \ (v \prec_i v')\}| }{( |C_i| - 1 )}\right)
			\end{equation}
	
	
		A value $v$ is called satisfiable if:
		 	$sat(v, \prec_i) \geq pow$.
		
		
		\par For example, for  a mental state where $pow =0.6$, the criterion which domain is  $D =\{A, B, C, D\}$ is defined with the set of preferences $\{A \rightarrow  B, C \rightarrow  D , B \rightarrow D \}$. The values of satisfiability are depicted in the table \ref{sat}. We can compute that the set of satisfiable values is composed by the value $sat = \{B, C, D\}$ 
			 \begin{table}
			 	\centering
			 	\begin{tabular}{ |c|c|c|c|c| }
			 		\hline				
			 		value & A & B & C & D \\
			 		\hline

			 		\hline
			 		Sat(value) & 0.3 & 0.6 & 0.6 & 1 \\
			 		\hline
			 		
			 	\end{tabular}
			 	\caption{Value of satisfiability for the model $D$.}
			 	\label{sat}
			 \end{table}
		
		\item \textit{Decision based on power: } During the negotiation, the agent makes decisions about the proposals it receives. However, when the negotiation is not converging, the agent  has to make concessions, which means that the agent might accept proposals which are not satisfiables. We compute the notion of concession with a function called $self$, which is a time varying function of $pow$ that decreases over time. In the beginning, $self = pow$, with the negotiation is evolving $self < pow$. 
		
		We call the set $Acc$, the set of acceptable values such that $ v \in Acc$ if $sat(v) > pow$ 
	\end{enumerate}
	
\end{document}
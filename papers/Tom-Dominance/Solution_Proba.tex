\documentclass{llncs}
\usepackage{subcaption}
\usepackage{subfig} 
\usepackage{usual}
\usepackage{graphicx}
\usepackage[rflt]{floatflt}
\usepackage[francais]{babel}
%\pagestyle{plain}
%
\begin{document}
	
	\section{Introduction}
	Nous avons propos\'e un mod\`ele de dialogue de n\'egociation coop\'erative  o\`u un agent peut adapter sa strat\'egie de n\'egociation en fonction de la relation de pouvoir qu'il cherche \`a exprimer. 
	Le but final de ce mod\`ele est  de mod\`eliser la relation de dominance que deux n\'egociateurs \'etablissent lors de leur interaction. 
	\par La relation de dominance est d\'efini par \emph{Dunbar et \textit{al}} comme la capacit\'e  \`a manifester des comportements de pouvoirs o\`u l'assertion de pouvoir d'un parti est forc\'ement accept\'e par l'autre parti. Par d\'efinition la relation de dominance est compl\'ementaire. 
	Par cons\'equent, si on veut impl\'ementer la relation de dominance lors de la n\'egiciation, l'agent doit "deviner" la relation de pouvoir exprim\'ee par son interlocuteur afin de s'y adapter. 
	
	Nous avons propos\'e d'utiliser la th\`eorie de l'esprit afin de reconna\^itre la relation de pouvoir exprim\'ee par l'autre. 
	
	La premi\`ere solution visait \`a r\'eutiliser le mod\`ele d\'ecisionnel de l'agent pour raisonner sur l'autre. Le mod\`ele existant de n\'egociation coop\'erative sur des pr\'eferences, repose sur trois \'elements (Le contexte de la n\'egociation, la relation de pouvoir \textbf{pow} et les pr\'eferences).
	
	Il fallait donc formuler des hypoth\`eses sur chaque \'element du mod\`ele d\'ecisionel pour pouvoir deviner le \textbf{pow} de l'autre. La limite de cette proposition r\'esidait dans le fait de devoir mod\`eliser tous les mo\`eles de pr\'eferences pour un sujet donn\'e, d'autant plus qu'on ne cherche pas \`a apprendre les pr\'eferences de l'autre.
	
	Pour cette raison, je propose une solution probabiliste où une repr\'esentation partielle des pr\'eferences de l'autre est faite. 
	
	\section{Adaptation de l'algorithme de d\'ecision}
	
	\subsection{Rappel de l'algorithme}
	L'algorithme de d\'ecision est bas\'e sur trois fonctions:
	\subsubsection{Satisfiabilit\'e:}
	
	Cette fonction est utilis\'e pour exprimer les valeurs aim\'es par l'agent (dans le cadre de StatePreference)
	\begin{equation}
	sat_{self}(v, \prec_i) =	1 - \left( \frac{|\{v' : v' \neq v \  \wedge \ (v \prec_i v')\}| }{( |C_i| - 1 )}\right)
	\end{equation}
	
	Une valeur $v$ est dite satisfiable si $ sat_{self}(v, \prec_i) > pow$.
	
	\subsubsection{Acceptabilit\'e:}
	La notion d'acceptabilit\'e est utilis\'ee pour soit accepter ou rejeter des propositions. Une proposition est dite acceptable: 
	\begin{equation}
	\vspace{-.5em} 
	acc(pow,v, t) = sat_{self}(v, \prec_i) \geq  (\beta \cdot self(pow,t))
	\end{equation}
	et Self repr\'esente le poid donn\'e \`a la satisfaction de ses pr\'ef\'erences \`a un moment \textbf{t} de la n\'egociation. 
	
	\begin{equation}
	self(pow, t) = \left\{\begin{array}{ll}
	pow & \mathrm{if\ } (t \leq \tau)\\
	max(0, pow - (\frac{\delta}{pow} \cdot (t - \tau))) & \mathrm{otherwise}
	\end{array}\right.
	\end{equation}
	
	\subsubsection{Tol\'erabilit\'e}
	Cette notion est utilis\'ee pour calculer la valeur de proposition que l'agent va \'enoncer. L'agent prend en compte ses pr\'eferences ainsi que celle de l'autre. A partir de la liste des valeurs acceptables, l'agent calcule la proposition la plus tol\'erable. 
	
	\begin{equation}
	tol(o, t, \prec, A, U, pow) = \frac{ \sum_{i}^{n} tol(v_i, t, \prec_i, A_i, U_i, pow) } {n}
	\end{equation}
	
	
	\subsection{Adaptation du syst\`eme}
	Le but de cette solution est de pr\'esenter un mod\`ele d\'ecisionnel qui calcule la probabilt\'e d'obtenir l'utterance ennonc\'e pour chaque hypoth\`ese de pouvoir \textbf{pow}. 
	
	J'ai formul\'e une premi\`ere hypoth\`ese  qui sugg\`ere qu'au lieu de mod\`eliser toutes les relations de pr\'eferences, on mod\`elisait l'ensemble de valeurs satisfiables.
	
	Par exemple, supposons qu'on ait un mod\`ele compos\'e de l'ensemble des valeurs \{A, B, C, D\}, l'ensemble des relations des pr\'eferences qu'on peut g\'en\'erer de ce mod\`ele est de l'ordre de 4! = 24. 
	
	Au lieu de g\'enerer l'ensemble des relations de pr\'eferences, je propose de g\'enerer l'ensemble des valeurs satisfiables possibles. 
	Rappelons qu'une valeur $v$ est satisfiable si $ sat_{self}(v, \prec_i) > pow$. En supposons qu'on ait un ordre total sur les pr\'ef\'erences, on peut conna\^itre le pourcentage des valeurs satisfiables d'un ensemble donn\'e. 
	
	Par exemple, pour l'ensemble ordonn\'e\{A, B, C, D\} et pow =0.4, a partir des valeurs de satisfiabilit\'e present\'e dans la table \ref{table:conditions}, on peut conclure que les valeurs satisfiables sont \{C,D\}. 
	Les valeurs de satisfiabilit\'e sont normalis\'e entre [0,1] on peut dire que pour un que pow =0.4, 60\% des valeurs  sont satisfiables.
	\begin{table}[h]
		\centering
		\begin{tabular}{ |l|c|c|c|c| }
			\hline
			\textbf{Values}& \textbf{A} & \textbf{B} & \textbf{C} & \textbf{D}\\ 
			\hline
			\newline  \textbf{Sat }& 0 & 0.3 & 0.6 & 1\\ 
			\hline
			
		\end{tabular}
		\caption{Valeurs de sat}
		\label{table:conditions}
	\end{table}
	On peut g\'en\'eraliser comme suit:
	
	Pour une valeur de $pow$ donn\'ee, le domaine de valeurs $D$. Le nombre de valeurs satisfiable est:
	\begin{equation}
	Satisfiable(pow, D)= card(D) * (1-pow)
	\end{equation}
	
	Avec cette proposition, nous avons une mod\'elisation partielle des pr\'ef\'erences mais qui r\'eduit consid\'erablement la taille des mod\`eles à $C_{card(D)}^{1-pow} $  
	
	\subsection{Adaptation d'Acceptabilit\'e}
	Etant donn\'ee qu'on a maintenant une repr\'esentation approximative des valeurs de satisfiabilit\'e. Il faut aussi adapter le calcul de l'acceptabilit\'e.
	Cette notion qui prend en compte le principe de concession mod\`elis\'e avec la fonction $Self$. Si la valeur de $Self$ d\'ecroit lors de la n\'egociation, des valeurs non satisfiables deviennent acceptables.
	Reprenons l'exemple pr\'ec\'edent, supposons qu'au court de la n\'egociation $Self = 0.3$, donc, \textbf{B} est maintenant acceptable. 
	Dans le cas o\`u on ne conna\^it pas l'ordre de pr\'ef\'erences, on peut dire qu'une des deux valeurs \textbf{A ou B} est maintenant acceptable.
	
	\begin{equation}
	Acceptability(V) =  \left\{\begin{array}{ll}
	1 & \mathrm{if\ } (sat(V) = true)\\
	m/n & \mathrm{otherwise}
	\end{array}\right.
	\end{equation}
	\begin{itemize}
		\item m: le nombre de valeurs devenus acceptables (dans l'exemple 1 (\textbf{B})
		\item n: le nombre de valeurs non satisfiables (dans l'exemple 2 (\textbf{A, B}))
	\end{itemize}
	
	\subsection{Tol\'erabilit\'e}
	Pour l'instant, je bloque un peu sur le mod\`ele, car les valeurs de satisfiabilit\'e sont soit vrais(1) ou fausse (0). 
	On pourrait essayer de calculer quand même.
	
	\subsection{D\'ecision}
	Pour le choix de l'utterance, au lieu de calculer l'utterance, je propose de calculer la probabilit\'e d'obtenir l'utterance \'ennonc\'e par l'autre. 
	On combine les probabilit\'es obtenues pour chaque mod\`ele pour chaque valeur de pow. 
	
\end{document}
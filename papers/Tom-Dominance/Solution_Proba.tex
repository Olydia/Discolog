\documentclass{llncs}
\usepackage{subcaption}
\usepackage{subfig} 
\usepackage{usual}
\usepackage[utf8]{inputenc}

\usepackage{graphicx}
\usepackage[rflt]{floatflt}

\usepackage[noend]{algpseudocode}
\usepackage{subfig} 
\usepackage{frame, caption}
\usepackage{amsmath}
\usepackage{eulervm}
\usepackage{fontenc}
\usepackage{mathrsfs}
\usepackage{multirow, enumitem, longtable, rotating,lipsum, scrextend}
\usepackage{array}
\usepackage{makecell}
\usepackage{xcolor, soul}
\sethlcolor{yellow}	
\usepackage{floatrow}
%\pagestyle{plain}
%
\begin{document}
	
	\section{Introduction}
	Nous avons proposé un modèle de dialogue de négociation coopérative  qui permet a un agent d'adapter sa stratégie de négociation en fonction de la relation de pouvoir qu'il cherche à exprimer. 
	Le but final de ce modèle est  de modéliser la relation de dominance que deux négociateurs établissent lors de leur interaction. 
	\par La relation de dominance est défini par \emph{Dunbar et \textit{al}} comme la capacité  à manifester des comportements de pouvoirs o\`u l'assertion de pouvoir d'un parti est forcément accepté par l'autre parti. Par définition la relation de dominance est complémentaire. 
	Afin de modéliser la relation de dominance lors de la négociation, l'agent doit "deviner" le comportement de pouvoir exprimé par son interlocuteur afin d'adopter un comportement complémentaire. 
	Nous proposons d'utiliser la théorie de l'esprit afin de simuler le raisonnement de 'Other' et donc d'être en mesure de le comprendre et reconnaître les comportements de pouvoir exprimés. la Tom assume que 'Other'  utilise les même principes raisonnement que 'Self'. 
	
	Dans la prochaine section, nous présentons un rappel du modèle décisionnel de l'agent. 
	
	%	La première solution visait à réutiliser le modèle décisionnel de l'agent pour raisonner sur le comportement de l'autre (prédir le comportement de pouvoir éxprimé). Le modèle existant de négociation coopérative sur des préférences, est composé sur trois élements (Le contexte de la négociation, la relation de pouvoir \textbf{pow} et les préférences).
	%	
	%	Nous visons donc à formuler des hypothèses sur chaque élement du modèle décisionel pour pouvoir deviner le \textbf{pow} de l'autre. La limite de cette proposition résidait dans le fait de devoir modéliser des hypothèses tous les moèles de préférences pour un sujet de négociation donné, d'autant plus qu'on ne cherche pas à apprendre les préférences de l'autre.
	%	
	%	Pour cette raison, nous propose une solution probabiliste où une représentation partielle des préférences de l'autre est faite. 
	%	
	%	\section{Adaptation de l'algorithme de décision}
	
	\section{Rappel du modèle décisionnel}
	Pour prendre une décision, l'agent prend en compte ses préférences ainsi que la valeur de pouvoir qu'il cherche à éxprimer. Nous avons donc modéliser un agent qui a un ensemble de préférences (binaire, transitive) sur chaque critère de la négociation. De plus, l'agent est initié avec une valeur de pouvoir $pow \in [0, 1]$. L'algorithme de décision est basé sur trois principales fonctions:
	
	\subsection{Satisfiabilité:}
	Cette fonction est utilisé pour exprimer les valeurs aimés par l'agent. Par exemple, dans le cas o\`u l'agent énonce ses préférences (\textit{i.e StatePreference}). La satisfiabilité d'une valeur $v \in C_i$ est calculée à partir des nombre de prédécesseurs de $v$ dans la relations de préférences sur la totalité $|C_i| - 1$ valeurs à comparer avec $v$. 
	
	
	\begin{equation}
	sat_{self}(v, \prec_i) =	1 - \left( \frac{|\{v' : v' \neq v \  \wedge \ (v \prec_i v')\}| }{( |C_i| - 1 )}\right)
	\end{equation}
	
	Une valeur $v$ est dite satisfiable si $ sat_{self}(v, \prec_i) > pow$.
	
	\subsection{Acceptabilité:}
	\label{sec:acc}
	La notion d'acceptabilité est utilisée pour décider d'accepter ou rejeter des propositions. Une proposition est dite acceptable: 
	\begin{equation}
	\vspace{-.5em} 
	acc(pow,v, t) = sat_{self}(v, \prec_i) \geq  (\beta \cdot self(pow,t))
	\end{equation}
	et Self est une fonction qui modélise la variation de \textit{pow} dans le temps. Elle représente le poid donné à la satisfaction de ses préférences à un moment \textbf{t} de la négociation. 
	
	\begin{equation}
	self(pow, t) = \left\{\begin{array}{ll}
	pow & \mathrm{if\ } (t \leq \tau)\\
	max(0, pow - (\frac{\delta}{pow} \cdot (t - \tau))) & \mathrm{otherwise}
	\end{array}\right.
	\end{equation}
	
	\subsubsection{Tolérabilité}
	Cette notion est utilisée pour calculer la valeur de proposition que l'agent va énoncer. L'agent prend en compte ses préférences ainsi que celle de l'autre. A partir de la liste des valeurs acceptables, l'agent calcule la meilleure proposition à énoncer. 
	
	\begin{equation}
	\begin{split}
	tol(v, t, \prec_i, A_i, U_i, pow) & = self(pow, t)  \cdot sat_{self}(v, \prec_i) \\
	& +  (1 - self(pow, t)) \cdot sat_{other}(v, A_i, U_i)
	\end{split} 
	\end{equation}
	
	
	\subsection{ToM: Raisonnement sur l'autre}
	
	
	Pour calculer la valeur de pouvoir exprimé par l'autre, nous proposons de reproduire son raisonnement à partir de l'utterance qu'il énoncée.
	
	Le raisonnement lors de la négociation étant basé  sur l'état mental de l'agent (\textit{pow} et le modèle de préférences), nous proposons de formuler des hypothèses sur les possibles états mentaux de l'agent et d'élaguer les fausses hypothèses au fur et mesure des informations collectées lors de la negociation (Utterances communiquées).
	
	Un inconvénient important se pose, le nombre d'hypothèses formulées pour une négociation donnée. En effet, pour chaque critère $C$, le nombre de modèles de préférences possibles est de l'ordre de $|C| !$. Donc pour un nombre de $n$ critères le nombre de modèles de préférences qu'on poeut générer est de l'ordre de $ M = \prod_{k=1}^n C_k!$
	
	La taille importante du possible représente une limite computationnelle du modèle de la ToM, d'autant qu'on ne cherche pas à connaître les préférences de l'autre. Pour rappel, notre objectif était de donner un \emph{score} à chaque valeur possiblde $pow_{other}$ et la méthode brute consiste à compter combien des $M$ modèles possibles il reste pour chaque valeur de $pow$.
	
	Pour contourner la limite computationnelle, nous proposons une adaptation de notre modèle de décision avec connaissance \emph{partielle} des préférences de l'autre. 
	
	Le but de cette solution est de présenter un modèle décisionnel qui calcule un score pour chaque valeur de $pow$ en travaillant sur des hypothèses d'état mental qui ne sont définies qu'en terme de valeur satisfiabilité pour les éléments, au lieu de représenter les ordres totaux. Cela réduit la taille des possibles puisque, si on a $sa$ éléments satisfiables dans un ensemble à $d$ éléments, il y a seulement $C_d^{sa}$ combinaisons.
	Comme dans la version avec les ordres totaux, l'agent révise ses hypothèses en fonction des informations collectées durant la négociation.
	
	
	\subsection{Satisfiabilité ou modèle partiel des préférences}
	\label{sec:sat}
	La première étape est d'adapter la modélisation des préférences et le calcul de satisfiabilité des valeurs à partir des préférences. Le but de notre modélisation est d'être em mesure de raisonner (e reproduire le calcul des fonctions de raisonnement) avec une connaissance partielle sur les préférences. 
	
	La première hypothèse forte suggère que \emph{'l'autre'} possède un ordre \emph{total} sur ses préférences. Par conséquent, pour un domaine de valeurs $D$, on génère un modèle $M$ de préférences composé de $(|D| -1)$ relations de préférences binaires. Par conséquent, toutes les valeurs sont comparables entre elles, ce qui permet de les trier par ordre croissants.
	Pour rappel la fonction $Sat$ se calcule à partir du nombre de prédécesseurs dans l'ordre de préférences. Comme les valeurs de $D$ peuvent être triées par ordre croissants due à l'ordre total, on peut anticiper les valeurs de satisfiabilité obtenues pour un domaine donné $D$ à partir de sa taille $|D|$.
	Par exemple, pour un domaine composé de quatre valeurs $ |D|=4$, dont le modèle de préférences $M$ contient un ordre total de préférences. On peur prédir les valeurs de satisfiabilité comme présenté dans la table \ref{poss}.
	 
	 \begin{table}
	 	\centering
	 	\begin{tabular}{ |c|c|c|c|c| }
		 	\hline				
		 	rang(Valeur) & 1 & 2 & 3 & 4 \\
		 	\hline
		 	Nombre prédécesseurs & 3 & 2 & 1& 0 \\
		 	\hline
		 	Sat(Valeur) & 0 & 0.33 & 0.66 &1 \\
		 	\hline
		 	
	 \end{tabular}
	 \caption{Valeurs de satisfiabilité pour un domaine $D$.}
	 \label{poss}
	 \end{table}
	
	Nous pouvons généraliser le calcul de satisfiabilité pour une domaine $D$ possédant un modèle $M$ de préférences d'ordre total. Les valeurs pouvant être triés, chaque valeur $v \in D$ est possède donc un rang $i(v)$ dans l'ordre des préférences. La satisfiabilité d'une valeur $v$ avec un rang $i(v)$ :
	
	\begin{equation}
		sat(v) = \frac{i(v)-1}{|D|-1}
	\end{equation} 
	
	Maintenant, qu'on peut peut calculer les valeurs de satisfiabilité d'un domaine, nous pouvons donc, pour un état mental donné \emph{(M, pow)} calculer la proportion de valeurs satisfiables. 
	Par exemple, pour un état mental défini sur $|D|$ avec un modèle de préférences d'ordre total $M$ et un $pow =0.3$ $M_1 = \{M, 0.3\}$, en visualisant les valeurs de satisfiabilité de $|D|$ (voir table \ref{poss}), on peut conclure que pour un état mental où $pow =0.3$, $3/4$ des valeurs dans $D$ sont satisfiables. 
	
	Cette information nous permet, donc, de générer des hypothèses sur les états mentaux sur un domaine donnée sans pour autant connaître l'ordre de préférences.
	Par exemple, pour l'état mental $M_1$,  supposons que les valeurs de $D$ sont $D =\{A, B, C, D\}$.
	
	Pour un $pow$ donné, nous pouvons générer les hypothèses suivantes sur les valeurs satisfiables: 
	$Hyp(pow) = \{(A,B,C) , (A,B,D), (A,C,D), (B,C,D) \}$
	
	Ce processus est généralisable à toute valeur de $pow$ donné. Nous proposons donc une première adaptation qui consiste à générer des hypothèses sur l'ensemble des valeurs satisfiables pour un $pow$ fixé, au lieu de générer des hypothèses sur l'ensemble des préférences pour le domaine. Cette proposition réduit considérablement l'ensemble du possible.
	
	Soit $\mathcal{M}_H$ une hypothèse sur un état mental, c'est-à-dire un ensemble des valeurs satisfiable pour un $pow$ fixé (par exemple, $\{B,C\}$). Le calcul de satisfiabilité pour une valeur $v\in D$ dans l'hypothèse $\mathcal{M}_H$ est alors le suivant: 
	
		\begin{equation}
		sat_{\mathcal{M}_H}(v)= \left\{\begin{array}{ll}
		True	 & \mathrm{if\ }  v \in \mathcal{M}_H\\
		False & \mathrm{otherwise}
		\end{array}\right.
		\end{equation}
		
	Les hypothèses sur les états mentaux possibles du domaine $D =\{A, B, C, D\}$ sont présenté dans la table \ref{table:poss} 
	
	\begin{table}[h]
		\centering
		\begin{tabular}{ |c|c|c| }
			\hline
			& \multicolumn{2}{c|}{État mental}  \\
			\hline
			Hypothèse & Pow & Modèles possibles Hyp(pow) \\
			\hline
			H1&0.3&$\{ (A,B,C) , (A,B,D), (A,C,D), (B,C,D) \}$ \\
			\hline
			H2&0.4&$\{ (A,B), (A,C), (A,D), (B,C), (B,D), (C,D) \}$ \\
			\hline
			H3&0.5&$\{ (A,B), (A,C), (A,D), (B,C), (B,D), (C,D) \}$\\
			\hline
			H4&0.6&$\{ (A,B), (A,C), (A,D), (B,C), (B,D), (C,D) \}$ \\
			\hline
			H5&0.7&$\{ (A), (B), (C), (D) \}$\\
			\hline
			H6&0.8&$\{ (A), (B), (C), (D) \}$ \\
			\hline
			H7&0.9&$\{ (A), (B), (C), (D) \}$ \\
			\hline
		\end{tabular}
		\caption{Hypothèses de l'état mental de l'autre pour le domaine $D=\{A, B, C, D\}$}
		\label{table:poss}
	\end{table}

	\subsubsection{Mise à jour des hypothèses:}
		Le but de la ToM est de deviner l'état mental de l'autre par le biais des utterances communiquées par ce dernier. Quand l'autre énonce un statePreference, il communique des informations sur ces préférences (ce qu'il aime ou n'aime pas). Cette information permet à  l'agent peut supprimer les hypothèses qui sont en contradictions avec le state énoncé ce qui nous permettrait de réduire le nombre des hypothèses rapidement.  Par exemple,  quand \emph{'l'autre'} énonce $StatePreference(v)$, l'agent met à jour ses hypothèses en supprimant les $\{\mathcal{M}_H/ v \notin \mathcal{M}_H\}$.  
	
	\subsubsection{Proposition pour la ToM}
	Notre objectif est de calculer, à chaque tour de parole, un score sur chaque valeur de pouvoir de l'interlocuteur en partant de ses réponses précédentes et des hypothèses sur les modèles possibles. (rappel)
	En l'état actuel, une première proposition pourrait être de dire que le score pour la valeur $pow$ dépend du nombre d'hypothèses restantes comparées au nombre d'hypothèses initiales: plus nous avons supprimé d'hypothèse (suite à des State), plus le score doit être faible.
	
	La proposition numéro 1 est la suivante:
	$$score(pow)(t) = \frac{|Hyp(pow)(t)|}{|Hyp(pow)(init)|}$$
	
	\subsection{Acceptabilité}
	
	Une valeur $v$ est dite acceptable si $sat(v) > Self(t, pow)$. Nous reprenons le même principe utilisé dans la section \ref{sec:sat} qui nous permet de connaître les valeurs de satisfiabilité d'un domaine. Nous pouvons donc formuler des hypothèses sur la proportion de valeurs acceptables à un moment donné dans la négociation. Nous notons $A$ le nombre de valeurs acceptables. Le modèle de préférences d'ordre total permet de connaître les valeurs de satisfiabilité, nous pouvons calculer le nombre de valeurs acceptables à n'importe quel moment de la négociation. Supposons que pour le domaine $D$, $self =0.3$, nous pouvons calculer à partir des valeurs présentées dans la table \ref{table:poss} que $A =3$. 
	
	
	Une première conclusion peut être faite \emph{une valeur \textit{satisfiable} est forcément \textit{acceptable}} car $(Self(t,pow) <= pow)$ ( voir section \ref{sec:acc}). Cependant, avec les concessions exprimées, la valeur de $Self$ est amenée à décroître durant de la négociation, si $Self<pow$, des valeurs non satisfiables deviennent acceptables. Nous notons $m$ le nombre de valeurs acceptables mais non satisfiables $ m= A - ||$

	Dans le cas où on ne connaît pas l'ordre de préférences, nous ne pouvons connaître les valeurs non satisfiables qui sont maintenant acceptables. En effet, dans le cas où $Self$ décroît, de nouvelles valeurs sont acceptables. Pour un $pow$ donné, c'est-à-dire pour un ensemble $Hyp(pow)$ de modèles possibles à un moment donné de la négociation, et pour chaque modèle $\mathcal{M}_H\in Hyp(pow)$, notons $ac(pow)$ le nombre de valeurs acceptables (c'est-à-dire qui vérifient $sat(v)\geq self$, qui est le même pour tous les $\mathcal{M}_H$) et $sa(pow)$ le nombre de valeurs satisfiables ($sat(v)\geq pow$, qui est aussi le même pour tous les $\mathcal{M}_H$). La probabilité qu'une valeur $v$ soit acceptable pour chacun des modèles $\mathcal{M}_H\in Hyp(pow)$ possibles se calcule alors par:
	
	
	
	% Reprenons l'exemple précédent, pour une hypothèse où $pow =0.4$ et $M= \{A,B\}$. Supposons que $Self = 0.3$ au termes de concessions, donc, des valeurs dont la satisfiabilité est inférieure à $pow$ deviennent acceptables.  
	
	\begin{equation}
	Acceptability_{\mathcal{M}_H}(v) =  \left\{\begin{array}{ll}
	1 & \mathrm{if\ } (sat_{\mathcal{M}_H}(v) = true)\\
	\frac{m}{n} & \mathrm{otherwise}
	\end{array}\right.
	\end{equation}
	\begin{itemize}
		\item $m= (ac(pow) - sa(pow)) $: le nombre de valeurs devenus acceptables
		\item $n= (|D| - sa(pow)) $ le nombre de valeurs non satisfiables
	\end{itemize}
	
	Par conséquent, pour chaque hypothèse d'état mental, l'agent peut calculer la probabilité qu'une valeur soit acceptable durant la négociation.
	
	\subsubsection{Mise à jour des hypothèses:}
		La fonction d'acceptabilité est utilisé dans le cas où l'autre énonce un Accept ou un Reject. En fonction de l'utterance énoncé, l'agent met à jour ses hypothèses. 
		
		\textbf{Reject:} Dans le cas où une valeur est rejetée, la valeur n'est pas acceptable. On peut déduire que cette valeur n'est forcément pas satisfiable (puisque, par définition de $self$, toute valeur acceptable est forcément satisfiable). Cette information permet à l'agent de réviser cette hypothèse en suppriment tous les modèles où cette valeur est satisfiable, comme elle le fait avec des $State$.
		
		\emph{Proposition reste la même}
		
		\textbf{Accept:} TODO
%		\hl{ Pour pow fixé, pour chaque model possible de pow M$\in$ Poss(pow), j'ai la proba d'avoir un accept(v) si c'était ce pow et ce modèle qui était le bon
%			Pour pow fixé, je fais la moyenne des probas des M (* problème pour la moyenne il faudrait que tous les M aient le poids... or quand j'en retire est-ce ça reste vrai ? peut-être ?)	Je considère que cette moyenne est la probabilité qu'il ait eu un accept avec cette valeur de pow.}
%	

	\subsubsection{Proposition pour la ToM}
	Une deuxième proposition, ne tenant compte que du accept, est de faire la moyenne des $acc_{M}(v)$ pour tous les $M$ d'un $pow$ donnné et de dire que c'est le score de $pow$. Et pour tenir compte de la proposition numéro 1, on fait cette moyenne en comptant aussi ce qui a été supprimé (valeur O).
	
	La proposition numéro 2 est la suivante:
	$$score(pow)(t) = \frac{\sum_{M\in Hyp(pow)(t)} acc_M(v)}{|Hyp(pow)(init)|}$$


	\subsection{Tolérabilité}
	
	
%	\subsection{Décision}
%	Pour le choix de l'utterance, au lieu de calculer l'utterance, je propose de calculer la probabilité d'obtenir l'utterance énoncé par l'autre. 
%	On combine les probabilités obtenues pour chaque modèle pour chaque valeur de pow. 
	
\end{document}

\documentclass {article}
\usepackage[francais]{babel}
\usepackage[utf8]{inputenc}

\begin{document}
		\section{Evaluation protocol}
		Evaluate the impact of dominance complementarity on the negotiation outcomes in the context of human/agent negotiation.
		
		\subsection{Hypotheses}
		Previous works on social psychology  stated that the complementarity in dominance improves the negotiation outcomes and enhanced value creation.
		Moreover, It facilitates reciprocal liking. 
		
		We define our hypotheses based on theses works:
		\begin{itemize}
			\item \textbf{H1:} Dyads which exhibit complementarity create more value than do dyads where negotiators shows similar behaviors.
			\item \textbf{H2:} Dyads which exhibit complementarity converges quickly.
			\item \textbf{H3:} Users feel more comfortable with agents displaying complementary behaviors. 

		\end{itemize}
		
		
		
	\subsection{Conditions}
	We take into account two conditions as presented in table \ref{tab:cond} . First, the \textit{simulation condition} in which the agent has a model of ToM to adapt its strategy in order to complement the user's strategy. 
	
	The second condition is the \textit{control condition}, in which the agent has an initial strategy and does not adapt to the user.
	We consider in this condition, two initial values. The first value, the agent will have a dominant behaviors. Whereas, in the second value, the agent expresses submissive behaviors.
	
	\begin{table}[h]
		\centering
		\caption{Conditions of the experiment}
		\label{tab:cond}
		\begin{tabular} {|c|c|c|}
			\hline
			\textbf{Condition 1} & Simulation condition & Adaptive agent \\
			\hline
			\textbf{Condition 2} & Control condition & Dominant Agent \\
			\hline
			 \textbf{Condition 3} & Control condition & Submissive agent \\
			\hline
			
		\end{tabular}
	\end{table}
	
	\subsection{Protocol}
		First, we have to explain the goal of the experiment to participants.
		They have to negotiate with three agents about the topic of restaurant. The have to imagine that they will indeed have dinner with each agent. For this reason,they have to choose which restaurant have dinner.
		The choice of a restaurant is based on the several criteria. They have to use their own preferences on the different criteria to choose a restaurant. 
		
		Second, we ask them to fill information about their preferences for the sake of the experiment.
		For this case, two options are possibles. 
		\begin{itemize}
				\item  Fix the participant's preferences. Give them the list of 'their' preferences. The problem I see with this solution, is that it might change the user's behaviors to a rigid behavior with the goal to only satisfy the given preferences. 
				\item Ask the participants to fill their preferences for each criterion as presented in table \ref{tab:ex_pref}.
		\end{itemize}
		
		\begin{table}[h]
			\centering
			\label{tab:ex_pref}
			\caption{sort the values by order of preferences. (1 is for the most preferred value)}
			\begin{tabular} {|c|c|c|c|}
				\hline
				 values of Price& cheap & affordable & expensive \\
				\hline
				Preference &  &  &    \\
				\hline
			\end{tabular}
		\end{table}
		
		Third, I propose to do an within-subject study in order to evaluate to evaluate \textbf{H3}. 
	
	\subsection{Questionnaire}
		
\end{document}
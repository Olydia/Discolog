%%%%%%%%%%%%%%%%%%%%%%%%%%%%%%%%%%%%%%%%%
% Wenneker Article
% LaTeX Template
% Version 2.0 (28/2/17)
%
% This template was downloaded from:
% http://www.LaTeXTemplates.com
%
% Authors:
% Vel (vel@LaTeXTemplates.com)
% Frits Wenneker
%
% License:
% CC BY-NC-SA 3.0 (http://creativecommons.org/licenses/by-nc-sa/3.0/)
%
%%%%%%%%%%%%%%%%%%%%%%%%%%%%%%%%%%%%%%%%%

%----------------------------------------------------------------------------------------
%	PACKAGES AND OTHER DOCUMENT CONFIGURATIONS
%----------------------------------------------------------------------------------------

\documentclass[10pt, a4paper, twocolumn]{article} % 10pt font size (11 and 12 also possible), A4 paper (letterpaper for US letter) and two column layout (remove for one column)

\input{structure.tex} % Specifies the document structure and loads requires packages

%----------------------------------------------------------------------------------------
%	ARTICLE INFORMATION
%----------------------------------------------------------------------------------------

\title{Impact of interpersonal relation of dominance in collaborative negotiation} % The article title
%
%\author{
%	\authorstyle{John Marston\textsuperscript{1,2,3} and Bonnie MacFarlane\textsuperscript{2,3}} % Authors
%	\newline\newline % Space before institutions
%	\textsuperscript{1}\institution{Universidad Nacional Autónoma de México, Mexico City, Mexico}\\ % Institution 1
%	\textsuperscript{2}\institution{University of Texas at Austin, Texas, United States of America}\\ % Institution 2
%	\textsuperscript{3}\institution{\texttt{LaTeXTemplates.com}} % Institution 3
%}

% Example of a one line author/institution relationship
%\author{\newauthor{John Marston} \newinstitution{Universidad Nacional Autónoma de México, Mexico City, Mexico}}

\date{\today} % Add a date here if you would like one to appear underneath the title block, use \today for the current date, leave empty for no date

%----------------------------------------------------------------------------------------

\begin{document}

\maketitle % Print the title

\thispagestyle{firstpage} % Apply the page style for the first page (no headers and footers)

%----------------------------------------------------------------------------------------
%	ABSTRACT
%----------------------------------------------------------------------------------------

\lettrineabstract{}

%----------------------------------------------------------------------------------------
%	ARTICLE CONTENTS
%----------------------------------------------------------------------------------------

\section{Introduction}

%------------------------------------------------

	Several studies in social psychology have explored the impact of the dominance relationship on the experience of negotiation and the produced outcomes. Some studies have shown that the expression of complementary behaviors of dominance allows to improve coordination and thus the common gain of negotiators. As a result, negotiators felt more comfortable (\cite{tiedens2003power,wiltermuth2009benefits,olekalns2013dyadic}). 
	In parallel, other studies have studied the impact of similarity in dominance behaviors in negotiation. The latter suggest that similarity in non-verbal dominance behaviors improves interaction and the negotiation process because individuals are attracted to those who express similar behaviors. Therefore, the sens of increases the sense of affiliation (\emph{\cite{olekalns2013dyadic}}).
	
	Giving theses contradictions in the literature on the impact of complementarity or similarity of dominance in negotiation, researchers conducted studies to compare the two approaches and investigate which improve interactions of negotiation (\cite{tiedens2003power,dryer1997opposites}). These studies have shown that the complementarity in the non-verbal behaviors of dominance is unconsciously established between individuals. In addition, participants preferred to interact with individuals who engaged in complementary behaviors and felt more comfortable compared to those who exhibited similar behaviors.
	
	Based on these studies, our objective is to explore the impact of these  behaviors of dominance, whether complementary or similar, on negotiation strategies. We focus our investigation in the context of collaborative negotiation between an agent and a user. 
	 
	However, social psychologists define the interpersonal relation of dominance in interactions as necessarily complementary, we believe that complementary strategies will have a greater impact on the negotiation than similar strategies.
 
%----------------------------------------------------------------------------------------
\section{Dominance and negotiation}	
	
	Negotiation has been defined as an interpersonal decision-making process in which two or more people agree on how to allocate scarce resources\cite {thompson2000mind}.  Social psychology literature has shown the importance of dominance in the negotiation process \cite{de1995impact,van2006power,fiske1993controlling}. For this reason, several works have been focused on investigating the impact of dominance in negotiation in various fields and disciplines, including communication, economics, social psychology and sociology.
	
	\subsection{What are the impact of dominance behaviors in negotiation ?}
	
			The community interested in the negotiation process first analyzed negotiator behaviors in order to propose models that optimize negotiation strategies \cite{thompson2010negotiation}.
			Nevertheless, work has progressed to address  social behaviors that influence negotiators' strategies. 
			Among the dimensions of social psychology studied, dominance (called "power" in the negotiating literature), had been the subject of several studies that showed the influence of the interpersonal relationship of dominance on the behavior and strategies of negotiators, and consequently on their performance and the outcome of negotiation \cite{de1995impact,van2006power}.
			
			First, research has shown that dominant negotiators (with high power) have higher aspirations, demand more, concede less. In addition, they tend to use threats and bluffs more often than submissive negotiators (with low power) \cite{de1995impact}.
			
			Dominance also increases action orientation and encourages goal-oriented behavior \cite{van2006power}. Indeed, Galinsky  \cite{Galinsky 2003power} argues that dominant negotiators are more likely to initiate the negotiation, make the first offer and try to control the flow of the negotiation. 
			In addition, he showed that dominant negotiators showed a greater propensity to act compared to the submissive negotiators and their actions are consistent with the final goal.
			
			In addition, dominance influences information search strategies during negotiation \cite{de2004influence}. Submissive negotiators have a strong desire to develop accurate impression on their negotiating partner, which leads them to ask more \emph{diagnostic questions}.
			On the other hand, dominant negotiators are likely to ask more \emph{leading questions}. This type of question suggests an answer that seems consistent with a belief or assumption, whether that belief or assumption is correct or not \cite{galinsky2003power}.
			
			As a result of these dominance behaviors, the literature suggests that dominant negotiators are self-centered and tend not to pay attention to the preferences of less dominant negotiators \emph{(\cite{fiske1993controlling, de1995impact})}. Indeed, dominant individuals have many resources and can often act at will without serious consequences \emph{(\cite{van2006power})}. 
			
			Conversely, submissive negotiators are interested in their partners, exhibit a higher level of equity and consideration \emph{(\cite{de1995impact})}. This strategy consists of acquiring or regaining control of their results by paying particular attention to the people on whom they depend \emph{(\cite{fiske1993controlling})}.
			
			\emph{Giebels} \emph{(\cite{giebels2000interdependence})} shows that all these behaviors lead the dominant negotiators to end up with the largest share of the pie.
			The work presented above focuses on the impact of dominance behaviors on an individual scale. However, to understand the dynamics of negotiation, we must consider the dominance at a dyadic or interpersonal level. 
			
			In the following section, we will present the literature in psychology that has examined the influence of interpersonal dominance on negotiation. 

		\subsection{Dominance complementarity in negotiation}
			 In the previous section, we presented the manifestation of dominance behaviors in negotiation through specific strategies.
			 In order to fully understand the relationship between dominance behaviors, strategies and negotiation results, researchers must simultaneously consider the individual behaviors of dominance expressed by each negotiator, but also the relation of dominance on an interpersonal level, (i.e. how the dominance behaviors expressed by an individual will influence those of his interlocutor.)
				For this reason, social psychologists have focused on the complementarity of dominance behaviors expressed by negotiators. For instance, researchers on the interpersonal circumplex theory \emph{(\cites{wiggins1979psychological, kiesler19831982})}, which organizes behavior according to the two orthogonal dimensions of affiliation and control, assert that the behaviors of individuals will be similar to those expressed by its interlocutor in the affiliation dimension and inverse in the control dimension which is the dominance dimension \cite{tiedens2003power}.
				
				\emph{Estroff and Nowicki} \emph{(\cite{estroff1992interpersonal})} found that subjects placed in complementary pairs (reciprocal on the control dimension and similar in affect) performed better together in a puzzle task than couples placed in non-complementary dyads (similar on control and inverse in affect). These results suggest that complementarity between two partners strengthens their mutual attractiveness.
				On the other hand, this coordination dynamic is not found in dyads where both partners are dominant or both are subject. 
				In the first case, the interaction partners struggle for control, which makes collaboration difficult. Although the task control trait can facilitate the evolution of the process in case of agreement, it can easily make coordination difficult in case of conflict caused by mutual confrontation. In the case where both interaction partners are submitted, little can be accomplished because no direction is defined and the partners do not take initiatives.
				On the other hand, this dynamic of coordination  is not found in dyads where both partners are dominants or both are submissives. 
				In the first case, the interlocutors struggle for control, which makes collaboration difficult. Although the control trait can facilitate the evolution of the process in case of agreement, it can easily makes the coordination difficult in case of conflict caused by mutual confrontation. In the case where both interaction partners are submissive, little can be accomplished because no direction is defined and the partners avoid to take initiatives \emph{(\cite{wiltermuth2015benefits})}.
				
		\subsection{Value creation}
		
				Value creation  allow to identify solutions that. It occurs when there are differences in negotiators' preferences, including the utility attributed to the options of negotiation \emph{(\cite{lax1986managerial,wiltermuth2015benefits})}.
				
				It is important to better understand this value creation process since it is defined ad the key to sustainable agreements \emph{(\cite{wiltermuth2015benefits})}.
				Moreover, \emph{Olekalns et al} asserts that the success or failure of any strategy in the value creation process is partly determined by the social context including the dominance relationship \cite{olekalns2013dyadic}.
				
					\emph{Wiltermuth et al.} argue that increased value creation occurs when the behaviors of negotiators create a dynamic of complementarity in dominance relation, characterized by one interlocutor behaves in a dominant manner and his or her counterpart behaves in a submissive manner\emph{(\cite{wiltermuth2015benefits}.)} In addition, negotiators who express dominance facilitate the process of finding mutually beneficial solutions when their dominance elicits the submission of their counterparts. As a result, complementary dyads achieve a greater common gain compared to other dyads.
					
					This improvement in gain is also due to a better distribution of roles in the complementary dyads which allows them to increase and improve the exchange of information. 
					
					All of these results suggest that complementarity enhances the feeling of comfort and affection between negotiators. This is confirmed by the work of \emph {Tiedens et al} \emph{(\cite{tiedens2003power})} who has shown through studies that when a complementarity of dominance is established between two individuals, they feel more comfortable during the interaction.
					
					Based, on theses findings, we have set up our study which aims to analyze these behaviors in the context of a negotiation between an agent and a human user.
%----------------------------------------------------------------------------------------
%	BIBLIOGRAPHY
%----------------------------------------------------------------------------------------

\printbibliography[title={Bibliography}] % Print the bibliography, section title in curly brackets

%----------------------------------------------------------------------------------------



\end{document}

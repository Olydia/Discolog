	\documentclass [french]{paper}
	\usepackage[francais]{babel}
	\usepackage[utf8]{inputenc}
	\usepackage{graphicx}
	\usepackage{color}
	
	\begin{document}
		\section{Explication de l'étude}
			\textbf{ Version FR}

		Bonjour, l’expérimentation à laquelle vous allez participer à pour objectif d'analyser les comportements d’un agent conversationnel dans le cadre d’un dialogue de négociation avec un humain.
		Dans le cadre de cette étude, vous allez dialoguer avec un ou plusieurs agents conversationnels.
		
		Imaginez que vous allez dîner avec un ami ce soir. Vous allez discuter avec lui afin de choisir un restaurant qui vous convienne à tous les deux.
	
		Le choix d'un restaurant est basé sur vos préférences personnelles. Nous allons prendre en compte 4 critères préalablement défini. En effet, vous prendrez en compte le type de \textbf{cuisine}, \textbf{le prix} , \textbf{l'ambiance} et \textbf{la localisation} pour le choix du restaurant. Par exemple le restaurant Barocco est défini avec les critères présenté dans la table \ref{teb:ex}.
			\begin{table}[h]
			\begin{tabular} {|c|c|c|c|c|}
				\hline
				Nom du restaurant & cuisine & prix & ambiance& localisation \\
				\hline
				Borocco & Italien & Grande table & Calme & à coté de la rivière \\
				\hline 
			\end{tabular}
			\label{teb:ex}
			\caption{Présentation du restaurant Borocco}
		\end{table}
		
		L'étude se déroulera en trois étapes:
		
		Premièrement, vous allez effectué une session d'entraînement afin de vous familiariser avec l'interface d'interaction.
		
		Ensuite, pour les besoins de l'étude, nous vous demanderons de renseigner vos préférences par critères. Les agents avec qui vous allez dialoguer ne connaissent pas vos préférences. Ils seront utilisés pour analyser les dialogues. Merci de bien vouloir respecter les préférences renseigné dans vos interactions.
		
		Finalement, vous allez discuter avec trois personnes différentes représentées par des agents conversationnels.  Comportez vous \textbf{naturellement} avec chacun des trois agents, comme vous l'auriez fait avec une de vos connaissance pour choisir un restaurant.  
		
		
		 Si vous avez des questions, c'est le moment de les poser à l'expérimentateur. Sinon, nous allons commencer la phase d'entraînement 
		\textcolor{blue}{Ici se termine le texte de présentation et nous pouvons lancer la partie tutoriel avec la présence de l'expérimentateur pour toute explication}.
		
		Dans ce qui suit je vais présenter la page d'accueil pour la seconde partie.
		
		\textcolor{red}{Remarque générale : tu ne dis pas trop à quoi va ressembler l'interaction. Je pense qu'il faudrait expliquer (choisir des types de phrase, on voit la phrase, etc). Comme tu tu rédigeais un "tuto" sur l'utilisation de l'interface d'Hatem. }
		
		\textbf{EN version}
			Hello, the experiment in which you will participate aims to analyze the behavior of conversational agents in the context of negotiation dialogues with  humans.
			
			As part of this study, you will interact with one or several conversational agents.
			
			Imagine that you are having dinner with a friend tonight. For this reason, you will discuss with him in order to choose a restaurant that suits you both.
			
			The choice of a restaurant is based on your personal preferences. We will take into account 4 criteria. You'll take into account the type of \textbf {cuisine}, \textbf {the price}, \textbf {the athmosphere} and \textbf {the location} to choose a restaurant. For example the restaurant Barocco is defined with the criteria presented in the table \ref{teb:exEn}
			
				\begin{table}[h]
			\begin{tabular} {|c|c|c|c|c|}
				\hline
				Restaurant name & cuisine & Price & Athmosphere & location \\
				\hline
				Borocco & Italian & Expensive & Calm & River side \\
				\hline 
			\end{tabular}
			\label{teb:exEn}
			\caption{Presentation of the restaurant Borocco}
		\end{table}
		
		The study will take place in three stages:
		
		First, you will do a training session to get familiar with the interaction interface.
		
		Then, for the purposes of the study, we will ask you to fill in your preferences for each criterion.  Note that the agents you are going to interact with do not know your preferences. They will be used to analyze the dialogues. Please, respect the preferences that you would have communicated to us during the interactions.
	
		
		Finally, you will discuss with three different people represented by conversational agents. Please, behave \textbf{naturally} with each of the three agents, as you would have done with one of your relations to choose a restaurant.
	
		If you have any question, its time to ask the experimenter. Otherwise, we can start the first stage of the study.
	
		\section{Reseignement des préférences}
			\subsection{Ecran d'acceuil}
			
				\textbf{Version FR}
				
				
				Pour les besoin de l'étude, nous vous demandons de bien vouloir renseigner vos préférences pour chaque critère sur lesquels portera la négociation.
				
				Les critères étant indépendants, chaque critère sera présenté indépendamment des autres. Vous devez donc spécifiez vos préférences en vous focalisant sur le critère présenté et en ignorant les autres critère.
				
				Nous vous invitons à classer les valeurs \textbf{par ordre décroissant}de vos préférences, en commençant par la valeur qui vous aimez le plus en allant vers la valeur que vous aimez le moins.
				
			 Si vous avez des questions, c'est le moment de les poser à l'expérimentateur. Sinon, nous allons commencer l'étude.
				
				$<<Commencer>>$
				
	
		\subsection{Pour chaque critère}
			Toutes choses égales par ailleurs, classez par ordre croissant de vos préférences les valeurs suivantes: 
		

			
			\textbf{En version}
			For the purposes of this study, we ask you to enter your preferences for each criterion that will be discussed during the negotiation.
			
			Since the criteria are independent, each criterion will be presented independently of the others. You have to focus your rank only on the criterion presented and ignore the other criteria.
			
			We invite you to rank your preferences in an \textbf {descending} order, starting with the value you like the most,  to the value you like the least.
			
			If you have any question, you can call the experimenter. Otherwise, we can strat the study.
		
			
				\textbf{For each criterion:}  All things being equal,sort the following values in ascending order of your preferences.
		
	\end{document}
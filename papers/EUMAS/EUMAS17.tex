
%%%%%%%%%%%%%%%%%%%%%%% file typeinst.tex %%%%%%%%%%%%%%%%%%%%%%%%%
%
% This is the LaTeX source for the instructions to authors using
% the LaTeX document class 'llncs.cls' for contributions to
% the Lecture Notes in Computer Sciences series.
% http://www.springer.com/lncs       Springer Heidelberg 2006/05/04
%
% It may be used as a template for your own input - copy it
% to a new file with a new name and use it as the basis
% for your article.
%
% NB: the document class 'llncs' has its own and detailed documentation, see
% ftp://ftp.springer.de/data/pubftp/pub/tex/latex/llncs/latex2e/llncsdoc.pdf
%
%%%%%%%%%%%%%%%%%%%%%%%%%%%%%%%%%%%%%%%%%%%%%%%%%%%%%%%%%%%%%%%%%%%


\documentclass[runningheads,a4paper]{llncs}
\usepackage{amssymb}
\setcounter{tocdepth}{3}
\usepackage[noend]{algpseudocode}
\usepackage{subfig} 
\usepackage{graphicx}
\usepackage{frame,caption}
\usepackage{amsmath}
\usepackage{eulervm}
\usepackage{fontenc}
\usepackage{mathrsfs}
\usepackage{multirow, enumitem, longtable, rotating,lipsum, scrextend}
\usepackage{array}
\usepackage{floatflt}
\usepackage{makecell}
\usepackage{xcolor,soul}
\sethlcolor{yellow}	
\usepackage{floatrow}
\usepackage{setspace}
\newcommand{\argmax}{\operatornamewithlimits{arg\,max}}

%\usepackage{url}
%\urldef{\mailsa}\path|{ouldouali, nicolas.sabouret}@limsi.fr
%\urldef{\mailsb}\path{rich@wpi.edu}
%\urldef{\mailsc}\path{hatem.dhouib@ensta-paristech.fr}
%\newcommand{\keywords}[1]{\par\addvspace\baselineskip


\begin{document}
	
	\mainmatter  % start of an individual contribution
	
	% first the title is needed
	\title{Effect of Interpersonal Power on Negotiation Strategies in Collaborative Negotiation Dialogues}
	
	% a short form should be given in case it is too long for the running head
	
	
	% the name(s) of the author(s) follow(s) next
	%
	% NB: Chinese authors should write their first names(s) in front of
	% their surnames. This ensures that the names appear correctly in
	% the running heads and the author index.
	%
	\author{Lydia Ould Ouali\inst{1}%
		%
		\and Nicolas Sabouret\inst{1}\and Hatem Dhouib\inst{1}\and Charles Rich\inst{2}}
	%
	
	% (feature abused for this document to repeat the title also on left hand pages)
	
	% the affiliations are given next; don't give your e-mail address
	% unless you accept that it will be published
	\institute{LIMSI-CNRS, UPR 3251, Orsay, France \\
		Universit\'e Paris-Sud, Orsay, France \\
		\email{\{ouldouali, nicolas.sabouret\}@limsi.fr, hatem.dhouib@ensta-paristech.fr}
		\and
		Worcester Polytechnic Institute\\Massachusetts, USA\\
		\email{rich@wpi.edu}
	}
	
	%
	% NB: a more complex sample for affiliations and the mapping to the
	% corresponding authors can be found in the file "llncs.dem"
	% (search for the string "\mainmatter" where a contribution starts).
	% "llncs.dem" accompanies the document class "llncs.cls".
	%
	
	\toctitle{Lecture Notes in Computer Science}
	\tocauthor{Authors' Instructions}
	\maketitle
	
	
	\begin{abstract}
		
		% Negotiation has drawn considerable attention in the IA and human-computer interaction fields where several researches proposed efficient decision making models which improves the outcomes of the negotiation. However,negotiation with a human is a process that involves social interaction as well as personal preferences. Indeed, social psychology have provided evidence that social aspects such as interpersonal relation or emotions play an important role in negotiation. For instance, the interpersonal relation of dominance affects the way the negotiators build their strategy of negotiation in term of demand and concessions. This paper present a model of negotiation where the agent is able to express different strategies of negotiation based on its relation of power with the user. The underlying strategies of negotiation are defined from the literature in social psychology. An experiment that studies the effect of the power in the strategy displayed by the agent during its interaction with human participants showed that participants perceived the different strategies of the agent in respect with the power it intended to express. 
		
		This paper presents a model of negotiation in which the agent is capable of expressing different negotiation strategies depending on its relation of power with the user. The underlying strategies of negotiation are defined from the literature in social psychology. We present an experiment that studies the effect of power in the strategies displayed by the agent during a human-agent collaborative negotiation. Our results show that participants correctly perceive the differences in the agent strategies depending on the relation of power it intendended to express.
		
		
	\end{abstract}
	
	
	\section{Introduction}
	Negotiation as a daily activity in life
	interest of building agents that negotiate when defining agents or help to learn negotiation
	Social behaviors in negotiation in psychology
	proposed implementation 
	One important dimension is dominance that we propose to stdy
	
	
	\section{Model of negotiation based on the relation of power}
	We present a model of \textit{collaborative} negotiation in which an agent and a human user negotiate to reach an agreement, based on each one's \textit{preferences}. In addition, the strategy of negotiation deployed by the agent is based or impacted by the interpersonal relation of power established.
	We present in this section the different elements of the collaborative negotiation model based on the relation of power.
	\subsection{The negotiation domain}
	
	The overall goal of the negotiator agent is to reach an agreement about an \textbf{option} in a set of possible options $\mathcal{O}$. 
	Each option is characterized by a set of \textbf{criteria} denoted $\mathcal{C}$. We define for each criterion its domain value $C_i$.
	The set of options $\mathcal{O}$ can be simply defined as the cross-product $C_1\times\ldots\times C_n$ and each option $o\in\mathcal{O}$ is a tuple $(v_1,\ldots,v_n)$, making the simplifying assumption that all options are available. For instance, in a dialogue about restaurants, the criteria might be the type of cuisine and the price, we could have the option: $(French,expensive)$.
	
	\subsubsection{Preference model} 
	We enable the agent to have preferences on each criterion of $\mathcal{C}$, formalized as a set of partial orders $\prec_i$ on each $C_i$. For instance, if the agent prefers affordable restaurants to expenssive, $Expenssive\prec_{Price}Affordable$.
	
	Based on the relation of preferences, we define a function of \emph{satisfaction} that allows the agent to compute whether its like a value or not. Therefore, for a given value $v\in C_i$, the agent computes its \emph{satisfaction} $sat_{self}(v \prec_i)$ for this value as the number of values it prefers less in the partial order $\prec_i$, normalized in [0,1]:
	\vspace{-.5em} 
	\begin{equation}
	sat_{self}(v, \prec_i) =	1 - \left( \frac{|\{v' : v' \neq v \  \wedge \ (v \prec_i v')\}| }{( |C_i| - 1 )}\right)
	\end{equation}
	
	This notion of satisfaction is generalized to any option $o= (v_1, \ldots, v_n)\in \mathcal{O}$ as a simple average
	\vspace{-1em} 
	\begin{equation}
	sat_{self}(o, \prec) = \frac{\sum_{i=1}^{n} sat_{self}(v_i, \prec_i) }{n}
	\vspace{-1.5em} 
	\end{equation}
	
	%In addition to its preferences, we allow the agent to construct a model of its interlocutor preferences
	
	\subsubsection{Communication model}
	\label{Comm}
	Agent and user communicate through \emph{utterances}. Each utterance type has a specific set of arguments and is associated with a specific expression in natural language (NL). We use five utterance types, based on the work of Sidner \cite{sidner1994artificial} and two additional utterances to close the negotiation (see Table \ref{table:utt}). The NL generation adapts to the discussed topic. The value /$v$/ in Table \ref{table:utt} refers to this NL format to express a value.
	
	
	Each utterance type takes as parameter either a criterion value $v \in C_i$, an option $o \in \mathcal{O}$ or a criterion type $i \in \mathcal{C}$. They can be separated into three groups. 
	
	\begin{itemize}
		\item Information moves (\textit{AskValue/AskCriterion} and \textit{StateValue}) are used to share knowledge about the participant's likings.
		\item Negotiation moves (\textit{Propose}, \textit{Accept} and \textit{Reject}). Throughout the course of the negotiation process, the agent makes offers for both values. Criterion(``Let's go to a Chinese restaurant'') or options (``Let's go to \emph{Chez Francis}''). It has also to respond to offers, by accepting or rejecting them.
		
		\item Closure moves (\textit{NegotiationSuccess} or \textit{NegotiationFailure}) are used to end the negotiation.
	\end{itemize}
	
	
	%The RejectPropose utterance type is used to clearly reject an option and make a counter-proposal in the same dialogue move. Similarly, the RejectState utterance type is used to make a reject with an explanation. The AcceptPropose is used to accept a criteria and propose a compatible option. 
	
	
	\begin{table}[t]
		{\scriptsize
			\begin{tabular} {|p{2.75cm}|p{4cm}|p{3cm}|}
				\hline
				\textbf{Utterance type}  &\textbf{ NL generation} & \textbf{Postcondition}\\
				\hline
				StateValue(v) &  I (don't) like /$v$/. & Speaker : $v \in S_i$ \newline Hearer:  \newline $v\in A_i$ is likable, $v\in U_i$ otherwise \\
				\hline
				AskValue(v)& Do you like /$v$/ ? & \multirow{2}{*}{} \\
				
				AskCriterion(i) &  What kind of /$i$/ do you like ? & \\
				\hline
				ProposeOption(o)  & Let's go to /$o$/. & $o \in P$\\
				
				ProposeValue(v) & Let's go to a /$v$/. & $v \in P_i$\\
				\hline
				AcceptOption(o)& Okay, let's go to /$o$/.& $o \in T$ \\
				
				AcceptValue(v) & Okay, let's go to a /$v$/.& $v \in T_i$ \\
				\hline
				RejectOption(o) & I'd rather choose  something else. & $o \in R$\\
				
				RejectValue(v) &  I'd rather choose  something else. & $v \in R_i$ \\
				\hline
				NegotiationSuccess &  We reached an agreement. & \multirow{2}{*}{}\\
				\cline{1-2}
				NegotiationFailure &  Sorry, but I no longer want to discuss this. & \\
				\hline
				% Counter Propose & $(r,p)\in C_i^2 \vee (r,p) \in \mathcal{O}^2 $ & I don't want to go to $r$. Let's rather go to $p$ \\
				% \hline 
				% RejectState & $x \in \mathcal{O} \vee x\in C_i$ &  I don't like /$x$/, let's choose something else. \\
				% \hline
				% AcceptPropose & $o \in \mathcal{O}$ & Okay. Let's go to /$o$/.\\
				% \hline
			\end{tabular}
		}
		\caption{\label{table:utt}List of utterance types in the model of dialogue.}
	\end{table}
	
	
	\subsection{Dialogue context}
	In order to make cohrent decisions during the negotiation, the agent keeps track of the information shared. 
	\begin{itemize}
		\item \emph{Other statements}: We build the sets $A_i\subseteq C_i$ and $U_i\subseteq C_i$ of values which have been stated by the other as liked or disliked through \emph{StateValue} utterances. The agent can thus computes the satisfaction for each value.
		\vspace{-0.5em} 
		\begin{equation}
		sat_{other}(v)= \left\{\begin{array}{ll}
		1	 & \mathrm{if\ }  c \in A_i\\
		0    & \mathrm{if\ }c \in U_i\\
		0.5	 & \mathrm{otherwise}
		\end{array}\right.
		\end{equation}
		\vspace{-0.5em} 
		\item \emph{Self statements:} We build the set $S_i \subseteq C_i$ of statements that the agent has made about this criterion which avoids repetition.
		
		\item \emph{Shared proposals:} We define the sets $P$, $T$ and $R$ of respectively open, accepted and rejected proposals in the dialogue.
		
	\end{itemize}
	
	\subsection{the decisional model based on power}
	
	The agent builds its strategy of negotiation with respect to the relation of power its has with the user. Therefore, we initiate the agent with it's belief of power $pow \in [0,1]$. Furthermore, its strategy of negotiation is driven by three main behaviors identified from researches in social psychology. 
	\vspace{-0.5em} 
	\subsubsection{Level of demand and concessions}
	Dedreu proved in his work \cite{de1995impact} that high-power negotiators show a higher level of demand than the low-power ones. In addition, low-power negotiator's demand decrease over time and the negotiator tends to make larger concessions than high-power negotiators. 
	The notion of concession is implemented by a \emph{concession curve} denoted  $self(pow, t)$ which decreases over time ($t$). \footnote{For detailed explanation, please refer to the article []}
	%illustrated in \ref{fig:conc}
	
	\begin{equation}
	self(pow, t) = \left\{\begin{array}{ll}
	pow & \mathrm{if\ } (t \leq \tau)\\
	max(0, pow - (\frac{\delta}{pow} \cdot (t - \tau))) & \mathrm{otherwise}
	\end{array}\right.
	\end{equation}
	
	%		\begin{floatingfigure}[r]{1.8in}
	%			\includegraphics[width=1.8in]{graphs/sv3.png}
	%			\caption{\label{fig:conc}Concession curve}
	%		\end{floatingfigure} 
	The level of demand shows up in negotiation when the agent has to decide wheather a proposal is acceptable.
	Therefore, we implemented a function that computes the acceptability of proposals. Let be $v$ a proposal, thus its acceptability is computed:
	\vspace{-0.5em} 
	\begin{equation}
	\vspace{-.5em} 
	acc(pow,v, t) = sat_{self}(v, \prec_i) \geq  (\beta \cdot self(pow,t))
	\end{equation}
	
	
	\subsubsection{Self vs other:} Low-power negotiators consider the preferences when making decisions, whereas high-power negotiators are self-centered and only interested in satisfying their own preferences. \cite{fiske1993controlling,de1995impact}.
	Therefore, when the agent makes proposals, it has to take into account its preferences and its interlocutor's preferences with respect of the weight it gives to the satisfaction of its preferences and the other preferences.	
	Let us consider the subset $V_i\subseteq C_i$ of values that are acceptable for the agent:
	\vspace{-0.5em} 
	\begin{equation}
	V_i(pow,t) = \{ v\in C_i : acc(pow,v,t) \}
	\vspace{-0.5em}
	\end{equation}
	
	We propose a function for a proposal selection as follows:
	\begin{equation}
	\begin{split}
	tol(v, t, \prec_i, A_i, U_i, pow) & = self(pow, t) . sat_{self}(v, \prec_i) \\
	& +  (1 - self(pow, t)) . sat_{other}(v, A_i, U_i)
	\end{split}
	\end{equation}
	
	\subsubsection{Controlling the flow of the negotiation:}
	High-power negotiators tend to make the first move \cite{magee2007power} and take the lead in the negotiation. Low-power negotiators aim to construct an accurate model of other preferences, which leads them to ask more questions about other preferences rather than keeping the negotiation going (\emph{e.g} by making proposals)\cite{de2004influence}.
	The lead of the negotiation is reflected throughout the utterances expressed during the negotiation. Therefore, we defined rules that control the choice of the utterance and reflects the cited behaviors. A high-powerful agent will try to keep the negotiation going by priorizing \emph{Negotiation moves} (see section \ref{Comm}). In addition, we designed additional utterances (\textit{i.e} \emph{RejectPropose, AcceptPropose}) allowing the agent to express the control of the negotiation.
	In the contrary, low power agent will at first, focus on \emph{information moves} in order to learn more about the preferences of its interlocutor then make proposals that suits the knowledge gathered. 
	The details of the algorithm are discussed in a previous work []
	
\section{Experiment}
	We conducted a study to evaluate the negotiation model during interactions with human users. We aim to analyze the user's perception of the agent behavior during the negotiation.
	
	\subsection{Study design}
		
		Participants negotiate with the agent on the social topic of \emph{"restaurants"}. Our aim was to define a social topic which does not require specific expertise and participants have personal preferences.
		
		The criteria defined to choose a restaurant(see section ) were \{ \textit{cuisine, price, ambiance, location}\}. Each criterion was defined with a domain of values, and a total of 420 of restaurants was generated from the values of each criterion. (figure)
		
		We defined two agents Bob and Arthur which play different behaviors. We designed Bob to follow a high power behavior (\textit{i.e} pow(Bob) = 0.8) and Arthur to play the low power negotiator (\textit{i.e} pow(Arthur) = 0.4).
		Participants interacted with agents thought a HMI that we designed for the experiment. They communicated using the utterances that we defined see( section .. ). Firgure 2 shows the HMI with the possible utterances to enunciate.
		
		Two conditions were considered in order to not bias the perception of participants. In the first condition, participants interacted first with Bob, the hight power agent, following by the interaction with the low power agent Arthur. In the contrary, the second condition, participants interacted first with Arther then with Bob. 
		 
		\subsection{Hypotheses}
			The hypotheses closely follows those studied in a previous work. We aim to validate theses hypotheses in the context of a human machine interaction
			
			\begin{itemize}
					\item  \textbf{H1:} The higher-power speaker will more strongly be perceived as self-centered than the lower-power speaker.  
					
					\item \textbf{H2:} The lower-power speaker will be more strongly perceived as making larger concessions than the higher-power speaker.
					
					\item \textbf{H3:}  The higher-power speaker will more strongly be perceived as demanding than the lower-power speaker.
					
					\item \textbf{H4:}  The higher-power speaker will more strongly be perceived as taking the lead in the negotiation than the lower-power speaker.
					
				\end{itemize}
				
		\subsection{Experimental Procedure}
		We conducted a within-subject where each participant interacted with both  agents Bob and Arther.
		
		
		Upon arrival, participants were asked to sign an informed consent form and the negotiation task was explained. Following this, training session begins, participants were given instruction on the use of the HMI to interact with the agent, the experimenter started the training session and left the room until training was complete and participants were familiar with the interface. Following training, the experiment begins and participants negotiate with the agents. Upon completion of the negotiation, participants were asked to fill out a questionnaire about their experience. They were asked to report their perception of the agents behaviors during the negotiation. 
		
		\subsection{Results}
		
		
	%  within-subject
\scriptsize{	
	\bibliographystyle{splncs}
	\bibliography{Library}}
\end{document}

% !TeX spellcheck = <none>
\section{Introduction}
Les agents conversationnels connaissent un essort important  dans plusieurs domaines d'applications. Ces agents jouent différents rôles allant du rôle de tuteur pour enfant au compagnon pour les personnes agées isolées.
Dans le contexte de ces d'interactions,  l'agent et l'utilisateur humain  sont amenés à collaborer afin de satisfaire des tâches ou buts communs. Par exemple, un agent tuteur qui collabore avec un eleve pour résoudre des exercices. Ils comparent leur connaissances respectives sur l'exercice à résoudre et discutent des solutions possibles. Une telle confrontation  permet d'offrir un enseignement personnalisé à l'éleve. 

En effet, la collaboration entre interlocuteurs pour la satisfaction de buts implique la confrentation des préférences et expertises de chacun des partis ce qui les conduit à négocier sur la manière de satisfaire le but en prenant en compte les préférences des deux partis. Ce type de négociation est appelée \emph{négociation collaborative.} A la différence de la négociation compétitive, la négociation collaborative fait l'hypothèse que chaque participant est motivé par le but de trouver le meilleur compromis satisfaisant les interêts des deux participants au lieu de maximiser ses propres gains.

Notre but est de modéliser un agent conversationnel capable d'avoir des négociations collaboratives crédibles avec un utilisateur humain.
Pour ce faire, nous avons besoin de comprendre les comportements adoptés durant les négociations humains/humain. En effet, ... présente la négociation comme un procéssus multidimensionnel qui implique  une interaction sociale, des affects ainsi que des préférences et des opinions. par conséquent, la compréhension des comportements sociaux est cruciale dans l'étude du procéssus de négociation. De plus, la litterature en psychologie sociale  a déjà démontré que la \emph{relation de pouvoir} a un effet direct sur les stratégies des négociateurs. 


Nous proposons dans ce papier, un modèle de négociation collaborative 




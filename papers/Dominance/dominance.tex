\documentclass{llncs}
\usepackage{subcaption}
\usepackage{subfig} 
\usepackage{usual}
\usepackage{graphicx}
\usepackage[rflt]{floatflt}
%\pagestyle{plain}
%
\begin{document}
\title{\vskip -10pt}

\author{Lydia Ould Ouali\inst{1}, Charles Rich\inst{2} \and
Nicolas Sabouret\inst{1} }

\institute{LIMSI-CNRS, UPR 3251, Orsay, France \\
Universit\'e Paris-Sud, Orsay, France \\
\email{\{ouldouali, nicolas.sabouret\}@limsi.fr}
\and
Worcester Polytechnic Institute\\ Worcester, Massachusetts, USA\\
\email{rich@wpi.edu}
}
\maketitle 
\begin{abstract}\vskip -20pt
  
\end{abstract}

\section{Introduction}
	1. Present the evolution of research on conversational agents with social abilities. Diversity of application fields.
	
	
	2. the crucial impact of social skills on behavior responses, decision making during the  dialogue [Strohkorb,leite, warren].
	
	3. Our interest concerns the relation of dominance, and its influence on the strategies adopted during the dialogue. 
	
	4. plan of the paper 

\section{Related works}

Papers that get interested in the relation of dominance  
	Psychology
	Conversational agents [Strohkorb,leite, warren] \cite{BickmoreEtAl2010}[Lina Zhou]

\section{Nature of dominance}

\subsection{Definition of dominance}
	Explain that the dominance that we are intersted in, is a Dominance is a function of particular relationships (Bernstein, 1981). It is more of 	a social phenomenon involving inter-individual dynamics than a simple phenomenon descriptive of an individual quality (Omark, 1980). Dominance–submission is
	positioned as one of two superordinate dimensions by which people come to define and understand their interpersonal relationships (Burgoon \& Hale, 1984).

\subsection{Dominance indicators}
Dominance can appears in different manners in dialogue :
\subsubsection{Verbal behavior}

\subsubsection{Non-verbal behavior}


\subsubsection{Nature of dominance in text based conversation}

Terminate by explaining that our contribution is to study the impact of dominance in text based dialogue and how it will affect the strategies of negotiation.


\section{Contribution}
	\subsection{Model of dialogue}
	Explain briefly that we construct a model of dialogue that allows to the agent to conduct a cooperative negotiation about preferences during the dialogue.
	therefore, we have a preference model that the agent keeps in its mental state. 
		\subsubsection{Preference model}
			What is a preference
			Decision based on preferences
				
		\subsubsection{Utterances types}
	
	\subsection{Utterance choice based on dominance}
		Use the paper that I wrote about the subtrees for each utterance that take in accout in addition to the current mental state, the perception of the dominance, and explain the different cases.
	
	\subsection{Implementation}		
\section{Validation}

Process of validation with the perception of dominance. 	
\vskip 4pt
\bibliographystyle{plain}
\bibliography{abbrevs,Library}
\end{document}

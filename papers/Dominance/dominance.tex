\documentclass{llncs}
\usepackage{subcaption}
\usepackage{subfig} 
\usepackage{usual}
\usepackage{graphicx}
\usepackage[rflt]{floatflt}
%\pagestyle{plain}
%
\begin{document}
\title{\vskip -10pt}

\author{Lydia Ould Ouali\inst{1}, Charles Rich\inst{2} \and
Nicolas Sabouret\inst{1} }

\institute{LIMSI-CNRS, UPR 3251, Orsay, France \\
Universit\'e Paris-Sud, Orsay, France \\
\email{\{ouldouali, nicolas.sabouret\}@limsi.fr}
\and
Worcester Polytechnic Institute\\ Worcester, Massachusetts, USA\\
\email{rich@wpi.edu}
}
\maketitle 
\begin{abstract}\vskip -20pt
  
\end{abstract}

\section{Introduction}
The notion of dominance as presented in the literature is wide. It includes the dominance as a trait of personality, a social role in the society or a dimension of interpersonal relationship. (Omark,1980) defines the dominance more as social phenomenon that involves inter-individual dynamics rather than a simple description of individual quality. In this study, we present a dominance as interpersonal relationship that affects the behavior of interlocutors in social conversation. 

\par Social dialogue can be presented as process of opinions and preferences exchange about a subject of discussion \cite{laver1981linguistic}. This exchange of personal preferences  can leads interlocutors to conduct a cooperative negotiation especially if at the end they have to take a decision about a goal to satisfy together. For example, find a restaurant. Prior works already proved that social aspect can have an important impact on the way in which interlocutors  develop their strategies of negotiation. This notion constitute our interest of research: we aim to develop a conversational agent that can perceive and understand its relationship with the user which allow him to adapt his strategy of negotiation in the conversation in order to better meet its goals. To this end, we first provide an outline on the social aspect of dominance-submissiveness that we focus on, and we discuss the features of dominant behavior in the context of cooperative negotiation. Then, we propose the objective of this investigation and our hypothesis regarding the effect of dominance in cooperative negotiation, and report our analysis. 

\section{Features of dominance}
There a wide range of research that investigate on the definition of dominance as dimension of interpersonal relationships. This works converge to define the dominance as the power to influence the behavior of others which can be either manifest or latent way. Dominance is manifest when the attempt of dominant person to assert his power and control is accepted by the partner in the conversation. Whereas, dominance is latent where the dominant person is not aware about his position of dominance. Such kind of behaviors can affects positively the interaction. For example, the dominant person leads the conversation and keeps it going,  take efficient and quick decisions. In the other hand, this same behavior can be perceived as negative during a conversation. For example, a dominant person doesn't give enough space to other participant to express his opinions and ideas and be not open to criticism. This type of verbal dominance expression wan be perceived as offensive and unjustified if both participants seek for the position of dominance with divergent opinions. 

\subsubsection{??}
Many approaches exists to detect and identify dominant behavior in social conversations. In conversation, dominance can appear with verbal or non verbal behavior. Non verbal behaviors include facial expressions such as facial expressions, gazes, posture control and invasive gestures.  Verbal indicators includes speaking frequency, number of words used and repeated during the conversation, argumentation, ignoring, interruption and change the subject of discussion.     

\subsection{Dominance in cooperative negotiation}
\label{DN}

\section{Objective of the investigation}

The main objective of this study is to investigate the perception of people about the behavior related to dominance during a negotiation. Indeed, we implemented a conversational agent that car produce a dominant / submissive behavior as described in the section \ref{DN} during a negotiation. Bernstein (1980) argued that dominance relationships can never be assumed but must be demonstrated in each application. Therefore, we aim to validate the implemented dominant behavior by investigating a perceptual experiment in which participants have to determine the behavior of an agent during a negotiation with another agent. 



%	1. Present the evolution of research on conversational agents with social abilities. Diversity of application fields.
%	
%	
%	2. the crucial impact of social skills on behavior responses, decision making during the  dialogue [Strohkorb,leite, warren].
%	
%	3. Our interest concerns the relation of dominance, and its influence on the strategies adopted during the dialogue. 
%	
%	4. plan of the paper 
%
%\section{Related works}
%
%Papers that get interested in the relation of dominance  
%	Psychology
%	Conversational agents [Strohkorb,leite, warren] \cite{BickmoreEtAl2010}[Lina Zhou]
%
%\section{Nature of dominance}
%
%\subsection{Definition of dominance}
%	Explain that the dominance that we are intersted in, is a Dominance is a function of particular relationships (Bernstein, 1981). It is more of 	a social phenomenon involving inter-individual dynamics than a simple phenomenon descriptive of an individual quality (Omark, 1980). Dominance–submission is
%	positioned as one of two superordinate dimensions by which people come to define and understand their interpersonal relationships (Burgoon \& Hale, 1984).
%
%\subsection{Dominance indicators}
%Dominance can appears in different manners in dialogue :
%\subsubsection{Verbal behavior}
%
%\subsubsection{Non-verbal behavior}
%
%
%\subsubsection{Nature of dominance in text based conversation}
%
%Terminate by explaining that our contribution is to study the impact of dominance in text based dialogue and how it will affect the strategies of negotiation.
%
%
%\section{Contribution}
%	\subsection{Model of dialogue}
%	Explain briefly that we construct a model of dialogue that allows to the agent to conduct a cooperative negotiation about preferences during the dialogue.
%	therefore, we have a preference model that the agent keeps in its mental state. 
%		\subsubsection{Preference model}
%			What is a preference
%			Decision based on preferences
%				
%		\subsubsection{Utterances types}
%	
%	\subsection{Utterance choice based on dominance}
%		Use the paper that I wrote about the subtrees for each utterance that take in accout in addition to the current mental state, the perception of the dominance, and explain the different cases.
%	
%	\subsection{Implementation}		
%\section{Validation}
%
%Process of validation with the perception of dominance. 	
\vskip 4pt
\bibliographystyle{plain}
\bibliography{abbrevs,Library}
\end{document}

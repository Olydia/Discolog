	\documentclass{article}
		\usepackage[noend]{algpseudocode}
		\usepackage{subcaption}
		\usepackage{subfig} 
		\usepackage{usual}
		\usepackage{amsmath}
		\usepackage{graphicx}
		\usepackage{eulervm}
		\usepackage{fontenc}
		\usepackage{mathrsfs}
		\usepackage{multirow}
		\usepackage{array}
		\usepackage[rflt]{floatflt}
		\usepackage{makecell}
		\usepackage{breqn}
		\usepackage{xcolor, soul}
		\sethlcolor{yellow}	
		\usepackage{pbox}
	
		\renewcommand\theadalign{cb}
		\renewcommand\theadfont{\bfseries}
		\renewcommand\theadgape{\Gape[4pt]}
		\renewcommand\cellgape{\Gape[4pt]}
		%\pagestyle{plain}
		%
		\begin{document}
		\title{\vskip -10pt}
		
		\author{Lydia Ould Ouali\inst{1}, Charles Rich\inst{2} \and
		Nicolas Sabouret\inst{1} }
		
	
	\section{Domain model}
		Consider a sequence of criterion sets $C_1, C_2, ..., C_n$ and an options set defined as the cross-product:
				$O = C_1 \times C_2 \times \ldots C_n$.
		
		For example, for restaurants, the criteria might be: \\
			Cuisine = \{Chinese, French, ...\} \\
			Cost = \{cheap, expensive, ...\} \\
			Atmosphere = \{quiet, lively, ...\} \\ 
		
		and a restaurant option might be: \\
		 Ginza: [Chinese, cheap, lively].  \\ Jiliya[Chinese, expensive, quit].   \\ Ying [Chinese, expensive, lively].\\Beaurepaire[French, cheap, quit].  \\ Parisien[French, expensive, quit].   \\ Gramophone [Chinese, expensive, lively]$\ldots$ \\ 
		 
	
	\subsection{Preferences for self}
	Preferences are pre-orders (transitive, antisymmetric binary
	relations) on the criterion sets.  The preference relation on the
	criterion set $C_i$ is denoted by $\prec_i$.
	
	Let P denote the sequence of preferences $ \prec_i, \prec_2, ..., \prec_n$
	
		\subsubsection{	Self satisfaction function} 
	
	Satisfaction is a function normalized to [0,1] that evaluates an element of a criterion set relative to the corresponding preference relation, defined as follows.
	
		For $c \in C_i$ and $\prec_i$:
	
	$$sat(c, \prec_i) =	1 - \left( \frac{|{d : (d \neq c)  \wedge (c \prec_i d)}| }{( |C_i| - 1 )}\right)  $$
	
	Satisfaction is generalized to options as a weighted sum.
	
	For $o \in O$ and where $o_i$ is the i-th element of $o$ and $P_i$ the i-th element of $P$
	
	$$sat(o, P) = \frac{\left( sat(o_1, P_1) + sat(o_2, P_2) + ...  \right)}{|P|} $$
	
	\hl{I don't have any opposition in removing criteria since they are never expressed in the negotiation, and only used in the calculation}
	
	\subsection{Acceptability for other}
	
	The agent do not have detailed knowledge of the preferences of the other; it only knows which criteria values are acceptable to the other from previous conversations. 
	
	Let the collection of $A_i \subseteq C_i$ be the criterion values that
	are acceptable to the other.
	 \hl{So if $ o \notin A_i$ means that o is not acceptable or the agent don't know ? I need a more precise representation for the utterance type section } 
	
	Then the boolean function for acceptability of an option $o \in O$ \hl{ or a criterion, we have knowledge about the criteria also}
	to the other is:
	
	$$other(o, A) = \forall i,  o_i \in A_i$$

	
	
	\section{Principles of dominance in negotiation}
	
	\subsection{Principle 1: Satisfaction of Self preferences Vs Other preferences}
%	Scholars in psychology already demonstrate that the relation of dominance affect negotiator's motivation in maximizing self preferences versus other preferences. We took an interest in DeDreu work in which he studies the impact of social value orientation \textit{(i.e  individual differences in	how people evaluate outcomes for themselves and others	in interdependent situation)} on negotiator behavior. We keep three main mechanisms:
%	\begin{enumerate}
%		\item Submissive negotiators consider the well-being of other in the negotiation, whereas dominant negotiators are only interested in their own well-being.
%		\item Dominant negotiators show a greater levels of demand than the submissive one. Moreover, submissive negotiators show a \textbf{greater decline} of their preferences after the second round of negotiation. 
%		\item Submissive negotiators conceded more on low priority issues.
%	\end{enumerate}   
	
	\subsubsection{Dominance and concession}
	Let  $dom \in [0, 1] $ denotes the dominance of an agent (self) in the 	relationship.  It is a constant for a given agent in a given	relationship.
	
	The weight that an agent gives to its self-satisfaction relative to	the satisfaction of the other is a time-varying function normalized to 	[0,1] of its dominance, defined as below.
		$$self(dom, t) = \left\{\begin{array}{ll}
		dom & \mathrm{if\ } (t <= \tau)\\
		max(0, dom - (\frac{\delta}{dom} * (t - \tau))) & \mathrm{otherwise}
		\end{array}\right.$$
		
	
	where is $t >= 0$ is the number of open or rejected proposals.
	
	This is called the "concession curve", where $\tau > 0$ and $\delta > 0$
	are parameters of the theory in general and are initially assumed to
	be 2 and 0.1, respectively. \hl {I'm not sure of whether present the values or the definition of the variables? }
	\subsubsection{Acceptability for self}

	The self-acceptability of a \hl{proposal $x$ (a criterion $x \in C_i)$ or an option $x  \in O$} is a function normalized to [0,1] and defined as below.

	$$acc(dom, t, x, P, A) = self(dom, t) * sat(x, P)+ (1 - self(dom, t)) * other(x, A)$$

	\hl{acc should also be calculated for the criteria, in a negotiation agent makes proposals of criteria not only options}
	
	
	\section{Principle 2: Lead of the negotiation}
%	
	
%	\begin{itemize}
%		\item Powerful negotiators tend to make the first move. %\cite{magee2007power}
%	
%		\item Carsten, De Dreu and Van Kleef demonstrate that high power negotiators are high in their propensity to negotiate relative to participants with low power. (leading individuals to focus on the rewards available to them in situations and to bargain for	greater rewards than were initially offered to them.)
%	
%		\item Dominance affects the way that negotiators gather information about their partners. Negotiators with less power have a stronger desire to develop an accurate understanding of their negotiation partner, which would lead them to ask more \emph{diagnostic} rather than \emph{leading} questions.
%		
%	\end{itemize} 
	
	\subsection{Choosing an utterance type}
		
		Let $V$ be the set of available options (values), such that option :$ V\rightarrow O$	i.e., each value is a "name" for an option.  For example, V is the set of restaurants that are mutually known to the participants. 
	
	\begin{itemize}
		\item Let be $ACCEPTED$ the list of proposals which have been accepted in the negotiation. (After an accept utterance).
		\item Let be $REJECTED$ the list of proposals which have been rejected in the negotiation.(After a reject utterance).
		\item Let be $OPEN$ the list of proposals which have been proposed in the negotiation.(after a propose utterance).
		\item $\sigma \in $[0,1] : boundary between submissive and dominant used in
		choosing an utterance type
		\item $\beta$:  a value that represent the minimum score that a value has to get to be positively satisfiable to the agent preferences in the negotiation.
		\item $\tau > 0$ : the minimum number of open or rejected proposals before
		concession begins
		\item $\delta > 0$ : parameter in slope of concession curve
	\end{itemize}
	
	We present in the following the possible responses and their applicability condition. Note that each line (utterance)  assumes that the previous one are already false.
	
	$ chooseUtterance(dom, t, V, P, A) = $ \\
	\textbf{if(\textbf{dom  $>\sigma$})} \\
	\begin{tabular}{|p{3cm}|p{9cm}|}
		\hline
		\textbf{Utterance type} & Condition \\
		\hline
		 Negotiation success &  $\exists x/ x \in O$  \emph{and} $x \in \{ACCEPTED\}$  \newline \emph{OR} \newline $x \in \{OPEN\}$ and \newline $acc(dom,t,x,P,A) >= \beta$ \\
		\hline
		Negotiation failure & $ \forall x \notin \{REJECTED\}$, \newline  $acc(dom,t,x,P,A) < \beta $ \\
		\hline
		State & $OtherAsks$ \\
		\hline
		Propose & $\exists y \notin \{REJECTED\}$ such that \newline $acc(dom,t,y,P,A) >= \beta $  \\
		
	\hline
	\end{tabular}
	\\ \\
		
	\textbf{if(\textbf{dom  $<\sigma$})} \\
	\begin{tabular}{|p{3cm}|p{9cm}|}
		\hline
		\textbf{Utterance type} & Condition \\
		\hline
		Negotiation success &  $\exists x/ x \in O$ \emph{and} $x \in \{ACCEPTED\}$ \\
		\hline
		Accept & $\exists x \in \{OPEN\} /$ \newline $acc(dom,t,x,P,A) >= \beta$ \\
		\hline
		Reject & $\exists x \in \{OPEN\} /$ \newline $ acc(dom,t,x,P,A) < \beta$  \emph{and} $t<n$.\\
		\hline
		Propose & $\exists x$ / $P_{Other} (x)= true $  \emph{and}
		\newline ($acc(dom,t,x,P,A) >= \beta$
		\newline \emph{OR}  
		\newline $\forall c \in C_i$,  $c \in \{ACCEPTED\}$)\\
		\hline
		Ask &  \textbf{(}$t> \tau,$ \emph{and} 
		\newline $\exists x \in \{OPEN\} /$
		\newline $ acc(dom,t,x,P,A) < \beta $\textbf{) }
		\newline \emph{OR}
		\newline \textbf{(}$ \forall c \in C_i, A_i(c) = Unkown$\textbf{)}\hl{$A_i$ needs to be edited}
		\newline \emph{OR} 
		\newline \textbf{(}$\exists x \in \{ACCEPTED, REJECTED\}$ / 
		\newline $acc(dom,t,x,P,A) >= \beta$ \textbf{)} \\
		\hline
		
		State & $OtherAsks$
		\newline \emph{OR}
		\newline $\exists x,A_i(x) \not = Unkown $ and \hl{$P_{OAS}(x) = Unkown$, In this model we didn't define a representation of what other knows about me }
		\newline \emph{OR}
		\newline $ \exists c \in \mathcal{C}, , A_i(c) = Unkown$
		\\
		\hline
	\end{tabular}
	

	\subsection{Choosing a value for an utterance}
	
	For available options V, \hl {Same, has also to be applicable for criteria}
	
	 $$ chooseValue(dom, t, V, P, A) =	\operatorname*{arg\,max}_{v \in V} acc(dom, t, option(v), P, A) $$
	 \hl {What is option(v) ?}
	
	Note that argmax may not be unique.  This function may be used to help
	decide utterance type and/or choose the argument after the utterance
	type is chosen.

	
%	% BIBLIO %
%	
%	\vskip 4pt
%	\bibliographystyle{plain}
%	\bibliography{abbrevs,Library}
	\end{document}
\documentclass{article}
	\usepackage[noend]{algpseudocode}
	\usepackage{subcaption}
	\usepackage{subfig} 
	\usepackage{usual}
	\usepackage{amsmath}
	\usepackage{graphicx}
	\usepackage{eulervm}
	\usepackage{fontenc}
	\usepackage{mathrsfs}
	\usepackage{multirow}
	\usepackage{array}
	\usepackage[rflt]{floatflt}
	\usepackage{makecell}

	\usepackage{xcolor, soul}
	\sethlcolor{yellow}	

	\renewcommand\theadalign{cb}
	\renewcommand\theadfont{\bfseries}
	\renewcommand\theadgape{\Gape[4pt]}
	\renewcommand\cellgape{\Gape[4pt]}
	%\pagestyle{plain}
	%
	\begin{document}
	\title{\vskip -10pt}
	
	\author{Lydia Ould Ouali\inst{1}, Charles Rich\inst{2} \and
	Nicolas Sabouret\inst{1} }
	

	\begin{abstract}\vskip -20pt
	  
	\end{abstract}

\section{Preliminary introduction: what would the dialogue be without dominance}

\subsection{Data structure: preferences for self}
Let $P_{self}$ be the preferences of the agent, represented as set of directed acyclic graphs (DAG) for each criterion, computed from the list of binary preferences.

For each value $v_i$ attached to criteria $c_i$ in proposal $x=(v_0,v_1,\ldots,v_m)$, $value(v_i)\in[-(n_i-1),(n-1)]$ is the score attached to $v_i$ in the preference graph (number of successors minus number of predecessors), with $n=card(D_{c_i})$

E.g.: assume Dragon is a quiet and cheap Chinese restaurant. The utterance "Let us go the Dragon restaurant" corresponds to $propose(x)$ with $x=(quiet,cheap,chinese)$. "Let us go to a cheap restaurant" corresponds to $propose(x)$ with $x=(cheap)$.

\subsection{Data structure: representation of other}
Let $A_{other}$ be the list of acceptability values, computed from the previous interactions. Each element in $A_{Other}$ is a triple $<criterion,value,acceptability>$ where $value\in D_{criterion}$ and $acceptability\in\{true,false,unknown\}$. The acceptability value is $true$ if and only if the agent knows, from a previous $state$ utterance in the negotiation, that the corresponding value for the given criterion is acceptable. Similarly, it is false only if the agent knows that this value is not acceptable.

Similarly, let $A_{OAS}$ (other about self) represent the set of acceptability values that have been communicated \textbf{by the agent to its interlocutor} during the dialogue.

For simplification, we note $A_{model}(x)$ the acceptability value for $x$ in the referenced model $model$ (Other or OAS).

\subsection{Acceptability for self}

We can compute a direct value for the acceptability for self:

$$acceptability_{self}(x) = \left(\sum_{i[1,card(x)]} rank(c_i)\times score(v_i)\right)\geq0$$

Similarly to other and OAS, we denote $A_{self}$ the acceptability set for self.

\subsection{Basic decision}

The basic decision algorithm for a given utterance $u$ with value $x$, given the current mental state $(A_Self,A_Other,A_OAS...)$ will be computed from the previous utterance $u ^{-1}(x)$ as the following:

\begin{itemize}
	\item If $u ^{-1}=ask (x)$
	\begin{itemize}
		\item $state(x, A_{Self}(x))$
		\\
	\end{itemize}	
	\item If $u ^{-1}=state (x, v)$ with $v = A_{Other}(x) $:
	\begin{itemize}
		\item $stop$ \textbf{if} $isMaxUtterances$
		\item $Propose (x)$ \textbf{ if } $ v = true$ \textbf{ and }
		$ A_{Self}(x) = true$
		\item $state(x,A_{Self}(x))$ \textbf{ if } $A_{OAS}(x)= unkown$
		\item $state(y,A_{Self}(x))$ \textbf{ if } $A_{OAS}(y)= unkown$ \textbf{ and } $x, y \in D_c$
		\\
	\end{itemize}
	
	\item If $u ^{-1}=propose(x)$
	\begin{itemize}
		\item $accept(x)$ \textbf{ if } $A_{Self}(x) = true$
		\item $reject(x)$ \textbf{ if } $A_{Self}(x) = false$
		\\
	\end{itemize}
	
	\item If $u ^{-1}=accept(x)$
	\begin{itemize}
		\item $stop$ \textbf{ if } $x$ is an option.
		\item $propose(o)$ \textbf{ if } $ x$  is a value, such that $x \in value(o, C_x)$ \textbf{ and } $A(o)_{self} = true$
		\\	
	\end{itemize}
	
	\item If $u ^{-1}=reject(x)$
	\begin{itemize}
		\item $ask(y)$ such that $A_{Other}(y)=unkown$
		\item $propose(y)$ \textbf{ if } $A_{Self}(y) = true$
		\item $stop$ \textbf{if} $isMaxUtterances$
		\\	
	\end{itemize}
	
\end{itemize}

\subsection{Dominance}

However,in several scholars(De Dreu, Van Kleef, Bugroon), it has been proved that the interpersonal dominance affect interlocutors strategies of negotiation. We propose, thus to introduce the dominance as an additional variable in the agent reasoning. (See the next section for more details)


\section{Principle of dominance in negotiation}
\par Dominance as a property of interpersonal relationship is defined as the power to produce intended effects, and the ability to influence the behavior of other person in the conversation. (Burgoon et al.,1998).
Moreover, in the context of communication, dominance is a dyadic variable where one individual's attempt of control is necessarily acquainted by the partner in the interaction.(Rogers-Millar and Millar, 1979,Dunba and Burgoon, 2005). 

\par Such behaviors in a conversation can contribute either positively or negatively to the discussion. For example, positive contributions include actions such as keeping the conversation going, orient the task decision, by making quick decisions and conclusions etc. Negative contribution may include not considering the partner in the conversation, for example, not giving the occasion to express his opinion, not open to criticism. In addition, expressing verbally the dominance can be viewed as offensive and unjustified (K,Zablotskaya). Giving these contributions to the conversation, several researches get interested to detect  behaviors related to the dominance during the conversation. In our work, we focus essentially on the context of conversations of negotiation, where several researches already proved the impact of dominance on the negotiation(VAN KLEEF, 2005, De Dreu, 1995). 

\par Scholars of social psychology dedicated to the negotiation, already proved that the relation of dominance impact negotiators strategies in different ways. We distinguish these behaviors in two main categories or principles.

\hl{Preliminary note: we have changed the order: principle 1 is now "self vs other", because it is much simpler than principle 2}


\section{Principle 1: Self Vs Other}
In DeDreu works, submissive negotiators consider the well-being of other in the negotiation, whereas dominant negotiators are only interested in their own well-being. 

We propose implement this mechanism by balancing the satisfaction of self-preferences with the satisfaction of the other, during the decision making process.

\subsection{Implementation}
For a given utterance $u(x)$ that is a negotiation move ($propose(x)$, $reject(x)$ or $accept(x)$), we compute $score_s(u(x))\in[0,1]$ such that $score_s(u(x))=0$ when the proposed option of value $x$ is one of the least preferred choice and $1$ when it is one of the most preferred choice.

We then compute: 

$$score_s(u(x)) = \frac{1}{\sum_{i[1,card(x)]} i} \sum_{i[1,card(x)]} rank(c_i)\times \frac{1}{card(D_{c_i})-1} \frac{score(v_i)+card(D_{c_i})-1}{2} $$

or simply:

$$score_s(u(x)) = normalize \left( \sum_{i[1,card(x)]} rank(c_i)\times normalize ( score(v_i) ) \right) $$


Similarly, we compute:

$$score_o(u(x)) = \left\{\begin{array}{ll}
0 & \mathrm{if\ }A_{other}(x)= false\\
1 & \mathrm{otherwise}
\end{array}\right.$$

Then, we compute a decision value for the first principle as follows:

$$principle1(u(x),P_{self},A_{other},dominance) = w_{self} * score_o(u(x)) + w_{other} * score_s(u(x))$$

with $w_{self}$ and $w_{other}$ are values in $[0,1]$ defined after the dominance value such that:

$$dominance = \frac{w_{self}}{w_{self}+w_{other}}$$

Concretely, $w_{self}>>w_{other}$ when the agent is dominant, whereas $w_{self}\simeq w_{other}$ for a peer or submissive agent.


\hl{Lydia would prefer to work in -1,1 : the formulas would be easier and this does not change anything but it would be easier for the code}

\hl{Lydia thinks that, maybe, $\alpha$ and $\beta$ (see below) could also be considered for Principle 1}


\section{Principle 2}

In DeDreu works, powerful or dominator negotiators adopt different strategies comparing to less powerful negotiators. He demonstrates a panel of different behaviors in a negotiation as presented bellow:


\textbf{Representation of demands}
\begin{itemize}
	\item Dominant negotiators show a higher level of demand than the submissive ones. In addition, submissive negotiator's demand decrease over time. 
	\item Submissive negotiators give a lower level of demand to low priority issues.
\end{itemize} 



\textbf{Control the flow of the negotiation}
\begin{itemize}
	\item Powerful negotiators tend to make the first move. \cite{magee2007power}

	\item Based on Carsten, De Dreu and Van Kleef demonstrate that high power negotiators are high in their propensity to negotiate relative to participants with low power. (leading individuals to focus on the rewards available to them in situations and to bargain for	greater rewards than were initially offered to them.)

	\item Dominance affects the way that negotiators gather information about their partners. Negotiators with less power have a stronger desire to develop an accurate understanding of their negotiation partner, which would lead them to ask more \emph{diagnostic} rather than \emph{leading} questions.
	
\end{itemize} 

\subsection{Implementation}
We propose to implement values that reproduce agent level of demand with respect of its perception of dominance. Dominant agent is more demanding and restricted list of preferences whereas, submissive agent is has a larger set of preferences. Moreover, his level of demand decreases over time.

\subsubsection{Choosing the utterance type}
The decision tree proposed here is inspired from the current implementation. 

\begin{itemize}
	\item If $u ^{-1}=ask (x)$ 
	\begin{itemize}
		\item $utility(state) = 1$
		\item $utility(propose) = 0$
		\item $utility(accept) = 0$
		\item $utility(reject) = 0$		
		\item $utility(ask) = 0$
		\\
	\end{itemize}
	
	\item If $u ^{-1}=state(x^{-1})$
	\begin{itemize}
		\item\textbf{ask}
		\subitem $utility(ask) = 1$ \textbf{ if } $ isSub_{self}$ \textbf{ and } {$\forall x \in C_{x^{-1}} /$ $ A_{Other} (x) = unkown$ }
		\subitem $utility(ask) = 0$ \textbf{Otherwise}
		
		\item \textbf{state}
		\subitem $utility(state) = 1$ \textbf{ if } $ \exists y \in C_{x^{-1}} / A_{OAS} (x) = unkown$ \textbf{ and } $ (isSub_{self}$  \textbf{ or } $\neg maxStatements)$
		\subitem $utility(state) = 0$ \textbf{Otherwise}.
		
		\item \textbf{propose}
		\subitem $utility(propose) = 1$ \textbf{ if } $\neg isSub_{self}$ \textbf{ and } ($maxStatements$ \textbf{ or } $A_{Self} (x^{-1}) = true$
		\subitem $utility(propose) = 0$ \textbf{Otherwise}.
	
		\item $utility(accept) = 0$
		
		\item $utility(reject) = 0$
		\\
	\end{itemize}
	\item If $u ^{-1} =propose$ 
	\begin{itemize}
		\item \textbf{propose}
			\subitem $utility(propose) = 1$ \textbf{ if } $A_{Self} (x^{-1})= false$ \textbf{ and } $isDom_{self}$
			\subitem $utility(propose) = 0$  \textbf{Otherwise}.
		\item \textbf{ask}	
			\subitem $utility(ask) = 1$ \textbf{ if } $isSub_{Self}$ \textbf{ and } $isMaxProposed()$ (see the definition bellow)
			\subitem $utility(ask) = 0$ \textbf{Otherwise}.
		\item \textbf{reject}	
			\subitem $utility(reject) = 1$  \textbf{ if } $A_{Self} (x^{-1}) = false$ \textbf{ and } $\neg isDom_{Self}$.
			\subitem $utility(reject) = 0 $ \textbf{otherwise}.
		\item \textbf{accept}	
			\subitem $utility(accept) = 1$ \textbf{ if } $ \neg isDom_{Self}$ \textbf{ and }  $A_{Self} (x^{-1}) = true$ 
			\subitem $utility(accept) = 0 $ \textbf{Otherwise}
			
		\item \textbf{ state}
			\subitem $utility(state) = 0$
			\\
	\end{itemize}
	\item If $u^{-1}=accept$
	\begin{itemize}
	\item \textbf{propose}
		\subitem $utility(propose) = 1$ \textbf{ if } $ \neg isisSub_{Self}$ \textbf{ and } $x^{-1}$ is a value of criterion. 
		\subitem $utility(propose) = 0$  \textbf{ if } $isSub_{Self}$ \textbf{ and } $OpenNewCriterion \not= null$.
	\item \textbf{ask}	
		\subitem $utility(ask) = 1$ \textbf{ if } $isSub_{Self}$
		\subitem $utility(ask) = 0$ \textbf{Otherwise}.
	\item \textbf{reject}	
		\subitem $utility(reject) = 0 $ 
	\item \textbf{accept}	
		\subitem $utility(accept) = 0 $
	
	\item \textbf{ state}
		\subitem $utility(state) = 1$ \textbf{ if } $isPeer_{Self}$
		\subitem $utility(state) = 0$ \textbf{Otherwise}.
	\\		
	\end{itemize}
	\item If $u^{-1}=reject$
	\begin{itemize}
		\item \textbf{ask}	
			\subitem $utility(ask) =1 $ \textbf{ if } $\neg isDom_{Self}$ \textbf{ and } $x^{-1}$ is an option.
			\subitem $utility(ask) = 0$ \textbf{Otherwise}.
		\item \textbf{state}
			\subitem $utility(state) = 1$ \textbf{ if } $isPeer_{Self}$
			\subitem $utility(state) = 0$ \textbf{ Otherwise}
		\item \textbf{propose}
			\subitem $utility(propose) = 1$ \textbf{ if } $isDom_{Self}$
			\subitem $utility(propose) = 0$ \textbf{Otherwise}
		\item \textbf{accept}
			\subitem $utility(accept) = 0$
		\item \textbf{reject}	
			\subitem $utility(reject) =0 $	
	\end{itemize}
	
\end{itemize}

- $isMaxProposed$: return true, if the last three utterances are negotiation moves.
- $OpenNewCriterion$: opens a new topic of negotiation, which is a new criterion to choose an option.


\subsubsection{Choosing the utterance values}
The choice of the utterance value, depends both on the current state of negotiation (the mental state) and the representation of preferences. We present first how the preferences on values are computed during the negotiation with respect of principles extracted for psychology scholars. 
Second, we present how the values of utterances are affected with respect of the current state of the negotiation.
\subsubsection{Implementation: compute a value of preferences}
\begin{enumerate}
	\item Dominant negotiators show a higher level of demands than the submissive ones.
	
	We propose to define a parameter $\alpha$ to compute the level of acceptability.
	
	$\alpha = \{ 1/4$ if dominant, $1/2$ if peer or submissive$\}$ 
	
	\item In addition, submissive negotiator's demand decrease over time. 
	
	$ \alpha$ increases over the negotiation if submissive, which means that the range of acceptable values increase over time.
	
	\item Submissive negotiators give a lower level of demand to low priority issues.
	
	We propose to define a parameter $\beta$ to ignore low ranked criteria 
	
	$\beta = n/2$ if submissive, $0$ otherwise
\end{enumerate}

Concretely, let $o$ be an option of the form $(v_1,\ldots,v_n)$ for criteria $(c_1,\ldots,c_n)$. We compute:
\begin{itemize}
	\item $value(o) = \sum_i rank_\beta(c_i)\times score(v_i)$
	
	with $rank_\beta(c_i)\in[1,n]$ as follows:
	
	let $rank(c_i)\in[1,n]$ the position of $c_i$ in the ascending list of $c_i$ (1 for the least important criteria, $n$ for the most important), based on the list of preferences. Then, $rank_\beta(c_i)\in[0,n] = 0$ if $rank(c_i)\leq \beta$, $rank(c_i)$ otherwise.
	
	\item $acceptable(o) = true$ iff $o$ is in the $\alpha$ upper part of the list of options sorted by descending $value(o)$.
	\item For the change of $\alpha$ over time, we propose a simple linear function: after $nb\_min$ proposals, the agent will increase $\alpha$ every $nb\_new$ proposals, in a manner that it reaches $1$ after $nb\_criteria$ increments, where $nb\_criteria = |C|$ is the number possible criteria in the negotiation.
	
	This leads to the following formula, with $n$ the current number of proposals in the dialogue:
	$$\alpha(n) = \alpha + (1-\alpha)\times\frac{max(0,\frac{n-nb\_min}{nb\_new})}{nb\_criteria}$$
\end{itemize}

\par For example, consider, four restaurants with the following values:
\begin{itemize}
	\item	Dragon = Quiet, cheap, Chinese restaurant.
	\item	CASA = Quiet, cheap, Italian restaurant.
	\item	Salento = Lively, expensive, Italian restaurant.
	\item   Jiliya = Lively, cheap Chinese restaurant.
\end{itemize}

The preference model contains the list of ranked preferences $C = \{Cost, Ambiance, Cuisine\}$. Each criterion is defined with preferences for its domain values: $ Cost = \{P(Cheap, Expensive)\}$ (which means that Cheap is less preferred than Expensive), $Ambiance= \{P(Lively, Quiet)\}$, and $Cuisine= \{P(Chinese, Italian)\}$.

In order to calculate the acceptability of the restaurants above, we use the function $acceptable(o)$, for both the relation of dominant and submissive. We first compute the value of each option:
\begin{itemize}
	\item \emph{Dominant:}	Value(Dragon) = 2 $\times$ 1 + 1 $\times$ -1 + 3 $\times$-1 = -2
	\\ \emph{Submissive:} Value(Dragon) = 2 $\times$ 1 + 0 + 3 $\times$-1 = -1
	
	\item \emph{Dominant:}	Value(Casa) = 2 $\times$ 1 + 1 $\times$ -1 + 3 $\times$1 = 4
	\\ \emph{Submissive:} Value(Dragon) = 2 $\times$ 1 + 0 + 3 $\times$ 1 = 5
	
	 \emph{Dominant:}	Value(Salento) = 2 $\times$ -1 + 1 $\times$ 1 + 3 $\times$ 1 = 2
	 \\ \emph{Submissive:} Value(Salento) = 2 $\times$ -1 + 0 + 3 $\times$ 1 = 1
	 
	\item \emph{Dominant:}	Value(Jiliya) = 2 $\times$ -1 + 1 $\times$ -1 + 3 $\times$-1 = -6
	\\ \emph{Submissive:} Value(Jiliya) = 2 $\times$ -1 + 0 + 3 $\times$-1 = -5
\end{itemize}
We obtain the ordered list of preferences on options (i.e the same for both relations dominant vs submissive) $option$ = {Casa, Salento, Dragon, Jiliya}.

Therefore, with  the function $acceptable(o)$, we can compute the acceptable restaurants from the list $options$. $Acceptables_{Restaurant}(Dominant) = \{$Casa$\}.$ If dominant, only Casa restaurant is an acceptable restaurant. While, if submissive $Acceptables_{Restaurant}(submissive) = \{$Casa, Salento$\}.$ With this toy example, wa can show that dominant negotiators in our model are more demanding than submissive one.


Now, for criteria values only, we define $acceptable(v)$ as: $true$ if $rank_\beta(c)=0$, $true$ if $v$ is in the $\alpha$ upper part of the list of criterion values sorted by descending $score(v)$, $false$ otherwise.

\par For example, suppose the following list of preferences $ Cost = \{$P(Cheap, Expensive), P(Affordable, Expensive )$\}$.
We obtain the sorted list of scores for each value of cuisine   $Cost{Scores}= \{Expensive=2; Affordable= -1; Cheap= -1;\}$.
We can compute the acceptable values for the criteria cuisine with the function $acceptable(v)$ presented above. 
For the submissive agent, because the criterion $Cost$ isn't important, all its values are acceptable: $Acceptables_{Cost}(submissive) = \{$Expensive, Affordable, Cheap$\}.$. However, for the dominant agent, the list of acceptable values is as follow: $Acceptables_{Cost}(submissive) = \{$Expensive$\}.$.

\subsubsection{Implementation: compute a value from current state}
let $u(x)$ be the current utterance with $u$ is a type of utterance and $x$ the stated value of $u$.
The value of each type of utterance is computed as follow: 
\begin{itemize}
	\item $u= propose$
	\begin{itemize}
		\item $utility (x)=1$ \textbf{ if } $A_{Self}(x)= true$ \textbf{ and } $ x \notin {Accepted, Rejected}$
		
		\item $utility (x)=0$ \textbf{Otherwise}
	\end{itemize}
	
	\item $ u= accept$
	\begin{itemize}
		\item $utility (x)=1$ \textbf{ if } $A_{Self}(x)= true$ 
		\item $utility (x)=0$ \textbf{Otherwise}	
	\end{itemize}
	
	\item $ u= reject$
	\begin{itemize}
		\item $utility (x)=1$ \textbf{ if } $A_{Self}(x)= false$ 
		\item $utility (x)=0$ \textbf{Otherwise}	
	\end{itemize}
	
	\item $ u= ask$
	\begin{itemize}
		\item $utility (x)=1$ \textbf{ if } $A_{Other}(x)= Unkown$ 
		\item $utility (x)=0$ \textbf{Otherwise}	
	\end{itemize}
	
	\item $ u= state$
	\\ This case is more complex due to the different values that can be stated, and the coherence in the dialogue. 
	\begin{itemize}
		\item The generic case: $utility (x)=1$ \textbf{ if } $A_{OAS}(x)= unkown$ 
		\item $utility (x)=0$ \textbf{Otherwise}	
		\item However, depending on the previous utterance that branches to the utterance type $state$ the algorithm choose different values:
			\begin{itemize}
				\item $u^{-1} =$\textbf{state(x)}
					\subitem $utility (x)=1$ \textbf{ if } $A_{OAS}(x)= unkown$  \textbf{ or } $openNewCriterion()$
					\subitem $utility (x)=0$ \textbf{Otherwise}	
				\item $u^{-1} =$\textbf{ask(x)}
					\subitem $utility (x)=1$ 
				\item $u^{-1} =$\textbf{accept(x)}
					\subitem $utility (y)=1$  \textbf{ if }	$A_{Self} (y) = true$ \textbf{ and } $A_{OAS}(y)= unkown$  \textbf{and}  $x, y \notin D_cx.$ which means that $y \in D_{OpenNewCriterion()}.$
					\subitem 	$utility (x)=0$ \textbf{Otherwise}
					
				\item $u^{-1} =$\textbf{reject(x)}
				\subitem $utility (y)=1$  \textbf{ if }	$x$ is an option , \textbf{ and } $ y \in D_{OpenNewCriterion()}$ \textbf{ and } $A_{Self} (y) = true$ \textbf{ and } $A_{OAS}(y)= unkown$
				\subitem 	$utility (x)=0$ \textbf{Otherwise}		
					
			\end{itemize}

	\end{itemize}
	
\end{itemize}
Thus:

$$principle2(u(x),dominance,mental\_state) = utility(TypeOf(u(x))) + utility(instance(u(x)))$$
\section{Decision}

Finally, the agent decides for the utterance based on the relative weight of principle 1 and principle 2. For each utterance type $u$ and for each possible utterance value $x$, we can compute the utility of the utterance:

$utility(u(x)) = w_1 * principle1(u(x),P_{self},A_{other},dominance)$

             $ + w_2 * principle2(u(x),dominance,mental\_state)$

such that $w_2 > w_1$: in the context of cooperative negotiation, we should give a higher level to the success of negotiation. \hl{ to confirm ?}


% BIBLIO %

\vskip 4pt
\bibliographystyle{plain}
\bibliography{abbrevs,Library}
\end{document}
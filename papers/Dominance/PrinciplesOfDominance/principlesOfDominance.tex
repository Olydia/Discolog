\documentclass{article}
	\usepackage[noend]{algpseudocode}
	\usepackage{subcaption}
	\usepackage{subfig} 
	\usepackage{usual}
	\usepackage{amsmath}
	\usepackage{graphicx}
	\usepackage{eulervm}
	\usepackage{fontenc}
	\usepackage{mathrsfs}
	\usepackage{multirow}
	\usepackage{array}
	\usepackage[rflt]{floatflt}
	\usepackage{makecell}

	\usepackage{xcolor, soul}
	\sethlcolor{yellow}	

	\renewcommand\theadalign{cb}
	\renewcommand\theadfont{\bfseries}
	\renewcommand\theadgape{\Gape[4pt]}
	\renewcommand\cellgape{\Gape[4pt]}
	%\pagestyle{plain}
	%
	\begin{document}
	\title{\vskip -10pt}
	
	\author{Lydia Ould Ouali\inst{1}, Charles Rich\inst{2} \and
	Nicolas Sabouret\inst{1} }
	

	\begin{abstract}\vskip -20pt
	  
	\end{abstract}


\subsection{Dominance}

However,in several scholars(De Dreu, Van Kleef, Bugroon), it has been proved that the interpersonal dominance affect interlocutors strategies of negotiation. We propose, thus to introduce the dominance as an additional variable in the agent reasoning. (See the next section for more details)


\section{Principles of dominance in negotiation}
\par Dominance as a property of interpersonal relationship is defined as the power to produce intended effects, and the ability to influence the behavior of other person in the conversation. (Burgoon et al.,1998).
Moreover, in the context of communication, dominance is a dyadic variable where one individual's attempt of control is necessarily acquainted by the partner in the interaction.(Rogers-Millar and Millar, 1979,Dunba and Burgoon, 2005). 

\par Such behaviors in a conversation can contribute either positively or negatively to the discussion. For example, positive contributions include actions such as keeping the conversation going, orient the task decision, by making quick decisions and conclusions etc. Negative contribution may include not considering the partner in the conversation, for example, not giving the occasion to express his opinion, not open to criticism. In addition, expressing verbally the dominance can be viewed as offensive and unjustified (K,Zablotskaya). Giving these contributions to the conversation, several researches get interested to detect  behaviors related to the dominance during the conversation. In our work, we focus essentially on the context of conversations of negotiation, where several researches already proved the impact of dominance on the negotiation(VAN KLEEF, 2005, De Dreu, 1995). 

\par Scholars of social psychology dedicated to the negotiation, already proved that the relation of dominance impact negotiators strategies in different ways. We distinguish these behaviors in two main categories or principles.

\hl{Preliminary note: we have changed the order: principle 1 is now "self vs other", because it is much simpler than principle 2}


\section{Principle 1: Self Vs Other}
In DeDreu works, submissive negotiators consider the well-being of other in the negotiation, whereas dominant negotiators are only interested in their own well-being. 

We propose implement this mechanism by balancing the satisfaction of self-preferences with the satisfaction of the other, during the decision making process.



\section{Principle 2}

In DeDreu works, powerful or dominator negotiators adopt different strategies comparing to less powerful negotiators. He demonstrates a panel of different behaviors in a negotiation as presented bellow:


\textbf{Representation of demands}
\begin{itemize}
	\item Dominant negotiators show a higher level of demand than the submissive ones. In addition, submissive negotiator's demand decrease over time. 
	\item Submissive negotiators give a lower level of demand to low priority issues.
\end{itemize} 



\textbf{Control the flow of the negotiation}
\begin{itemize}
	\item Powerful negotiators tend to make the first move. \cite{magee2007power}

	\item Based on Carsten, De Dreu and Van Kleef demonstrate that high power negotiators are high in their propensity to negotiate relative to participants with low power. (leading individuals to focus on the rewards available to them in situations and to bargain for	greater rewards than were initially offered to them.)

	\item Dominance affects the way that negotiators gather information about their partners. Negotiators with less power have a stronger desire to develop an accurate understanding of their negotiation partner, which would lead them to ask more \emph{diagnostic} rather than \emph{leading} questions.
	
\end{itemize} 



% BIBLIO %

\vskip 4pt
\bibliographystyle{plain}
\bibliography{abbrevs,Library}
\end{document}
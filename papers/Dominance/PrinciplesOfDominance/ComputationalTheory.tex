	\documentclass{article}
		\usepackage[noend]{algpseudocode}
		\usepackage{subcaption}
		\usepackage{subfig} 
		\usepackage{usual}
		\usepackage{amsmath}
		\usepackage{graphicx}
		\usepackage{eulervm}
		\usepackage{fontenc}
		\usepackage{mathrsfs}
		\usepackage{multirow}
		\usepackage{array}
		\usepackage[rflt]{floatflt}
		\usepackage{makecell}
		\usepackage{breqn}
		\usepackage{xcolor, soul}
		\sethlcolor{yellow}	
		\usepackage{pbox}
	
		\renewcommand\theadalign{cb}
		\renewcommand\theadfont{\bfseries}
		\renewcommand\theadgape{\Gape[4pt]}
		\renewcommand\cellgape{\Gape[4pt]}
		%\pagestyle{plain}
		%
		\begin{document}
		\title{\vskip -10pt}
		
		\author{Lydia Ould Ouali\inst{1}, Charles Rich\inst{2} \and
		Nicolas Sabouret\inst{1} }
		
	
	\section{Domain model}
	Negotiation is based on choosing the best value from a set of options which concerns the topic of negotiation. We note $\mathcal{O}$ the set of possible options to negotiate about. $\forall o \in \mathcal{O}$ is defined with \textbf{criteria} that reflect options characteristics.
	Let $\mathcal{C}$ be the set of criteria. For example, criteria for choosing a restaurant could be cuisine, ambiance, price, and location, while the criteria for choosing a movie could be type and location. 
	
	Furthermore, each criterion has to be measurable, in the sense that it must be possible to rate an option even in qualitative way. Therefore, $\forall$ \emph{c $\in\mathcal{ C}$},  we note $\mathcal{D}_c$ its  domain of values. For example, the domain  values of the criterion cuisine is noted $\emph{D}_{cuisine} = \{chineese, italian\}$.
	
	Therefore, we note an option $o\in \mathcal{O}$:
	$o = \{c_1=v_1,..., c_n=v_n\}$ with $c_i \in \mathcal{C}, \forall i \in [1,n]$ and $v_i\in \emph{D}_{c_i} $. 
	
	\subsection{Preferences for self}
	$\forall c \in \mathcal{C}$ , an agent has a preference set $P_{self}(c)$, represented as set of directed acyclic graphs (DAG), computed from the list of binary,transitive and antisymmetric preferences on the values of $\mathcal{D}_c$.
	
	$\forall c \in \mathcal{C}$ and for each tuple$ (u,v) \in \mathcal{D}_c$ we define, for the purpose of notation, $(u,v) \in P_{self}$, the relation of preference $u<v$ in  $P_{self}(c)$. 
	
	
	We define a proposal as non-empty set of possible values  $x=(v_0,v_1,\ldots,v_m)$, attributed to the different criteria  $(c_0,c_1,\ldots,c_m) \in \mathcal{C}$.
	For example, we note a proposal of an option $Ginza = Chinese, lively, cheap$ such that $Chinese \in Cuisine$, $lively \in Ambiance$ and $cheap \in Cost$. In the same idea, we note a proposal of a criterion as a singleton $Chinese$.
	
	
	$\forall v_i \in c$, we note $Satisfaction_{self}(v_i)\in[0, 1]$ the score of the personal satisfaction of preferences attached to $v_i$ 
	
	 $$Satisfaction_{self}(v_i) = 1 - \left(\frac{|v_j / i \not= j, (v_i,v_j) \in P_{self}|}{|\mathcal{D}_c|-1}\right)$$

	
	For each proposal $o=(v_0,v_1,\ldots,v_m)$, we note, $Satisfaction_{self}(o)$ as the personal satisfaction attributed to $o$:  
	$$Satisfaction_{self}(o) = \left(\sum_{i[1,card(x)]} Satisfaction_{self}(c_i)\times  Satisfaction_{self}(v_i)\right)$$
	
		
	
	\subsection{Representation of other}
	Let $P_{other}(c)$ be the list of other preferences for each criterion $c \in \mathcal{C}$, computed from the previous interactions. Each element in $P_{Other}(c)$ is a triple $<criterion,value,satisfaction>$ where $value \in \mathcal{D}_{criterion}$ and $satisfaction\in\{true,false,unknown\}$. The satisfaction value is $true$ if and only if the agent knows, from a previous $state$ utterances in the negotiation, that the corresponding value for the given criterion satisfies other preferences. Similarly, it is false only if the agent knows that this value is not satisfiable.
	
	Similarly, let $P_{OAS}$ (other about self) represent the set of values that have been communicated \textbf{by the agent to its interlocutor} during the dialogue.
	
	
	We compute the $Satisfaction(x)$ in model of preferences $Other, OAS$ from preceding knowledge: 
	
	We note the model of preferences $ model \in \{Other, OAS\}$  such that,
		$$Satisfaction_{model}(x) = \left\{\begin{array}{ll}
			0 & \mathrm{if\ }P_{model}(x)= false\\
			1 & \mathrm{otherwise}
			\end{array}\right.$$
	
	
	\section{Principles of dominance in negotiation}
	%	\par Dominance as a property of interpersonal relationship is defined as the \textbf{power} to produce intended effects, and the ability to influence the behavior of other person in the conversation. (Burgoon et al.,1998).
	%	Moreover, in the context of communication, dominance is a dyadic variable where one individual's attempt of control is necessarily acquainted by the partner in the interaction.(Rogers-Millar and Millar, 1979,Dunba and Burgoon, 2005). 
	%	
	%	\par Such behaviors in a conversation can contribute either positively or negatively to the discussion. For example, positive contributions include actions such as keeping the conversation going, orient the task decision, by making quick decisions and conclusions etc. Negative contribution may include not considering the partner in the conversation, for example, not giving the occasion to express his opinion, not open to criticism. In addition, expressing verbally the dominance can be viewed as offensive and unjustified (K,Zablotskaya). Giving these contributions to the conversation, several researches get interested to detect  behaviors related to the dominance during the conversation. In our work, we focus essentially on the context of conversations of negotiation, where several researches already proved the impact of dominance on the negotiation(VAN KLEEF, 2005, De Dreu, 1995). 
	%	
	%	\par Scholars of social psychology dedicated to the negotiation, already proved that the relation of dominance impact negotiators strategies in different ways. We distinguish these behaviors in two main categories or principles.
	
	%\hl{Preliminary note: we have changed the order: principle 1 is now "self vs other", because it is much simpler than principle 2}
	
	
	\subsection{Principle 1: Satisfaction of Self preferences Vs Other preferences}
	Scholars in psychology already demonstrate that the relation of dominance affect negotiator's motivation in maximizing self preferences versus other preferences. We took an interest in DeDreu work in which he studies the impact of social value orientation \textit{(i.e  individual differences in	how people evaluate outcomes for themselves and others	in interdependent situation)} on negotiator behavior. We keep three main mechanisms:
	\begin{enumerate}
		\item Submissive negotiators consider the well-being of other in the negotiation, whereas dominant negotiators are only interested in their own well-being.
		\item Dominant negotiators show a greater levels of demand than the submissive one. Moreover, submissive negotiators show a \textbf{greater decline} of their preferences after the second round of negotiation. 
		\item Submissive negotiators conceded more on low priority issues.
	\end{enumerate}   
	
	
	\subsection{Implementation}
	In order to implement behaviors related the the relation of dominance. We first represent this relation of dominance $dom$ such that $dom \in [0, 1]$. For example, $dom$ can be initialized 
	to $dom =0.8$ for a dominant agent, $dom =0.5$ for a peer agent and $dom =0.2$ for a submissive agent.
	The next sections present the implementation for each mechanism.
	
	\subsubsection{Considering well being of other}
	
	For the first mechanism, we propose to implement a function capable to balance the satisfaction of self-preferences with the satisfaction of the other, during the decision making process.
	We define $w_{Self} \in [0, 1]$ as the weight the agent assign to the satisfaction its preferences. Thus, the relation of dominance can be initialized $dom = w_{Self}$.
	
	Therefore, for a given value $x$,  we compute a decision value as follows:
	\begin{dmath}
	$$Satisfiabiliy(x,P_{Self},P_{Other}) = w_{Self} * Satisfaction_{Self}(x) + (1- w_{Self}) * Satisfaction_{Other}(x))$$
	\end{dmath}
	
	\subsubsection{Level of demand} 
	\label{demand}
	When the negotiation is not converging, the agent should make concessions in his decision making, which can be interpreted as reducing the weight of its self satisfaction. We propose thus, to update $ w_{Self}$ to reduce over negotiation rounds. 
	
	
		$$w_{Self} (t) = \left\{\begin{array}{ll}
		max(Threshold,dom) & \mathrm{if\ }(t<=n)\\
		max(0, \frac{-k}{dom} (t-n)+ max(Threshold,dom)) & \mathrm{otherwise}
		\end{array}\right.$$ With: 
		\begin{itemize}
			\item $t$ is the number of proposals which have not been accepted (negotiation moves except Accept).
			\item $n$ is the minimum number of negotiation moves to start make concessions.
			\item $dom = w_{Self} (0)$ initialized to  
					$dom  \left\{\begin{array}{ll}
					> Threshold& \mathrm{if\ } dominant \\
					= Threshold & \mathrm{if\ } peer \\
					< Threshold & \mathrm{if\ } submissive 
					\end{array}\right.$
			\item $Threshold$ is a value to define that cut the spectrum of dominance in two (more dominant if $ dom > Threshold$, or submissive otherwise). For example, $Threshold = 0.5$.
		 \end{itemize}
	
	
	
	\subsubsection{Concessions }
	\hl{The proposed implementation is not validated yet. Start to wonder on the importance to model this mechanism}.
	
	The agent should present greater concessions on the values whose criterion type is not considered as important (or with high issue).
	
	$$Satisfaction^{'}_{Self} (x,dom) = \left\{\begin{array}{ll}
	1 & \mathrm{if\ }(dom< Threshold) , Satisfaction(c_x) < th.\\
	Satisfaction_{Self}(x) & \mathrm{otherwise}.
	\end{array}\right.$$
	
	We obtain the following function that capture the different mechanisms of principle 1. 
	\begin{dmath}
	$$Satisfiabiliy(x,P_{Self},P_{Other}) = w_{Self}(t) * Satisfaction^{'}_{Self}(x) + (1 - w_{Self}(t)) * Satisfaction^{'}_{Other}(x))$$
	\end{dmath}



	\section{Principle 2: Lead of the negotiation}
%	
	
	\begin{itemize}
		\item Powerful negotiators tend to make the first move. %\cite{magee2007power}
	
		\item Carsten, De Dreu and Van Kleef demonstrate that high power negotiators are high in their propensity to negotiate relative to participants with low power. (leading individuals to focus on the rewards available to them in situations and to bargain for	greater rewards than were initially offered to them.)
	
		\item Dominance affects the way that negotiators gather information about their partners. Negotiators with less power have a stronger desire to develop an accurate understanding of their negotiation partner, which would lead them to ask more \emph{diagnostic} rather than \emph{leading} questions.
		
	\end{itemize} 
	
	\subsection{Implementation}
	In order to choose an utterance, it has to satisfy one of the set of applicability conditions:
	We define first, a possible branching independent from the relation of dominance, which is agent states a preference if other asks him about. we name this condition $OtherAsks$.
	
	\begin{itemize}
		\item Let be $ACCEPTED$ the list of proposals which have been accepted in the negotiation. (After an accept utterance).
		\item Let be $REJECTED$ the list of proposals which have been rejected in the negotiation.(After a reject utterance).
		\item Let be $OPEN$ the list of proposals which have been proposed in the negotiation.(after a propose utterance).
		\item We note $Threshold$ a value that cut the spectrum of dominance in two. if $dominance >= Threshold$ means that agent is dominant. otherwise the agent is submissive.
		\item $satisfiable$ is a value that represent the minimum score that a value has to get to be positively satisfiable to the agent preferences in the negotiation.
		\item $t$ is the number of proposals which have not been accepted (negotiation moves except Accept).
		\item $n$ is the minimum number of negotiation moves to start make concessions.
	\end{itemize}
	
	\par In the following , we present all the possible responses that an agent can choose in function of its relation of dominance. These responses are sorted from the most restrictive one the least restrictive (the most likely to be chosen).
	\subsubsection{Relation (\textbf{dominance $>=$ Threshold})}
	\begin{tabular}{|p{3cm}|p{9cm}|}
		\hline
		\textbf{Utterance type} & Condition \\
		\hline
		 Negotiation success &  $\exists x/ x \in \mathcal{O}$  \emph{and} $x \in \{ACCEPTED\}$  \newline \emph{OR} \newline $x \in \{OPEN\}$ and \newline $Satisfiability(x, P_{Self}, P_{Other}) >= satisfiable$ \\
		\hline
		Negotiation failure & $ \forall x \notin \{REJECTED\}$, \newline  $Satisfiability(x, P_{Self}, P_{Other}) < satisfiable $ \\
		\hline
		State & $OtherAsks$ \\
		\hline
		Propose & $\exists y \notin \{REJECTED\}$ such that \newline $Satisfiability(y, P_{Self}, P_{Other}) >= satisfiable $  
	%	\emph{and}  \newline $[ $ 
%		 \hl{Not sure to keep the conditions bellow, because theses conditions just condition the value to propose. The most important is that it remains satisfiable values to propose}
%		\newline ($\exists x \in \{OPEN\} /$ 
%		 $ Satisfiability(x, P_{Self}, P_{Other}) < satisfiable$) 
%		\newline \emph{OR}
%		 \newline( $ \exists x \in \{REJECTED\} /$ 
%		\newline$ Satisfiability(x, P_{Self}, P_{Other}) >= satisfiable)$ 
%		\newline \emph{OR} 
%		\newline  ($\exists x \in \{ACCEPTED\} $) \newline$]$ 
\\
		
	\hline
	\end{tabular}
	
		\par Note that the agent focus only on negotiation move of proposing, which allows him to lead the negotiation. He doesn't ask the other about his preferences, because he does not give an important weight to them in its decision.
		
		In addition, the value to be chosen for the propose utterance, has to be computed from a set of values $X$ to determine. $X$ should be computed from the current state of the nevgotation , which means that it should includes values of the discussed criterion, exclude accepted and rejected values etc...
	
	\subsubsection{Relation (\textbf{dominance $<=$ Threshold})}
	\begin{tabular}{|p{3cm}|p{9cm}|}
		\hline
		\textbf{Utterance type} & Condition \\
		\hline
		Negotiation success &  $\exists x/ x$ is an option \emph{and} $x \in \{ACCEPTED\}$ \\
		\hline
		Accept & $\exists x \in \{OPEN\} /$ \newline $Satisfiability(x, P_{Self}, P_{Other}) >= satisfiable$ \\
		\hline
		Reject & $\exists x \in \{OPEN\} /$ \newline $ Satisfiability(x, P_{Self}, P_{Other}) < satisfiable$  \emph{and} $t<n$.\\
		\hline
		Propose & $\exists x$ / $P_{Other} (x)= true $  \emph{and}
		\newline ($Satisfiability(x, P_{Self}, P_{Other}) >= satisfiable$
		\newline \emph{OR}  
		\newline $\forall c \in \mathcal{C}$, $\exists v_c \in \mathcal{D}_c / v_c \in \{ACCEPTED\}$)\\
		\hline
		Ask &  \textbf{(}$t>n,$ \emph{and} 
		\newline $\exists x \in \{OPEN\} /$
		\newline $ Satisfiability(x, P_{Self}, P_{Other}) < satisfiable $\textbf{) }
		\newline \emph{OR}
		\newline \textbf{(}$ \exists c \in \mathcal{C},  \forall v_c \in \mathcal{D}_c, P_{Other}(v_c) = Unkown$\textbf{)}
		\newline \emph{OR} 
		\newline \textbf{(}$\exists x \in \{ACCEPTED, REJECTED\}$ / 
		\newline $Satisfiability(x, P_{Self}, P_{Other}) >= satisfiable$ \textbf{)} \\
		\hline
		
		State & $OtherAsks$
		\newline \emph{OR}
		\newline $\exists x,P_{Other}(x) \not = Unkown $ and $P_{OAS}(x) = Unkown$ 
		\newline \emph{OR}
		\newline $ \exists c \in \mathcal{C}, \exists v_c \in \mathcal{D}_c, P_{Other}(v_c) = Unkown$
		\\
		\hline
	\end{tabular}
	
	We can observe that the choice of utterance $ask$ has different conditions. For example, the first sub-condition represent the fact that a submissive agent won't reject several times in the negotiation. When he observes that the negotiation is not converging, he will prefer gather more information about other preferences. The third sub condition shows that if a proposal has been either accepted or rejected by the other, the submissive agent keep the negotiation going by asking him more about other preferences that the agent still ignores. 
	\subsection{Choosing a value for an utterance}
	Let $X$ be a set of possible values to choose, in a current state of a negotiation. We define a function that computes the value that maximizes agent satisfiability.  
	\begin{dmath}
		$$ ChooseValue(X,P_{Self},P_{Other}) = \operatorname*{arg\,max}_{x \in X} Satisfiabiliy(x,P_{Self},P_{Other})$$ 
	\end{dmath}
	
%	% BIBLIO %
%	
%	\vskip 4pt
%	\bibliographystyle{plain}
%	\bibliography{abbrevs,Library}
	\end{document}
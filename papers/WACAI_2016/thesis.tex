
\documentclass [french]{article}
\usepackage[francais]{babel}
\usepackage[utf8]{inputenc}




\begin{document}
	
	\section{Introduction}
		\begin{enumerate}
			\item Intérêt des agents relationnels 
			\subitem Prise en compte des relations sociales
			\item Présentation de la problématique : Étudier l'impact d'une relation interpersonnelle sur les stratégies de négociation.
				\subitem 
			\item Organisation du document
		\end{enumerate}
		
		
	\section{État de l'art}
	Dans cette partie, nous dressons un état de l'art des travaux qui s'intéressent à la dominance dans les interactions tant qu'en psychologie sociale qu'en informatique affective 
		\subsection{Dominance en psychologie sociale}
			La dominance a été largement étudié dans le domaine de la psychologie sociale. Nous retrouvons des travaux dans différentes veines du domaines qui étudient l'influence de la dominance sur les relations humaines, tant comme trait de personnalité, rôle social ou encore relation interpersonnelle que construisent les individus. Ces comportements se manifestent dans l'interaction avec des comportements verbaux, non verbaux ou encore sur un niveau stratégique.		
			Nous présentons dans cette section différentes études sur la dominance.
			
			\subsubsection{Relation interpersonnelle}
			\subsubsection{Trait de personnalité}
			\subsubsection{Statue social}
			\subsubsection{Attitude sociale} Travaux de Florian Pecune?
			\subsubsection{Conclusion} 
				Conclure sur les différents aspects de la dominance dans les interactions sociales.
				Définir le contexte qui nous intéresse à savoir la relation interpersonnelle de dominance dans le contexte de négociation collaborative présentée dans la prochaine section. 
		\subsection{Relation interpersonnelle de dominance et négociation}
			Il a été largement prouvé que la relation de dominance est la 1ère relation à s'établir dans le contexte de négociation. De  plus, elle  affecte directement les stratégies de négociation exprimés par les négociateurs. Nous présentons les comportements dans la négociation influencés par la dominance.
			
			\subsubsection{Comportements de pouvoir}
			
			\subsubsection{Complémentarité de dominance} 
				
			\subsubsection{Conclusion} Positionner mes travaux de:
			
			 1) Étudier la dominance dans l'interaction comme relation interpersonnelle. 
			
			2) Nous nous intéressons plus précisément à l'impact de la dominance dans les dialogues la négociation collaborative.
				 
				 \textbf{A quel moment définir la négociation collaborative... ?} 
				
				
		\subsection{Agents conversationnels sociaux}
			Des travaux en informatique affective se sont intéressées aux différents comportements de dominance dans la modélisation d'agent conversationnels.
			(Présenter les différents travaux en IA qui ont modélisé des comportements liés à la dominance).
			
			\subsubsection{Modélisation de l'autre: Théorie de l'esprit} Modéliser l'état mental de l'autre afin de comprendre son raisonnement. Cela permet d'une part d'adapter le comportement de l'agent afin de simuler une relation interpersonnelle. D'une autre part , cela permet d'anticiper les prochains comportement de l'autre. 
		
		\subsubsection{Conclusion}
		
			Notre but est de modéliser les comportements de pouvoir qui traduisent une tendance de dominance. De plus, nous souhaitons simuler une relation interpersonnelle de dominance avec l'utilisateur. A cette fin, nous proposons un modèle de la théorie de l'esprit basé sur la simulation  pour deux raisons. En premier lieu, comprendre les comportements de pouvoir exprimés par l'utilisateur. Deuxièmement, adapter le comportement de l'agent aux comportements de pouvoir exprimés par l'utilisateur afin de créer une relation de dominance dite complémentaire. 	
			
	
	\section{Contributions}
		
		\subsection{Modèle de négociation collaborative}
			\subsubsection{Domaine de négociation }
				 Critère , Option, Préférences, Satisfiabilité
				Contexte de négociation: préférences de l'autre, historique de la négociation (proposition, préférences exprimés)
			
			\subsubsection{Communication}
			 Présentation des actes de dialogues avec leurs catégories et conditions d'applicabilité. 


		\subsection{Décision basée sur le pouvoir}
			\subsubsection{Principes de pouvoirs }
			
			\subsubsection{Algos de décision pour les 3 principes}
			
		\subsection{Évaluation des comportements de pouvoir}
			
\end{document}
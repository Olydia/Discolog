
\documentclass [french]{article}
\usepackage[francais]{babel}
\usepackage[utf8]{inputenc}




\begin{document}
	
	\section{Introduction}
		\begin{enumerate}
			\item Intérêt des agents relationnels 
			\subitem Prise en compte des relations sociales
			\item Présentation de la problématique : Étudier l'impact d'une relation interpersonnelle sur les stratégies de négociation.
				\subitem 
			\item Organisation du document
		\end{enumerate}
		
		
	\section{État de l'art}
	Dans cette partie, nous dressons un état de l'art des travaux qui s'intéressent à la dominance dans les interactions tant qu'en psychologie sociale qu'en informatique affective 
		\subsection{Dominance en psychologie sociale}
			La dominance a été largement étudié dans le domaine de la psychologie sociale. Nous retrouvons des travaux dans différentes veines du domaines qui étudient l'influence de la dominance sur les relations humaines, tant comme trait de personnalité, rôle social ou encore relation interpersonnelle que construisent les individus. Ces comportements se manifestent dans l'interaction avec des comportements verbaux, non verbaux ou encore sur un niveau stratégique.		
			Nous présentons dans cette section différentes études sur la dominance.
			
			\subsubsection{Relation interpersonnelle}
			\label{ri}
			\subsubsection{Trait de personnalité}
			\subsubsection{Statue social}
			\subsubsection{Attitude sociale} Travaux de Florian Pecune?
			\subsubsection{Conclusion} 
				Conclure sur les différents aspects de la dominance dans les interactions sociales.
				Définir le contexte qui nous intéresse à savoir la relation interpersonnelle de dominance dans le contexte de négociation collaborative présentée dans la prochaine section. 
		\subsection{Relation interpersonnelle de dominance et négociation}
			Il a été largement prouvé que la relation de dominance est la 1ère relation à s'établir dans le contexte de négociation. De  plus, elle  affecte directement les stratégies de négociation exprimés par les négociateurs. Nous présentons les comportements dans la négociation influencés par la dominance.
			
			\subsubsection{Comportements de pouvoir}
			
			\subsubsection{Complémentarité de dominance} 
				
			\subsubsection{Conclusion} Positionner mes travaux:
			
			 1) Étudier la dominance dans l'interaction comme relation interpersonnelle. 
			
			2) Nous nous intéressons plus précisément à l'impact de la dominance dans les dialogues la négociation collaborative.
				 
				 \textbf{A quel moment définir la négociation collaborative... ?} 
				
				
		\subsection{Agents conversationnels sociaux}
			Des travaux en informatique affective se sont intéressées aux différents comportements de dominance dans la modélisation d'agent conversationnels.
			(Présenter les différents travaux en IA qui ont modélisé des comportements liés à la dominance).
			
			\subsubsection{Modélisation de l'autre: Théorie de l'esprit} Modéliser l'état mental de l'autre afin de comprendre son raisonnement. Cela permet d'une part d'adapter le comportement de l'agent afin de simuler une relation interpersonnelle. D'une autre part, cela permet d'anticiper les futurs comportements de l'autre. 
		
		\subsubsection{Conclusion}
		
			Notre but est de modéliser les comportements de pouvoir qui traduisent une tendance de dominance. De plus, nous souhaitons simuler une relation interpersonnelle de dominance avec l'utilisateur. A cette fin, nous proposons un modèle de la théorie de l'esprit basé sur la simulation  pour deux raisons. En premier lieu, comprendre les comportements de pouvoir exprimés par l'utilisateur. Deuxièmement, adapter le comportement de l'agent aux comportements de pouvoir exprimés par l'utilisateur afin de créer une relation de dominance dite complémentaire. 	
			
	
	\section{Contributions}
		
		\subsection{Modèle de négociation collaborative}
			\subsubsection{Domaine de négociation }
				 Critère , Option, Préférences, Satisfiabilité
				Contexte de négociation: préférences de l'autre, historique de la négociation (proposition, préférences exprimés)
			
			\subsubsection{Communication}
			 Présentation des actes de dialogues avec leurs catégories et conditions d'applicabilité. 


		\subsection{Décision basée sur le pouvoir}
			\subsubsection{Principes de pouvoirs }
			
			\subsubsection{Algos de décision pour les 3 principes}
			
		\subsection{Évaluation des comportements de pouvoir}
			Dans cette section, nous présentons une évaluation qui vise a valider les comportements de pouvoirs implémentés. Nous avons effectué deux types d'évaluation. Nous avons d'abord chercher à faire une évaluation externe des comportements dans le cadre d'une interaction agent-agent.  Ensuite, nous nous sommes intéressés à évaluer la perception des comportements dans le cadre d'une interaction agent-humain. 
			
			\subsubsection{Dialogue Agent-Agent}
				\begin{enumerate}
					\item Conditions
					\item Hypothèses
					\item Protocole expérimental
					\item Résultat et discussion 
				\end{enumerate}
			\subsubsection{Dialogue Humain- Agent}
					\begin{enumerate}
						\item Conditions
						\item Hypothèses
						\item Protocole expérimental (Interface, participant, et pool de participants)
						\item Résultat et discussion 
					\end{enumerate}
					
			\subsubsection{Discussion générale}
					Discuter la perception des comportements de pouvoir de manière générale, et introduire la partie implémentation de la relation de dominance complémentaire.
	\subsection{Modèle de la relation interpersonnelle de dominance}
	\textbf{But}: simuler une relation interpersonnelle de dominance entre l'utilisateur et l'agent en se basant sur leur comportements de pouvoir. 
	
	\textbf{Intérêt}: complémentarité de dominance améliore les résultats de la négociation. Par ailleurs, elle permet de renforcer la relation d'appréciation (\emph{liking}) entre les partenaires.  
	
	\textbf{Résumé objectifs}: dans le but de modéliser la relation de dominance, il faut, en premier temps, que l'agent puisse reconnaître les comportements de pouvoir exprimé par l'utilisateur en utilisant un modèle de la théorie de l'esprit. En second temps, l'agent adapter sa stratégie de négociation afin d'adopter un comportement de pouvoir complémentaire a celui exprimé par l'utilisateur. Cela permettra de créer une relation complémentaire de dominance 
	
	\subsubsection{Modèle de l'interlocuteur}
		\begin{itemize}
			\item Utilisation de la théorie de l'esprit basée simulation. 
			\item Méthode: Formuler des hypothèses sur l'état mental de l'interlocuteur.
			\item Problème: Taille importante des hypothèses sur les préférences pour un sujet donné. 
			\item Solution: Adaptation de la solution vers un modèle de raisonnement avec connaissances partielle sur les préférences. 
		\end{itemize}
		
		\textbf{Adaptation de l'algorithme de décision}
		
			\par Expliquer la démarche comme dans le papier AAMAS. Pour chaque module, expliquer l'adaptation utilisée.
			
		\textbf{Évaluation du modèle}
		
			Le but est de valider la pertinence du modèle de l'interlocuteur dans le cadre d'une interaction agent-agent. 
			\par Présenter l'évaluation purement informatique des performances de l'algorithme.
			\begin{enumerate}
					\item \textbf{Conditions:} Présenter les paramètres évalués à savoir : l'efficacité de prédiction, le temps de convergence et d'exécution.
					\item \textbf{Protocole:} nombre de dialogues, conditions d'initialisations (pow, pref).
					\item \textbf{Résultats et discussion}
			\end{enumerate}
	
	\subsubsection{Adaptation du comportement de l'agent : Simulation de la relation de dominance}		
		Expliquer rapidement qu'a chaque tour l'agent calcule le $other_{pow}$ pour s'adapter $ self_{pow}= 1 - other_{pow}$
		
		Présenter l'expérimentation:
		
		\begin{enumerate}
			\item Conditions étudiée: Interaction (avec/sans) adaptation.
			\item Hypothèses
			\item Protocole expérimental
			\item Résultats
			\item Discussion
		\end{enumerate}
		
	\subsubsection{Conclusion}
		Parler de notre démarche à chaque étape.
		
		Les études menées
		
		Discuter globalement des résultats
	
	\section{Conclusion}
		\subsection{Résumé des contributions}
		
		\subsection{Limites observés ? (pas sure)}
			Faire une autocritique de notre solution parce que bien-sure elle n'est pas parfaite ni complète et qu'on en a conscience. 
		
		\subsection{Respectives et futurs travaux}
			(Je reprends tes idées ... )
			\subsubsection{Perspectives à court termes}
				Intégrer le modèle de dialogue dans un ACA pour prendre en compte les comportements non verbaux. Il a été montré que la dominance est fortement véhicule a travers des comportements non verbaux (comme présenté dans la section \ref{ri}).
			
			\subsubsection{Perspectives à moyen termes}
			\begin{itemize}
				\item	Ajouter la notion d'argumentation dans le processus de négociation
				\item	Étudier la dominance comme trait de personnalité et son impact sur les stratégies de négociations et d'argumentation.

			\end{itemize}

			\subsubsection{Perspectives à long termes}
				Étudier les autres dimensions de relations [Svennivig] et leur impact sur le processus de négociation
				
				Projet ANR (Nicolas Sabouret et Magalie Ochs).
\end{document}
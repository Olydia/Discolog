	\documentclass{article}
		\usepackage[noend]{algpseudocode}
		\usepackage{subcaption}
		\usepackage{subfig} 
		\usepackage{amsmath}
		\usepackage{graphicx}
		\usepackage{eulervm}
		\usepackage{fontenc}
		\usepackage{mathrsfs}
		\usepackage{multirow}
		\usepackage{array}
		\usepackage[rflt]{floatflt}
		\usepackage{makecell}	
		\usepackage{xcolor, soul}


		
	\begin{document}
	\section{Questionnaire :}
	
	
		\begin{enumerate}
			

		\item  \textbf{H1:} In the negotiation, the more dominant speaker will more strongly be perceived as self-centered
				\begin{itemize}
				\item Speaker (A/B) is self-centered.
				\item Speaker (A/B) takes into account the preferences of the other speaker. 
					\end{itemize}  
		
		\item \textbf{H2:} Agents with lower dominance is more strongly perceived as making concessions. (concession \& level of demand principle)
		
			The more an agent is dominant, the less he is perceived as making concessions.
			

				\begin{itemize}
						\item Speaker (A/B) makes concessions in the negotiation.
						\item Speaker (A/B) gives up his position in the negotiation.
				\end{itemize} 
				
				
		\item \textbf{H3:} Agents with lower dominance are perceived as having a lower level of demand comparing to agents with higher dominance. (concession \& level of demand principle)

				\begin{itemize}
						\item Speaker (A/B) is demanding.
						\item Speaker (A/B) presses his position in the negotiation.
				\end{itemize} 
				
				
		\item \textbf{H4:} Agents with higher dominance are more likely to take the lead of the negotiation. (leading the dialogue principle)

		\begin{itemize}
				\item Speaker (A/B) takes the lead in the negotiation.
				\item Speaker (A/B) takes the initiative in the negotiation. 
		\end{itemize}
		
		
		\item \textbf{Number of dialogues:} .
		\begin{itemize}
				\item dominant high/low, submissive fixed
				\item submissive high/low, dominant fixed
				
				=> H5 to H8 - The more dominant the speaker, the more strongly he is perceived as self-centered (in different negotiations) + 3 others
				
				\item \textbf{H9:} In the condition where initial preference sets of agents are similar, the behaviors of dominance (except for taking the lead) will not be visible, because the negotiation converges quickly.
				
		\end{itemize}
		

		
		\item Manipulation check questions (2 for each dialogue)
		\begin{itemize}
			\item Speaker (A/B) proposes X
			\item The parcipant chose X 
		\end{itemize}
		
	\end{enumerate}
	
	\end{document}
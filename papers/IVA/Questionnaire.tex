	\documentclass{article}
		\usepackage[noend]{algpseudocode}
		\usepackage{subcaption}
		\usepackage{subfig} 
		\usepackage{amsmath}
		\usepackage{graphicx}
		\usepackage{eulervm}
		\usepackage{fontenc}
		\usepackage{mathrsfs}
		\usepackage{multirow}
		\usepackage{array}
		\usepackage[rflt]{floatflt}
		\usepackage{makecell}	
		\usepackage{xcolor, soul}


		
	\begin{document}
	\section{Questionnaire :}
	
	
		\begin{enumerate}
			

		\item  \textbf{H1:} The more an agent is dominant the more he is perceived as individualist and self-centred.
				\begin{itemize}
				\item Speaker (A/B) is self centered.
				\item Speaker (A/B) takes into account the preferences and the interests of the other speaker. 
					\end{itemize}  
		
		\item \textbf{H2:} Agents with lower dominance will make larger concessions.
				\begin{itemize}
						\item Speaker (A/B) makes concessions in the negotiation.
						\item Speaker (A/B) gives up his position in the negotiation.
				\end{itemize} 
				
				
		\item \textbf{H3:} Agents with lower dominance are perceived as having a lower level of demand comparing to agents with higher dominance. 
				\begin{itemize}
						\item Speaker (A/B) have high demands.
						\item Speaker (A/B) reduces his demands during the negotiation.
				\end{itemize} 
				
				
		\item \textbf{H4:} Agents with higher dominance are more likely to take the lead of the negotiation. In addition they are perceived to adopt a goal-directed behavior. 
		\begin{itemize}
				\item Speaker (A/B) takes the lead of the negotiation.
				\item Speaker (A/B) takes the initiative in the negotiation. 
				\item Speaker (A/B) is goal oriented.
		\end{itemize}
		
		
		\item \textbf{H5:} In the condition where initial preference sets of agents are similar, the behaviors of dominance will not be visible, because the negotiation converge quickly.
			\subitem The same questions above will be asked for this condition. 
			
		\item Manipulation check questions (2 for each dialogue)
		
	\end{enumerate}
	
	\end{document}
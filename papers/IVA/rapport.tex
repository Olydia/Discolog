\documentclass  [french] {article}
\usepackage{color}
\usepackage[francais]{babel}
\usepackage[utf8]{inputenc}
\usepackage{lmodern}
\usepackage[noend]{algpseudocode}
\usepackage{subcaption}
\usepackage{subfig} 
\usepackage{graphicx}
\usepackage[rflt]{floatflt}
\usepackage{mathrsfs}
\usepackage[francais]{babel}


	%	
	%		\setlength\extrarowheight{3pt}
	%		% Table float box with bottom caption, box width adjusted to content
	%		\newfloatcommand{capbtabbox}{table}[][\FBwidth]

	\begin{document}
		\title{\vskip -10pt Rapport sur l’avancement des travaux de thèse: Impact de la relation de dominance dans les stratégies de négociation collaborative}
				\maketitle
		
	\textbf{Doctorant:} Lydia OULD OUALI.
	\vspace{1em}


	\textbf{Directeur de thèse :} Nicolas SABOURET
	\vspace{1em}
	
	
	
	
	\par Aujourd’hui, les agents conversationnels animés (ACA) connaissent un essor important dans différent domaines d’applications, plaçant ces derniers dans des rôles de plus en plus variés tel que compagnon, tuteur  ou encore un recruteur virtuel. Dans ce genre de situation, l'ACA est amené à partager des buts et des tâches avec l'utilisateur et donc collaborent afin de les satisfaire.
	
	De plus, de nombreux travaux ont déjà démontré que les utilisateurs humains tendent à interagir naturellement et socialement avec ces ACA comme s’ils étaient humains. Ce rapprochement a mené la communauté à s'intéresser d'avantages aux comportements sociaux et leur impacts sur le comportement durant l'interaction. En effet, il a été démontré que la dimension sociale affecte nos comportements. Un nombre important de recherches visent actuellement à modéliser des ACA   capables d'exprimer des comportements socio-émotionnelles tant sur l'aspect verbal que non-verbal.
	
	Mon projet de thèse s'inscrit dans la continuité de ses travaux, où je m'intéresse à l'impact des relations sociales sur les stratégies de prise de décision durant une interaction, plus particulièrement dans le contexte d'une négociation collaborative avec l'utilisateur. 
	
	La première étape consistait à définir le contexte de la recherche et à faire une étude bibliographique du domaine. Ensuite, j'ai mené une étude afin de collecter des données sur le comportements des interlocuteurs dans un dialogue social. L'étude consistait à enregistrer deux personnes discutant pour choisir un restaurant où aller dîner. Cette étude nous a permis de détecter un nombre de comportements tels que l’aspect de la négociation collaborative dans le choix du restaurant ainsi l’apparition de certain comportements de dominance. De plus, elle nous a permis de définir les actes de dialogues pour le module de communication de l'ACA. Ensuite, une étude bibliographique en psychologie sociale a été effectué afin de déterminer les comportements de dominance qui apparaissent dans un dialogue de négociation et comment ces derniers affectent les stratégies des négociateurs dans le dialogue.
	
	En se basant sur l'état de l'art effectué et les données collectées, j'ai pu durant ma deuxième année de thèse proposer un modèle d'ACA capable de mener une négociation collaborative avec un utilisateur et d'adapter ses stratégies de négociation en fonction de sa dimension de dominance.  Ce modèle a été ensuite implémenté dans une plate-forme d'agent conversationnel. En début de troisième année, j'ai mi en place une expérimentation afin de validé le modèle proposé. Les résultats obtenues confirment la cohérence du modèle proposé ainsi que les comportements perçus.
	
	Actuellement, je travaille a étendre mon ACA avec un modèle de la théorie de l'esprit afin qu'il soit capable de percevoir sa relation de dominance avec son interlocuteur et pouvoir adapter son comportement à la relation perçue. La dernière étape consistera à valider ce modèle de dialogue.
	 
	  
	
	

	\end{document}
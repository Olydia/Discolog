	\documentclass{llncs}
	\usepackage{llncsdoc}
	\usepackage[noend]{algpseudocode}
	\usepackage{subfig} 
	\usepackage{graphicx}
	\usepackage{frame, caption}
	\usepackage{amsmath}
	\usepackage{eulervm}
	\usepackage{fontenc}
	\usepackage{mathrsfs}
	\usepackage{multirow, enumitem, longtable, rotating}
	\usepackage{array}
	\usepackage[rflt]{floatflt}
	\usepackage{makecell}	
	\usepackage{xcolor, soul}
	\sethlcolor{yellow}	
	\usepackage{floatrow}

	\setlength\extrarowheight{3pt}
	% Table float box with bottom caption, box width adjusted to content
	\newfloatcommand{capbtabbox}{table}[][\FBwidth]
	
	\begin{document}
	\title{\vskip -10pt Dominance in dialogue of cooperative negotiation}
	
	\author{Lydia Ould Ouali\inst{1}, Charles Rich\inst{2} \and
		Nicolas Sabouret\inst{1} }
	
	\institute{LIMSI-CNRS, UPR 3251, Orsay, France \\
		Universit\'e Paris-Sud, Orsay, France \\
		\email{\{ouldouali, nicolas.sabouret\}@limsi.fr}
		\and
		Worcester Polytechnic Institute\\ Worcester, Massachusetts, USA\\
		\email{rich@wpi.edu}
	}

	\maketitle
	\section{Introduction}
	\par With the rise of artificial conversational agents capable to hold a conversation with a human user and potentially display social behaviors. It become important to understand which social behaviors appear in human-human (H-H) dialogues and how they affect interlocutors behaviors. In particular the social relationship that interlocutors establish during the dialogue.
	
	Previous researches have shown that people tend to respond to computers as social actors \cite{bickmore2005establishing}, which lead the community to assess the psychosocial relationship between the user and the agent during their interaction. 
	A growing body of research is investigating the use of appropriate social behavior for virtual agents in different roles and types of user-computer relationship \cite{bickmore2005s,bickmore2005establishing,kidd2005sociable}.

	Moreover, many applications require the user and agent to cooperate in order to achieve a common objective. For example, an conversational agent playing a role of collaborator which have to cooperate with other employees to fulfill different tasks, or a companion agent for elderly that helps the elderly to respect a specific diet. However, in many cases, interlocutors have to negotiate in cooperative manner about the way to achieve the common task (\emph{i.e. Cooperative negotiation}). 	Indeed, we believe that individuals that collaborate to fulfill an objective, have their own preferences in the way to do it (based on their experiences), which lead them to conduct a \emph{cooperative negotiation} about their preferences in order to determine a trade-off that satisfy both interlocutors. Moreover, scholars in social psychology and communication investigated \cite{dunbar2005perceptions,de1995impact} the impact interactive effects of relations and emotion in negotiation, and proved that  \emph{interpersonal power} directly affects the strategies of negotiators. Our objective, in this paper, is to build a conversational agent that can deploys strategies of negotiation based its perceived relation of power with the user. 

	The next section will relate on existing works around interpersonal power based on both researches on social psychology and affective computing. Section 3, will present the different modules our model of dialogue. Section 5 will discuss our experiment and the analysis of the obtained results. 
	
	
	\section{Related works}
	\par Despite the various definitions of power available in the fields of interpersonal communication and psychology, scholars are converging to a general definition of power as the capacity to produce intended effects and ability to influence the behavior of other person in the conversation. \cite{dunbar2005perceptions}. Power may be viewed as a personality trait, or describe the social role of an individual inside a group \cite{kecskes2013research}. In the context of communication, power is a dyadic variable that takes place during the dialogue, where the interlocutor who exerts power is viewed as \textit{dominant}, while the interlocutor with low power's behaviors is viewed as \textit{submissive}. 
% where one individual's attempt of control is necessarily acquainted by the partner in the interaction.\cite{dunbar2005perceptions}
	\par Behaviors related to power in a conversation can contribute either positively or negatively to the discussion. For example, positive contributions include actions such as keeping the conversation going, orient the task decision, by making quick decisions and conclusions etc. Negative contributions may include not considering the partner in the conversation, for example, not giving the occasion to express his opinion, not open to criticism. In addition, expressing verbally the power can be viewed as offensive and unjustified \cite{zablotskaya2012relating}. Giving these contributions to the conversation, several researches get interested to detect behaviors related to power during the conversation. We focus essentially on the context of conversation of negotiation, where several researches already proved the impact of power on the negotiation\cite{van2006power}.
	
	\subsection{Behaviors of power in dialogue}
	\label{domDialogue}
	Several scholars studied behaviors related to the relation of power in order to gain a better understanding to social relations. During a conversation, power can manifest through verbal and nonverbal behaviors.
	
	At the nonverbal level, a wide range of behaviors have been associated with the relation of power. \cite{burgoonnonverbal} divides nonverbal behaviors into  different classes such as kinesics behaviors that includes facial expression, body movements, gestures. For instance, dominant individuals are related with high visual ratio.
	
	Voice cues are another class of behaviors which manifest by speaking duration, speaking intensity, voice control and pitch.
	
	
	Verbal behaviors of power in the dialogue are related to the type of \textit{strategies} that individuals choose in order to take control of the other especially during a negotiation. Therefore, dominant negotiators tend to end up with the larger share of the pie \cite{giebels2000interdependence}. 
	 A considerable body of research has documented the effects of power on negotiation behaviors and outcomes. First, in De Dreu researches about the impact of social value orientation in negotiation strategies, he demonstrated that \cite{de1995impact} dominant negotiators have higher aspirations, demands more and concede less. 
	Moreover, Galinsky \cite{galinsky2003power} affirms that power increases the action orientation. Dominant negotiators control the flow of the negotiation. In addition, high power increase task orientation and goal-directed behavior. 
	
	Furthermore, power affects the way that negotiators gather information about their partners. Submissive negotiators have a stronger desire to develop an accurate understanding of their negotiation partner, which would lead them to ask more \emph{diagnostic} rather than \emph{leading} questions.
	
	It was shown \cite{fiske1993controlling},\cite{de1995impact} that dominant negotiators are self-centered and tend to not pay attention to submissive negotiators .
	%					 The idea is that high-dominant individuals have many resources and can often act at will without serious consequences, while submissive individuals, have to be more careful because they are more dependent on other people. In addition, they are motivated to gain or regain control over their outcomes by paying close attention to the people on whom they depend.
	%					
	\par We are interested in this paper reproduce  verbal behaviors for two main reasons. First, verbal behaviors are directly related to \emph{the strategies} deployed during the negotiation. In addition, our dialogue system is text oriented, which make the nonverbal behavior impossible to reflect during the dialogue. 
	In order to implement those behaviors, we extracted three principle behaviors related to the relation of power that impact strategies of negotiation.

	
%	\subsection{Principles of power in negotiation}
	Based on psychology research, we selected three main principles of behaviors in order to implement our conversational agent.
	\begin{enumerate}
	\item \textbf{Self Vs Other:} Submissive negotiators consider the preferences of other in the negotiation, whereas dominant negotiators  are self-centered and only interested by satisfying their own preferences.
	
	\item \textbf{Representation of demands:} Dominant negotiators show a higher level of demand than the submissive ones. In addition,  Submissive negotiator's demand decrease over time, and tends to make larger concessions than dominant negotiators.
	
	\item \textbf{Control the flow of the negotiation:}
	Dominant negotiators tend to make the first move \cite{magee2007power}. In addition, they take the lead of the negotiation. Otherwise, submissive negotiators aim to construct an accurate model of other preferences, which lead them to ask more questions about other preferences rather than keeping the negotiation going (makes proposals).
	
	\end{enumerate}
		We will present in the next section the decision model based on the behaviors of power. 
	%						\item Based on Carsten, De Dreu and Van Kleef demonstrate that high power negotiators are high in their propensity to negotiate relative to participants with low power. (leading individuals to focus on the rewards available to them in situations and to bargain forgreater rewards than were initially offered to them.)			
	
	\subsection{Power in affective computing}
	Different models enabling conversational agents to exhibit social power behaviors through their verbal and nonverbal behavior have been proposed. 
	
	Most of the existing works investigate the nonverbal behavior of power. For instance, \cite{lance2008relation} proposed an animated virtual agent with a gaze model for emotional expression based on Mehrabians PAD model \cite{mehrabian1996analysis}. They demonstrated that power was associated to a raised head and fast movements. These results were also \cite{mignault2003many} works.
	 Further on, \cite{gebhard2014exploring,callejas2014computational} demonstrated that head-tilt as well as the use of spacial movements is related to power and submissive perception. 
	  
	\par Other works investigated the perception of power combined to other social aspects. For instance,\cite{strassmann2016effect} investigate the perception of nonverbal behaviors of virtual agents with a focus on power and cooperation. On the same vain,  \cite{bee2010bossy} combines in their model the perception of nonverbal behaviors from the PAD model \cite{mehrabian1996analysis} with verbal behaviors related to personality traits from the big five model.
	 They demonstrated that the linguistic personality traits influence the perception of power. Reciprocally, gaze-based power influences the perception of personality traits.
	
	 
	These researches differ from ours in two main points. First, they model the power as an individual trait, while in our research, we consider the power as an interpersonal relation, which means that it takes place between the agent and the user(dyadic trait). 
	Second, most of theses research focused on nonverbal behaviors of power, while we are interested in the verbal behaviors. Moreover, we aim to study the effect of power at a decisional level, where we investigate how power affects the strategies of negotiation in dialogue.
	
	
	\section{Model of negotiation based on the relation of power}
	In this section, we present our model of dialogue of negotiation of preferences.	
	First, we present the domain of model that represent the agent's preferences and the representation of topics of negotiation. Second, we present the implementation of the principles of behaviors of power in negotiation (see section \ref{domDialogue}).

	\subsection{Domain model}
	The overall goal of a negotiation in our dialogue model is to choose an \textbf{option} in a set of possible options for a given topic. For instance, on the topic ``Restaurant'', we have a set of options: ``Chuck's cake'', ``Ginza''\ldots 
	 Let $\mathcal{O}$ be the set of options, and $V$ be the set of available options, such that option :$ V\rightarrow O$. For example, V is the set of restaurants that are mutually known to the interlocutors. 
		
	\par To be able to compare these options, interlocutors base their evaluation of each option on a set of \textbf{criteria} that reflect options characteristics. We consider a sequence of criterion sets $C_1, C_2, ..., C_n$ such as options set are defined as the cross-product:
	$O = C_1 \times C_2 \times \ldots C_n$.
	
	Furthermore, each criterion has to be measurable, in the sense that it must be possible to rate an option. Therefore, $\forall$ \emph{c $\in C_i$},  we note \emph{$V_i$} its  domain of values mutually knowns by interlocutors. For example, the domain values of the criterion cuisine is noted $\emph{V}_{cuisine} = \{chineese, italian, \ldots\}$.
%	For example, for restaurants, the criteria might be:
%	
%	Cuisine = \{Chinese, French, ...\} /Atmosphere = \{quiet, lively, ...\}. 
%	
%	and a restaurant options might be: 
%	
%	$[Chinese, cheap, lively]$. / $[Japanese, expensive, quit]$.   
%	%\item $[Chinese, expensive, lively]$.
%	%					\item $[French, cheap, quit]$.  
%	%					\item $[French, expensive, quit]$.   
%	%					\item $[Chinese, expensive, lively]\ldots$ 
	\subsection{Self model} 
		In order to allow the agent to negotiate about its preferences, we define a model about agent's preferences for each criterion of the topic. Preferences are represented as partial orders on the criterion sets. The preference relation on the criterion set $C_i$ is denoted by $\prec_i$.			
		Let $\prec$ denotes the sequence of preferences $\{ \prec_i, \prec_2, ..., \prec_n\}$.
	
		Using the set of binary preferences$\prec_i$, an agent is able to rate he likes a value $v \in V_i$ compared to other values in $V_i$. we denote a function \emph{Satisfaction} which is normalized to [0,1] that evaluates an element of a criterion set relative to the corresponding preference relation.

	\begin{equation}
	sat(v, \prec_i) =	1 - \left( \frac{|\{d : d \neq c \  \wedge \ (v \prec_i d)\}| }{( |V_i| - 1 )}\right)
	\end{equation}
	
	
	Satisfaction is generalized to options as a weighted sum.
	For $o \in O$ and where $o_i$ is the i-th element of $o$ and $\prec_i$ the i-th element of $\prec$.
	\begin{equation}
	sat(o, \prec) = \frac{\sum_{i}^{n} sat(o_i, \prec_i) }{n}
	\end{equation}
	
	\subsection{Dialogue model}
		Negotiators communicate during the negotiation via utterances. We generated nine types of utterance based on Sidner work. \cite{sidnerartificial}. 
		
		\begin{table}[h]
			\begin{tabular} {|p{2.5cm}|p{3.5cm}|p{6cm}|}
				\hline
				Utterance type & input & Format \\
				\hline
				StatePreference & $c \in V_i$ & I (don't) like $c$.\\
				\hline
				 \multirow{2}{*}{AskPreference} &$c \in V_i$ & do you like $c$ ?\\
				 \cline{2-3}
				 & $c \in C_i$ & What kind of $c$ do you like ? \\
				 \hline
				 Propose & $p / p\in V_i \vee p \in V$ & Let's go to $p$. \\
				 \hline
				 Reject & $p / p\in V_i \vee p \in V$ & I'd rather choose  something else \\
				 \hline
				 RejectPropose & $r, p / r,p\in Vi \vee r,p \in V $ & I don't want to go to $r$. Let's rather go to $p$ \\
				 \hline 
				 RejectState & $ p,c / p,c\in Vi \vee p \in V$ &  I don't like $c$, let's choose something else \\
				 \hline
				 Accept& $p / p\in V_i \vee p \in V$& Okay, let's go to $p$	 \\
				 \hline
				 AcceptPropose & $a,p / a,p\in Vi \vee a,p \in V $ & Okay. Let's go to $p$\\
				 \hline
			\end{tabular}
			\caption{The list of possible utterances in the model of dialogue}
		\end{table}
	
	In order to produce coherent dialogues, the agent keeps track of the different states of the dialogue updated after each move. 
	The negotiation have to take into account each criterion of the discussed topic. For example, in a negotiation on restaurants, speakers might negotiate about the type of \textit{cuisine}, and the \textit{ambiance} to choose.  
	
	First, the agent processes a model about the other interlocutor's preferences during the negotiation (\textit{i.e. model of other}). Let the collection of $A_i \subseteq C_i$ be the criterion values that are satisfiable to the other, and $U_i \subseteq C_i$ be the criteria values that are not.  We assume $A_i \cap U_i = \emptyset$.  Note that some values are thus unknown.
	
	Then the satisfiability of a criterion $c \in C_i$ is a function normalized to [0,1] and defined as follows.
	\begin{equation}
	other(c, A_i, U_i)= \left\{\begin{array}{ll}
	1	 & \mathrm{if\ }  c \in A_i\\
	0    & \mathrm{if\ }c \in U_i\\
	0.5	 & \mathrm{otherwise}
	\end{array}\right.
	\end{equation}
	
	This function is generalized to option $o \in O$ as a weighted sum.
	
	\begin{equation}
	other(o, A, U) = \frac{ \sum_{i}^{n} other(o_i, A_i, U_i) } {n}
	\end{equation}
	
	Second, information about the \textit{negotiation state} are processed. \\
	We note  $s_i, p_i, t_i, r_i \subseteq C_i$, respectively, the criterion values stated by the agent in the dialogue, values which have been proposed, values which have been accepted and the values rejected in the negotiation. 
		
	Similarly, we define the same structure of proposals for the options called $P, R, T$
	
	\subsection{Decision based on power in negotiation}
	\label{decision}
	We extracted from social psychology works three mains principles related to the relation of power which affects negotiators strategies and behaviors (see section \ref{domDialogue}). We present in this section, the formal theory built for each principle. 
	
	\subsubsection {Self vs other}
		It was proved that dominant negotiators are self-centered, whereas submissive negotiators care about fairness and the well being of the other negotiator. Therefore, we can conclude, that when the power get higher,the higher a negotiator give weight to the satisfaction of its preferences 
	
		Let  $pow \in [0, 1] $ denotes the agent's perception of its relationship of power. It is a constant for a given agent in a given relationship.
	The weight that an agent gives to its self-satisfaction relative to	the satisfaction of the other is a function normalized to 	[0,1] of its power, defined as below.
	%time-varying
	\begin{equation}
		self(pow, t) = \left\{\begin{array}{ll}
		pow & \mathrm{if\ } (t \leq \tau)\\
		max(0, pow - (\frac{\delta}{pow} . (t - \tau))) & \mathrm{otherwise}
		\end{array}\right.
	\end{equation}
	
	where is $t \geq 0$ is the number of open or rejected proposals and $\tau > 0$ and $\delta > 0$
	are parameters of the theory in general and are initially assumed to
	be 2 and 0.1, respectively.
	
	Let be	$\{V' \subset V$ / $\forall v \in V', acc(v,t) = true\}$ be the list of agent's acceptable values. 
	
		In the context of cooperative negotiation, the value of a proposal should be \textit{tolerable} for both interlocutors. Therefore, in addition to self preferences, the agent should considers other's preferences when proposing a value, with respect to the weight an agent gives to itself.
	%	We define a tolerance function of a criterion $c \in C_i$ as a function normalized to [0,1]:
	\begin{equation}
	 tolerable(pow, t, c, \prec, A, U) = self(pow, t) . sat(c, \prec_i)  +  (1 - self(pow, t)) . other(c, A_i, U_i)
	\end{equation}
	
	\begin{equation}
	tolerable(pow, t, o, \prec, A, U) = \frac{ \sum_{i}^{n} tolerable(pow, t, o_i, \prec_i, A_i, U_i) } {n}
	\end{equation}
	
	\subsection{Level of demand and concessions}
	
	\hl{level of demand}
				Therefore, the acceptability of a value $c \in V_i$  is relative to the weight an agent gives to its self satisfaction:
				\begin{equation}
				acc(c, t) = sat(c, \prec_i) \geq  \beta . Self(t).
				\end{equation}
				
				This function is generalized to option $o \in O$:
				\begin{equation}
				acc(o, t) = sat(o, \prec) \geq  \beta . Self(t)
				\end{equation}
	
		We represent the notion of concession, by making the agent decreases in the weight that its gives to self satisfaction during the negotiation. Indeed, we place our dialogue model in the context of \textit{cooperative negotiation}. Therefore, we aim to make the negotiation succeed which lead the negotiators  to make concessions if the negotiation is not converging. We define a \emph{"concession curve"}, where 
		

			

	\subsection{Lead of the negotiation}
	

	\subsubsection{Choosing an utterance type}
	
	%This can be a many-to-one function, e.g., there could be two restaurants with same criteria.		
	
	We present in table \ref{utt} the possible utterance responses and their applicability conditions. Note that each line (utterance) assumes that the previous ones are already not applicable.% We name this function 	$ chooseUtterance(dom, t, V, \prec, A, U)$ 
	
	
	\begin{table} [t]
	\centering
	\begin{tabular}{|p{.45cm}|p{3cm}|p{8cm}|}
	\hline
	\parbox[t]{2mm}{\multirow{5}{*}{\rotatebox[origin=c]{90}{\textbf{pow  $>\sigma$}}}}&\textbf{Utterance type} & Condition \\
	\cline{2-3}
	&Negotiation success & $\exists o \in T$   \textbf{ OR } $o \in P$ such that  $acc(o,t) = true$ \\
	\cline{2-3}
	& Negotiation failure & $ \forall o \in O$,  $o \in O$  \textbf{ OR } $acc(o,t) = false$\\
	\cline{2-3}
	& State & $type(u^{-1}) = AskPreference$  \textbf{ and }
	$n < \alpha$ \newline(with $n$ the number of successive statement moves)\\
	\cline{2-3}
	& AcceptPropose & $\exists c \in P_i$ / $acc(c,t)= true$ \\
	\cline{2-3}
	& RejectPropose & $\exists c \in P_i$ / $acc(c,t)= false$ \\
	\cline{2-3}
	& Propose & Otherwise  \\
	
	\hline
	\end{tabular}

	\begin{tabular}{|p{.45cm}|p{3cm}|p{8cm}|}
	\hline
	\parbox[t]{2mm}{
	\multirow{5}{*}{\rotatebox[origin=c]{90}{ \textbf{pow  $ \leq \sigma$}}}} & \textbf{Utterance type} & \textbf{Condition} \\
	\cline{2-3}
	& Negotiation success &  $\exists o \in T$ \\
	\cline{2-3}
	& Accept & $\exists c \in P_i$, $acc(c, t)=true $  \textbf{ OR } $ \exists o \in P$ ,  $acc(o, t) =true$ \\
	\cline{2-3}
	& RejectState & $ [\exists c \in P_i$, $acc(c, t)= false $ 
	\newline \textbf{ OR }$ \exists o \in P$ ,  $acc(o, t)=false]$  \textbf{and} $t<\tau$.\\
	\cline{2-3}
	& Propose & $\exists c$ / $other(c, A_i, U_i)  = 1 $  \textbf{ and }$acc(c, t)=true$
	\newline \textbf{ OR }  $\forall c \in C_i$,  $c \in T_i$\\
	\cline{2-3}
	& Ask &  \textbf{(}$t> \tau,$ \textbf{ and } 
	$\exists c \in P_i /$
	$ acc(c, t)=false$\textbf{) }
	\newline \textbf{OR}  $ \forall c \in C_i,other(c, A_i, U_i)=0.5$ \\
	
	\cline{2-3}
	
	& State & $type(u^{-1}) = AskPreference$
	\newline \textbf{OR}
	\newline $\exists x,other(x, A_i, U_i) \not = 0.5 $ 
	\newline \textbf{OR}
	\newline $ \exists C \in \mathcal{C}, A_i(C) = Unkown$
	\\
	\cline{2-3}
	& Propose & Otherwise \\
	\hline
	\end{tabular}
	\caption{Conditions to choose an utterance type}
	\label{utt}
	\end{table}

	\subsection{Summary of general parameters }
	\begin{itemize}[noitemsep]
	
	\item $\sigma \in $[0,1] : boundary between submissive and dominant used in
	choosing an utterance type
	%		\item $\beta$:  a value that represent the minimum score that a value has to get to be positively satisfiable to the agent preferences in the negotiation. Note that $\beta = const \times self(dom,t)$.
	\item $\tau > 0$ : the minimum number of open or rejected proposals before concession begins
	\item $\delta > 0$ : parameter in slope of concession curve.
	\item $u^{-1}$ refers to the previous utterance.
	\item $\alpha> 0$: the maximum number of successive statement moves.
	
	
	\end{itemize}
	
	
	
	\section{Evaluation}
	
	We built a conversational agent able to deploy a strategy of negotiation based on its perception of interpersonal relationship of power. 
	
	In order to validate our model, we conducted a perceptual study in which participants have to determine the behaviors of two agents generated using our model. 
	
	\subsection{Study design}
	%	We developed an experimental scenario in which we generated dialogues of negotiation between two agents implemented with the proposed model. We used a social topic of "negotiation about a restaurant where to have dinner".
	We used our model to develop two conversational agents which have to negotiate about a social topic of "negotiation about a restaurant where to have dinner".
	
	We generated a set of dialogues where we manipulated two main conditions. First, the initial value of the power \textbf{pow} (see section \ref{decision}). The purpose was to define the behavior of agents in negotiation.
	We aimed to make the first agent \emph{(Agent 1)} adopt more a dominant behavior by initiating the value of its power \textbf{pow} in the higher range of the power spectrum ($pow>\sigma$). Complementarity, the second agent \emph{(Agent 2)} had to adopt a submissive behavior. Therefore we initiated its \textbf{pow} to be in the lower range. ($ pow\leq \sigma$).
	
	The second condition involved varying the initial preferences of both agents. We manipulated the initial preferences to be either \textit{similar} or \textit{different}. We used the metric of Kendall distance \cite{bra2013Kendall} in order to compute the distance between two preferences sets (see section \ref{decision}).  
	We figured out, that the dialogues generated with the \textit{similar} preferences condition are quite similar, because the negotiation converges quickly. Therefore, we presented only one dialogue in the \textit{similar preferences} condition for the experiment. 
	The conditions manipulated to generate the dialogues are depicted in table \ref{Conditions}
	
	
	\begin{table}
	
	\centering
	\begin{tabular}{ |l|c|c|l| }
	\hline
	\multicolumn{3}{ |c| }{Conditions} & \multirow{2}{*}{Dialogue's label}  \\ \cline{1-3}
	
	\newline \multirow{2}{*} {\textbf{Initial preferences}}& \multicolumn{2}{ c| } {\textbf{Dominance}} & \\ \cline{2-3}
	
	\newline  & pow(Agent 1) & pow(Agent 2) &  \\ 
	\hline
	\newline\multirow{3}{*} {Different preferences} & 0.9 & 0.4 & Dialogue 1 \\ \cline{2-4}
	
	\newline  & 0.7 & 0.4 & Dialogue 2\\ \cline{2-4}
	
	\newline   &0.7 & 0.2 & Dialogue 3\\ 
	\hline
	\newline Similar preferences & 0.7 & 0.4 & Dialogue 4\\
	\hline
	\end{tabular}
	\caption{Initial condition's setting for generating dialogues} 
	\label{Conditions}
	\end{table}
	
	
	
	%	We present in table the initial setting for agents for the presented dialogues in the experiment.
	
	
	\subsection{Hypotheses}
	We investigated four main hypotheses about the perception of agents behaviors of power during the negotiation. 
	\begin{itemize}
	\item  \textbf{H1:} In the dialogue, the speaker with the higher power will more strongly be perceived as self-centered.  
	
	\item \textbf{H2:} The speaker with the lower power is more strongly perceived as making concessions.
	
	\item \textbf{H3:} The speaker with the higher power is perceived as having a higher level of demand comparing to the speaker with a lower power.
	
	\item \textbf{H4:} The speaker with the higher power is more strongly perceived as taking the lead of the negotiation.
	
	\end{itemize}
	
	\subsection{Experimental Procedure}
	
	We conducted a between-subject study using an online crowdsoursing website \emph{CrowdFlower} \footnote{https://www.crowdflower.com/}. 
	
	%Each generated dialogue had a questionnaire. There were 12 questions (including 2 test questions).
	
	Each participant was shown only one dialogue, where the task was to judge each agent behaviors. Agents were described as two friends (Speaker A as Agent1, Speaker B as Agent2) trying to find a restaurant where to have dinner. We wanted to avoid skewing the participant's perception by the fact that negotiators are artificial agents. Participants were invited to read the assigned dialogue and answer the corresponding questionnaire. 
	
	We defined for each hypothesis two questions (asked for both speakers) to analyze the speaker's behaviors related to the hypothesis. 
	Two test questions were included to check the sanity of the answers. Therefore, the questionnaire was defined with 18 questions.
	
	A total of 120 subjects participated to the experiment. They were recruited through the website \emph{Crowdflower.com}, for which each subject received \textit{25 cents}. 
	
	We limited the participant pools to native English speakers. We excluded participants providing wrong answers to our sanity questions. The final number of accepted participants was 105. 
	
	\subsection{Results}
	\begin{table}[t]
		\centering
		\begin{tabular}{|ll|c|c|c|c|c|c|c|c|} 
			\cline{3-10}
			
			\multicolumn{1}{c} {}	& \multirow{2}{*} {}& \multicolumn{2}{c|} {Dialogue1} & \multicolumn{2}{c|} {Dialogue2} & \multicolumn{2}{c|} {Dialogue3} &\multicolumn{2}{c|} {Dialogue4} \\ 
			\cline{3-10}
			
			
			\multicolumn{1}{c} {} & & Agent1 & Agent2 & Agent1 & Agent2 & Agent1 & Agent2 & Agent1 & Agent2 \\
			\hline 
			%\multicolumn{9}{|c|}{ \textbf{Results for H1}} \\
			%	\hline
			\newline \multirow{2}{*} {\textbf{H1}}  & \multicolumn{1}{|l|}{ \textit{Mean} $\pm$ \textit{SD} } & 3.9 $\pm$ 1.1 & 2.2$\pm$ 0.9  & 3.6 $\pm$0.9 & 2.2 $\pm$0.8  &2.8 $\pm$1.1  & 2.13$\pm$ 0.7 & 3.4 $\pm$ 1 & 2 $\pm$0.9 \\
			\cline{2-10}	
			\newline & \multicolumn{1}{|l|}{p-value} & \multicolumn{2}{c|}{ $<<0.01$} & \multicolumn{2}{c|}{ $<<0.01$} & \multicolumn{2}{c|}{ $<0.01$}& \multicolumn{2}{c|}{ $<<0.01$}\\
			\hline	
			
			\newline \multirow{2}{*} {\textbf{H2}} &\multicolumn{1}{|l|}{ \textit{Mean} $\pm$ \textit{SD} } & 2.2 $\pm$ 1.1 & 4.3$\pm$ 0.8  & 2.5 $\pm$1.2 & 3.8 $\pm$1.04 &2.7 $\pm$1.2  & 3.6$\pm$ 0.8 & 2.3 $\pm$ 1 & 3.2 $\pm$1.2 \\
			\cline{2-10}	
			\newline & \multicolumn{1}{|l|}{p-value} & \multicolumn{2}{c|}{ $<<0.01$} & \multicolumn{2}{c|}{ $<<0.01$} & \multicolumn{2}{c|}{ $=0.01$}& \multicolumn{2}{c|}{ $<<0.01$}\\
			\hline	
			
			\newline \multirow{2}{*} {\textbf{H3}} &\multicolumn{1}{|l|}{ \textit{Mean} $\pm$ \textit{SD} } & 4.1 $\pm$ 0.8 & 2.6$\pm$ 1.1 & 4.03 $\pm$ 0.8 & 2.7 $\pm$0.9 &3.5 $\pm$1.1 & 2.3$\pm$ 1 & 3.8 $\pm$ 1.8 & 1.8 $\pm$0.8 \\
			\cline{2-10}	
			\newline & \multicolumn{1}{|l|}{p-value}  & \multicolumn{2}{c|}{ $<<0.01$} & \multicolumn{2}{c|}{ $<<0.01$} & \multicolumn{2}{c|}{ $<0.01$}& \multicolumn{2}{c|}{ $<<0.01$}\\
			\hline	
			%				
			
			
			\newline \multirow{2}{*} {\textbf{H4}} & \multicolumn{1}{|l|}{ \textit{Mean} $\pm$ \textit{SD} } & 4.2 $\pm$ 0.9 & 2.3$\pm$ 1.1  & 3.8 $\pm$0.9 & 2.6 $\pm$1.07 & 3.8 $\pm$0.9  & 2.8$\pm$ 1.1  & 4.5 $\pm$0.5  & 1.9 $\pm$ 0.9\\
			\cline{2-10}
			\newline & \multicolumn{1}{|l|}{p-value} & \multicolumn{2}{c|}{ $<<0.01$} & \multicolumn{2}{c|}{ $<<0.01$} & \multicolumn{2}{c|}{ $<0.05$}& \multicolumn{2}{c|}{ $<<0.01$}\\
			\hline	
		\end{tabular}
		\caption{Summary of the obtained results for each hypothesis}
	\end{table}
	In order to analyze participant's responses, we computed the correlation for each pair of questions of the same hypothesis. The obtained results showed a high correlation between each pair of questions for all the hypotheses.
	
	The obtained data was analyzed using a non parametric test, and all pairwise comparison of of agent 1 and agent 2 behaviors were analyzed using Wilcoxon signed-rank test for paired data. All the results are resumed in the tables in the Appendix.
	
	\par Our first hypothesis (H1) stated that participants would perceive agents with the higher power to be self-centered and only care about their own preferences. Our analysis confirmed our prediction. For example, results obtained for \textit{dialogue2} showed that in average Agent1 was perceived as more self-centered (M=3.6, SD=0.9) than Agent 2 (M=2.2, SD=0.8) and (\textit{p$<$0.01}). The same results were observed for all the dialogues. 
	Our second hypothesis (H2) predicted that agents with the lower power will be perceived to make larger concessions. For all the proposed dialogues, participants responses supported our hypothesis. 
	%For instance, in \textit{dialogue 1} participants reported that Agent2 made larger concessions (M=4.3, SD=0.8), than did the Agent 1(M=2.19, SD=1.1), (\textit{p$<$0.01}).  (see figure  \ref{H2})

	\par The third hypothesis was also supported in all different the dialogues. H3 stated that agents with the higher power in dialogue are more strongly perceived as having higher level of demand. In all the dialogues, Agent 1 was perceived as expressing a higher level of demand than Agent 2 does. 
	%For instance, in dialogue 1, the level of demand was rated for Agent 1(M=4.07, SD = 0.8) versus Agent 2(M= 2.65, SD =1.1), with a \textit{p-value $<$ 0.01}.
	Finally, for the last hypothesis H4, the analysis revealed a significant main effect of the power in the lead of the dialogue, indicating that participants perceived agents with the higher power (Agent 1) as more leading the dialogue than agents with lower power (Agent2). 
	%In dialogue 1, participants indicated that Agent 1 is the one who leads the dialogue (M=4.2, SD=0.8) comparing to Agent 2 (M=2.3, SD= 1.1), \textit{p$<$ 0.01}. These results are depicted in the figure \ref{H4}.
	
	\subsection{Discussion and conclusion}
	
	Our research aims to model a conversational agent which is able to deploy strategies of negotiation in function of its representation of power. 
	We proposed a model which allows a conversational agent to generate strategies of negotiation in function of its relation of power, inspired from researches in social psychology
	Our on-line study had the purpose to validate the implemented behaviors and did provide strong support to the claim that the relation of power affects agent's behaviors in negotiation.
	
	We found that participants perceived a significant difference in the agent's behaviors depending on their respective relation of power.  
	Indeed, all of our four hypotheses were confirmed. We predicted that agents with higher power will be perceived as more individualist and self-centered, whereas agents with lower power would care about other preferences. Furthermore, agents with lower power were perceived to make larger concessions than did agents with higher power. Consistent with this prediction, agents with higher power showed a greater level of demand in the negotiation. Finally, we obtain evidence that agents with higher power tends to lead the dialogue and take control of the negotiation's flow. 
	
	In addition to these finding, we extended our analysis to understand the perception of power in dialogue.
	
	First, we studied the influence of the initial preferences in the generation and perception of behaviors of power.
	We supposed that when the initial preferences of agents are similar (\textit{similar preferences} condition), behaviors related power won't be perceived due to the high aspect of cooperation in the negotiation. However, as our results demonstrated for dialogues generated in the similar preferences condition, (see results of Dialogue 4), all the hypotheses were supported even when the preferences were quiet similar. Therefore, the behaviors of power implemented are not affected by the initial preferences of agents.
	
	Second, we get interested in analyzing the behaviors of power in the spectrum of power. We supposed that the highest the power gets in the spectrum, the better behaviors of power will be perceived. We did compare participant's judgments of Agent 1 in dialogues where Agent 1 had different setting of power whereas Agent 2 had the same setting. We specially compared Agent1 behaviors in Dialogue 1 and Dialogue 2. The results obtained are presented in \hl{the figure ... }.	
	The obtained results did not supported our hypothesis. no significant difference in agents behaviors was observed. This might be explained by the interpersonal nature of power, which means that participants rated the power of Agent 1 in opposition of Agent2, which makes the comparison of agents behaviors from different dialogues irrelevant. 
	
%	\par  The results presented have a number of limitations. Participants judged the dialogues from an external point of view which might influence the engagement in the process of judging agent behavior. 
	Our findings validate our model of dialogue in general and specifically confirmed the coherence of the generated behaviors of power. 
	The next step, would be to experiment our model in a realistic human-agent interaction.
	Further on, the interaction with a user should take into consideration the user's representation of its relation with the agent. Therefore, we aim to extend our model with a module of theory of mind that would allow the agent to have a mental representation of the user, and by consequence adapt its behavior to complement the user behavior perceived during the dialogue.
	
%	\section{Conclusion}
	
		

	
	% ================== BIBLIO ===============
	\vskip 4pt
	\bibliographystyle{plain}
	\small{\bibliography{Library}}



	\end{document}
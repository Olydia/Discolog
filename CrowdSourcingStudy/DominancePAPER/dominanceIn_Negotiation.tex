\documentclass{llncs}
\usepackage[noend]{algpseudocode}
\usepackage{subcaption}
\usepackage{subfig} 
\usepackage{usual}
\usepackage{amsmath}
\usepackage{graphicx}
\usepackage{eulervm}
\usepackage{fontenc}
\usepackage{mathrsfs}
\usepackage{multirow}
\usepackage{array}
\usepackage[rflt]{floatflt}
\usepackage{makecell}

\renewcommand\theadalign{cb}
\renewcommand\theadfont{\bfseries}
\renewcommand\theadgape{\Gape[4pt]}
\renewcommand\cellgape{\Gape[4pt]}
\pagestyle{plain}

%
\begin{document}
\title{ Interpersonal dominance in dialogue of cooperative negotiation}
\maketitle 
\section{Introduction}
Dialogue systems are artificial systems capable to hold a conversation with a human user, usually to achieve certain objective or fulfill a task.
% Existing conversational agents provided with dialogue system play different roles like 

During a human-human dialogue, interlocutors establish a social relationship that affect their behaviors. Previous researches have shown that people tend to respond to computers as social actor [bickmore], which lead the community to assess the psychosocial relationship between the person and the agent during their interaction. 
A growing body of research is investigating the use of appropriate social behavior for virtual agents in different roles and types of user-computer relationship.
For example, Bickmore studied the impact of trust ...


In this paper we are interested in modeling a dialogue system for social dialogue in which user and agent cooperate in order to achieve a common objective.  We believe that individuals that collaborate to fulfill an objective (for example: find a restaurant for dinner), have both preferences in the way to do it, which lead them to conduct a \emph{cooperative negotiation} about their preferences in order to determine a trade-off that satisfy both interlocutors. Moreover, scholars in psychology and communication investigated the impact interactive effects of relations and emotion in negotiation, and proved that  \emph{interpersonal dominance} adopt particular strategies depending on their perception of the relation of dominance. Our objective is to define an agent that can perceive the established relation of dominance and adapt its strategy of negotiation to the relation. 
 
We present in section 1 our model of negotiation on preferences. In section 2, we present the agent strategies of negotiation during the dialogue. 
%
%We present our model of dialogue for a cooperative negotiation, where the agent is defined with a panel of behaviors related to his representation of his relation with the user. We present in section 1, section 2 etc 
%\par Dominance have been regarded by psychology research as a fundamental dimension of interpersonal relationships \cite{burgoon}. 
%social influence the strategies of dialogue


\section{Model of negotiation on preferences}

\par The overall goal of a negotiation in our dialogue model is to choose an \textbf{option} in a set of possible options for a given topic. For instance, on the topic ``Restaurant'', we have a set of options: ``Chuck's cake'', ``The ducking Duck'', ``Ginza sushis''\ldots 
\par Each interlocutor makes its decision by consulting its preferences about these different options. Let $\mathcal{O}$ be the set of options.

To be able to compare these options, interlocutors base their evaluation of each option on a set of \textbf{criteria} that reflect options characteristics.
Let $\mathcal{C}$ be the set of criteria. 
Note that this set is dependent on the topic: the criteria for choosing a restaurant could be cuisine, ambiance, price, and location, while the criteria for choosing a movie could be type and location. 
Furthermore, each criterion has to be measurable, in the sense that it must be possible to rate an option even in qualitative way. Therefore, $\forall$ \emph{c $\in\mathcal{ C}$},  we note \emph{D$_c$} its  domain of values. For example, the domain  values of the criterion cuisine is noted $\emph{D}_{cuisine} = \{chineese, italian\}$.

Each option $O\in \mathcal{O}$ is characterized by a value for each criterion:
 $O = \{c_1=v_1,..., c_n=v_n\}$ with $c_i \in \mathcal{C}, \forall i \in [1,n]$ and $v_i\in \emph{D}_{c_i}$. 
 Thus, we define $\{v(c,O) \in \emph{D}_{c} / \forall O \in \mathcal{O}, \forall c \in \mathcal{C}\}$ as the \emph{objective} value of the criterion $c$ attributed  the option $O$. 
By objective, we mean that this value is independent from the preferences: both interlocutors evaluate the option with the same values, regardless of their preferences. 
For example, Ginza is an expensive Japanese restaurant: $v(price, Ginza) = expensive$ and $(cuisine, Ginza) = japanese$. 

In the following we will present the interlocutor preference model that allows him to represent and decide on preferences.

\subsection{Representation of preferences}

 Preference is a transitive antisymmetric binary relation \emph{P} defined on a set of elements \emph{A}, such that:
 
 \[ \left \{
     \begin{array}{l}
 	\emph{P(a,b)} $ means \emph{a}  is preferred over $\emph{b}. \emph{ a,b} \in \emph{A}\\
 	\emph{P(b,a)} $ means  \emph{b} is preferred over $\emph{a}. \emph{ a,b} \in \emph{A}\\
 $Otherwise, neither is preferred. $\\
     \end{array}
    \right .\]

For example $P_{cuisine} (Japanese, French)$ means that the interlocutor prefers the Japanese cuisine over the French. 

\par  We define the notation \emph{P(a,*)}  = \{$\forall$ \emph{x}$\in$\emph{A}, \emph{P(a,x)}\}, which means that \emph{a} is the \textit{most preferred} element in \emph{A}. 
By opposition, \emph{P(*,b)} = \{$\forall$ \emph{x}$\in$\emph{A}, \emph{P(x,b)}\} means that \emph{b} is the \textit{least preferred} element in \emph{A}.
 
\par Our goal is to be able to define preferences over options of negotiation. 
In the literature \cite{dodgson2009multi}, preference of one option over others depends on the options performances on the set of criteria. This is called multi-criteria decision. In general \cite{dodgson2009multi}, preferences on options are built by inference from preferences over criteria values. This inference can be done using different methods such as ordered weight average (OWA \cite{yager2012ordered}) or Choquet's integrals \cite{chouquet1953}.


\subsection{Decision based on preferences in our model}
\label{scores}
\par To produce dialogue with cooperative negotiation, interlocutors have to take coherent decisions about their preferences over options of a certain topic. Decision on options imply comparing theses options based on interlocutors preferences.  
 Since defining preferences over options is a multi criteria decision that involves calculating how well each option performs on its set of criteria. We have to build a \textbf{preference model }that contains interlocutors preferences on criteria, from which we can infer option preferences.  
In the following, we present the different elements of the preference model noted \textbf{$\mathcal{P}$} : 
\begin{itemize}
	\item \textit{Preferences on option's criteria} noted $\mathcal{P}_O$ is a partially ordered list of interlocutor preferences about the importance of each criterion in the choice of an option. For example, $ P_{Restaurant} = \{(Cuisine, Cost), \ldots\} $
	means that the criterion of cuisine is more important than the Cost to choose a restaurant. 
	\item  \textit{Preferences on the values of criteria}, $\forall c \in \mathcal{C}$, the interlocutor has a list of preferences noted $\mathcal{P}_C = \{(a_1, a_2), ..., (a_i,a_n) | a_i \in D_C \}$ that contains all its pair of preferences on the values of this criterion. 
	\item The preference model \textbf{$\mathcal{P}$} is thus the aggregation of all the preferences defined on the different criteria that define the option $O$. 

\end{itemize}
 \subsubsection{Running example:} 
 To illustrate our approach, we shall use the following running example:
 
 
 User and agent discuss about the topic \textit{Restaurant}. \textit{Restaurant} options are defined with the set of criteria $\mathcal{C} $= \{\textit{Cuisine, Cost, Location, Ambiance}\} and each criterion has its domain of values. We define in this example  a simple representation of the  preference model on the topic \textit{Restaurant} $\mathcal{P}$  that contains the following information: 
 \begin{itemize}
 \item $\mathcal{P}_{Restaurant} = $ \{(Cuisine, Location), (Cost, Ambiance), (Location, Ambiance)\}. $\mathcal{P}_{Restaurant}$ defines the agent preference model  about the criteria that define a restaurant. In this case, the agent believes that the criterion of  \textit{Cuisine} is more important than the criteria  \textit{Location} to choose a restaurant.
 \item In the following, we define  agent preferences on the values of each restaurant criterion: 
 \subitem $\mathcal{P}_{Cuisine} = $\{(French, Japanese), (Italian, Japanese)\}
 \subitem $\mathcal{P}_{Cost} = $ \{(Cheap, Expensive)\}
 \subitem $\mathcal{P}_{Location} = $\{(Paris01, Paris02), (Paris09, Paris01), (Paris01, Paris14)\}
  \subitem $\mathcal{P}_{Ambiance} = $ \{(Calm, Noisy)\}
  \end{itemize}
 
 Thus the preference model of our interlocutor is the aggregation of the all the  preferences that defines the topic Restaurant.  \\$\mathcal{P}_= \{\mathcal{P}_{Restaurant}, \mathcal{P}_{Cuisine}, \mathcal{P}_{Cost}, \mathcal{P}_{Location}, \mathcal{P}_{Ambiance}\} $

 \subsubsection{Selection based on preferences}
\par Once preferences on criteria of the topic are identified and the preference model $\mathcal{P}$ of the interlocutor is built, the interlocutor is provided with enough information to be able to compare two options and calculate the relation 
\\$P(O_1, O_2) / O_1, O_2 \in \mathcal{O} $.
 This comparison is done by calculating each option utility with a multi-criteria decision function. We selected for our model the WA \cite{yager2012ordered} function (weighted averaging) that offers a way of aggregating the preferences calculated on individual criteria to provide an overall utility rate for options. 
  \par We note  $score(a)$ the number of $a$  successors in the preference model $\mathcal{P}$, which means $|\{x \in \mathcal{D} / (a,x) \in \mathcal{P}\}|$. 
  $rank(a)$ is the normalized  score of the criterion $a$, which is calculated by ranking all the values of the specified domain $\mathcal{D}$ by their scores. 
  
Therefore, calculating the utility of an option using the WA is performed as follows: 

 \[U(O) = \sum_{c_j \in \mathcal{C}}  rank_R(c_j) \times score\left( v(O, c_j) \right) \] 
 
 
 \par The relation of preferences between two options is calculated by comparing  options utilities. 
  \[ P(O_1, O_2)  = \left \{
    \begin{array}{l}
	P(O_1, O_2)$ \textit{if}  $U(O_1) > U(O_2) \\
	P(O_2, O_1)$  \textit{if}  $U(O_2) > U(O_1) \\
	$  \textit{Neither is preferred if}  $U(O_2) = U(O_1)\\
    \end{array}
    \right .\]
  
 
\subsubsection{Example of decision}

Suppose that an interlocutor intents to calculate a relation of preference  P(Clementine, Mogoroko), such that 
\{Mogoroko, Clementine\} $\in$ Restaurant. Theses restaurants are described as follow: 
\begin{itemize}
\item Clementine=(Cuisine =\textit{French}, Cost=\textit{Expensive}, Location=\textit{Paris02},
 \\Ambiance=\textit{Calm}).
\item Mogoroko=(Cuisine=\textit{Japanese}, Cost=\textit{Cheap}, Location=\textit{Paris09}, 
\\Ambiance=\textit{Calm}).
\end{itemize}

Thus the utility of each restaurant is calculated as described bellow: 
\begin{itemize}
\item U(Clementine)=$rank$(Cuisine)$\times score$(\textit{French})+$rank$(Cost)$\times score$(\textit{Expensive})\\+$rank$(Location)$\times score$(\textit{Paris02})
+$rank$(Ambiance)$\times score$(\textit{Calm}).
\item U (Mogoroko)= $rank$(Cuisine)$\times score$\textit{Japanese}+$rank$(Cost)$\times score$(\textit{Cheap})\\+$rank$(Location)$\times score$(\textit{Paris09}) +  
$rank$(Ambiance)$\times score$(\textit{Calm})..
\end{itemize}
After calculating both restaurants utilities: U(Clementine)=-3 and U(Mogoroko)=5.
\\  We conclude that $P(Mogoroko, Clementine)$ is true.
 (i.e Mogoroko is more preferred to Clementine).

%Trois étapes :
%- préférences entre les valeurs pour chaque critère (ex) $->$ j'ai des P_c (un pour chaque c)
%- j'ai aussi des préférences entre les critères (ex) $->$ j'ai un P^criteria
%- je calcule les préférences sur les options à partir de ces P_c et de P^criteria (voir fonction section machin)


 \subsection{Mental model of interlocutors}
 
% Dans l'état mental, j'ai deux ensembles de préférences tels que définis ci-dessus ( Pc et P^criteria) : les miennes (je préfère le japonais au chinois) et celles de l'interlocuteur (mon interlocuteur préfère le chinois au japonais) que j'acquiert au fur et à mesure du dialogue.
 
% Notation : Pself et Pother sont des ensembles de prefs
% 
% De plus, je dispose d'un ensemble d'informations partagées: les propositions qui ont été faites (aussi bien en terme de critères (si on allait au chinois) que d'options (si on allait au ducking duck), celles qui ont été rejetées (idem), mais aussi ce que je sais que l'autre a déjà appris de mes préférences (je ne veux pas lui répéter)
 
% Notations: Proposed_option, Proposed_criterai, Rejected_option, Rejected_criteria, Pother-about-self (ToM) : ce que je crois que l'autre sait de moi
% Les quatre premiers sont des listes de critères/options ; le dernier est un ensemble de préférences
% 
% Notations: si Pself est un ensemble de préférences, on note Pself_c et Pself^criteria les sous-ensembles de Pref correspondants...
 
 
 To be able to negotiate in dialogue, the agent  needs a formal representation of its environment namely, its preferences,  user preferences. In addition, the agent has to take into account the current context of the dialogue with all the information shared during the conversation (i.e both interlocutors preferences). 
 
 In the following, we present the formal representation of an interlocutor mental model for dialogue.  
 
 \subsubsection{Preferences model}
 
 \begin{itemize}
 	\item agent preference model is defined with the notation $\mathcal{P}_{self}$.
 	\item User preference model as perceived by the agent during the dialogue is  defined with the notation $\mathcal{P}_{other}$.
 	\item In addition, the agent has to model  the user  knowledge about him based on the interaction (i.e Theory of mind; what the user knows about me). We note it   $\mathcal{P}_{other-about-self}$.
 \end{itemize}

 
\subsubsection{Dialogue context}
 % explain the notion of preferences models and how they are communicated in the conversation
During dialogue, interlocutors  negotiate  about a topic and its criteria. The final decision is to select an option (\emph{e.g.} let's go to Ginza restaurant). Therefore, interlocutors share information about their preferences and propose decisions to their negotiation (\emph{e.g.} let's go to a Japanese restaurant, we can't afford expensive restaurant, or let's go to Ginza). To represent these elements of negotiation, we use the following notations to define the status of proposals.

We first define a proposal as a tuple $Proposal(Type, Value)$ where  $Type$ is either the the topic (for example Restaurants) or a criterion $c \in \mathcal{C}$ and value is:
\begin{itemize}
	\item an option $O \in \in \mathcal{O}$ if $Type \in Topic$ 
	\item a value $v \in \emph{D}_c$ if $Type \in \mathcal{C}$
\end{itemize}
    

In order to keep track of all the proposals made during the dialogue, we define the following structures that defines the different status of a proposal can take during a negotiation:
 \begin{itemize}
	 	\item $Proposed$ is the set of all the open proposals made during the dialogue.
	 	\item $Rejected$  is the set of rejected proposals.
	 	\item $Accepted$  is the set of accepted proposals.
 \end{itemize}


\subsubsection{Utterances semantic}
Agents communicate using utterances that encapsulate the message. In dialogue, messages are represented as actions, they are defined with precondition, postconditions and effects that update both interlocutors mental states. The preconditions are all optional, because the message selection depends first on the interlocutor dialogue strategy. For utterance effects, we only represent the  agent's perception of its environment (i.e. his preferences and user preferences). We cannot represent the user belief, we only represent the agent perception of the user belief. For example, suppose that the agent states that he prefers Japanese cuisine over Chinese. By consequence, the agent adds this preference to its $\mathcal{P}_{o-a-s}$ model, but we don't have no certainty that the user adds this preference to its $\mathcal{P}_{other}$  model.

 Note that the effect of an utterance updates only the belief of the agent about a preference, but in any case it can update the values of the preference model.

\begin{table}

\caption{\label{tab: utt} dialogue utterances semantic}
\begin{tabular}  {|m{0.40cm}|m{4cm}|m{2cm}|m{2.5cm}|m{3cm}|m{3cm}|}

\hline 
N & \thead{Utterance} & \multicolumn{2}{c|} {\thead{Preconditions}  } &  \multicolumn{2}{c|} {\thead{Effects}  } \\
\hline 
1 & \makecell{State.Preference(\textit{$a,b$}):\\ ``I prefer $a$ over $b$''}& \multicolumn{2}{c|} {\makecell{ $(a,b)\in$ $\mathcal{P}_{self}$  \\ $(a,b) \notin$ $\mathcal{P}_{o-a-s}$} }&\makecell{ \textit{(hearer case)} \\ $add((a,b)$, $\mathcal{P}_{other})$ } & \makecell{\textit{(speaker case)} \\ $add((a,b)$, $\mathcal{P}_{o-a-s})$}  \\
\hline
2 & \makecell{Ask.Preference(\textit{$a,b$}):\\``Do you prefer $a$ to $b$'' ?}& \multicolumn{2}{c|} {\makecell{ $ (a,b) \notin \mathcal{P}_{other} $} }&
\multicolumn{2}{c|} {\makecell{None} } \\
\hline
3 & \makecell{Propose(\textit{Proposal(T,V)}):\\``Let's choose  \textit{V}''}& \multicolumn{2}{c|} {\makecell{$ Proposal(T,V) \notin  Proposed$} }&\multicolumn{2}{c|} {\makecell{ $add(Proposal(T,V), Proposed)$ }} \\
\hline
4 & \makecell{Accept(\textit{Proposal(T,V)}):\\``Okay, let's choose \\ \textit{V} for \textit{T}''}& \multicolumn{2}{c|} {\makecell{$Proposal(T,V) \in Proposed$ \\ $ Proposal(T,V)\notin Accepted$} } & \multicolumn{2}{c|} { \makecell{$add(Proposal(T,V), Accepted)$\\$ remove(Value, Proposed)$}  }  \\
\hline
5 & \makecell{Reject(\textit{Proposal(T,V)}):\\`` Sorry, I would choice \\ something else.''}& \multicolumn{2}{c|} { \makecell{$Proposal(T,V) \in Proposed$ \\$Proposal(T,V)\notin Rejected$}  } & \multicolumn{2}{c|} {\makecell{ $add(Proposal(T,V),Rejected)$ \\$remove(Proposal(T,V), Proposed)$}} \\
\hline
\end{tabular}
\end{table}
\par  Our aim was to define utterances with semantic that allows interlocutors to handle cooperative negotiation in dialogue. The proposed utterances are summarized in the table \ref{tab: utt}. We define in details the semantic of each utterance.
\begin{enumerate}
	\item  State.Preference utterance allows the agent to express his preference. For example: State.Preference$_{cuisine}(\textit{Japanese , Chinese})$ : ``I prefer japanese cuisine over Chinese''. 
	\\The  sender of the utterance (speaker) has to belief the preference statement (i.e it has to be defined in his preference model). Note that effect of the state utterance is different whether the agent is the speaker or the hearer. Thus,  the speaker updates its mental state about the hearer's knowledge about him. In addition, $(a,b) \in \mathcal{P}_{other-about-self}$ does not prevent from sending redundant information (I can try to insist on an already stated preferences): this depends on the dialogue strategy (see section X below). In parallel, The hearer updates its mental state about the speaker preferences.	
	\par We define two variant valuations on stating preferences: 
	 	\subitem State.Preference(\textit{$a, *$}): ``I prefer the most $a$''.
	 	\subitem State.Preference(\textit{$*, a$}): ``I don't like /hate $a$''.
	 	\subitem In this model, we cannot express indifference about preferences.%
	 	\\
	\item Ask.utterance is defined to enrich speaker knowledge about the hearer preferences. For  example: Ask.Preference$_{cuisine}$(\textit{$Japanese , Chinese$}) : Do you prefer japanese cuisine or chinese?, is sent because the speaker has no belief about the hear preference on those elements. 
	\par We define two variant valuations as follows: 
	 	\subitem Ask.Preference(\textit{$Pref_{j}(a, *)$}): ``Do you like $a$?''
	 	\subitem Ask.Preference(\textit{$*$}): ``What do you like ?.'' This case appear when the speaker has any belief on the hear preferences on the current topic$\mathcal{P}= \emptyset$. 
	\\
	\item Propose (Proposal(T,V)) allows the speaker to make a proposal. The effect of this utterance is to update interlocutors shared information about the negotiation.
	\\ For example: Propose.Preference(\textit{$cuisine,Japanese$}) : ``Let's choose japanese cuisine'', will update both interlocutor information about the dialogue context, in the way that \textit{Japanese} is added to \textit{Proposed}.
	\\
	\item Accept (Proposal(T,V)), indicates to the hearer that the speaker accepts the proposal \textit{V} for \textit{T} which has been the subject of a previous  proposal. We belief that there is no update of the belief on preference model because the speaker accepts a proposal if it is consistent with its strategy of dialogue and not necessarily with its preferences. For example : 
	
		\subitem Other: Propose.Criterion (Proposal(Cuisine, Indian)):`` Let's choose Indian cuisine.''
		\subitem Self: Accept.Criterion (Proposal(Cuisine, Indian)): ``Okay, lets choose Indian cuisine.''
		\\The agent accepts this proposal while Indian$\notin \mathcal{P}_{self_{Cuisine}}$ because in its strategy of dialogue, he prioritizes the other preferences (submissive interlocutor).
	\\	
	\item Reject (Proposal(T,V)): this utterance has the same semantic of an Accept utterance, it occurs after a proposal and the proposed value doesn't respect the speaker goals. For example: 
		\subitem Other: Propose.Criterion (Proposal(Cuisine, Indian)): ``Let's choose Indian cuisine.'' 
		\subitem Self: Accept.Criterion (Proposal(Cuisine, Indian)):`` Sorry, I would choose something else.'' 
	\\
	
\end{enumerate}
 
 \section{Dominance in dialogue}
\par Despite the various definitions of dominance available in the fields of interpersonal communication and psychology, scholars are converging to a general definition of dominance as the power to produce intended effects, and the ability to influence the behavior of other person in the conversation. (Bachrach \& Lawler, 1981; Berger, 1994; Burgoon et al.,
1998; Foa\& Foa, 1974; French \& Raven, 1959; Gray-Little \& Burks, 1983;
Henley, 1995; Olson \& Cromwell, 1975; Rollins \& Bahr, 1976). Dominance is generally viewed as a personality trait, or to describe the social role of an individual inside a group. However, in the context of communication, dominance is a dyadic variable where one individual's attempt of control is necessarily acquainted by the partner in the interaction.(Rogers-Millar and Millar, 1979,Dunba and Burgoon, 2005). 
\par Dominant behaviors in a conversation can contribute either positively or negatively to the discussion. For example, positive contributions include actions such as keeping the conversation going, orient the task decision, by making quick decisions and conclusions etc. Negative contribution may include not considering the partner in the conversation, for example, not giving the occasion to express his opinion, not open to criticism. In addition, expressing verbally the dominance can be viewed as offensive and unjustified (K,Zablotskaya). Giving these contributions to the conversation, several researches get interested to detect  behaviors related to the dominance during the conversation. We focus essentially on the context of conversation of negotiation, where several researches already proved the impact of dominance on the negotiation(VAN KLEEF, 2005)

 \subsection{Behaviors of dominance in dialogue}
  Several scholars studied behaviors related to the dominance in order to gain a better understanding to social relations. During a conversation, dominance can be perceived through verbal and nonverbal behaviors. We present in this section behaviors related to dominance that can be perceived in a conversation of cooperative negotiation.
 \subsubsection{Non-verbal behavior}
At the non-verbal level, a wide range of behaviors have been associated with dominance. (Dunbar \& Burgoon) divide non-verbal behaviors into classes such (kinesics, vocalic, ). First, kinesics behaviors are considered as the richest of all the codes that includes facial expression, body movements, gestures. Indeed, dominant individuals are related with high visual ratio (high looking while speaking / listening)(burgoon).  Furthermore, (Dunbar, Burgoon 2005) found that the more body control an individual had the more observers perceived her as dominant. In addition, dominant individual are more susceptible to use gesture when they talk.  
\par  Second, voice cues of dominance manifest by speaking duration, speaking intensity, voice control and pitch. Dominant individuals speak loudly and more frequently and might cause interruption while other are talking (Dunbar, Burgoon 2005).

 \subsubsection{Verbal behavior}
verbal behavior of dominance in the dialogue is related to the type of \textit{strategies} that individuals choose in order to take control of the other especially during a negotiation. Therefore, dominant negotiators tend to end up with the larger share of the pie (Giebels, De Dreu, \& Van de Vliert, 2000). 
\par A considerable body of research has documented the effects of dominance on negotiation behaviors and outcomes. First, dominant negotiator have higher aspirations, demands more and concede less(De Dreu, 1995). Second, dominant negotiator control the flow of the negotiation. Indeed, dominant behavior increase task orientation and goal-directed behaviour (Galinsky,Gruenfeld, \& Magee, 2003). Finally, dominant negotiator tend to not pay attention to less dominant negotiators Fiske (1993). The idea is that high-dominant individuals have many resources and can often act at will without serious consequences, while submissive individuals, have to be more careful because they are more dependent on other people. In addition, they are motivated to gain or regain control over their outcomes by paying close attention to the people on whom they depend.

\par We are interested in this paper to the verbal behaviors for two main reasons. First, verbal behaviors are directly related to \emph{the strategies} deployed during the negotiation. In addition our dialogue system is text oriented, which make the non-verbal behavior impossible to reflect during the dialogue. 
We present in the next section the decision model based on dominance behaviors.

\section{Decision in negotiation of preferences}
During the dialogue, agent and user enunciate utterances, in order to express and negotiate about their preferences. Therefore, at each talk turn, the agent has to reason about the utterance to choose, by observing  the current state of the negotiation and its perception of the relation of dominance. In addition, the user is allowed to choose any of the five utterance. We, thus built a model based on dialogue trees that consider all theses parameter to compute the best agent response. 

\par For each possible utterance chosen by the user, we define a dialogue tree with all the agent possible responses. The tree root is the user utterance. The branches of the tree aim to reflect the possible behaviors related to the dominance in negotiation. 
 The verbal behaviors of dominance that we represent in our model cover three principals:

\begin{enumerate}
	\item Dominant negotiator tends to not pay attention to the other negotiator and is only concerned by satisfying its preferences. While, the submissive negotiator is dependent and take in consideration the other negotiator preferences  to make a decision.
	\item Dominant negotiator is very demanding and  refuse to make concessions, while the submissive agent is the opposite.
	\item Dominant negotiator tends to take the control of the dialogue, in term of controlling the flow of the dialogue and orient the negotiation. 
\end{enumerate}

We note that a peer negotiator, since it is a \emph{cooperative negotiation} adopts a behavior in which he tries to find a threshold between its preferences and the other negotiator preferences.

\par We explicit in the following our dialogue trees and explain how they capture the specific behaviors of dominance. First, each branch of a tree  is defined with an applicability condition that respects the perceived relation of the dominance in addition to the current state of the negotiation. 

\par The first principal we investigate is the level of demand and concession of the agent depending on the relation of dominance. Indeed, in our model of negotiation, the agent is able to rate each criterion and option using functions presented in section \ref{scores}. We define a function (depicted in \fig{pseudo}) that calculates either a value is acceptable for the agent. A dominant agent only likes values with the best scores. Whereas, a submissive agent make concession. Thus, if the other agent insists on a value, the agent will feel obliged to accept this value, because he is dependent to the other. This function is used in several trees in order to compute either a discussed might be accepted.

 	\begin{figure}[]
 		\begin{algorithmic}[1]\small
 			\Function{isAcceptable}{$value$, $relation$}
 			
 			\If{($relation = dominant$)} 
 			\State return $(score(value)> bestScoreOfPreference \times 0.7)$
 			\EndIf
 			\If{($relation = peer$)} 
 			\State return $(score(value)> 0)$
 			\EndIf
 			\If{($relation = sub$)} 
 			\State return $(score(value)> 0$  or $isInOAS(value))$
 			\EndIf
 			\EndFunction
 		\end{algorithmic}
 		\vskip 8pt
 		\defig{pseudo}{Function to compute the acceptability of a value of preference}
 	\end{figure} 
\subsection{Open a negotiation}
When a cooperative negotiate starts, the agent and the other will follow a specific strategy depending on their perception of their relation of dominance. Therefore, we defined a tree dialogue that define how the agent opens a negotiation with respect of its perception of the relation.
\subsubsection{Dominant / submissive start}

The third principal of dominance explains that a dominant agent tend to control the conversation and the first one explains that he only cares about his preferences. Therefore, a dominant agent will propose his most preferred value to open the negotiation. In the contrary, a submissive agent is dependent. Therefore, he will wait to the other to open the negotiation.

\subsubsection{Peer start} Since, the negotiation is cooperative, a peer agent will tend to share his knowledge in order to find 
the best compromise between both preferences. Thus, a peer agent will state his preferences about his most preferred or important criterion.
\subsection{StatePreference (less,more)}
When the user states a preference that we note StatePreference(less,more), where $less, more \in C$.  We modeled the tree of choices, where each branch depends on the satisfiability of an applicability condition. Each branch of the trees is explained bellow:
 		\subsubsection{Propose(More):} The agent might propose the stated value. A propose is made only if the value is \emph{acceptable} for the agent. The acceptability of a value is computed with the function \emph{isAcceptable} (see \fig{pseudo}) as explained previously. Furthermore, this behavior reproduce the first principal of dominant behavior, the dominant agent only proposes something he likes, whereas the submissive agent is able to make concession if the other keeps proposing this value. 
 		
 		\subsubsection{StatePreference(less', more')} :If the stated preference don't meet the agent preferences. The agent will express its preferences about the expressed preferences using the algorithm depicted in \fig{react}, in order to make the dialogue going and share knowledge with the other.
 	\begin{figure}[h]
 		\begin{algorithmic}[1]\small
 			\Function{reactToUser}{$less$, $more$}
 			
 			\If{($more = MostPrefferedValue()$ and $ (*, more) \notin OAS$)} 
 			\State return $(*, more)$
 			\EndIf
 			\If{($more = LeastPrefferedValue()$ and $ (more, *) \notin OAS$)} 
 			\State return $(more, *) $
 			\EndIf
 			\State The conditions $(2,4)$ are also applied for $less$
 			\State $(less',more') = computePreference(less,more)$
 			\If {$(less',more') \in OAS$}
 			\State $(less1,more1)= reactToCriterion(more)$
 			\If {$(less1,more1)=null$}
 			\State $(less2,more2)= reactToCriterion(less)$
 			\If{ $(less2,more2)=null$}
 			\State return $(less',more')$
 			\Else  
 			return $(less2,more2)$
 			\EndIf
 			\EndIf
 			\EndIf
 			\State return $(less', more')$
 			\EndFunction
 			
 		\end{algorithmic}
 		\vskip 8pt
 		\defig{react}{Function to react to a preference (less,more)}
 	\end{figure}
 
 \subsection{Propose(Proposal)}
 
 \subsubsection{Booking:} In the case where the other proposes an option which \emph{isAcceptable} for the agent, this later calls the task \textit{Booking}, to ask the user to book a table for the restaurant. In addition, the agent has to be dominant, which represents the third principal, where the agent controls the flow the conversation.
 
 \subsubsection{Accept(Proposal):} the user proposes a value for a criterion that \emph{isAcceptable} for the agent. Thus, the agent expresses an accept.
 
 
 \subsubsection{Propose(Proposal'):} The other proposes a proposal which is not acceptable for the agent. This branch is applicable only if the agent is dominant. Indeed, the dominant agent is demanding and self centered. He will thus try to control the flow of the negotiation in order to satisfy his preferences and  make other proposal which he's susceptible to accept.
 
 \subsubsection{StatePreference(less,more):} The user proposes a proposal which is not acceptable for the agent, and the agent is submissive and dependent in its relation with the user. Instead of expressing a reject, the submissive agent will express a statement to express that the proposed value doesn't suit its preferences. In the case of an option, the agent computes the least scored value of this option to explain why he doesn't accept the proposal.
 
 \subsubsection{Reject(Proposal):} In the same case where the proposed value is not acceptable but the agent has not to be submissive to express his reject.
 
 \subsection{Accept(Proposal)}
 \subsubsection{Booking: } When the other accepts an option proposal , the agent closes the negotiation by proposing to book a table for the accepted restaurant (option).
 
 \subsubsection{Propose(ProposalO):} With respect of the third principal, when a user accepts a criterion proposal, a non-submissive agent keeps  the negotiation going and proposes an option that is defined with the accepted value. Moreover, the proposed option suits it preferences as presented in the first principal. 
 \par A submissive agent will only propose an option, if he collects knowledge on the preferences of the other for all the values of each criterion of the topic.  The proposed option is defined with all the accepted values.
 \par after accepting a value for a criterion, the agent opens a negotiation about another criterion of the topic. We distinguish three different responses with respect of the first principal of dominant behavior, that we present in the next sections.
 
 \subsubsection{StatePreference(less,more)}:  A peer agent opens a new subtopic or a negotiation on a criterion by expressing his preferences on this criterion.
 \subsubsection{AskPreference(less,more)}:  A submissive agent considers the preferences of the other in his decision making. Thus, to open a new negotiation on a criterion, he asks the other about his preferences on this criterion.
 \subsubsection {Propose(Proposal):} a dominant agent only considers his preferences. Thus, he directs the negotiation to satisfy them, by proposing his most preferred value on the new criterion. 
 
 \subsection{Reject(Proposal)}
 	When the other rejects an agent proposal, the agent reacts differently if the proposal is either is an option or a criterion proposal.
 		\subsubsection{Reject(criterion)}
 		\begin{itemize}
 			\item \textbf{Propose(Proposal):} This branch of the tree is called if the rejected proposal value is the agent most preferred value for the discussed criterion, in addition, the agent is dominant. Therefore, with respect of all the principals of dominant behaviors , the agent will repropose over the rejected proposal.
 			\item \textbf{AskPreference(less,more):} In the opposite of the previous branch, when a submissive agent receives a reject, and because of his dependence to the other. he will assume that his knowledge is not sufficient to negotiate. Therefore, he asks the user about his preferences on the current discussed negotiation. 
 			\item \textbf{StatePreference(less,more):} A peer agent will keep the negotiation going by sharing his preferences on the current discussed criterion.
 			\item\textbf{Propose(Proposal'):} A dominant agent leads the negotiation and keeps on proposing values on the discussed criterion. This behavior respects out third principal of dominant behavior.
 			
 		\end{itemize}

 		
 		\subsubsection{Reject(Option)}
 		\begin{itemize}
 			\item \textbf{Propose(Proposal):} If the reject option is the most preferred option of the agent, and the agent is dominant, he will keep proposing this option, which depicted the first and third principals of dominant behaviors.
 			
 			\item \textbf{AskPreference(less,more)}: An option proposal is always proposed by the agent when at least a criterion proposal has been accepted. Therefore, if the user rejects an option proposal, means that the agent still ignores the other preferences about the different criteria of an option. The submissive agent which is dependent to the other will try to gather more knowledge.
 			
 			\item \textbf{StatePreference(less,more)} : the peer agent follows the same behavior of gathering knowledge on preferences about the remaining  criteria oh the options. Therefore, he opens the negotiation by stating his preferences on the new discussed criterion. By consequence, the agent manages the flow of the negotiation. 
 			
 			\item \textbf{Propose(Proposal'):} when a non submissive agent receive a reject, he will want to continue the negotiation by proposing other proposals. The agent continues the negotiation by proposing other proposals, which the third principal.
	\end{itemize}
\subsection{AskPreference(less, more)} Inependently to the relation of dominance, when the agent has been asked to express his preferences. The only possible response is to state his preferences \textbf{StatePreference(less, more)}.



\section{Procedure of validation}
We defined in our model a set of behaviors related to the dominance. We aim in this study to validate the perception of these behaviors and their coherence during the dialogue.
\par we thus conduct a first study in which participants analyze the behavior of  two agents (agentA, agentB)  that participate to a cooperative negotiation on the topic of restaurants. Each agent runs our model of negotiation and share a perception of their relation of dominance. We present in this section the conditions of our study.

\subsection{Conditions of the study}
We define two conditions for our study. Fist, the perception of the relation of dominance which is asymmetric and static in time. Second, the difference between the preferences of both agents. Indeed, agents which share similar preferences will tend to converge easily to a solution. 


\noindent 
\vskip 4pt
\bibliographystyle{plain}
\bibliography{abbrevs,Library}
\end{document}
